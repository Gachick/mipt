\documentclass{article}

\usepackage{amsmath}
\usepackage[utf8]{inputenc}
\usepackage[T2A]{fontenc}

\usepackage{hyperref}
\hypersetup{
    colorlinks=true, %set true if you want colored links
    linktoc=all,     %set to all if you want both sections and subsections linked
    linkcolor=black,  %choose some color if you want links to stand out
}

\begin{document}

\tableofcontents

\section{Электрическое поле}
\[
    \vec{E} = k\frac{q}{r^3}\vec{r} ,\;\text{где }
    k=\frac{1}{4\pi\varepsilon_0}=9\cdot10^9 \,\text{Ф/м} ,\;
    \varepsilon_{0} = 8,85\cdot10^{-12} \,\text{Ф/м}
\]
В случае непрерывного распределения заряда:
\[\vec{E} = \frac{1}{4\pi\varepsilon_0}\int\frac{\rho\vec{r}dV}{r^3}\]

\subsection{Теорема Гаусса}
\[\Phi=\int\limits_{S}\vec{E}\vec{dS}\]
\[\oint\vec{E}\vec{dS}=\frac{1}{\varepsilon_{0}}q_\text{внутр}\]
\subsubsection{Частные случаи}
\begin{itemize}
    \item Плоскость: $E=\frac{\sigma}{2\varepsilon_{0}\varepsilon}$
    \item Стена ширины d: $E=\begin{cases}
                  \frac{\rho x}{\varepsilon_{0}}   & ,x<\frac{d}{2}   \\
                  \frac{\rho d}{2\varepsilon_{0}} & ,x\ge\frac{d}{2}
              \end{cases}$
    \item Цилиндр: $E=\begin{cases}
        \frac{\rho r}{2 \varepsilon_0} & ,r<R \\
        \frac{\sigma R}{\varepsilon_0 r} & ,r>R
    \end{cases}$
    \item Шар: $E=\begin{cases}
                  \frac{Qr}{4\pi\varepsilon_{0}R^3}  & ,r<R \\
                  \frac{Q}{4\pi\varepsilon_{0}r^{2}} & ,r>R
              \end{cases}$
\end{itemize}
\subsubsection{Дифференциальный вид}
\[\nabla\cdot\vec{E}=\frac{\rho}{\varepsilon_{0}}\]

\subsection{Потенциал}
Теорема о циркуляции вектора E:
\[\oint_{1}^{2}\vec{E}\vec{dl}=0\]
Электростатическое поле является потенциальным.
\[\varphi_{1}-\varphi_{2}=\int_{1}^{2}\vec{E}\vec{dl}\]
Потенциал - величина, численно равная потенциальной энергии единичного положительного заряда в данной точке.
\[
    -d\varphi=\vec{E}\vec{dl} \qquad
    \varphi=\frac{1}{4\pi\varepsilon_{0}}\frac{q}{r}
\]
Потенциал на бесконечности ($r\to\inf$) полагаем равным 0.
\[\vec{E}=-\nabla\varphi\]

\subsection{Электрический диполь}
Момент диполя:
\[\vec{p}=q\vec{l}\]
где $\vec{l}$ направлен от - к +, q - положительный заряд
\subsubsection{Потенциал поля диполя}
\[
    \varphi=\frac{1}{4\pi\varepsilon_{0}}(\frac{q}{r_{+}}-\frac{q}{r_{-}})=
    \frac{1}{4\pi\varepsilon_{0}}\frac{p\cos\theta}{r^2}
\]
где $\theta$ - угол между p и r
\subsubsection{Напряженность поля диполя}
\[
    E_{r}=-\frac{\partial\varphi}{\partial r}=\frac{1}{4\pi\varepsilon_{0}}\frac{2p\cos\theta}{r^3} \qquad
    E_{\theta}=-\frac{\partial\varphi}{r\partial\theta}=\frac{1}{4\pi\varepsilon_{0}}\frac{p\sin\theta}{r^3} \qquad
\]
\[E=\sqrt{E_{r}^2+E_{\theta}^2}=\frac{1}{4\pi\varepsilon_{0}}\frac{p}{r^3}\sqrt{1+3\cos^2\theta}\]
В частности при $\theta = 0$ и $\theta = \frac{\pi}{2}$
\[
    E_{\parallel} = \frac{1}{4\pi\varepsilon_{0}}\frac{2p}{r^3} \qquad
    E_{\perp} = \frac{1}{4\pi\varepsilon_{0}}\frac{p}{r^3}
\]
\subsubsection{Сила действующая на диполь}
\[\vec{F}=p\frac{\partial\vec{E}}{\partial\vec{l}}\]
\subsubsection{Момент сил действующих на диполь}
\[M=qEl\sin\alpha=pE\sin\alpha\]
\[\vec{M}=[\vec{p}\times\vec{E}]\]
\subsubsection{Энергия диполя в поле}
\[W=q(\varphi_{+}-\varphi_{-})=q\frac{\partial\varphi}{\partial l}l=-qE_{l}l=-\vec{p}\vec{E}\]


\section{Проводник в электростатическом поле}

\subsection{Поле внутри проводника}
Внутри проводника:
\[
    \vec{E}=0 \qquad \rho_{\text{внутр}}=0 \qquad \varphi=const
\]
Поверхность проводника эквипотенциальна!

\subsection{Поле у поверхности проводника}
\[E_{n}=\frac{\sigma}{\varepsilon_{0}}\]
где $\sigma$ - локальная плотность заряда.

\subsection{Силы, действующие на поверхность проводника}
\[\varDelta\vec{F}=\sigma\varDelta S\vec{E}_{0}\]
где $\sigma\varDelta S$ - заряд элемента, $\vec{E}_{0}$ - напряженность, создаваемая остальными зарядами.
\[
    E_{\sigma}=E_{0} \qquad
    \vec{E}_{0}=\frac{\vec{E}}{2} \qquad
    \varDelta\vec{F}=\frac{\sigma\varDelta S\vec{E}}{2}
\]
\[
    \vec{F}_{\text{ед}}=\frac{\sigma\vec{E}}{2}=\frac{\sigma^2\vec{n}}{2\varepsilon_{0}}=\frac{\varepsilon_{0}E^2\vec{n}}{2}
\]

\subsection{Замкнутая оболочка}
Замкнутая проводящая оболочка разделяет все пространство на внешнюю и внутреннюю части, в электрическом отношении совершенно не зависящие друг от друга

\subsection{Конденсатор}
Для изолированного проводника:
\[C=\frac{q}{\varphi}\]
Для кондесатора:
\[C=\frac{q}{U}\]
\subsubsection{Частные случаи ёмкостей}
\begin{itemize}
    \item Плоский: $C=\frac{\varepsilon_{0}S}{h}$
    \item Сферический: $C=4\pi\varepsilon_{0}\frac{ab}{a-b}$
    \item Цилиндрический: $C=\frac{2\pi\varepsilon_{0}l}{\ln(b/a)}$
\end{itemize}


\section{Электрическое поле в веществе}
Свзянные заряды и их поле помечаются штрихом ($q', \rho', \sigma', \vec{E}'$), сторонне поле обозначено как $\vec{E}_{0}$.

\subsection{Поляризованность P}
- дипольный момент объёма вещества
\[\vec{P}=\frac{1}{\Delta V}\sum\vec{p_{i}} \qquad \vec{P}=\eta\langle\vec{p}\rangle\]
Для изотропного диэлектрика:
\[\vec{P}=\kappa\varepsilon_{0}\vec{E} ,\;\text{где } \kappa=\varepsilon-1\ \;\text{- диэлектрическая восприимчивость}\]
\subsubsection{Теорема Гаусса}
\[\oint\vec{P}\vec{dS}=-q'_{\text{внутр}}\]
\subsubsection{Граничные условия}
\[P_{2n}-P_{1n}=-\sigma'\]
где индекс n означает проекцию на нормаль

\subsection{Вектор D}
\[\vec{D}=\varepsilon_{0}\vec{E}+\vec{P} \qquad \vec{D}=\varepsilon\varepsilon_{0}\vec{E}\]
\subsubsection{Теорема Гаусса}
\[\oint\vec{D}\vec{dS}=q_{\text{внутр}}^{\text{сторонние}}\]

\subsection{Условия на границе}
\subsubsection{Два диэлектрика}
\[\oint\vec{E}\vec{dS}=0 \qquad \oint\vec{D}\vec{dS}=q_{\text{внутр}}\]
\[E_{1\tau}=E_{2\tau} \qquad D_{2n}-D_{1n}=\sigma=0\]
\[\frac{\tan\alpha_{2}}{\tan\alpha_{1}}=\frac{\varepsilon_{2}}{\varepsilon_{1}}\]
\subsubsection{Проводник - диэлектрик}
\[D_{n}=\sigma \Rightarrow \sigma'=-\frac{\varepsilon-1}{\varepsilon}\sigma\]

\subsection{Поле в однородном диэлектрике}
\[\vec{E}=\frac{\vec{E}_{0}}{\varepsilon} \qquad \vec{D}=\vec{D}_{0}\ \qquad \vec{E}'=-\vec{P}/\varepsilon_{0}\]


\section{Энергия электрического поля}

\subsection{Электрическая энергия системы зарядов}
\[\delta A=-dW \qquad W=\frac{1}{2}\sum q_{i}\varphi_{i}\]
где $q_{i}$ - i-й заряд системы, $\varphi_{i}$ - потенциал, создаваемый в месте нахождения i-ro заряда всеми остальными зарядами.
\[W=\frac{1}{2}\int\rho\varphi dV\]
где $\varphi$ - потенциал, создаваемый всеми зарядами системы в объеме dV.

\subsection{Энергия заряженных проводников и кондесатора}
Уединенный проводник:
\[W=\frac{q\varphi}{2}=\frac{C\varphi^2}{2}=\frac{q^2}{2C}\]
Конденсатор:
\[W=\frac{qU}{2}=\frac{CU^2}{2}=\frac{q^2}{2C}\]

\subsection{Энергия поля}
\[W=\int\frac{\varepsilon\varepsilon_{0} E^2}{2}dV=\int\frac{\vec{E}\vec{D}}{2}dV\]
\subsubsection{Объёмная плотность}
\[w=\frac{\varepsilon\varepsilon_{0} E^2}{2}=\frac{\vec{E}\vec{D}}{2}\]


\section{Постоянный электрический ток}
\[I=\frac{dq}{dt}\]

\subsection{Плотность тока, уравнение непрерывности}
\[j=\frac{dI}{dS_{\perp}} \qquad I=\int\vec{j}\vec{dS}\]
Вектор плотности тока - отношение силы тока и площадки в данной точке, перпендикулярной ему. Сонаправлен с вектором скорости положительных частиц.
\[\oint\vec{j}\vec{dS}=-\frac{dq}{dt}\]
В случае постоянного тока, распределение не изменяется:
\[\oint\vec{j}\vec{dS}=0\]
Дифференциальная форма:
\[\nabla\cdot\vec{j}=-\frac{\partial\rho}{\partial t}\]

\subsection{Закон Ома для однородного проводника}
\[I=\frac{U}{R} \qquad R=\rho\frac{l}{S}\]
\[\vec{j}=\frac{1}{\rho}\vec{E}=\lambda\vec{E}\]
$\lambda$ - удельная электропроводность, сименс на метр (См/м)
\subsubsection{Закон Ома для неоднородного участка}
При наличии сторонних (некулоновских) сил ($\vec{E}^{*}$), обобщённый закон Ома:
\[\vec{j}=\lambda(\vec{E}+\vec{E}^{*})\]
\[I\int_{1}^{2}\rho\frac{dl}{S}=\int_{1}^{2}\frac{\vec{j}\vec{dl}}{\lambda}=\int_{1}^{2}\vec{E}\vec{dl}+\int_{1}^{2}\vec{E}^{*}\vec{dl}\]
\[RI=\varphi_{1}-\varphi_{2}+\varepsilon_{12}\]
$\varepsilon_{12}$ - электродвижущая сила

\subsection{Правила Кирхгофа}
В узле алгебраическая сумма токов равна 0:
\[\sum I_{k}=0\]
В замкнутом контуре:
\[\sum I_{k}R_{k}=\sum\varepsilon_{k}\]

\subsection{Закон Джоуля-Ленца}
\[Q=RI^{2} \qquad Q_{\text{удельная}}=\rho j^{2}=\vec{j}\vec{E}=\lambda E^{2}\]
В неоднородной среде:
\[Q=\varepsilon I \qquad Q_{\text{удельная}}=\rho j^{2}=\vec{j}(\vec{E}+\vec{E}^{*})\]

\subsection{Переходные процессы в цепи с конденсатором}
\subsubsection{Разрядка конденсатора}
\[RI=U \qquad \frac{dq}{dt}+\frac{q}{RC}=0\]
\[q=q_{0}e^{-t/\tau} \qquad \tau=RC\]
\[I=-\frac{dq}{dt}=I_{0}e^{-t/\tau} \qquad I_{0}=\frac{q_{0}}{\tau}\]
\subsubsection{Зарядка кондесатора}
\[RI=\varphi_{1}-\varphi_{2}+\varepsilon \qquad \frac{dq}{dt}=\frac{\varepsilon-q/C}{R}\]
\[RC\ln(1-\frac{q}{\varepsilon C})=-t \qquad q=q_{m}(1-e^{-t/\tau})\]
\[I=\frac{dq}{dt}=I_{0}e^{-t/\tau} \qquad I_{0}=\frac{\varepsilon}{R}\]


\section{Магнитное поле}
Магнитное поле равномерно движущегося заряда:
\[
    \vec{B}=\frac{\mu_{0}}{4\pi}\frac{q[\vec{v}\times\vec{r}]}{r^3}
    ,\; \text{где } \mu_{0}=4\pi\cdot10^{-7} \, \text{Гн/м}
\]
\[\vec{B}=\varepsilon_{0}\mu_{0}[\vec{v}\times\vec{E}]\]

\subsection{Сила Лоренца}
\[
    \vec{F}_{L}=[\vec{v}\times\vec{B}]q
\]

\subsection{Закон Био-Савара}
\[\vec{dB}=\frac{\mu_{0}}{4\pi}\frac{I[\vec{dl}\times\vec{r}]}{r^3}\]
Магнитное поле на оси кругового тока: $B=\frac{\mu_{0}}{4\pi}\frac{2\pi R^2I}{(z^2+r^2)^{\frac{3}{2}}}$

\subsection{Теорема Гаусса}
\[\oint\vec{B}\vec{dS}=0 \qquad \nabla\cdot\vec{B}=0\]

\subsection{Теорема о циркуляции B}
\[\oint\vec{B}\vec{dl}=\mu_{0}I\]
Циркуляция вектора $\vec{B}$ по произвольному контуру Г равна произведению $\mu_{0}$ на алгебраическую сумму токов, охватываемых контуром Г.
\subsubsection{Частные случаи}
\begin{itemize}
    \item Прямой провод: $B=\frac{\mu_{0}I}{2\pi r}, \; r \ge R$
    \item Внутри длинного соленоида: $B=\mu_{0}\mu nI$
    \item Плоскость с током: $B=\frac{\mu_{0}l}{2}$, где l сторона контура, параллельная плоскости
\end{itemize}
\subsubsection{Дифференциальный вид}
\[\lim_{S \to 0}\frac{\oint\vec{B}\vec{dl}}{S}=(rot\,\vec{B})_{n}\]
Правая часть - проекция ротора на нормаль площадки, полученную по правилу правого винта.
\[
    \nabla \times \vec{B} = \begin{vmatrix}
        e_{x}                       & e_{y}                       & e_{z}                       \\
        \frac{\partial}{\partial x} & \frac{\partial}{\partial y} & \frac{\partial}{\partial z} \\
        B_{x}                       & B_{y}                       & B_{z}
    \end{vmatrix}
\]
\[
    \nabla \times \vec{B} = \mu_{0}\vec{j}
\]
Где $\vec{j}=\frac{\vec{I}}{S}$ - плотность тока.

\subsection{Сила Ампера}
\[
    \vec{dF}_{A}=\rho[\vec{v}\times\vec{B}]dV \qquad
    \vec{dF}_{A}=I[\vec{dl}\times\vec{B}] \qquad
    \vec{F}_{A}=l[\vec{I}\times\vec{B}]
\]
\subsubsection{Частные случаи}
\begin{itemize}
    \item Параллельные токи (на расстояние h) на единицу длины: $F = \frac{\mu_{0}}{4\pi}\frac{2I_{1}I_{2}}{h}$
    \item Контур с током при постоянном B: $\vec{F}=I\oint[\vec{dl}\times\vec{B}]=I[(\oint\vec{dl})\times\vec{B}]=I[0\times\vec{B}]=0$
\end{itemize}
\subsubsection{Работа при перемещении контура}
\[\delta A=Id\Phi \]
где $d\Phi$ - приращение магнитного потока сквозь контур.\\
В случаем подвижной перемычки:
\[\delta A=Fdx=IBldx=IBdS\]

\subsection{Элементарный контур (магнитный диполь)}
Поведение плоского малого контура описывается магнитным моментом: $\vec{p}_{m}=IS\vec{n}$, где S - площать контура, $\vec{n}$ - нормаль по правилу правого винта.
\subsubsection{Сила действующая на контур}
В неоднородном магнитном поле по закону ампера получаем:
\[\vec{F}=p_{m}\frac{\partial\vec{B}}{\partial\vec{n}}\]
где $p_{m}$ - модуль момента, $\frac{\partial\vec{B}}{\partial\vec{n}}$ - производная по направлению нормали $\vec{n}$.
\subsubsection{Момент сил}
\[\vec{M}=[\vec{p}_{m}\times\vec{B}]\]


\section{Магнитное поле в веществе}

\subsection{Намагниченность J}
\[\vec{J}=\frac{1}{\Delta V}\sum \vec{p}_{m} \qquad \vec{J}=n\langle\vec{p}_{m}\rangle\]
\subsubsection{Циркуляция}
\[\oint\vec{J}\vec{dl}=I' \qquad I'=\int\vec{j}'\vec{dS}\]
где $I'$ - алгебраическая сумма токов намагничивания в контуре, а \\ $\vec{j}'$ - объёмная плотность тока намагничивания, интегрирование по \\ произвольной поверхности, натянутой на контур.
\[\nabla\times\vec{J}=\vec{j}'\]
\subsubsection{Когда j'=0}
\begin{itemize}
    \item магнетик однородный
    \item внутри магнетика нет токов проводимости
\end{itemize}

\subsection{Вектор H}
\[\oint\vec{B}\vec{dl}=\mu_{0}(I+I')\]
\[\vec{H}=\frac{\vec{B}}{\mu_{0}}-\vec{J} \qquad \oint\vec{H}\vec{dl}=I\]
где $I$ - алгебраическая сумма токов проводимости
\[\nabla\times\vec{H}=\vec{j}\]
где $\vec{j}$ - плотность тока проводимости
\subsubsection{Связь J и H}
\[\vec{J}=\chi\vec{H}\]
$\chi$ - магнитная восприимчивость
\begin{itemize}
    \item пармагнетики $\chi > 0, \; \vec{J}\uparrow\uparrow\vec{H}$
    \item диамагнетики $\chi > 0, \; \vec{J}\uparrow\downarrow\vec{H}$
    \item ферромагнетики, $J$ зависит от предыистории (гистерезис)
\end{itemize}
\subsubsection{Связь B и H}
\[\vec{B}=\mu\mu_{0}\vec{H} \qquad \mu=1+\chi\]

\subsection{Граничные условия B и H}
\[\oint\vec{B}\vec{dl}=0 \qquad \oint\vec{H}\vec{dl}=I\]
\[B_{2n}\Delta S + B_{1n}\Delta S=0 \qquad B_{2n}=B_{1n}\]
\[H_{2\tau}l+H_{2\tau}l=i_{N}l \qquad H_{2\tau}-H_{1\tau}=i_{N}\]
$\vec{N}$ - нормаль к контуру, $i$ - плотность токов проводимости
\subsubsection{Преломление линий B (H)}
\[\frac{\tan\alpha_{2}}{\tan\alpha_{1}}=\frac{\mu_{2}}{\mu_{1}}\]


\section{Относительность полей}
Система отсчёта $K'$ движется относительно системы $K$, тогда \textbf{локально} верны следующие соотношения:
\[\vec{E}'_{\parallel}=\vec{E} \qquad \vec{B}'_{\parallel}=\vec{B}\]
\[\vec{E}'_{\perp}=\frac{\vec{E}_{\perp}+[\vec{v}_{0}\times\vec{B}]}{\sqrt{1-\beta^{2}}} \qquad \vec{B}'_{\perp}=\frac{\vec{B}_{\perp}+[\vec{v}_{0}\times\vec{E}]}{\sqrt{1-\beta^{2}}}\]
где $\beta = v_{0}/c $

\subsection{Простые следствия}
\begin{itemize}
    \item В системе K только E: $\vec{B}'=-[\vec{v}_{0}\times\vec{E}']$
    \item В системе K только B: $\vec{E}'=[\vec{v}_{0}\times\vec{B}']$
\end{itemize}

\subsection{Инварианты}
\[\vec{E}\vec{B}=inv \qquad E^2-c^2B^2=inv\]


\section{Электромагнитная индукция}

\subsection{Правило Ленца}
Индукционный ток направлен так, чтобы противодействовать причине, его вызывающей.

\subsection{Закон Фарадея}
\[\varepsilon=\oint\vec{E}\vec{dl}=-\frac{d\Phi}{dt} \qquad \nabla\times E=-\frac{\partial B}{\partial t}\]
При нескольких витках $\varepsilon=-N\frac{d\Phi}{dt}$
В полном виде:
\[\oint\vec{E}\vec{dl}=-\frac{\partial\Phi}{\partial t}+\oint[\vec{v}\times\vec{B}]\vec{dl}\]
Первое слагаемое связано с изменением магнитного поля во времени, второе - с движением контура.

\subsection{Самоиндукция}
\[\Phi=LI\]
$L$ - индуктивность (e.g. соленоид $L=\mu\mu_{0}n^{2}V$)
\[\varepsilon_{s}=-\frac{\partial\Phi}{\partial t}=-\frac{d}{dt}(LI)=-L\frac{dI}{dt}\]
\subsubsection{Исчезновение тока при размыкании цепи}
\[RI=\varepsilon_{s}=-L\frac{dI}{dt} \qquad I=I_{0}e^{-t/\tau}\]
где $\tau=L/R$ - время релаксации.
\subsubsection{Установление тока при замыкании цепи}
\[RI=\varepsilon-L\frac{dI}{dt} \qquad I=I_{0}(1-e^{-t/\tau})\]

\subsection{Взаимная индукция}
Два неподвижных контура достаточно близких к друг другу:
\[\Phi_{2}=L_{21}\cdot I_{1} \qquad \Phi_{1}=L_{12}\cdot I_{2}\]
Где $L_{12}=L_{21}=M$ - взаимная индуктивность, при отсутствии поблизости ферромагнетиков. (Может быть и отрицательна, в отличие от L.)

\subsection{Энергия магнитного поля}
Работа совершаемая сторонними силами против эдс самоиндукции:
\[\varepsilon_{0}=RI-\varepsilon_{s} \; |\cdot Idt \; \Rightarrow \; \delta A_{\text{стор}}=\delta Q + Id\Phi\]
\[\delta A^{\text{доп}}=Id\Phi\]
Считаем что ферромагнетиков нет:
\[d\Phi=LdI \; \Rightarrow \; A^{\text{доп}}=\frac{LI^2}{2}\]
\[W=\frac{LI^2}{2}=\frac{I\Phi}{2}=\frac{\Phi^2}{2L}\]
\[W=\int\frac{\vec{B}\vec{H}}{2}dV \qquad w=\frac{\vec{B}\vec{H}}{2}=\frac{B^2}{2\mu\mu_{0}}\]
\[L=\frac{1}{I^2}\int\frac{B^2}{\mu\mu_{0}}dV\]

\subsection{Магнитная энергия двух контуров}
\[W=\frac{L_1I_1^2}{2}+\frac{L_2I_2^2}{2}+MI_1I_2\]
\[W=\int\frac{B_1^2}{2\mu\mu_0}dv+\int\frac{B_2^2}{2\mu\mu_0}dv+\int\frac{\vec{B}_1\vec{B}_2}{\mu\mu_0}dv\]
\[M=\frac{1}{I_1I_2}\int\frac{\vec{B}_1\vec{B}_2}{\mu\mu_0}dv\]

\subsection{Энергия и силы в магнитном поле}
\[\delta A^*=\delta Q + dW + \delta A_{\text{мех}} \qquad \delta A^{\text{доп}}=I_1d\Phi_1+I_2d\Phi_2\]
где $A^*$ - работа источника тока, $\delta Q$ - потери тепловыделения, $\delta A_{\text{мех}}$ - работа на перемещение и деформацию контуров, dW - прирост магнитной энергии.
\[I_1d\Phi_1+I_2d\Phi_2=dW+dA_{\text{мех}}\]
Следствия:
\begin{itemize}
    \item При постоянных потоках: $\delta A_{\text{мех}}=-dW \Big\rvert_{\Phi}$
    \item При постоянных токах: $\delta A_{\text{мех}}=-dW \Big\rvert_{I}$
\end{itemize}


\section{Уравнения Максвелла}

\subsection{Ток смещения}
\[\oint \frac{\partial \vec{D}}{\partial t}\vec{dS}=\frac{\partial q}{\partial t} \qquad \oint \vec{j}\vec{dS}=-\frac{\partial q}{\partial t}\]
\[\oint (\vec{j} + \frac{\partial \vec{D}}{\partial t})\vec{dS}=0\]
$\vec{j}_{\text{см}}=\partial\vec{D}/\partial t$ - ток смещения, $\vec{j}_{\text{полн}}=\vec{j}+\frac{\partial \vec{D}}{\partial t}$ - полный ток
\[\oint \vec{H}\vec{dl}=I_{\text{полн}}=\int(\vec{j}+\frac{\partial \vec{D}}{\partial t})\vec{dS}\]
\[\nabla \times \vec{H}=\vec{j}+\frac{\partial \vec{D}}{\partial t}\]

\subsection{Система уравнений Максвелла}
\[\boxed{\begin{aligned}
& \int \vec{E}\vec{dl}=-\int \frac{\partial \vec{B}}{\partial t}\vec{dS}  & & \oint \vec{B}\vec{dS}=0 \\
& \oint \vec{H}\vec{dl}=\int(\vec{j}+\frac{\partial \vec{D}}{\partial t})\vec{dS} & & \oint\vec{D}\vec{dS}=\int\rho dV 
\end{aligned}}\]
\[\boxed{\begin{aligned}
& \nabla \times \vec{E}=-\frac{\partial \vec{B}}{\partial t} & & \nabla \cdot \vec{B}=0 \\
& \nabla \times \vec{H} = \vec{j} + \frac{\partial \vec{D}}{\partial t} & & \nabla \cdot \vec{D} = \rho
\end{aligned}}\]
Вместе с силой Лоренца $d\vec{p}/dt=q\vec{E}+q[\vec{v}\times\vec{B}]$ составляют фундаментальную систему

\subsection{Вектор Пойнтинга. Энергия и её поток}
\subsubsection{Теорема Пойнтинга}
\[-\frac{dW}{dt}=\oint\vec{\Pi}\vec{dS}+P\]
где $W=\int wdv$, $P=\int \vec{j}\vec{E}dV$ - мощность силы прикладываемой к зарядам, $\vec{\Pi}$ - вектор Пойнтинга
\[w=\frac{\vec{E}\vec{D}}{2}+\frac{\vec{B}\vec{H}}{2} \qquad \vec{\Pi}=[\vec{E}\times\vec{H}]\]

\subsection{Импульс электромагнитного поля}
\subsubsection{Давление электромагнитной волны}
Электромагнитная волна распространяется в однородной среде, \\ обладающей поглощением (объёмная плотность выделяемой теплоты $\rho j^2=\lambda E^2$)
\[\vec{j}=\lambda \vec{E} \qquad F_{\text{ед}}=[\vec{j} \times \vec{B}]=\lambda [\vec{E}\times\vec{B}]\]
$F_{\text{ед}}$ - сила на единицу объёма, ответсвенная за появление давления и \\ сонаправленная с вектором скорости волны.
\subsubsection{Импульс электромагнитного поля}
Плотность импульса $\vec{G}$ - импульс поля в единице объёма
\[\vec{G}=\vec{\Pi}/c^2\]
В вакууме $G=w/c$
\section{Электрические колебания}
\subsection{Уравнение колебательного контура}
\subsubsection{
    Условие квазистационарности.
}
\[ \tau=\frac{l}{c} << T\]
где T - период изменений 
l - длина цепи 
c - скорость света
\subsubsection{
    Колебательный контур.
}
уравнение колебательного контура
\[L\frac{d^2q}{dt^2}+R\frac{dq}{dt} + \frac{1}{c}q = \varepsilon\] 
иной вид
\[\ddot{q} + 2\beta\dot{q} + \omega_0^2q = \frac{\varepsilon}{L}\]
\[2\beta = \frac{R}{L}, \qquad \omega_0^2 = \frac{1}{LC} \]
\subsection{
    Свободные электрические колебания
}
\subsubsection{
    Свободные незатухающие колебания
}
\[\ddot{q} + \omega_0^2q = 0 \qquad q= q_m\cos(\omega_0t + \alpha)\]
Период (формула Томпсона)
\[T_0 = 2\pi\sqrt{LC}\]
\subsubsection{
    Свободные затухающие колебания
}
\[\ddot{q} + 2\beta\dot{q} + \omega_0^2q = 0 \qquad q= q_m e^{-\beta t} \cos(\omega t + \alpha)\]
\[\omega = \sqrt{\omega_0^2 - \beta^2} \qquad T = \frac{2\pi}{\omega}\]
Напряжение на конденсаторе и ток в контуре
\[U_C = \frac{q}{C} = \frac{q_m}{C}e^{-\beta t}cos(\omega t + \alpha)\]
\[I = \omega q_m e ^ {-\beta t} cos(\omega t + \alpha + \delta)\]
\subsection{
    Величины характеризующие затухание
}
\subsubsection{
    Коэффицент затухания и время релаксации
}
время релаксации  $\tau$ - время за которое амплитуда колебаний\\
уменьшается в e раз
\[\tau = \frac{1}{\beta}\]
\subsubsection{
    Логарифмический декремент затухания
}
Определяется как натуральный логарифм отношения двух значений \\
амплитуд взятых через период колебания T
\[\lambda = \ln{\frac{\alpha(t)}{\alpha(t + T)}} = \beta T\]
если затухание мало 
\[\lambda \approx \beta \frac{2\pi}{\omega_0} = \pi R \sqrt{\frac{C}{L}}\]
\subsubsection{
    Добротность
}
\begin{itemize}
    \item По определению: $Q = \frac{\pi}{\lambda}$
    \item При слабом затухании: $Q \approx \frac{1}{R}\sqrt{\frac{L}{C}}$
    \item Энергетический смысл: $Q \approx 2\pi \frac{W}{\delta W}$
\end{itemize}
\subsection{
    Вынужденные электрические колебания
}
\[L\frac{dI}{dt} + RI + \frac{q}{C} = \varepsilon_m cos(\omega t)\]
или
\[\ddot{q} + 2\beta\dot{q} + \omega_0^2 q = \frac{\varepsilon_m}{L} cos(\omega t) \qquad q = q_m cos(\omega t - \psi)\]
где $q_m$ - амплитуда заряда на конденсаторе\\
$\psi$ - разность фаз между колебаниями заряда и внешней э.д.с $\varepsilon$
ток в цепи
\[I = I_m cos (\omega t - \varphi)\]
где $I_m$ амплитуда тока $\varphi$ сдвиг по фазе между током и\\
внешней э.д.с $\varepsilon$
\[I_m = \omega q_m, \qquad \varphi = \psi - \frac{\pi}{2}\]
напряжения на индуктивности сопротивлении и емкости
\[U_R = RI_mcos(\omega t - \varphi)\]
\[U_C = \frac{I_m}{\omega C} cos(\omega t - \varphi - \frac{\pi}{2})\]
\[U_L = {I_m\omega L} cos(\omega t - \varphi + \frac{\pi}{2})\]
\subsubsection{
    Векторная диаграмма
}
\[I_m = \frac{\varepsilon_m}{\sqrt{R^2 + (\omega L - \frac{1}{\omega C})^2}} \qquad \tan \varphi = \frac{\omega L - \frac{1}{\omega C}}{R}\]
\subsubsection{
    Резонансные кривые
}
\[\omega_{I\text{рез}} = \omega_0 = \frac{1}{\sqrt{LC}} \qquad \omega_{q\text{рез}} = \sqrt{\omega_0^2 - 2\beta^2}\]
резонансные частоты для:
\begin{itemize}
    \item $U_R$: $\omega_{R\text{рез}} = \omega_0$
    \item $U_C$: $\omega_{C\text{рез}} = \omega_0\sqrt{1 - 2(\frac{\beta}{\omega_0}) ^ 2}$
    \item $U_L$: $\omega_{R\text{рез}} = \frac{\omega_0}{\sqrt{1 - 2(\frac{\beta}{\omega_0}) ^ 2}}$ 
\end{itemize}
\subsubsection{
    Резонансные кривые и добротность
}
если $\beta << \omega_0$
\[\frac{U_{C\text{рез}}}{\varepsilon_m} = Q \qquad Q = \frac{\omega_0}{\delta \omega}\]
где $\omega_0$ - резонансная частота$\delta \omega$ - ширина резонансной кривой на\\ 
высоте 0.7 от максимальной
\subsection{
    Переменный ток
}
\subsubsection{
    Полное сопротивление (импенданс)
}
\[Z = \sqrt{R^2 + (\omega L - \frac{1}{\omega C}) ^ 2}\]
при $\omega = \omega_0 = \frac{1}{\sqrt{LC}}$ это сопротивление минимально и равно активному \\
сопротивлению R.
X - реактивное сопротивление
\[X = \omega L - \frac{1}{\omega C}\]
$\omega L$ - индуктивное сопротивление $ \frac{1}{\omega C}$ - емкостное сопротивление \\
их обозначают $X_L \quad X_C$ соответсвенно
\subsubsection{
    Мощность, выделяемая в цепи переменного тока
}
\[P(t) = UI = U_mI_mcos(\omega t)cos(\omega t - \varphi)\]
можно представить в виде
\[P(t) = U_mI_m(\cos^2(\omega t)cos(\varphi) + sin(\omega t) cos(\omega t) sin(\varphi))\]
практический интерес имеет среднее за период колебаний значение мощности.
\[\langle P\rangle = \frac{U_mI_m}{2}cos(\varphi) = \frac{RI^2_m}{2}\]
такую же мощность развивает постоянный ток с постоянными величинами: 
\[I = \frac{I_m}{\sqrt{2}} \qquad U = \frac{U_m}{\sqrt{2}} \qquad \langle P\rangle = UIcos(\varphi)\]

\section{Скин эффект}
Толщина скин-слоя:
\[l \approx \frac{1}{\sqrt{2\mu_0\mu\lambda\nu}}\]

\section{Разложение Фурье}
\[f(x)=\frac{a_0}{2} + \sum_{n=1}^{\inf}(a_n \cos (n \omega t) + b_n \sin (n \omega t))\]
\[a_n = \frac{2}{T}\int_{t_1}^{t_1+T}f(t)cos(n\omega t)dt \qquad b_n = \frac{2}{T}\int_{t_1}^{t_1+T}f(t)sin(n\omega t)dt\]
\begin{itemize}
    \item Прямоугольник ($0\le U \le U_0$): $b_n=0 \qquad a_n=2U_0\frac{2U_0}{n\pi}\sin(\frac{\tau}{T}\pi n)$
\end{itemize}

\end{document}
