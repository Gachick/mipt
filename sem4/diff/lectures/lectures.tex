\documentclass{article}

\usepackage{amsmath, amsthm, amsfonts, amssymb}
\usepackage[utf8]{inputenc}
\usepackage[T2A]{fontenc}
\usepackage[english, russian]{babel}

\usepackage{import}
\usepackage{pdfpages}
\usepackage{transparent}
\usepackage{xcolor}

\usepackage{parskip}
\usepackage{systeme}

\newcommand{\incfig}[2][1]{%
    \def\svgwidth{#1\columnwidth}
    \import{./figures/}{#2.pdf_tex}
}

\pdfsuppresswarningpagegroup=1

\usepackage{hyperref}
\hypersetup{
    colorlinks=true, %set true if you want colored links
    linktoc=all,     %set to all if you want both sections and subsections linked
    linkcolor=black,  %choose some color if you want links to stand out
}

\newcommand\hr{
    \noindent\rule[0.5ex]{\linewidth}{0.5pt}
}

% All the environments
\usepackage{mdframed}
\mdfsetup{skipabove=1em,skipbelow=0em}
\theoremstyle{definition}
\newmdtheoremenv[nobreak=true]{theorem}{Теорема}
\numberwithin{theorem}{section}
\newmdtheoremenv[nobreak=true]{lemma}{Лемма}
\numberwithin{lemma}{section}
\newmdtheoremenv[nobreak=true]{definition}{Определение}
\numberwithin{definition}{section}
\newmdtheoremenv[nobreak=true]{corollary}{Следствие}
\numberwithin{corollary}{section}
\newtheorem*{eg}{Пример}
\newtheorem*{remark}{Замечание}

\numberwithin{equation}{section}

% Defs
\let\phi\varphi
\let\epsilon\varepsilon
\let\kappa\varkappa
\let\implies\Rightarrow
\let\iff\Leftrightarrow
\let\true\hookrightarrow

\newcommand{\pd}[2]{\frac{\partial{#1}}{\partial{#2}}}
\newcommand{\pdd}[2]{\frac{\partial^2{#1}}{\partial{#2^2}}}
\newcommand{\pdm}[3]{\frac{\partial^2{#1}}{\partial{#2}\partial{#3}}}
\newcommand\R{\ensuremath{\mathbb{R}}}


\begin{document}

% lection 1
\section{Формула Остроградского-Лиувилля (ФОЛ) для лин. сис.}

\begin{gather*}
  \dot{X}=A(t)x \qquad x\in \R \\
  A(t)= \begin{pmatrix}
     a_{11}(t) & \dots & a_{1n}(t) \\ 
     \vdots & & \vdots \\ 
     a_{n1}(t) & \dots & a_{nn}(t)
  \end{pmatrix} \qquad a_{ij}(t) \in C(a,b)
\end{gather*}
$x_1(t),\dots, x_n(t)$ - решение системы
\[
  W(t) = |x_1(t) \dots x_n(t)| = \begin{vmatrix}
     x_{11}(t) & \dots & x_{1n}(t) \\ 
     \vdots & & \vdots \\ 
     x_{n1}(t) & \dots & x_{nn}(t)
  \end{vmatrix}
\]
\begin{theorem}
  Пусть $x_1(t),\dots,x_n(t)$ решение системы и $W(t)$ - определитель Вронского,
  тогда:
  \[
    W(t)=W(t_0)e^{\int_{t_0}^{t}trA(\tau)d\tau} \qquad \forall t \in (a,b), \; t_0 \in (a,b)
  \]
\end{theorem}
\begin{proof}
  \[
    \dot{W}(t)=\underset{=a_{11}W(t)}{\begin{vmatrix}
      \dot{x}_{11} & \dot{x}_{12} & \dots & \dot{x}_{12} \\
      {x}_{21} & {x}_{22} & \dots & {x}_{22} \\
      \dots & \dots & \dots & \dots \\
      {x}_{n1} & {x}_{n2} & \dots & {x}_{nn}
    \end{vmatrix}} + \dots + \underset{=a_{nn}W(t)}{\begin{vmatrix}
      {x}_{11} & {x}_{12} & \dots & {x}_{12} \\
      {x}_{21} & {x}_{22} & \dots & {x}_{22} \\
      \dots & \dots & \dots & \dots \\
      \dot{x}_{n1} & \dot{x}_{n2} & \dots & \dot{x}_{nn}
  \end{vmatrix}}
  \]
  1) решения лин. завис $\implies \; W(t) =0$ на $(a,b)$

  2) решения лин. независ:

  $\dot{x}_1=Ax_1 \; \dots \; \dot{x}_n=Ax_n$,
  ФМР $\Phi(t)=(x_1(t),\dots,x_n(t))$, $\dot{\Phi}(t)=A\Phi$
  \[
    \begin{pmatrix}
      \dot{x}_{11} & \dot{x}_{12} & \dots & \dot{x}_{1n} \\
      \dot{x}_{21} & \dot{x}_{22} & \dots & \dot{x}_{2n} \\
      \dots & \dots & \dots & \dots \\
      \dot{x}_{n1} & \dot{x}_{n2} & \dots & \dot{x}_{nn} \\
    \end{pmatrix} = \begin{pmatrix}
      {a}_{11} & {a}_{12} & \dots & {a}_{1n} \\
      {a}_{21} & {a}_{22} & \dots & {a}_{2n} \\
      \dots & \dots & \dots & \dots \\
      {a}_{n1} & {a}_{n2} & \dots & {a}_{nn}
    \end{pmatrix} \begin{pmatrix}
      
      {x}_{11} & {x}_{12} & \dots & {x}_{1n} \\
      {x}_{21} & {x}_{22} & \dots & {x}_{2n} \\
      \dots & \dots & \dots & \dots \\
      {x}_{n1} & {x}_{n2} & \dots & {x}_{nn}
    \end{pmatrix}
  \]
  \begin{gather*}
    \dot{x}_{11}=a_{11}x_{11}+a_{12}x_{21}+\dots+a_{1n}x_{n1} \\
    \dot{x}_{12}=a_{11}x_{12}+a_{12}x_{22}+\dots+a_{1n}x_{n2} \\
    \dots \\
    \dot{x}_{1n}=a_{11}x_{1n}+a_{12}x_{2n}+\dots+a_{1n}x_{nn} \\
  \end{gather*}
  \begin{gather*}
    (\dot{x}_{11} \, \dot{x}_{12} \, \dots \, \dot{x}_{1n})=
    a_{11}(x_{11}\, x_{12} \, \dots \, x_{1n}) + a_{12}(x_{21}\, x_{22} \, \dots \, x_{2n}) \\
    + \dots + a_{1n}(x_{n1}\, x_{n2} \, \dots \, x_{nn})
  \end{gather*}
  \begin{gather*}
    W_1 = a_{11} \underset{=W(t)}{\begin{vmatrix}
      {x}_{11} & {x}_{12} & \dots & {x}_{1n} \\
      {x}_{21} & {x}_{22} & \dots & {x}_{2n} \\
      \dots & \dots & \dots & \dots \\
      {x}_{n1} & {x}_{n2} & \dots & {x}_{nn}
  \end{vmatrix}}
  + a_{12} \underset{=0}{\begin{vmatrix}
      {x}_{21} & {x}_{22} & \dots & {x}_{2n} \\
      {x}_{21} & {x}_{22} & \dots & {x}_{2n} \\
      \dots & \dots & \dots & \dots \\
      {x}_{n1} & {x}_{n2} & \dots & {x}_{nn}
  \end{vmatrix}} + \\
    + \dots
    + a_{1n} \underset{=0}{\begin{vmatrix}
      {x}_{n1} & {x}_{n2} & \dots & {x}_{nn} \\
      {x}_{21} & {x}_{22} & \dots & {x}_{2n} \\
      \dots & \dots & \dots & \dots \\
      {x}_{n1} & {x}_{n2} & \dots & {x}_{nn}
  \end{vmatrix}}
  \end{gather*}
  \[
    \dot{W}(t)=trA\cdot W(t) \qquad W(t)=W(t_0)e^{\int_{t_0}^{t}trA(\tau)d\tau}
  \]
\end{proof}
\section{Формула Остроградского-Лиувилля для линейного уравнения}
\[
  a_n(x)y^{(n)}+ a_{n-1}(x)y^{(n-1)}+ \dots +  a_1(x)y' +  a_0(x)y = 0
\]
$a_i(x)/a_n(x) \in C(a,b)$; $y_1,y_2,\dots,y_n$ - решения
\[
  W(x) = \begin{vmatrix}
    y_1 & \dots & y_n \\ 
    y_1' & \dots & y_n' \\ 
    \dots & \dots & \dots \\
    y_1^{(n-1)} & \dots & y_n^{(n-1)}
  \end{vmatrix} 
\]
\begin{theorem}
  Пусть $y_1, \dots , y_n$ - решения уравнения и $W(x)$ опр. Вронского, тогда
  \[
    W(x)=W(x_0)e^{-\int_{x_0}^{x}\frac{a_{n-1}(\xi)}{a_n(\xi)}d\xi}
  \]
\end{theorem}
\begin{proof}
  1) $y_1,\dots,y_n$ лин завис $\implies \; W(x) = 0$

  2) $y_1,\dots,y_n$ - лин независ
  \begin{gather*}
    t=x \; x_1=y\; x_2=y' \; \dots \; x_n=y^{n-1} \\ 
    \left\{\begin{aligned}
      & \dot{x}_1=y'=x_2 \\
      & \dot{x}_2=y''=x_3 \\
      & \dots \\ 
      & \dot{x}_n=y^{n}=-\frac{a_{n-1}(t)}{a_n(t)}x_n-\dots- -\frac{a_{0}(t)}{a_n(t)}x_1 \\
    \end{aligned}\right. \\ 
    A(t) = \begin{pmatrix}
      0 & 1 & 0 & \dots & 0 \\ 
      0 & 0 & 1 & \dots & 0 \\ 
      \dots & \dots & \dots & \dots & \dots \\ 
      -\frac{a_0}{a_n} & -\frac{a_1}{a_n} & \dots & \dots & -\frac{a_{n-1}}{a_n}
    \end{pmatrix} \\ 
    W(t) = W(t_0) e^{-\int_{t_0}^{t}\frac{a_{n-1}(t)}{a_0(t)}d\tau} \qquad W(x) = W(x_0) e^{-\int_{x_0}^{x}\frac{a_{n-1}(\xi)}{a_0(\xi)}d\xi}
  \end{gather*}
\end{proof}
\begin{remark}
  \begin{gather*}
    W(x_0)=0 \rightarrow W(x) =0 \text{ на } (a,b) \\ 
    W(x_0)\neq0 \rightarrow W(x)\neq0 \text{ на } (a,b) \\ 
  \end{gather*}
\end{remark}
Частные случаи:
\begin{enumerate}
  \item $a_1(x)y'+a_0(x)y=0$, $W(x)=y_1(x)=y_1(x_0)e^{-\int_{x_0}^{x}\frac{a_0(\xi)}{x_1(\xi)}d\xi}$
  \item $a_2(x)y''+a_1(x)y'+a_0(x)y=0$
    \[
      W(x)=\begin{vmatrix}
        y_1(x) & y_2(x) \\ 
        y_1'(x) & y_2'(x)
      \end{vmatrix}=y_1y_2'-y_2y_1'=\left(\frac{y_2}{y_1}\right)'y_1^2=W(x_0)e^{-\int_{x_0}^{x}\frac{a_1(\xi)}{a_2(\xi)}d\xi}
    \]
\end{enumerate}

\section{Схема решения уравн (линейного) второго порядка}
\[
  a_2(x)y''+a_1(x)y'+a_0(x)y=0 \qquad \frac{a_1}{a_2},\frac{a_0}{a_2},\frac{f}{a_2} \in C(a,b)
\]
\begin{enumerate}
\item однородное уравнение
  \begin{enumerate}
    \item угадали $y_1(x)$ чаще всего в виде $P_n(x) \quad e^{ax} \quad x^{a}$
    \item $y_2(x)$ по ФОЛ $\left(\frac{y_2}{y_1}\right)'=\frac{1}{y_1^2}e^{-\int^{x}\frac{a_1(\xi)}{a_2(\xi)}d\xi} \rightarrow y_2(x)$
      - линейно незав. решение $y_0=C_1y_1+C_2y_2$
  \end{enumerate}
\item неоднородное уравнение МВП
$\tilde y=C_1(x)y_1(x)+C_2(x)y_2(x)$ - частное решение неоднородного
\[
  \begin{pmatrix}
    y_1(x) & y_2(x) \\ 
    y_1'(x) & y_2'(x)
  \end{pmatrix}\begin{pmatrix}
    C_1' \\ 
    C_2'
  \end{pmatrix} = \begin{pmatrix}
    0 \\ 
    \frac{f(x)}{a_2(x)}
  \end{pmatrix}
\]
\[
  y=y_0+\tilde y
\]
\end{enumerate}

\section{Качественное исследование лин. однор. ур. 2-го порядка}
\subsection{Виды уравнений}
\[
  a_2(x)y''+a_1(x)y'+a_0(x)y=0
\]
\begin{enumerate}
  \item нормальный вид 
    \[
  y''+b_1(x)y'+b_0(x)y=0 \qquad b_0(x),b_1(x) \in C(a,b)
    \]
  \item самосопряжённый вид
    \[
      (p(x)y')'+q(x)y=0
    \]
    $p(x)>0$ на $(a,b)$, $p(x)\in C'(a,b)$, $q(x)\in C(a,b)$.

    Как приводить?
    \begin{gather*}
      py''+p'y'+qy=0 \\ 
      y'' + \frac{p'}{p}y'+\frac{q}{p}y=y''+b_1y'+b_0y=0 \\
      \left\{\begin{aligned}
        \frac{p'}{p}=b_1 \\ 
        q=pb_0
      \end{aligned}\right.
    \end{gather*}
    $p(x)=e^{\int_{x_0}^{x}b_1(\xi)d\xi} > 0$, $p(x)\in C'(a,b)$, $q=b_0e^{\int_{x_0}^{x}b_1(\xi)d\xi}\in C(a,b)$
    \begin{eg}
      $xy''+2y'+y=0$ нормальный вид $y''+\frac{2}{x}y'+\frac{1}{x}y=0$ 
      смотрим на $(-\infty,0)$ или $(0,+\infty)$ (у нас второе)
      самосопр вид $(x^2y')'+xy=0$, $p=x^{2}$, $q=x$
    \end{eg}
  \item канонический вид
    \[
      y''+r(x)y=0 \qquad r(x) \in C(a,b)
    \]    
    \begin{enumerate}
      \item $b_0 \in C(a,b)$ 
        и $b_1(x)\in C'(a,b)$ замена $y(x)=\phi(x)z(x)$
        \begin{eg}
          $b_1(x)=\frac{2}{x}$, $b_0=\frac{1}{x}$, $x>0$, $b_1 \in C'(0,+\infty)$
          \begin{gather*}
            y'=\phi'z+\phi z' \qquad y''=\phi''z+2\phi'z'+\phi z'' \\ 
            x\phi''z+2x\phi'z'+x\phi z''+2\phi'z+2\phi z'+\phi z=0
          \end{gather*}
          Зануляем коэфициент перед $z'$ $\implies 2x\phi'+2\phi=0$
          \begin{gather*}
            \frac{\phi'}{\phi}=-\frac{1}{x} \qquad \phi(x)=\frac{1}{x} \\ 
            \phi'=-\frac{1}{x^2} \qquad \phi''=\frac{2}{x^2} \\ 
            \frac{2}{x^2}z+z''-2z\frac{1}{x^2}+\frac{1}{x}z=0 \\ 
            z''+\frac{1}{x}z=0
          \end{gather*}
        \end{eg}
      \item $b_0(x)$ и $b_1(x) \in C(a,b)$, замена $t=\psi(x)$
        \begin{gather*}
          y'=\pd{y}{x}=\pd{y}{t}\pd{t}{x}=\dot{y}\psi'  \qquad
          y''=\dot{y}\psi''+\ddot y \psi'^2 \\
          py''+p'y'+qy=0 \qquad p \dot{y}\psi''+p \ddot y \psi'^{2}+p'\dot{y}\psi'+qy=0 \\ 
          p\psi''+p'\psi'=0  \qquad (p\psi')=0 \qquad p\psi'=1 \\ 
        \end{gather*}
        $\psi(x)=\int_{x_0}^{x}\frac{d\xi}{p(\xi)}\rightarrow$ сторого монот и непр,
        $\exists$ обратная функция $x=x(t)$ на $(t_1,t_2)$
        \[
          \psi'=\frac{1}{p} \qquad \psi''=-\frac{p'}{p^2}
        \]
        \[
          p\frac{1}{p^2}\ddot y + qy=0
        \]
        \[
          \ddot y(t) + q(x(t))p(x(t))y(t)=0
        \]
      % lecture 2 
      \item Преобразование Фурье-Лиувилля - привидение к канон. виду
      \begin{enumerate}
        \item $t=\phi(x)$ к виду $\ddot y +c(t)\dot{y}\pm y =0$
        \item $c(t)\in C'(t_1,t_2)$, $y(t)=\psi(t)z(t)$ к канон. виду $\ddot y +\alpha(t)y=0$
      \end{enumerate}
      \begin{eg}
        \begin{gather*}
          xy''+2y'+y=0 \quad x>0 \\ 
          y'=\dot{y}\phi'(x) \qquad y''=\dot{y}\phi''(x)+\ddot y \phi'^{2} \\ 
          x \dot{y}\phi'' + x\ddot y \phi'^{2}+2\dot{y}\phi'+y=0 \\ 
          x\phi'^{2}=1 \qquad \phi'=\frac{1}{\sqrt{x}} \qquad \phi=2\sqrt{x} \\ 
          t=2\sqrt{x} \qquad \phi''=-\frac{1}{2}\frac{1}{x^{3/2}} \\ 
          \ddot y + \dot{y}\left(-\frac{1}{2\sqrt{x}}+\frac{2}{\sqrt{x}}\right)+y=0 \\ 
          \ddot y + \dot{y}\frac{3}{2\sqrt{x}}+y=0 \qquad \ddot y + \dot{y}\underbrace{\frac{3}{t}}_{c(t)}+y=0
        \end{gather*}
        При $t>0$ непр дифф., делаем второй шаг
        \begin{gather*}
          y(t)=z(t)\psi(t) \\ 
          \ddot z \psi(t)+2 \dot{z}\dot{\psi} + z \ddot \psi + \frac{3}{t}\dot{z}\psi+\frac{3}{t}z\dot{\psi}+z\psi =0 \\
          2\dot{\psi}+\frac{3}{t}\psi = 0 \qquad \psi=\frac{1}{t^{3/2}} \\ 
          \dot{\psi} = -\frac{3}{2}\frac{1}{t^{5/2}} \qquad \ddot \psi=\frac{15}{4}\frac{1}{t^{7/2}} \\ 
          \ddot z \frac{1}{t^{3/2}}+z\left(\frac{15}{4}\frac{1}{t^{7/2}}-\frac{9}{2}\frac{1}{t^{7/2}}+\frac{1}{t^{3/2}}\right)=0 \\ 
          \ddot z + z\left(1-\frac{3}{4t^{2}}\right)=0 \qquad t>0
        \end{gather*}
      \end{eg}
    \end{enumerate}
\end{enumerate}

\subsection{Асимптотический вид решения}
\[
  \ddot y + y(m+\beta(t))=0 \qquad m \neq 0
\]
\begin{theorem}
  Если $\beta(t)$ непр на $[t_0,+\infty)$ и $\beta(t)=O(\frac{1}{t^{1+\epsilon}})$ при $t\to\infty$, $\epsilon>0$, то 
  \begin{itemize}
    \item $m>0$: $y(t)=C_1\cos\sqrt{m}t+C_2\sin \sqrt{m}t+O(\frac{1}{t^{\epsilon}})$
    \item $m<0$: $y(t)=C_1e^{\sqrt{|m|}t}\left(1+O(\frac{1}{t^{\epsilon}})\right) + C_2e^{-\sqrt{|m|}t}\left(1+O(\frac{1}{t^{\epsilon}})\right)$
  \end{itemize}
\end{theorem}
\begin{eg}
  \begin{gather*}
    \ddot z + z\left(1-\frac{3}{4t^{2}}\right)=0 \qquad t>0 \\ 
    m=1 \qquad \beta(t)=-\frac{3}{4t^{2}} \quad \text{непр} \; (0,+\infty) \quad \epsilon=1 \\ 
    z(t)=C_1\cos t + C_2\sin t +O(t) \\ 
    t=2\sqrt{x} \qquad y(t)=z(t)\frac{1}{t^{3/2}} \\
    y(t)=C_1\frac{\cos t }{3^{3/2}}+C_2\frac{\sin t}{t^{3/2}}+O\left(\frac{1}{t^{5/2}}\right) \\ 
    y(x)=\tilde C_1 \frac{\cos 2\sqrt{x}}{x^{3/4}}+ \tilde C_2 \frac{\sin 2\sqrt{x}}{x^{3/4}} + O\left(\frac{1}{x^{3/4}}\right) \\ 
  \end{gather*} 
  Верно при $x\to \infty$
\end{eg}

\subsection{Исследование нулей решения уравн. второго порядка}
\[
  y''+b_1(x)y'+b_0(x)y=0 \qquad b_0,b_1\in C(a,b)
\]
\begin{definition}
  Точка $x_0$ называется нулём решения $y(x)$, если $y(x_0)=0$
\end{definition}
\begin{theorem}
  Пусть $y(x)$ нетривиальное решение и $y(x_0)=0$ тогда $y'(x_0)\neq 0$
\end{theorem}
\begin{proof}
  Пусть $y'(x_0)=0$, получаем задачу Коши $y(x_0)=0$, $y'(x_0)=0$ $\implies y\equiv 0$
  единст. реш., противоречит с нетрив реш.
\end{proof}
\begin{theorem}
  Любое нетривиальное решение может иметь на отрезке $[c,d]\subset (a,b)$
  не более конечного числа нулей.
\end{theorem}
\begin{proof}
  Пусть число нулей бесконечно на $[c,d]$, счётное подмнож $x_1,x_2,\dots , x_n$ -
  ограниченная послед, выделяем сход. подпослед. $x_{n_{k}}\to x_0\in [c,d]$.

  $y(x_{n_{k}})=0$, $y(x_{n_{k}})\to y(x_0)=0$ непр. $y(x)$

  $y(x)$ - решение $\implies \exists y'(x_0)$
  \[
    y'(x_0)=\lim\limits_{k\to\infty}\frac{y(x_{n_{k}})-y(x_0)}{x_{n_{k}}-x_0}=0
  \]
  Противоречие с пред. теоремой.
\end{proof}

\begin{theorem}[Теорема сравнения Штурма]
  Пусть $(p(x)z')'+q(x)z=0$, $(p(x)y')'+Q(x)y=0$, $p(x)\in C'(a,b)$, $q,Q\in C(a,b)$,
  $p(x)>0$ на $(a,b)$ и пусть $x_1$ и $x_2$ $\in (a,b)$ 
  два последовательных нуля нетривиального решения $z(x)$
  и $q(x)\le Q(x)$ на $[x_1,x_2]$.

  Тогда любое решение $y(x)$ имеет хотя бы один нуль на $[x_1,x_2]$.

\end{theorem}

\begin{proof}
  \incfig{l2_shturm}
  Пусть $y(x)\neq 0$ на $[x_1,x_2]$
  \begin{gather*}
    (pz')'y+qzy'-(py')'z-Qyz=0 \\ 
    \underbrace{p'z'y+pz''y-p'y'z-py''z}_{(p(z'y-zy'))'}+yz(q-Q) = 0 \quad \int_{x_1}^{x_2} \\ 
    p(z'y-zy')\big|_{x_1}^{x_2}+\int_{x_1}^{x_2}yz(q-Q)dx=0 \\ 
    \underbrace{\underbrace{p(x_2)}_{>0}(\underbrace{z'(x_2)}_{<0}\underbrace{y(x_2)}_{>0}-\underbrace{z(x_2)}_{=0}y'(x_2))-\underbrace{p(x_1)}_{>0}(\underbrace{z'(x_1)}_{>0}\underbrace{y(x_1)}_{>0}-\underbrace{z(x_1)}_{=0}y'(x_1))}_{<0}+ \\ 
    +\underbrace{\int_{x_1}^{x_2}\underbrace{yz}_{\ge0}\underbrace{(q-Q)}_{<0}dx}_{\le 0}=0
  \end{gather*}
  Противоречие.
\end{proof}
\begin{corollary}[Теорема о премежаемости нулей]
  Пусть $y_1(x)$ и $y_2(x)$ два линейно независимых решения $(p(x)y(x))'+q(x)y=0$,
  $p\in C'(a,b)$, $p>0$ на $(a,b)$, $q\in C(a,b)$ и $x_1$ и $x_2$ два последовательных
  нуля $y_1(x)$ тогда $y_2(x)$ имеет ровно один нуль на $(x_1,x_2)$.
\end{corollary}
\begin{proof}
  Пусть $y_1(x_0)=0$ и $y_2(x_0)=0$
  \[
    W(x_0)=\begin{vmatrix}
      y_1(x_0) & y_2(x_0) \\
      y_1'(x_0) & y_2'(x_0) \\
    \end{vmatrix}=0
  \]
  Противоречие с линейной независ.
  \[
    (py_1')'+qy_1=0 \qquad (py_2')'+qy_2=0
  \]
  По теореме Штурма $y_2$ имеет хотя бы один нуль на $(x_1,x_2)$.
  
  Пусть два нуля $y_2(x_3)=y_2(x_4)=0$, тогда по теореме сравнения Штурма
  $\exists x_5: y_1(x_5)=0$ противоречит с соседством $x_1$ и $x_2$.
\end{proof}

\subsection{Оценка расстояние между нулями}
\begin{theorem}
  Пусть $y''+qy=0$, $q\in C(a,b)$,
  тогда для любого нетривиального решения расстояние между соседними нулями
  удовлетворяет неравенству:
  \[
    \frac{\pi}{\sqrt{M}} \le \Delta \le \frac{\pi}{\sqrt{m}} \qquad 0<m\le q(x) \le M \; \text{на} \; (a,b)
  \]
\end{theorem}
\begin{proof}
  Пусть $\Delta > \frac{\pi}{\sqrt{m}}$
  \begin{gather*}
    z''+mz=0 \qquad z(x)=\sin(\sqrt{m}(x+\alpha)) \qquad \forall \alpha
  \end{gather*}

  \incfig{l2_dist}
  $m\le q$ по т. ср. Штурма между двумя нулями $z(x)$ $\exists$ нуль $y(x)$.
  Противоречие.

  Аналогично $\Delta<\frac{\pi}{\sqrt{M}}$
\end{proof}

% lecture 3 

\subsection{Оценка числа нулей на интервале}
\begin{theorem}
  Пусть $y''+Q(x)y=0$, $Q(x)\in C(a,b)$, $0<m\le Q(x)\le M$ на $(a,b)$,
  тогда число нулей любого нетривиального решения на $(a,b)$
  удовлетворяет неравнству:
  \[
    \left[\sqrt{m}\frac{b-a}{\pi}\right]-1 \le N \le \left[\sqrt{M}\frac{b-a}{\pi}\right]+1
  \]
  где $[\dots ]$ - целая часть числа.
\end{theorem}
\begin{proof}
  $\frac{\pi}{\sqrt{M}}\le \Delta \le \frac{\pi}{\sqrt{m}}$
  - расстояние между соседними нулями.

  \incfig{l3_zeros}
\end{proof}
\begin{theorem}
  Пусть $y''+Q(x)y=0$, $Q(x)\in C(a,b)$, $Q(x) \le 0$ на $(a,b)$,
  тогда число нулей любого нетривиального решения на $(a,b)$
  удовлетворяет неравнству:
  \[
    0 \le N \le 1
  \]
  (не более одного нуля)
\end{theorem}
\begin{proof}
  Пусть 2 нуля $x_1$, $x_2$
  \[
    z''+0z=0 \;\Rightarrow\; z''=0
  \]
  По т. сравн. Штурма на $[x_1,x_2]$ лежит хотя бы один нуль
  $z''=0$ любого решения.
\end{proof}
\begin{remark}
  \begin{itemize}
    \item в Т. 1: $(a,b)$ - открытое и ограниченное множество
    \item в Т. 2 $(a,b), [a,b], (a,b]$ и может быть неогран.
  \end{itemize}
\end{remark}
\begin{eg}
  Доказать $\forall$ нетрив. реш $y''+\sqrt{4-x^{2}}y=0$
  имеет на $[-2,2]$ не более 2 нулей.
  \begin{gather*}
    0 \le Q(x)=\sqrt{4-x^{2}}\le 2  \\ 
    (-2,2) \quad N \le \underbrace{\left[\sqrt{2}\frac{(2-(-2))}{\pi}\right]}_{1.8}+1=2
  \end{gather*}
  Пусть 3 нуля на $[-2,2]$.

  \incfig{l3_eg_zeros}
  \begin{gather*}
    z''+2z=0 \\ 
    z=\sin \sqrt{2}(x+\alpha) \qquad \frac{\pi}{\sqrt{2}}\approx 2.22
  \end{gather*}
  Можем подобрать $\alpha$ так чтобы попадал только один ноль в $[-2,2]$.
\end{eg}

\subsection{Уравнение Бесселя}
\begin{gather*}
  x^{2}y''+xy'+(x^{2}-\nu^{2})y=0 \\ 
  x>0 \quad \nu \in \R \quad \nu \ge 0
\end{gather*}
Попытаемся привести к каноническому виду:
\begin{gather*}
  y=z(x)\phi(x) \\ 
  x^{2}(z''\phi+z\phi''+2z'\phi')+x(z'\phi+z\phi')+(x^{2}-\nu^{2})z\phi=0 \\ 
  2x^{2}\phi'+x\phi=0 \qquad \frac{d \phi}{\phi}=-\frac{dx}{2x} \\ 
  \phi = \frac{1}{\sqrt{x}} \qquad \phi'=-\frac{1}{2x^{3/2}} \qquad \phi''=\frac{3}{4}\frac{1}{x^{3/2}} \\
  z''x^{3/2}+z\left(\frac{3}{4\sqrt{x}}-\frac{1}{2\sqrt{x}}+x^{3/2}-\nu^{2}\frac{1}{\sqrt{x}}\right) =0 \\ 
  z''+z\left(1+\frac{.25-\nu^{2}}{x^{2}}\right)=0 \\ 
  \nu=\frac{1}{2} \qquad z''+z=0 \qquad y(z)=C_1\frac{\cos x}{\sqrt{x}}+C_2\frac{\sin x}{\sqrt{x}}
\end{gather*}
Одно решение ограничено (синус), а другое нет (косинус).
Может быть это характерно для всех решений уравнения?
\begin{gather*}
  W(x)=W(x_0)e^{-\int_{x_0}^{x}\frac{\xi}{\xi^{2}}d\xi}=W(x_{0})\frac{x_{0}}{x} \\ 
  W(x)=\frac{C}{x}=y_1y_2'-y_2y_1' \underset{x\to 0}{\to} \infty
\end{gather*}
Действительно что-то стремится к бесконечности (или производные или сама функция).
\begin{gather*}
  z''+z(1+\underbrace{\frac{.25-\nu^{2}}{x^{2}}}_{\alpha(x)})=0
\end{gather*}
Чтобы было почти с пост. коэф. $\alpha$ непр на $[x_{0},+\infty)$,
$\alpha(x)=O\left(\frac{1}{x^{1+\epsilon}}\right)$,$\epsilon>0$.
\begin{gather*}
  z(x)=C_1\cos x + C_2\sin x + \underbrace{O\left(\frac{1}{x^{\epsilon}}\right)}_{O\left(\frac{1}{x}\right) } \\ 
  y(x)=C_1\frac{\cos x}{\sqrt{x}} + C_2 \frac{\sin x}{\sqrt{x}} + \underbrace{O\left(\frac{1}{x^{3/2}}\right)}_{x\to\infty}
\end{gather*}
Обобщённый степенной ряд:
\[
  y(x)=x^{\alpha}(a_0+a_1x+a_2x^{2}+\dots )=\sum_{k=0}^{\infty}a_kx^{k+\alpha}
\]
Полагаем что дифф. нужно число раз и после нахождения решения задним числом смотрим так ли это.
\begin{gather*}
  y'(x)=\sum_{k=0}^{\infty}a_k(k+\alpha)x^{k+\alpha-1} \\ 
  y''(x)=\sum_{k=0}^{\infty}a_k(k+\alpha)(k+\alpha-1)x^{k+\alpha-2} \\ 
  \sum_{k=0}^{\infty}[a_k(k+\alpha)(k+\alpha-1)x^{k+\alpha}+a_k(k+\alpha)x^{k+\alpha}-\nu^{2}a_kx^{k+\alpha}+a_kx^{k+\alpha+2}]=0 \\ 
  \sum_{k=0}^{\infty}[(a_k(k+\alpha)^{2}-\nu^{2}a_k)x^{k}+a_kx^{k+2}]=0 \\ 
  k=0: \; a_0\alpha^{2}-a_0\nu^{2}=0 \\ 
  k=1: \; a_1(1+\alpha)^{2}-a_1\nu^{2}=0 \\ 
  k \ge 2: \; a_k(k+\alpha)^{2}-\nu^{2}a_k+a_{k-2}=0 \\ 
  a_0 \neq 0 \quad \alpha=\pm \nu \quad \alpha=\nu\ge0 \\ 
  a_1(1+\nu^{2}+2\nu-\nu^{2})=0 \; \Rightarrow \; a_1=0 \\
  a_k=\frac{a_{k-2}}{\nu^{2}-(k+\nu)^{2}}=\frac{-a_{k-2}}{k(k+2\nu)} \\ 
  a_{2n+1} = 0 \\ 
  a_{2n}=\frac{-a_{2n-2}}{2n(2n+2\nu)}=\frac{-a_{2n-2}}{4n(n+\nu)} \\ 
  a_2=\frac{-a_0}{4(a+\nu)} \qquad a_4=\frac{-a_2}{4\cdot2(2+\nu)}=\frac{a_0}{\underbrace{4\cdot4}_{2^{4}}\cdot 2(1+\nu)(2+\nu)} \\
  a_6=\frac{a_0}{2^{6}\cdot 2 \cdot 3 (3+\nu)(1+\nu)(2+\nu)} \\ 
  a_{2n}=\frac{(-1)^{n}a_0}{2^{2n}n!(1+\nu)(2+\nu)\dots (n+\nu)}\\ 
  \Gamma(s)=\int_{0}^{\infty}x^{s-1}e^{-x}dx \quad s>0 \quad \Gamma(s+1)=s\Gamma(s) \\ 
  \Gamma(n+\nu+1)=(n+\nu)\Gamma(n+\nu)=\dots =(n+\nu)(n+\nu-1)\dots (\nu+1)\Gamma(\nu+1)\displaybreak  \\ 
  a_0=\frac{1}{2^{\nu}\Gamma(\nu+1)} \qquad a_{2n}=\frac{(-1)^{n}}{2^{2n+\nu}n!\Gamma(n+\nu+1)} \\ 
  y(x)=\sum_{k=0}^{\infty}a_kx^{\overbrace{k+\alpha}^{\nu}} \\ 
  \underbrace{y(x)=\sum_{n=0}^{\infty}\frac{(-1)^{n}}{n!\Gamma(n+\nu+1)}\left(\frac{x}{2}\right)^{2n+\nu}}_{\text{ф. Бесселя 1го рода}}=J_\nu(x)
\end{gather*}
Постфактум доказываем дифферинцируемость (признак Доломбера):
\begin{gather*}
  \lim\limits_{n\to\infty}\left|\frac{u_{n+1}(x)}{u_n(x)}\right| < 1 \\ 
  \lim\limits_{n\to\infty}\left|\frac{(.5x)^{2}}{(n+1)(n+\nu+1)}\right| =0
\end{gather*}
Получаем что радиус сходимости бесконечен ($R=\infty$),
то есть можем бесконечно диф. где угодно.

$J_\nu(x)$ беск. дифф $x>0$.

Ищем второе решение через ФОЛ:
\[
  \left(\frac{y_2}{J_\nu(x)}\right)^{'}=\frac{1}{J_\nu^{2}(x)}\frac{1}{x}
\]
$y_2=Y_\nu(x)$ функция Бесселя второго рода.
\[
  y(x)=C_1J_\nu(x)+C_2Y_\nu(x)
\]

\incfig{l3_bessel}

% lecture 4 

\section{Автономные системы дифф. ур.}
\begin{definition}
  Нормальная система называется автономной, если правая часть
  не зависит явно от t.
  \begin{gather*}
    \dot{x}=F(x) \\ 
    \left\{\begin{aligned}
      & \dot{x}_1=f_1(x) \\ 
      & \dots \\
      & \dot{x}_2=f_2(x)
    \end{aligned}\right.
    \qquad f_i(x)\in C^{1}(D) \quad D\in \R^{n} \quad t\in(a,b)
  \end{gather*}
  через $\forall t_0 \in (a,b)$ и $\forall x_{0}\in D$ проходит единственная
  интегральная кривая.
\end{definition}
\begin{definition}
  Точка $\tilde{x} \in D$ называется положением равновесия автономной системы,
  если $F(\tilde{x})=0$.
\end{definition}

\incfig{l4_auto_bl}
\begin{definition}
  Фазовая траектория - проекция инт. кр. на $\R^{n}$, где $\R^{n}$
  - фазовое пространство.
\end{definition}

\subsection{Свойства фазовых траекторий}
\begin{enumerate}
  \item Если $x=\phi(t)$ решение системы на $(a,b)$ (автономн.),
    то $\forall c$ $x=\phi(t+c)$ тоже решение на $(a-c,b-c)$.
    \begin{proof}
      \begin{gather*}
        \frac{d\phi(t)}{dt}=F(\phi(t)) \qquad t= \tau+c \\ 
        \frac{d\phi(\tau+c)}{d(\tau+c)}=F(\phi(\tau+c))
      \end{gather*}
      $\phi(\tau+c)$ $\rightarrow$ решение
    \end{proof}
  \item Фазовые траектории не могут пересекаться
    \begin{proof}
      Пусть есть два решения $\phi(t)$ и $\psi(t)$ на $(a,b)$
      
      \hbox{\incfig{l4_intr}}
      $\exists t_1,t_2\in (a,b): \phi(t_1)=\psi(t_2)=x_0$,
      $\chi(t)=\phi(t+t_1-t_2)$ - решение 

      $\chi(t_2)=\phi(t_1)=\psi(t_2)=x_0$ противоречит Т. единст.
    \end{proof}
  \item Пусть $\tilde{x}$ положение равновесия, тогда $x=\phi(t)=\tilde{x}$
    - решение системы, а точка $\tilde{x}\in \R^{n}$ - фазовая траектория. 
    \begin{proof}
      \[
        F(\tilde{x})=0 \qquad \dot{x}=0
      \]
      Проекция фазовой троектории - точка.
    \end{proof}
  \item Фазовая траектория, отличная от положения равн. является гладкой кривой.
    \begin{proof}
      $x=\phi(t)$ - параметрически заданная кривая в фазовом простаранстве

      $\dot{x}=F(x)$, $F\in C'$, $\phi(t)$ - непр. дифф. на $(a,b)$,

      $\dot{x}=\dot{\phi}(t)\neq 0$ т.к. не явл. полож. равн.
    \end{proof}
  \item Фазовые траектории:
    \begin{itemize}
      \item точки
      \item незамкнутые гладкие кривые без самопересечения
      \item замкнутые гладкие кривые без самопересечения
    \end{itemize}
\end{enumerate}

\subsection{Классификация положений равновесия}
\subsubsection{n=1}
\[
  \dot{x}=-x \qquad x(t)=Ce^{-t} \qquad \tilde{x}=0
\]

  \incfig{l4_n1}
    
  \incfig{l4_n1_v}
\subsubsection{n=2}
\[
  \dot{x}=Ax \qquad
  A=\begin{pmatrix}
    a_{11} & a_{12} \\ 
    a_{21} & a_{22}
  \end{pmatrix}
\]
\subsubsection{Изолированные положения равновесия при n=2}
\begin{gather*}
  det A \neq0 \qquad Ax=0 \qquad \tilde{x}=0 \\ 
  \lambda_1\neq 0 \qquad \lambda_2 \neq 0
\end{gather*}

\hr
$\lambda_1$ и $\lambda_2$ действительны, $\lambda_1 \neq \lambda_2$,
$h_1$ и $h_2$ - базис
\begin{gather*}
  x(t)=C_1 e^{\lambda_1 t} h_1 + C_2 e^{\lambda_2 t} h_2 \\ 
  \xi_1 = C_1 e^{\lambda_1 t} \qquad \xi_2 = C_2 e^{\lambda_2 t} \\ 
  \left\{\begin{aligned}
    & \xi_2 = C_2 \left(\frac{\xi_1}{C_1}\right)^{\lambda_2/\lambda_1} &,\, C_1 \neq0 \\ 
    & \xi_1=0 &,\, C_1 = 0
  \end{aligned}\right. \qquad \alpha = \frac{\lambda_2}{\lambda_1}
\end{gather*}
\begin{enumerate}
  \item $\alpha>0$,

    \incfig{l4_knot}
    $h_1$ и $h_2$ - фазовые траектории

    $\lambda_2>0$ и $\lambda_2>0$ неустойчивый узел

    $\lambda_2<0$ и $\lambda_2<0$ устойчивый узел
  \item $\alpha<0$, $\xi_2=A\xi_1^{\alpha}$

    $\lambda_1$ и $\lambda_2$ разные знаки, седло

    \incfig{l4_saddle}
\end{enumerate}

\hr
  $\lambda_1$, $\lambda_2$ - действ., $\lambda_1=\lambda_2=\lambda$
    \begin{enumerate}
      \item Два собственных вектора $h_1$ и $h_2$
         
        \begin{gather*}
          x(t)=C_1 e^{\lambda t} h_1 + C_2 e^{\lambda t} h_2 \\ 
          \xi_1 = C_1 e^{\lambda t} \qquad \xi_2 = C_2 e^{\lambda t} \\ 
          \left\{\begin{aligned}
            & \xi_2 = \frac{C_2}{C_1}\xi_1 &,\, C_1 \neq0 \\ 
            & \xi_1=0 &,\, C_1 = 0
          \end{aligned}\right. 
        \end{gather*}

        \incfig{l4_dknot}

        $\lambda>0$ неустойчивый дикритический узел

        $\lambda<0$ устойчивый дикритический узел
        \item $h_1$ - собственный вектор, $h_2$ - присоед.
          \begin{gather*}
          x(t)=C_1 h_1 e^{\lambda t} + C_2(h_1 t+h_2)e^{\lambda t} \\ 
          \xi_1 = (C_1+C_2 t) e^{\lambda t} \qquad \xi_2 = C_2 e^{\lambda t} \\ 
          \left\{\begin{aligned}
            & \xi_1=C_1\frac{\xi_2}{C_2}+\xi_2\frac{1}{\lambda}\ln \frac{\xi_2}{C_2} &, \; C_2 \neq 0 \\ 
            & \xi_2=0 &, \; C_2=0
          \end{aligned}\right.
        \end{gather*}

        \incfig{l4_prop}
        $h_1$ - фазовая таектория, $h_2$ - не явл. фаз. тр.

        $\lambda>0$ неустойчивый вырожденный узел 

        $\lambda<0$ устойчивый вырожденный узел 
    \end{enumerate}

\hr
  $\lambda_1$, $\lambda_2$ - комплексные, $\lambda_{1,2}=a\pm ib$, $b>0$
    \begin{gather*}
      \lambda_1=a+ib \; \rightarrow \; h\\ 
      h_1=Re\, h \quad h_2=Im \, h \\ 
      x(t)=C_1 Re(he^{\lambda_1 t}) + C_2 Im(he^{\lambda_1t}) \\ 
      he^{\lambda t}=(h_1+ih_2)e^{at}(\cos bt + i\sin bt)= \\ 
      =e^{at}[(h_1\cos bt - h_2 \sin bt)+i(h_2\cos bt + h_1 \sin bt)] \\ 
      x(t)=h_1( \underbrace{C_1e^{at}\cos bt + C_2 e^{at} \sin bt}_{\xi_1}) + h_2 (\underbrace{-C_1 e^{at}\cos bt + C_2 e^{at} \cos bt}_{\xi_2} ) \\ 
      C_1 = A\cos \theta \qquad A\ge 0 \\ 
      C_2= A\sin \theta \qquad \theta \in [0, 2\pi) \\ 
      \xi + e^{at}\cos (\theta-bt)A \\ 
      \xi_2=e^{at}\sin (\theta-bt)A
    \end{gather*}
    \begin{enumerate}
      \item $a=0$, центр

        \incfig{l4_center}
      \item $a\neq 0$, фокус

        \incfig{l4_focus}

        $a>0$ неустойчивый фокус

        $a<0$ устойчивый фокус
    \end{enumerate}
\hr

% lecture 5

\subsubsection{Неизолированные полож равн. при n=2}
\begin{gather*}
  \dot{x}=Ax \qquad x \in \R^{2} \\ 
  Ax=0 \qquad det A =0
\end{gather*}
Хотя бы одно $\lambda=0$

\hr 
$\lambda_1=0$, $\lambda_2 \neq 0$, собст. век, $h_1$ и $h_2$
\begin{gather*}
  x=C_1h_1+C_2h_2e^{\lambda_2 t} \\ 
  J = S^{-1}AS = \begin{pmatrix}
    0 & 0 \\ 
    0 & \lambda_2
  \end{pmatrix} \\ 
  \begin{pmatrix}
    0 & 0 \\ 
    0 & \lambda_2
  \end{pmatrix} \begin{pmatrix}
    \xi_1 \\ 
    \xi_2
  \end{pmatrix} = \begin{pmatrix}
    0 \\ 0
  \end{pmatrix} \\
  \xi_2 = 0 \\ 
  \xi_1=C_1 \qquad \xi_2=C_2e^{\lambda_2 t}
\end{gather*}

\incfig{l5_1}

$\lambda_2 > 0$ нестабильная "антенна"

$\lambda_2 < 0$ стабильная "антенна"

\hr 
$\lambda_1=\lambda_2=0$, $h_1$ и $h_2$ собст. век.
\[
  x=C_1h_1 + C_2h_2
\]
Положение равновесия все точки фызовой плоскости.

\incfig{l5_dots}
"Точки"

\hr 
$\lambda_1=\lambda_2=0$, $h_1$ обст. век., $h_2$ присоед
\begin{gather*}
  x=C_1h_1+C_2(h_1t+h_2) \\ 
  S^{-1}AS=\begin{pmatrix}
    0 & 1 \\ 
    0 & 0 
  \end{pmatrix} \\ 
  \begin{pmatrix}
    0 & 1 \\ 
    0 & 0 
  \end{pmatrix}\begin{pmatrix}
    \xi_1 \\ 
    \xi_2
  \end{pmatrix} = \begin{pmatrix}
    0 \\ 0
  \end{pmatrix} \qquad \xi_2=0 \\ 
  \xi_1=C_1+C_2t \qquad \xi_2=C_2
\end{gather*}
$h_1$ - положение равновесия

\incfig{l5_street}
"Улица"

\hr

\subsection{Второй взгляд на классификацию}
\begin{gather*}
  det \begin{pmatrix}
    a_{11}-\lambda & a_{12} \\ 
    a_{21} & a_{22} - \lambda
  \end{pmatrix} \\ 
  \lambda^{2}-\lambda \underbrace{a_{11}+a_{22}}_{Tr A}+\underbrace{a_{11}a_{22}-a_{12}a_{21}}_{detA=0}=0 \\ 
  \lambda^{2}-T\lambda+D=0 \\ 
  \lambda_{1,2}=\frac{T\pm \sqrt{T^{2}-4D}}{2}
\end{gather*}

\incfig{l5_class}

\subsection{Третий взгляд на классификацию}
Насколько влияют нелинейные коэффициенты?

Грубые: седло, узел, фокус

Негрубые: остальные

\subsection{Устойчивость по Ляпунову}
\[
  \dot{x}=F(t,x) \qquad x \in \R^{n}
\]
$x(t_0)=x_0$, $\phi(t)$ - решение задачи Коши, продолжаемое на $[t_0, +\infty)$

\begin{definition}
Решение $\phi(t)$ задачи Коши называется устойчивым по Ляпунову,
если $\forall \epsilon >0 \, \exists \delta>0: \, \forall x(t)$ реш.
$: \, |x(t_0)-x_0|<\delta \; \forall t \ge t_0 \true x(t)$ определено на $[t_0, +\infty)$
и $|x(t)-\phi(t)|<\epsilon$.
\end{definition}
\begin{definition}
  Решение $\phi(t)$ з. К. называется асимпотически уст.,
  если оно устойчиво и $\exists \delta_0: \; \forall x(t)$ реш. с
  $|x(t_0)-x_0|<\delta_0 \true \lim \limits_{t \to +\infty} |x(t)-\phi(t)|=0$
\end{definition}
\begin{definition}[Неустойчивость]
  Решение $\phi(t)$ называется неуст. по Ляпунову,
  если $\exists \epsilon_0>0: \, \forall \delta>0 \; \exists x(t)$ реш.
  $: \, |x(t_0)-x_0|<\delta \; \exists \tilde{t} \ge t_0 \true x(\tilde{t})$
  неопределено или $|x(\tilde{t})-\phi(\tilde{t})|\ge \epsilon_0$.
\end{definition}
\begin{remark}
  \[
    |x(t_0)-x_0|=\sqrt{(x_1(t_0)-x_{01})^{2}+\dots +(x_n(t_0)-x_{0n})^{2}}
  \]
\end{remark}
\begin{eg}[889]
  \phantom{.}
 
  \incfig{l5_889}
  \[
    \left\{\begin{aligned}
      \dot{x}=P(x,y) \\ 
      \dot{y}=Q(x,y)
    \end{aligned}\right.
  \]
  $\phi(t)=0$ - тривиальное решение, $\phi(t_0)=0$, $x_0=0$

  Выполняются условия отрицания.
\end{eg}

\subsubsection{Исследование на устойчивость}
\[
  \dot{x}=F(t,x) \qquad x(t_0)=x_0
\] 
$\phi(t)$ - решение з. К., $x(t)=\phi(t)+\epsilon(t)$
\[
  \dot{\phi}+\dot{\epsilon}=F(t, \phi+\epsilon) \qquad \epsilon(t_0)=0 \\ 
\]
Система уравнений возмущений:
\[
\dot{\epsilon}=F(t,\phi(t)+\epsilon(t))-\dot{\phi}(t) \qquad \epsilon(t)=0
\]
\subsubsection{Теоремы об уст. (неуст.) полож. равн. авт. сис.}
\[
  \dot{x}=F(x) \qquad x \in \R^{n}
\]
$F(x)=0 \rightarrow\tilde{x}$ полож. равн., $\phi(t)=\tilde{x}$
\begin{theorem}[Об уст. по линейному прибл.]
  Пусть $\dot{x}=F(x)$, $F(\tilde{x})=0$, $F'(\tilde{x})$ матрица Якоби,
  $\lambda_i$ - собст. числа $F'(\tilde{x})$, тогда
  \begin{enumerate}
    \item Если $\forall \lambda_i: \, Re \lambda_i < 0$, то $\tilde{x}$ асимпт. уст.
    \item Если $\exists \lambda_i: \; Re \lambda_i>0$, то $\tilde{x}$ неуст.
  \end{enumerate}
\end{theorem}
\begin{definition}[Функция Ляпунова]
  $v(x)$ опр. и непр. дифф. в $O_{\epsilon}(\tilde{x})$ (окрестности)
  и $v(\tilde{x})=0$.

  Производная в силу системы:
  \[
    \frac{dv}{dt}=\pd{v}{x_1}\pd{x_1}{t}+\dots +\pd{v}{x_n}\pd{x_n}{t}=\pd{v}{x_1}f_1(x)+\dots +\pd{v}{x_n}f_n(x)
  \]
\end{definition}
\begin{theorem}[Ляпунова об устойчивости]
  Пусть $\exists v(x): \; v(x)>0$ в $\dot{O}_\epsilon(\tilde{x})$
  и $\frac{dv}{dt}\le 0$ в $O_{\epsilon}(\tilde{x})$,
  тогда $\tilde{x}$ - уст. по Ляпунову.
\end{theorem}
\begin{theorem}[Ляпунова об асимпт. уст.]
  Пусть $\exists v(x): \; v(x)>0$ в $\dot{O}_\epsilon(\tilde{x})$
  и $\frac{dv}{dt}< 0$ в $\dot{O}_{\epsilon}(\tilde{x})$,
  тогда $\tilde{x}$ - асимпт. уст. по Ляпунову.
\end{theorem}
\begin{theorem}[Ляпунова о неустойчивости]
  Пусть $\exists v(x):$ в $\forall O_\delta(\tilde{x})\in O_\epsilon(\tilde{x})$
  $\exists x^{*}: \; v(x^{*})>0$ и $\frac{dv}{dt}>0$ в $\dot{O}_\epsilon(\tilde{x})$,
  тогда $\tilde{x}$ - неуст.
\end{theorem}
\begin{theorem}[Четаева о неустойчивости]
  \phantom{.}

  \incfig{l5_chet}
  Пусть $\exists$ область $D\subset O_\epsilon(\tilde{x})$ и $\exists v(x)$,
  $\tilde{x}\in \partial_1 D$, $v(x)=0$ на $\partial_1 D$,
  $v(x)>0$ в D, $\frac{dv}{dt}>0$ в D,
  тогда $\tilde{x}$ неуст.
\end{theorem}

% lecture 6 

\subsection{Примеры на устойчивость}
\begin{eg}
  \begin{gather*}
    \left\{\begin{aligned}
      \dot{x}=y-x^{3} \\ 
      \dot{y}=-x-y^{3}
    \end{aligned}\right. \\ 
    A(0,0) \qquad F'(A)=\begin{pmatrix}
      0 & 1 \\ 
      -1 & 0
    \end{pmatrix} \qquad \lambda_{1,2}=\pm i \; \text{(центр)} \\ 
    v=x^{2}+y^{2} \\ 
    \frac{dv}{dt}=2x\dot{x}+2y\dot{y}=-2x^{4}-2y^{4} < 0
  \end{gather*}
  $v>0$ в $\dot{O}_\epsilon(A)$ и $\frac{dv}{dt}<0$ там же $\implies$ асимп. уст.
\end{eg}
\begin{eg}
  \begin{gather*}
    \left\{\begin{aligned}
      \dot{x}=-x \\ 
      \dot{y}=x+y^{2}
    \end{aligned}\right. \\ 
    A(0,0) \qquad F'(A)=\begin{pmatrix}
      -1 & 0 \\ 
      1 & 0
    \end{pmatrix} \\ 
    v=y(y+2x)
  \end{gather*}  
  В т. Четаева берём за $D$ область где $v>0$:
  $v(A)=0$, $v \big|_{\partial_1 D}=0$, $v>0$ в $D$
  \begin{gather*}
    \frac{dv}{dt}=2y^{3}+2x^{2}+2xy^{2}=y^{3}+2x^{2}+y^{2}(y+2x)>0 \; \text{в} \;D
  \end{gather*}
  Выполнены условия для Четаева $\implies$ неустойчива.
\end{eg}

\subsection{Исследование фаз. тр. нелин. автоном. сист.}
\[
  \dot{x}=F(x) \qquad x\in \R^{n} \qquad F(x)\in C'(D) \qquad D\in \R^{n}
\]
\begin{enumerate}
  \item полож. равновесия, иссл. методом линеаризации
    \begin{gather*}
      F(x)=\underbrace{F(\tilde{x})}_{0}+\underbrace{F'(x)}_{\text{м. Якоби}}(x-\tilde{x})+o(|x-\tilde{x}|) \\ 
      y=x-\tilde{x} \qquad \dot{y}=F'(\tilde{x})y+o(|y|)
    \end{gather*}
    (важно учитывать грубые или негрубые)
  \item иссл. устойчивость пол. равновесия
  \item иссл. предельные множества.
  \item иссл. фазовые траектории вне предельных множеств, метод первых интегралов
\end{enumerate}

\subsection{Первые интегралы автономных систем}
\[
  \dot{x}=F(x) \qquad x\in \R^{n} \qquad F(x)\in C'(D) \qquad D\in \R^{n}
\]
\begin{definition}
  Функция $u(x)\in C'(D)$ называется первым интегралом автономной системы,
  если для любого решения $g(t)$, фазовая траектория которого лежит в $D$
  $\true u(g(t))=const$.
\end{definition}
\begin{eg}
  $u=const$ - тривиальный первый интеграл
\end{eg}
\begin{theorem}[Критерий первого интеграла]
  Функция $u(x)\in C'(D)$ является первым интегралом тогда и только тогда,
  когда $(\nabla u,F(x))=0$ в $D$.
\end{theorem}
\begin{proof}
  $\rightarrow$: Пусть $u(x)$ первый инт:

  \incfig{l6_fi1}
  Возьмём $x_0 \in D$, $x(0)=x_0$ $\implies \; \exists$ решение $g(t)$,
  $u(g(t))=const$
  \begin{gather*}
    \frac{du}{dt}\Big|_{t=0}=\sum_{i=1}^{n}\underbrace{\pd{u}{x_i}}_{(\nabla u)_{i}}\underbrace{\frac{dx_i}{dt}\Big|_{t=0}}_{f_i(x_0)}=0 \\ 
    (\nabla u, F(x))=0
  \end{gather*}
  В силу произвольности $x_0$ выполнено в $D$.

  $\leftarrow$: Пусть $(\nabla u, F(x))=0$ в $D$.

  \incfig{l6_fi2}
  Произвольное решение $x=g(t)$
  \begin{gather*}
    \frac{du}{dt}\Big|_{\gamma}=\sum_{i=1}^{n}\pd{u}{x_i}f_{i}(x)=(\nabla u, F)\Big|_{\gamma}=0
  \end{gather*}
  Т.е. $u(g(t))=const$ на $\gamma$.
\end{proof}

\subsection{Независимость функция}
\begin{definition}
Система функций $u_1(x),\dots ,u_m(x)\in C'(D)$, $x\in R^{n}$, $m \le n$, $D \in \R^{n}$
- область., называется зависимой в $D$, если в $D$
\[
  \exists K, \exists G \in C': \: u_k(x)=G(u_1(x),\dots , u_{k-1}(x), u_{k+1}(x),\dots ,u_m(x))
\]
Если так сделать нельзя, то система называется независимой в $D$.
\end{definition}
\begin{theorem}[Необход. усл. зависимости]
  Пусть $u_1,\dots ,u_m$ зависимы в $D$, тогда $\nabla u_1,\dots , \nabla u_m$
  линейно зависимы в каждой точке $D$.
\end{theorem}
\begin{proof}
  \begin{gather*}
    \nabla u_k(x)=\pd{G}{u_1}\nabla u_1+\dots +\pd{G}{u_m}\nabla u_m
  \end{gather*}
  Получили опредление линейной зависимости $\nabla u_i$.
\end{proof}
\begin{corollary}[Достаточное условие независимости]
  Пусть $\exists x_0\in D$, в которой $\nabla u_i$ линейно незав.,
  тогда система $u_1,\dots, u_m$ незав. в $D$.
\end{corollary}
\begin{eg}
  \begin{gather*}
    u_1=x_1 \qquad u_2=x_2 \qquad u_3=x_1^{2}+u_2^{2} \\ 
    u_3=u_1^{2}+u_2^{2} \\ 
    \begin{pmatrix}
      2x_1 \\ 2x_2 \\ 0
    \end{pmatrix} = 2x_1 \begin{pmatrix}
      1 \\ 0 \\ 0
    \end{pmatrix} + 2x_2 \begin{pmatrix}
      0 \\ 1 \\ 0
    \end{pmatrix}
  \end{gather*}
\end{eg}
\begin{theorem}[Достаточное условие зависимости]
  Пусть в каждой точке области $D \true \nabla u_1,\dots ,\nabla u_m$ лин. зав.
  и не обращаются $0$ одновременно, тогда для каждой точки $D$
  существует окрестность, в которой $u_1, \dots ,u_m$ зависимы.
\end{theorem}
\begin{proof}(для $n=2$)

  $f(x,y)$, $g(x,y)$, $\nabla f$ и $\nabla g$ лин. зав. в $D$

  $(x_0,y_0)\in D$ пусть $\nabla f\big|_{A} \neq \vec{0}$, пусть $\pd{f}{x}\big|_A\neq 0$
  \begin{gather*}
    \xi=f(x,y) \qquad \eta=y \qquad J=\begin{vmatrix}
      \pd{f}{x} & \pd{f}{y} \\ 
      0 & 1
    \end{vmatrix}=\pd{f}{x}\big|_{A}\neq 0
  \end{gather*}
  $\exists O_{\epsilon}(A)$, $J\neq 0$, значит есть взаимооднозначное отображение
  \begin{gather*}
    x=x(\xi,\eta) \qquad y=y(\xi, \eta) \\ 
    f(x(\xi,\eta),y(\xi,\eta))=\xi \qquad g(x(\xi,\eta),y(\xi,\eta))  \\ 
    \underbrace{\begin{pmatrix}
          \pd{f}{\xi} & \pd{f}{\eta} \\ 
          \pd{g}{\xi} & \pd{g}{\eta}
          \end{pmatrix}}_{det = 0 \, \text{в} \, O_\delta(B)} = \underbrace{\begin{pmatrix}
          \pd{f}{x} & \pd{f}{y} \\ 
          \pd{g}{x} & \pd{g}{y}
        \end{pmatrix}}_{det = 0 \, \text{в}\, D} \begin{pmatrix}
      \pd{x}{\xi} & \pd{x}{\eta} \\ 
      \pd{y}{\xi} & \pd{y}{\eta}
    \end{pmatrix} \\ 
    \begin{vmatrix}
      1 & 0 \\ 
      \pd{g}{\xi} & \pd{g}{\eta}
    \end{vmatrix} = 0
  \end{gather*}
  $\pd{g}{\eta}=0$ в $O_\delta(B)$ $\implies$ $g=g(\eta)=h(f)$ завис в $O_\epsilon(A)$.
\end{proof}
\begin{theorem}[О существовании первых инт.]
  Пусть точка $a$ не является положением равновесия автономной системы,
  тогда в некоторой окрестности этой точки $\exists$ первые интегралы $u_1(x),\dots ,u_{n-1}(x)$
  и она независимы.
\end{theorem}
\begin{proof}
  \[
    \dot{x}=F(x) \qquad F(x) \in C'(x) \qquad F(a)\neq \vec{0}
  \]
  Пусть $f_n(a)\neq0$.

  \incfig{l6_fie}
  \begin{gather*}
    x(0)=b \qquad b = \begin{pmatrix}
      b_1 \\ \vdots \\ b_{n-1} \\\ a_n
    \end{pmatrix}
  \end{gather*}
  $x=g(t,b)$ - реш. задачи Коши
  \[
    \left\{\begin{aligned}
      & x_1 = g_1(t, b_1, \dots , b_{n-1}, a_n) \\ 
      & \dots \\
      & x_n = g_n(t, b_1, \dots , b_{n-1}, a_n) \\ 
    \end{aligned}\right.
  \]
  Можем смотреть как на решение з. К. и как на систему уравнений.

  Значит $g_i(t,b_1,\dots ,b_{n-1},a_n$ - по $t$ напр дифф как решение,
  а также по $b_j$ непр. дифф. по теор. о дифф. решения по параметру.
  \begin{gather*}
    A: t=0 \quad b_1=a_1, \dots , b_{n-1}=a_{n-1}
  \end{gather*}
  Дифф. по $b_j$, $t=0$:
  \[
    \left\{\begin{aligned}
      & b_1=g_1(0,b_1,\dots ,b_{n-1},a_n) \\ 
      & \dots \\ 
      & a_n=g_n(0,b_1,\dots ,b_{n-1},a_n) \\ 
    \end{aligned}\right. \qquad 
    \pd{g_1}{b_1}\Big|_{A}=1; \; \pd{g_1}{b_j}\Big|_{A}=0;\; \pd{g_n}{b_i}\Big|_{A}=0
  \]
  Дифф. по $t$ в $A$: $b_i=a_i$
  \[
    \left\{\begin{aligned}
      & x_1=g_1(t,a_1,\dots ,a_n) \\ 
      & \dots \\
      & x_n=g_n(t,a_1,\dots ,a_n) \\ 
    \end{aligned}\right. \qquad \dot{g}_1(t)=f_1(a)
  \]
  \[
    J=\frac{\partial(g_1,\dots ,g_n)}{\partial(b_1,\dots ,b_{n-1},t)}=\begin{vmatrix}
      1 & 0 & \dots & 0 & f_1(a) \\
      0 & 1 & \dots & 0 & f_2(a) \\
      \dots & \dots & \dots & \dots & \dots \\
      0 & 0 & \dots & 0 & f_n(a) \\
    \end{vmatrix}
  \]
\end{proof}

\end{document}
