\documentclass{article}


\usepackage{amsmath, amsthm, amsfonts}
\usepackage[utf8]{inputenc}
\usepackage[T2A]{fontenc}
\usepackage[english, russian]{babel}

\usepackage{import}
\usepackage{pdfpages}
\usepackage{transparent}
\usepackage{xcolor}

\usepackage{parskip}
\usepackage{systeme}

\newcommand{\incfig}[2][1]{%
    \def\svgwidth{#1\columnwidth}
    \import{./figures/}{#2.pdf_tex}
}

\pdfsuppresswarningpagegroup=1

\usepackage{hyperref}
\hypersetup{
    colorlinks=true, %set true if you want colored links
    linktoc=all,     %set to all if you want both sections and subsections linked
    linkcolor=black,  %choose some color if you want links to stand out
}

\newcommand\hr{
    \noindent\rule[0.5ex]{\linewidth}{0.5pt}
}

% All the environments
\usepackage{mdframed}
\mdfsetup{skipabove=1em,skipbelow=0em}
\theoremstyle{definition}
\newmdtheoremenv[nobreak=true]{theorem}{Теорема}
\newmdtheoremenv[nobreak=true]{lemma}{Лемма}
\newmdtheoremenv[nobreak=true]{definition}{Определение}
\newmdtheoremenv[nobreak=true]{corollary}{Следствие}
\newtheorem*{eg}{Пример}
\newtheorem*{remark}{Замечание}

% Defs
\let\phi\varphi
\let\epsilon\varepsilon
\let\implies\Rightarrow
\let\iff\Leftrightarrow
\let\true\hookrightarrow

\newcommand{\pd}[2]{\frac{\partial{#1}}{\partial{#2}}}
\newcommand{\pdd}[2]{\frac{\partial^2{#1}}{\partial{#2^2}}}
\newcommand{\pdm}[3]{\frac{\partial^2{#1}}{\partial{#2}\partial{#3}}}
\newcommand\R{\ensuremath{\mathbb{R}}}


\begin{document}

% lecture 1

\section{Введение}
\begin{definition}[Динамическая система]
  Система у которой
  \begin{itemize}
    \item Определено состояние как совокупность велечин или функция в данный момент времени
    \item Задан закон эволюции системы
  \end{itemize} 
  Изучаем вектор состояния $\vec{x}=\{x_1,\dots,x_n\}, \; \vec{x}\in X\in \R^n$
\end{definition}
\begin{definition}[Типы законов эволюции] /
  \begin{itemize}
    \item Системы с непрерывным временем (потоки) $\dot{\vec{x}}=\vec{f}(\vec{x})$, $\vec{x}$ - автономная
    \item Системы с дискретным временем (каскады) $\vec{x}_{k+1}=T_k \vec{x}_k$,
      здесь $T_k$ - оператор, в большинстве случаев - функция
  \end{itemize}
\end{definition}
\begin{eg}
  Для описание системы движущегося колеса используем два параметра: скорость и угол. 
  Вектор состояния $\vec{x}=\{x_0 \; x_1\}$.
  \[
    \frac{\vec{dx}}{t}=f(\vec{x},t)
  \]
  \[
    \left\{\begin{aligned}
        & x_1= f_1(\vec{x},t)  \\
        & \dots  \\
        & x_n= f_n(\vec{x},t)  \\
    \end{aligned}\right.
  \]
\end{eg}
\begin{definition}[Автономная система]
  Если правая часть системы дифф. уравнений не зависит от времени система называется автономной. 
  Когда система не автономная над ней находится сверхсистема,
  изменение которой нужно изучить и определить её воздействие на нашу систему в зависимости от времени.
\end{definition}
\begin{eg}
  Спутник на орбите - автономная система, сила воздействующая на него зависит только от координат.
  А если Земля не округлая и вращается нужно вводить начальное положение и изменение от времени, система уже не автономная.
\end{eg}

\begin{eg}
  Если мы смотрим на популяцию кроликов раз в пол года имеем следующий закон изменения. 
  \[
    x_{k+1}=f_1(x_k)
  \]
\end{eg}

\section{Описание системы}
\begin{definition}[Твёрдое тело]
  Система мат точек, у которых не изменяется расстояние между любыми 2мя точками.
\end{definition}
Имеем систему $N$ материальных точек, где $\nu$ - номер точки. 
Движение описывается в виде
\[
  \vec{r}_\nu(t), \; \nu=1,\dots,N
\]
\[
\vec{v}_\nu(t)=\dot{\vec{r}}_\nu(t) \qquad \vec{w}_\nu(t)=\dot{\vec{v}}_\nu(t) 
\]
Связи $f(t,\vec{r}_1,\dots,\vec{r}_\nu,\dots,\vec{r}_N, \dot{\vec{r}}_1, \dots,\dot{\vec{r}}_\nu, \dots, \dot{\vec{r}}_N) =0$, кратко $f(t,\vec{r}_\nu,\dot{\vec{r}}_N)=0$. \\ 
Конечные связи - не зависят от скорости $f(t,\vec{r}_\nu)=0$, 
стационарные - не зависят от времени $f(\vec{r}_\nu)=0$, интегрируемые $f(t,\vec{r}_\nu)=G$

\begin{eg}
\phantom{.}

\incfig{l1_wheel}
В колесе без проскальзывания:
\[
  \dot{x_1}=R\dot{\phi}_1
\]
\end{eg}

\begin{eg}
  \phantom{.}

  \incfig{l1_rod}
  Стационарная и нестационарная связи:
  \[
    (\vec{r}_1-\vec{r}_2)^2=l^2 \qquad\qquad (\vec{r}_1-\vec{r}_2)^2=l^2(t)
  \]
\end{eg}

\begin{definition}[Голономные системы]
  Системы мат точек, в которых нет дифференицальных неинтегрирумеых связей.
\end{definition}
\begin{remark}
  Если налолжено $d$ связей (голономных) для описание положения нужно $3N-d$ переменных.
\end{remark}
\begin{definition}[Кол-во степеней свободы]
  Минимальное количество независимых переменных которые однозначно определяют \textbf{положение} системы называется количеством степеней свободы.
\end{definition}
\begin{remark}
  Состояние отличается от положения тем, что в неё входят скорости и зная сотояние, мы может предсказать как система будет развиваться.
\end{remark}
\begin{definition}[Параметризация системы]
  Введение параметров \\ $q_1,\dots,q_n$ ($q=\{q_1,\dots,q_n\}$-обобщённые координаты) таких, что однозначнозначно описывают положение:
  \[
    \vec{r}_\nu=\vec{r}_\nu(t,q_1,\dots,q_n) \qquad \nu=1,\dots,N
  \]
\end{definition}

% lecture 2

\begin{eg}
  \phantom{.}

  \incfig{l2_wheel}
  Твёрдое тело круг на плоскости, три обощённые координаты $x,y,\varphi$ соответственно
  $q_1,q_2,q_3$.
\end{eg}
\begin{definition}
  Если нет нестационарных связей то $\vec{r}_\nu(q)$ склерономные системы
  (не зависят от времени).
\end{definition}

\section{Лагранжева механика}
\incfig{l2_lagrange} \phantom{.}
\[
  \left\{\begin{aligned}
    y = -x\tan \alpha \\ 
    m_{\nu}\ddot{\vec{r}}_{\nu}=\sum_{}^{}\vec{F}_{\nu}
  \end{aligned}\right. \qquad \ddot{q_1}=g\sin\alpha
\]
\subsection{Уравнения Лагранжа 2-го рода}
Применимо к голономным системам.
\subsubsection{Кинетическая энергия}
\[
  \frac{d}{dt}\left(\frac{\partial T}{\partial \dot{q}_i}\right)-\frac{\partial T}{\partial q_i}=Q_i \quad i=1\dots n 
\]
$T(\dot{q},q,t)$ - кинетическая энергия, $Q_i(\dot{q},q,t)$ - обобщённые силы
\[
  \delta A =\sum_{\nu=1}^{N}\vec{F}_\nu \delta \vec{r}_\nu=\sum_{i=1}^{N}Q_i \delta q_i
\]
$\delta \vec{r}_\nu$ - виртуальное перемещение.
\begin{definition}[Виртуальное перемещение]
  Виртуальное пермещение - перемещение точек системы с учётом наложенных связей
  не изменяющихся (замороженных) во времени.
\end{definition}
\begin{eg}
  \phantom{.}

  \incfig{l2_lag_eg}
  Одна степерь свободы - угол вращения. Начинаем варьировать $\phi$ чтобы выразить
  $\delta \vec{r}_\nu$ через $\delta q_i$ получаем работу и тогда можем выразить
  $Q_i$. В случае когда степеней свободы много нужно жёстко
  фиксировать все кроме одной.
\end{eg}
\subsubsection{Потенциальная энергия}
\[
  Q_i=-\frac{\partial \Pi(q,t)}{\partial q_i}
\]
\begin{eg}
  \phantom{.}
  \begin{itemize}
    \item Гравитационное поле: $\Pi=mgh$ или $\Pi=-\frac{\gamma Mm}{r}$ при нескольких телах.
    \item Пружина $\Pi=\frac{c(\Delta x)^{2}}{2}$. 
    \item Точка на стержне прикреплённого к оси вращения: возникает цетробежная сила
    если рассматривать систему отсчёта связанную с осью и стрежнем $F_x(x)=m\omega^{2}x$
    тогда потенциальная энергия центробежной силы $\Pi=-\-\frac{m\omega^{2}x^{2}}{2}$
  \end{itemize}
\end{eg}
\begin{remark}
  Потенциальная энергия всегда увеличивается в сторону обратную направлению силы.
\end{remark}
\subsubsection{Лагранжиан}
\[
  L = T - \Pi
\]
\[\boxed{
  \frac{d}{dt}\left(\frac{\partial L}{\partial \dot{q}}\right) - \frac{\partial L}{ \partial q_i}=0 \quad i=1\dots n
}\]
\subsubsection{Стректура кинетической энергии}
Для набора материальных точек ($\vec{r}_\nu=\vec{r}_\nu(q-t)$):
\begin{gather*}
  T=\frac{1}{2}\sum_{\nu=1}^{N}m_\nu(\dot{\vec{r}}_\nu)^2=
  \frac{1}{2}\sum_{\nu=1}^{N}m_\nu\left(\sum_{i=1}^{N}\pd{\vec{r}_\nu}{q_i}\dot{q}_i+\pd{\vec{r}_\nu}{t}\right)^2 \\ 
  a_{jk}=\sum_{\nu=1}^{N}m_\nu\pd{\vec{r}_\nu}{q_j}\pd{\vec{r}_\nu}{q_k} \qquad 
  a_{j}=\sum_{\nu=1}^{N}m_\nu\pd{\vec{r}_\nu}{q_j}\pd{\vec{r}_\nu}{t} \qquad
  a_{0}=\frac{1}{2}\sum_{\nu=1}^{N}m_\nu\left(\pd{\vec{r}_\nu}{t}\right)^{2} \\ 
  T=\underbrace{\frac{1}{2}\sum_{j,k=1}^{n}a_{jk}\dot{q}_j\dot{q}_k}_{T_2} + \underbrace{\sum_{j=1}^{n}a_j\dot{q}_j}_{T_1}+\underbrace{a_0}_{T_0} \\ 
  T=T_2+T_1+T_0
\end{gather*}
Для склерономной $T=T_2$

\begin{eg}
  \phantom{.}

  \incfig{l2_kinetic}
$q$ - расстояние от начала стержня до точки $a$
  \begin{gather*}
    \phi=\omega t \qquad x=q\cos(\omega t) \qquad y=q\sin(\omega t) \\ 
    \dot{x}=\dot{q}\cos(\omega t) - q \omega \sin(\omega t) \qquad
    \dot{y}=\dot{q}\sin(\omega t) + q \omega \cos(\omega t) \\ 
    T=\frac{1}{2}m(\dot{x}^{2}+\dot{y}^{2})^{2}=\frac{m}{2}[\dot{q}^{2}+\omega^{2}q^{2}] \\ 
    T=\frac{1}{2}m(v_r^{2}+v_{\phi}^{2})=\frac{m}{2}[\dot{q}^{2}+\omega^{2}q^{2}]
  \end{gather*}
\end{eg}
\begin{eg}
  \phantom{.}

  \incfig{l2_kinetic_dots}
  Теорема Кёнига:
  \begin{enumerate}
    \item Кёнигова система отсчёта
    \begin{enumerate}
      \item Центр находится в центре масс
      \item При движении системы кёнигова система точек движется поступательно
    \end{enumerate}
  \item $T=\frac{1}{2}m_cv_c^2+T^{\text{отн}}$, 
    для твёрдого тела (на плоскости) $T^{\text{отн}}=\frac{1}{2}I_{\text{ось вр}}\omega^{2}$
  \end{enumerate}
\end{eg}

%lecture 3

\subsubsection{Алгоритм решения}
\begin{enumerate}
  \item Определить количество степеней свободы
  \item Выбрать обобщённую систему координат
  \item $T(\dot{q},q,t)$
  \item $\Pi(q,t)$
  \item $L=T-\Pi$
  \item $\frac{d}{dt}\left(\pd{L}{q_i}\right)-\pd{L}{q_i}=0$, $i=1,\dots ,n$
  \item Вычислить производные
\end{enumerate}

\subsection{Классификация обобщённых сил}
\[
  Q_i=-\pd{\Pi}{q_i}+\tilde Q_i(\dot{q},q,t)
\]
$\tilde Q$ - непотенциальная сила
\[
  N=\sum_{i=1}^{n}Q_i\dot{q_i}
\]
- мощность обобщённых сил.
  \begin{gather*}
    \vec{Q}=-Q_qq-Q_{\dot{q}}\dot{q} \\ 
    C=\frac{1}{2}(Q_q+Q_q^{T})=C^{T} \qquad B=\frac{1}{2}(Q_{\dot{q}}+Q_{\dot{q}}^{T})=B^{T} \\
    P=\frac{1}{2}(Q_q-Q_q^{T})=-P^{T}  \qquad G=\frac{1}{2}(Q_{\dot{q}}-Q_{\dot{q}}^{T})=-G^{T} \\
    \vec{Q}=-Cq-Pq-B\dot{q}-G\dot{q}
  \end{gather*}
\begin{center}
\begin{tabular}{| c | c | c |}
  \hline
  матрица & $q$ & $\dot{q}$ \\ 
  \hline
  симметричные & консервативные $C$ & диссипативные $B$ \\ 
  \hline
  кососимметричные & существенно непотенц. $P$ & гироскопические $G$ \\
  \hline
\end{tabular}
\end{center}
\[
  T(\dot{q},q,t)=T_2+T_1+T_0
\]
Индекс показывает степень с которой $T_i$ зависит от $\dot{q}$. В склерономной 
$T_1$ и $T_0$ нет.
\[
  E=T+\Pi=T_2+T_1+T_0+\Pi
\]
\subsubsection{Изменение полной энергии}
\[
  \frac{dE}{dt}=\sum_{i=1}^{n}\tilde Q_i \dot{q}_i + \frac{d}{dt}(T_1+2T_0)-\pd{T}{t}+\pd{\Pi}{t}
\]
Условия консервативной системы (не учитывая экзотику когда слогаемые сокращаются):
\begin{enumerate}
  \item Если система склерономная $T_1=T_0=0 \; \implies \; \pd{T}{t}=0$
  \item Если $\pd{\Pi}{t}=0$.
  \item Если $\sum_{i=1}^{n}\tilde Q_i \dot{q}_i=0$
\end{enumerate}
\section{Гамильтонова механика}
\begin{gather*}
  L(\dot{q},q,t)=T-\Pi \\ 
  \frac{d}{dt}\left(\pd{L}{q_i}\right)-\pd{L}{q_i}=0 \\ 
  \left\{\begin{aligned}
      & a_{11}\ddot{q}_1 + a_{12}\ddot{q}_2 + \dots =0 \\ 
      & \dots 
  \end{aligned}\right.
\end{gather*}
В лагранжевой механике на очевидно что такое лагранжиан и итоговые
уравнения получаются в ненормальном виде из-за чего их сложно решать.
\subsection{Переход к новым переменным}
$q,\dot{q},t$ ($\dot{q}$-обобщённые скорости) $\Rightarrow$ $q,p,t$ ($p$ - обобщённый импульс)

$L(\dot{q},q,t) \Rightarrow H(q,p,t)$ функция гамильтона (гамильтониан)
\[
  p_i=\pd{L}{q_i} \qquad \dot{\hat{q}}=\dot{\hat{q}}(q,p,t)
\]
Преобразование Лежандра:
\[
  H(q,p,t)=\sum_{i=1}^{n}p_i \dot{\hat{q}}_i - L(q, \dot{\hat q},t)
\]
На место $\dot{\hat q}$ нужно подставить зависимость выше.

Канонические уравнения Гамильтона:
\[
  \left\{\begin{aligned}
    \dot{q}_i=\pd{H}{p_i} \\ 
    \dot{p}_i=-\pd{H}{q_i}
  \end{aligned}\right. \qquad i=1,\dots ,n
\]
Уравнения 1го порядка в нормальном виде!

\[
  H=T_2-T_0+\Pi
\]
В склерономной систему $T_0=0$ $\implies$ $H=T_2+\Pi=T+\Pi=E$

% lecture 4

\section{Первый интеграл}
\[
  \left\{\begin{aligned}
    \dot{x}_j=g_j(x,t) \qquad j=1,\dots ,m \qquad m=2n
  \end{aligned}\right.
\]
\[
  x_j^{*}=x_{j}^{*}(t,C_1,\dots ,C_m)
\]
\begin{definition}[Первый интеграл]
  $f(x,t)$ - первый итеграл, если при подстановке любого решения $x^{*}$,
  её значение констанка.
  \[
    \frac{df}{dt}=\pd{f}{t}+\sum_{j=1}^{m}\pd{f}{x}\dot{x}_j=0
  \]
\end{definition}
Законы сохранения $f(x)=const$ тоже являются первыми интегралами.

Если набрать $m$ функционально независимых первых интегралов можно выразить $x_j$
через соответствующие константы и $t$ ($f_k(x,t)=C_k$)
и тогда не нужно решать диффур.

Если же первые интегралы не зависят от $t$ ($f(x)=C_k$)
всего их может быть $m-1$.
\begin{definition}
  $k$ первых интегралов $f_k(x_1,\dots ,x_m,t)$ называются функционально
  независимыми в области $D$ если в каждой точке $D$ ранг матрицы
  $\left(\pd{f_i}{x_j}\right)$, $i=1,\dots, k$, $j=1,\dots ,n$
  равен k.
\end{definition}
\subsection{Гамильтоновы системы}
\[
  \frac{df}{dt}=\pd{f}{t}+\sum_{i=1}^{n}\left(\pd{f}{q_i}\pd{H}{p_i}-\pd{f}{p_i}\pd{H}{q_i}\right)=0
\]
\begin{definition}[Скобки Пуассона]
  Пусть $\exists$ дважды непр. дифф. функции гамильтоновых переменных
  $\phi(q,p,t)$ и $\psi(q,p,t)$
  \[
    (\phi,\psi)=\sum_{i=1}^{n}\left(\pd{\phi}{q_i}\pd{\psi}{p_i}-\pd{\phi}{p_i}\pd{\psi}{q_i}\right)
  \]
\end{definition}
\begin{remark}
  $(\phi,\phi)=0$, $(\phi,\psi)=-(-\psi,\phi)$
\end{remark}
\begin{theorem}[Критерий перв. инт. гам. сист.]
  \[
    \frac{df}{dt}=\pd{f}{t}+(f,H)=0
  \]
\end{theorem}
\begin{theorem}[Теоремя Якоби-Пуассона]
  Скобка Пуассона от двух первых интегралов гамильтоновой системы
  также является первым интегралом.
\end{theorem}
\subsection{Первые инт. гам. сист}
\begin{enumerate}
  \item Если $H$ не зависит от $t$, он явл. первым интегралом,
    $H(q,p)=h$ - обобщённый интеграл энергии. ($H=T+\Pi=E$, если $T=T_2$)
    \[
      \pd{H}{t}=0 \implies \frac{dH}{dt}=(H,H)=0
    \]
  \item Если $H$ не зависит от координаты $q_k$ она называется циклической
    и существует первый интеграл $p_k=const$.
  \item Если пара переменных $q_k$ и $p_k$ входит в $H$ в виде одной функции
    $z_k=z_k(q_k,p_k)$ то $z_k(q_k,p_k)=const$ является первым интегралом.
    Переменная $q_k$ называется отделимой.
\end{enumerate}
\begin{eg}
  \[
    H(q_1,q_2,p_1,p_2)=q_1^{2}+p_1^{2}+\frac{q_2p_2}{e^{q_1^{2}+p_1^{2}}}\sin(q_2p_2)
  \]
\end{eg}
\begin{eg}
  \phantom{.}

  \incfig{l4_fint_eg1}
  Материальная точка в постоянном поле тяжести.
  \[
    H=\underbrace{\frac{p_1^{2}}{2m}}_{const}+\underbrace{\frac{p_2^{2}}{2m}+mgq_2}_{const}=const
  \]
  Получаем циклическую переменную, отделимую перменную и обобщённый инт. энергии.
\end{eg}
\begin{eg}
  Мат. точка на парабалоиде.

  \incfig{l4_fint_eg2}
  $\phi$ добавляет степень свободы однако она не влияет на потенциальную энергию,
  поэтому она и называется циклической.
  \[
    \Pi=mgh
  \]
\end{eg}
\begin{eg}
  \[
    H(q,p)=T_2-T_0+\Pi=const
  \]
  \phantom{.}

  \incfig{l2_kinetic}
  \begin{center}
  \begin{tabular}{| c | c | c |}
    \hline
     & инерциальная & неинерц \\
    \hline
    $T$ & $\frac{m}{2}[\dot{q}^{2}+\omega^{2}q^{2}]$ & $\frac{m}{2}\dot{q}^{2}$  \\
    \hline
    $\Pi$ & $0$ & $-\frac{m\omega^{2}q^{2}}{2}$ \\ 
    \hline
    $E$ & $\frac{m}{2}[\dot{q}^{2}+\omega^{2}q^{2}]$ & $\frac{m}{2}\dot{q}^{2}-\frac{m\omega^{2}q^{2}}{2}$ \\
    \hline
    $H$ & $\frac{p^{2}}{2m}-\frac{m\omega^{2}q^{2}}{2}$ & $\frac{p^{2}}{2m}-\frac{m\omega^{2}q^{2}}{2}$\\
    \hline
  \end{tabular}
  \end{center}
\end{eg}
\subsection{Интегрируемость гамильтоновых систем}
  \[
    \{\dot{x}_j=g_j(x,t)
  \]
  Пусть известно $l$ первых интегралов
  \[
    f_k(x,t)=C_k \qquad k=1,\dots ,l
  \]
  Выразим $l$ переменных $x_j$ через остальные $x_{m-l}$.
  Понизим порядок системы дифф. уравнений.
  Если система автономна
  \[
    \{\dot{x}_j=g_j(x)
    \]
    и известно $n-1$ первых инт. получим
  \[
    \frac{dx_m}{dt}=g_m(x_m,C_1,\dots ,C_{m-1}) \qquad \int_{}^{}\frac{dx_m}{g_m(x_m,C_1,\dots ,C_{m-1})}=t+C_m
  \]

\begin{enumerate}
  \item $l$ циклических инт. понижают порядок системы на $2l$
    \begin{gather*}
      \left\{\begin{aligned}
        &\dot{q}_i=\pd{H}{p_i} \\ 
        &\dot{p_i}=-\pd{H}{q_i}
      \end{aligned}\right. \\ 
      p_i = C_k \\ 
      \dot{q}_k=\pd{\tilde{H}}{C_k}=F_k(c_1,\dots ,C_l,C_{2n-2l+1},\dots ,C_{2n},t)
    \end{gather*}
  \item Если $n$ координат отделимы. То гамильт. сист инт.
    Если сист. 2-го порядка имеет 1-интеграл $\implies$ интегрируема.
\end{enumerate}

\end{document}

