\documentclass{article}


\usepackage{amsmath, amsthm, amsfonts, amssymb}
\usepackage[utf8]{inputenc}
\usepackage[T2A]{fontenc}
\usepackage[english, russian]{babel}

\usepackage{import}
\usepackage{pdfpages}
\usepackage{transparent}
\usepackage{xcolor}

\usepackage{parskip}
\usepackage{systeme}

\newcommand{\incfig}[2][1]{%
    \def\svgwidth{#1\columnwidth}
    \import{./figures/}{#2.pdf_tex}
}

\pdfsuppresswarningpagegroup=1

\usepackage{hyperref}
\hypersetup{
    colorlinks=true, %set true if you want colored links
    linktoc=all,     %set to all if you want both sections and subsections linked
    linkcolor=black,  %choose some color if you want links to stand out
}

\newcommand\hr{
    \noindent\rule[0.5ex]{\linewidth}{0.5pt}
}

% All the environments
\usepackage{mdframed}
\mdfsetup{skipabove=1em,skipbelow=0em}
\theoremstyle{definition}
\newmdtheoremenv[nobreak=true]{theorem}{Теорема}
\numberwithin{theorem}{section}
\newmdtheoremenv[nobreak=true]{lemma}{Лемма}
\numberwithin{lemma}{section}
\newmdtheoremenv[nobreak=true]{definition}{Определение}
\numberwithin{definition}{section}
\newmdtheoremenv[nobreak=true]{corollary}{Следствие}
\numberwithin{corollary}{section}
\newtheorem*{eg}{Пример}
\newtheorem*{remark}{Замечание}

\numberwithin{equation}{section}

% Defs
\let\phi\varphi
\let\epsilon\varepsilon
\let\kappa\varkappa
\let\implies\Rightarrow
\let\iff\Leftrightarrow
\let\true\hookrightarrow

\newcommand{\pd}[2]{\frac{\partial{#1}}{\partial{#2}}}
\newcommand{\pdd}[2]{\frac{\partial^2{#1}}{\partial{#2^2}}}
\newcommand{\pdm}[3]{\frac{\partial^2{#1}}{\partial{#2}\partial{#3}}}
\newcommand\R{\ensuremath{\mathbb{R}}}


\begin{document}

% lecture 1

\section{Введение}
\begin{definition}[Динамическая система]
  Система у которой
  \begin{itemize}
    \item Определено состояние как совокупность велечин или функция в данный момент времени
    \item Задан закон эволюции системы
  \end{itemize} 
  Изучаем вектор состояния $\vec{x}=\{x_1,\dots,x_n\}, \; \vec{x}\in X\in \R^n$
\end{definition}
\begin{definition}[Типы законов эволюции] /
  \begin{itemize}
    \item Системы с непрерывным временем (потоки) $\dot{\vec{x}}=\vec{f}(\vec{x})$, $\vec{x}$ - автономная
    \item Системы с дискретным временем (каскады) $\vec{x}_{k+1}=T_k \vec{x}_k$,
      здесь $T_k$ - оператор, в большинстве случаев - функция
  \end{itemize}
\end{definition}
\begin{eg}
  Для описание системы движущегося колеса используем два параметра: скорость и угол. 
  Вектор состояния $\vec{x}=\{x_0 \; x_1\}$.
  \[
    \frac{\vec{dx}}{t}=f(\vec{x},t)
  \]
  \[
    \left\{\begin{aligned}
        & x_1= f_1(\vec{x},t)  \\
        & \dots  \\
        & x_n= f_n(\vec{x},t)  \\
    \end{aligned}\right.
  \]
\end{eg}
\begin{definition}[Автономная система]
  Если правая часть системы дифф. уравнений не зависит от времени система называется автономной. 
  Когда система не автономная над ней находится сверхсистема,
  изменение которой нужно изучить и определить её воздействие на нашу систему в зависимости от времени.
\end{definition}
\begin{eg}
  Спутник на орбите - автономная система, сила воздействующая на него зависит только от координат.
  А если Земля не округлая и вращается нужно вводить начальное положение и изменение от времени, система уже не автономная.
\end{eg}

\begin{eg}
  Если мы смотрим на популяцию кроликов раз в пол года имеем следующий закон изменения. 
  \[
    x_{k+1}=f_1(x_k)
  \]
\end{eg}

\section{Описание системы}
\begin{definition}[Твёрдое тело]
  Система мат точек, у которых не изменяется расстояние между любыми 2мя точками.
\end{definition}
Имеем систему $N$ материальных точек, где $\nu$ - номер точки. 
Движение описывается в виде
\[
  \vec{r}_\nu(t), \; \nu=1,\dots,N
\]
\[
\vec{v}_\nu(t)=\dot{\vec{r}}_\nu(t) \qquad \vec{w}_\nu(t)=\dot{\vec{v}}_\nu(t) 
\]
Связи $f(t,\vec{r}_1,\dots,\vec{r}_\nu,\dots,\vec{r}_N, \dot{\vec{r}}_1, \dots,\dot{\vec{r}}_\nu, \dots, \dot{\vec{r}}_N) =0$, кратко $f(t,\vec{r}_\nu,\dot{\vec{r}}_N)=0$. \\ 
Конечные связи - не зависят от скорости $f(t,\vec{r}_\nu)=0$, 
стационарные - не зависят от времени $f(\vec{r}_\nu)=0$, интегрируемые $f(t,\vec{r}_\nu)=G$

\begin{eg}
\phantom{.}

\incfig{l1_wheel}
В колесе без проскальзывания:
\[
  \dot{x_1}=R\dot{\phi}_1
\]
\end{eg}

\begin{eg}
  \phantom{.}

  \incfig{l1_rod}
  Стационарная и нестационарная связи:
  \[
    (\vec{r}_1-\vec{r}_2)^2=l^2 \qquad\qquad (\vec{r}_1-\vec{r}_2)^2=l^2(t)
  \]
\end{eg}

\begin{definition}[Голономные системы]
  Системы мат точек, в которых нет дифференицальных неинтегрирумеых связей.
\end{definition}
\begin{remark}
  Если налолжено $d$ связей (голономных) для описание положения нужно $3N-d$ переменных.
\end{remark}
\begin{definition}[Кол-во степеней свободы]
  Минимальное количество независимых переменных которые однозначно определяют \textbf{положение} системы называется количеством степеней свободы.
\end{definition}
\begin{remark}
  Состояние отличается от положения тем, что в неё входят скорости и зная сотояние, мы может предсказать как система будет развиваться.
\end{remark}
\begin{definition}[Параметризация системы]
  Введение параметров \\ $q_1,\dots,q_n$ ($q=\{q_1,\dots,q_n\}$-обобщённые координаты) таких, что однозначнозначно описывают положение:
  \[
    \vec{r}_\nu=\vec{r}_\nu(t,q_1,\dots,q_n) \qquad \nu=1,\dots,N
  \]
\end{definition}

% lecture 2

\begin{eg}
  \phantom{.}

  \incfig{l2_wheel}
  Твёрдое тело круг на плоскости, три обощённые координаты $x,y,\varphi$ соответственно
  $q_1,q_2,q_3$.
\end{eg}
\begin{definition}
  Если нет нестационарных связей то $\vec{r}_\nu(q)$ склерономные системы
  (не зависят от времени).
\end{definition}

\section{Лагранжева механика}
\incfig{l2_lagrange} \phantom{.}
\[
  \left\{\begin{aligned}
    y = -x\tan \alpha \\ 
    m_{\nu}\ddot{\vec{r}}_{\nu}=\sum_{}^{}\vec{F}_{\nu}
  \end{aligned}\right. \qquad \ddot{q_1}=g\sin\alpha
\]
\subsection{Уравнения Лагранжа 2-го рода}
Применимо к голономным системам.
\subsubsection{Кинетическая энергия}
\[
  \frac{d}{dt}\left(\frac{\partial T}{\partial \dot{q}_i}\right)-\frac{\partial T}{\partial q_i}=Q_i \quad i=1\dots n 
\]
$T(\dot{q},q,t)$ - кинетическая энергия, $Q_i(\dot{q},q,t)$ - обобщённые силы
\[
  \delta A =\sum_{\nu=1}^{N}\vec{F}_\nu \delta \vec{r}_\nu=\sum_{i=1}^{N}Q_i \delta q_i
\]
$\delta \vec{r}_\nu$ - виртуальное перемещение.
\begin{definition}[Виртуальное перемещение]
  Виртуальное пермещение - перемещение точек системы с учётом наложенных связей
  не изменяющихся (замороженных) во времени.
\end{definition}
\begin{eg}
  \phantom{.}

  \incfig{l2_lag_eg}
  Одна степерь свободы - угол вращения. Начинаем варьировать $\phi$ чтобы выразить
  $\delta \vec{r}_\nu$ через $\delta q_i$ получаем работу и тогда можем выразить
  $Q_i$. В случае когда степеней свободы много нужно жёстко
  фиксировать все кроме одной.
\end{eg}
\subsubsection{Потенциальная энергия}
\[
  Q_i=-\frac{\partial \Pi(q,t)}{\partial q_i}
\]
\begin{eg}
  \phantom{.}
  \begin{itemize}
    \item Гравитационное поле: $\Pi=mgh$ или $\Pi=-\frac{\gamma Mm}{r}$ при нескольких телах.
    \item Пружина $\Pi=\frac{c(\Delta x)^{2}}{2}$. 
    \item Точка на стержне прикреплённого к оси вращения: возникает цетробежная сила
    если рассматривать систему отсчёта связанную с осью и стрежнем $F_x(x)=m\omega^{2}x$
    тогда потенциальная энергия центробежной силы $\Pi=-\-\frac{m\omega^{2}x^{2}}{2}$
  \end{itemize}
\end{eg}
\begin{remark}
  Потенциальная энергия всегда увеличивается в сторону обратную направлению силы.
\end{remark}
\subsubsection{Лагранжиан}
\[
  L = T - \Pi
\]
\[\boxed{
  \frac{d}{dt}\left(\frac{\partial L}{\partial \dot{q}}\right) - \frac{\partial L}{ \partial q_i}=0 \quad i=1\dots n
}\]
\subsubsection{Стректура кинетической энергии}
Для набора материальных точек ($\vec{r}_\nu=\vec{r}_\nu(q-t)$):
\begin{gather*}
  T=\frac{1}{2}\sum_{\nu=1}^{N}m_\nu(\dot{\vec{r}}_\nu)^2=
  \frac{1}{2}\sum_{\nu=1}^{N}m_\nu\left(\sum_{i=1}^{N}\pd{\vec{r}_\nu}{q_i}\dot{q}_i+\pd{\vec{r}_\nu}{t}\right)^2 \\ 
  a_{jk}=\sum_{\nu=1}^{N}m_\nu\pd{\vec{r}_\nu}{q_j}\pd{\vec{r}_\nu}{q_k} \qquad 
  a_{j}=\sum_{\nu=1}^{N}m_\nu\pd{\vec{r}_\nu}{q_j}\pd{\vec{r}_\nu}{t} \qquad
  a_{0}=\frac{1}{2}\sum_{\nu=1}^{N}m_\nu\left(\pd{\vec{r}_\nu}{t}\right)^{2} \\ 
  T=\underbrace{\frac{1}{2}\sum_{j,k=1}^{n}a_{jk}\dot{q}_j\dot{q}_k}_{T_2} + \underbrace{\sum_{j=1}^{n}a_j\dot{q}_j}_{T_1}+\underbrace{a_0}_{T_0} \\ 
  T=T_2+T_1+T_0
\end{gather*}
Для склерономной $T=T_2$

\begin{eg}
  \phantom{.}

  \incfig{l2_kinetic}
$q$ - расстояние от начала стержня до точки $a$
  \begin{gather*}
    \phi=\omega t \qquad x=q\cos(\omega t) \qquad y=q\sin(\omega t) \\ 
    \dot{x}=\dot{q}\cos(\omega t) - q \omega \sin(\omega t) \qquad
    \dot{y}=\dot{q}\sin(\omega t) + q \omega \cos(\omega t) \\ 
    T=\frac{1}{2}m(\dot{x}^{2}+\dot{y}^{2})^{2}=\frac{m}{2}[\dot{q}^{2}+\omega^{2}q^{2}] \\ 
    T=\frac{1}{2}m(v_r^{2}+v_{\phi}^{2})=\frac{m}{2}[\dot{q}^{2}+\omega^{2}q^{2}]
  \end{gather*}
\end{eg}
\begin{eg}
  \phantom{.}

  \incfig{l2_kinetic_dots}
  Теорема Кёнига:
  \begin{enumerate}
    \item Кёнигова система отсчёта
    \begin{enumerate}
      \item Центр находится в центре масс
      \item При движении системы кёнигова система точек движется поступательно
    \end{enumerate}
  \item $T=\frac{1}{2}m_cv_c^2+T^{\text{отн}}$, 
    для твёрдого тела (на плоскости) $T^{\text{отн}}=\frac{1}{2}I_{\text{ось вр}}\omega^{2}$
  \end{enumerate}
\end{eg}

%lecture 3

\subsubsection{Алгоритм решения}
\begin{enumerate}
  \item Определить количество степеней свободы
  \item Выбрать обобщённую систему координат
  \item $T(\dot{q},q,t)$
  \item $\Pi(q,t)$
  \item $L=T-\Pi$
  \item $\frac{d}{dt}\left(\pd{L}{q_i}\right)-\pd{L}{q_i}=0$, $i=1,\dots ,n$
  \item Вычислить производные
\end{enumerate}

\subsection{Классификация обобщённых сил}
\[
  Q_i=-\pd{\Pi}{q_i}+\tilde Q_i(\dot{q},q,t)
\]
$\tilde Q$ - непотенциальная сила
\[
  N=\sum_{i=1}^{n}Q_i\dot{q_i}
\]
- мощность обобщённых сил.
  \begin{gather*}
    \vec{Q}=-Q_qq-Q_{\dot{q}}\dot{q} \\ 
    C=\frac{1}{2}(Q_q+Q_q^{T})=C^{T} \qquad B=\frac{1}{2}(Q_{\dot{q}}+Q_{\dot{q}}^{T})=B^{T} \\
    P=\frac{1}{2}(Q_q-Q_q^{T})=-P^{T}  \qquad G=\frac{1}{2}(Q_{\dot{q}}-Q_{\dot{q}}^{T})=-G^{T} \\
    \vec{Q}=-Cq-Pq-B\dot{q}-G\dot{q}
  \end{gather*}
\begin{center}
\begin{tabular}{| c | c | c |}
  \hline
  матрица & $q$ & $\dot{q}$ \\ 
  \hline
  симметричные & консервативные $C$ & диссипативные $B$ \\ 
  \hline
  кососимметричные & существенно непотенц. $P$ & гироскопические $G$ \\
  \hline
\end{tabular}
\end{center}
\[
  T(\dot{q},q,t)=T_2+T_1+T_0
\]
Индекс показывает степень с которой $T_i$ зависит от $\dot{q}$. В склерономной 
$T_1$ и $T_0$ нет.
\[
  E=T+\Pi=T_2+T_1+T_0+\Pi
\]
\subsubsection{Изменение полной энергии}
\[
  \frac{dE}{dt}=\sum_{i=1}^{n}\tilde Q_i \dot{q}_i + \frac{d}{dt}(T_1+2T_0)-\pd{T}{t}+\pd{\Pi}{t}
\]
Условия консервативной системы (не учитывая экзотику когда слогаемые сокращаются):
\begin{enumerate}
  \item Если система склерономная $T_1=T_0=0 \; \implies \; \pd{T}{t}=0$
  \item Если $\pd{\Pi}{t}=0$.
  \item Если $\sum_{i=1}^{n}\tilde Q_i \dot{q}_i=0$
\end{enumerate}
\section{Гамильтонова механика}
\begin{gather*}
  L(\dot{q},q,t)=T-\Pi \\ 
  \frac{d}{dt}\left(\pd{L}{q_i}\right)-\pd{L}{q_i}=0 \\ 
  \left\{\begin{aligned}
      & a_{11}\ddot{q}_1 + a_{12}\ddot{q}_2 + \dots =0 \\ 
      & \dots 
  \end{aligned}\right.
\end{gather*}
В лагранжевой механике на очевидно что такое лагранжиан и итоговые
уравнения получаются в ненормальном виде из-за чего их сложно решать.
\subsection{Переход к новым переменным}
$q,\dot{q},t$ ($\dot{q}$-обобщённые скорости) $\Rightarrow$ $q,p,t$ ($p$ - обобщённый импульс)

$L(\dot{q},q,t) \Rightarrow H(q,p,t)$ функция гамильтона (гамильтониан)
\[
  p_i=\pd{L}{q_i} \qquad \dot{\hat{q}}=\dot{\hat{q}}(q,p,t)
\]
Преобразование Лежандра:
\[
  H(q,p,t)=\sum_{i=1}^{n}p_i \dot{\hat{q}}_i - L(q, \dot{\hat q},t)
\]
На место $\dot{\hat q}$ нужно подставить зависимость выше.

Канонические уравнения Гамильтона:
\[
  \left\{\begin{aligned}
    \dot{q}_i=\pd{H}{p_i} \\ 
    \dot{p}_i=-\pd{H}{q_i}
  \end{aligned}\right. \qquad i=1,\dots ,n
\]
Уравнения 1го порядка в нормальном виде!

\[
  H=T_2-T_0+\Pi
\]
В склерономной систему $T_0=0$ $\implies$ $H=T_2+\Pi=T+\Pi=E$

% lecture 4

\section{Первый интеграл}
\[
  \left\{\begin{aligned}
    \dot{x}_j=g_j(x,t) \qquad j=1,\dots ,m \qquad m=2n
  \end{aligned}\right.
\]
\[
  x_j^{*}=x_{j}^{*}(t,C_1,\dots ,C_m)
\]
\begin{definition}[Первый интеграл]
  $f(x,t)$ - первый итеграл, если при подстановке любого решения $x^{*}$,
  её значение констанка.
  \[
    \frac{df}{dt}=\pd{f}{t}+\sum_{j=1}^{m}\pd{f}{x}\dot{x}_j=0
  \]
\end{definition}
Законы сохранения $f(x)=const$ тоже являются первыми интегралами.

Если набрать $m$ функционально независимых первых интегралов можно выразить $x_j$
через соответствующие константы и $t$ ($f_k(x,t)=C_k$)
и тогда не нужно решать диффур.

Если же первые интегралы не зависят от $t$ ($f(x)=C_k$)
всего их может быть $m-1$.
\begin{definition}
  $k$ первых интегралов $f_k(x_1,\dots ,x_m,t)$ называются функционально
  независимыми в области $D$ если в каждой точке $D$ ранг матрицы
  $\left(\pd{f_i}{x_j}\right)$, $i=1,\dots, k$, $j=1,\dots ,n$
  равен k.
\end{definition}
\subsection{Гамильтоновы системы}
\[
  \frac{df}{dt}=\pd{f}{t}+\sum_{i=1}^{n}\left(\pd{f}{q_i}\pd{H}{p_i}-\pd{f}{p_i}\pd{H}{q_i}\right)=0
\]
\begin{definition}[Скобки Пуассона]
  Пусть $\exists$ дважды непр. дифф. функции гамильтоновых переменных
  $\phi(q,p,t)$ и $\psi(q,p,t)$
  \[
    (\phi,\psi)=\sum_{i=1}^{n}\left(\pd{\phi}{q_i}\pd{\psi}{p_i}-\pd{\phi}{p_i}\pd{\psi}{q_i}\right)
  \]
\end{definition}
\begin{remark}
  $(\phi,\phi)=0$, $(\phi,\psi)=-(-\psi,\phi)$
\end{remark}
\begin{theorem}[Критерий перв. инт. гам. сист.]
  \[
    \frac{df}{dt}=\pd{f}{t}+(f,H)=0
  \]
\end{theorem}
\begin{theorem}[Теоремя Якоби-Пуассона]
  Скобка Пуассона от двух первых интегралов гамильтоновой системы
  также является первым интегралом.
\end{theorem}
\subsection{Первые инт. гам. сист}
\begin{enumerate}
  \item Если $H$ не зависит от $t$, он явл. первым интегралом,
    $H(q,p)=h$ - обобщённый интеграл энергии. ($H=T+\Pi=E$, если $T=T_2$)
    \[
      \pd{H}{t}=0 \implies \frac{dH}{dt}=(H,H)=0
    \]
  \item Если $H$ не зависит от координаты $q_k$ она называется циклической
    и существует первый интеграл $p_k=const$.
  \item Если пара переменных $q_k$ и $p_k$ входит в $H$ в виде одной функции
    $z_k=z_k(q_k,p_k)$ то $z_k(q_k,p_k)=const$ является первым интегралом.
    Переменная $q_k$ называется отделимой.
\end{enumerate}
\begin{eg}
  \[
    H(q_1,q_2,p_1,p_2)=q_1^{2}+p_1^{2}+\frac{q_2p_2}{e^{q_1^{2}+p_1^{2}}}\sin(q_2p_2)
  \]
\end{eg}
\begin{eg}

  \incfig{l4_fint_eg1}
  Материальная точка в постоянном поле тяжести.
  \[
    H=\underbrace{\frac{p_1^{2}}{2m}}_{const}+\underbrace{\frac{p_2^{2}}{2m}+mgq_2}_{const}=const
  \]
  Получаем циклическую переменную, отделимую перменную и обобщённый инт. энергии.
\end{eg}
\begin{eg}
  Мат. точка на парабалоиде.

  \incfig{l4_fint_eg2}
  $\phi$ добавляет степень свободы однако она не влияет на потенциальную энергию,
  поэтому она и называется циклической.
  \[
    \Pi=mgh
  \]
\end{eg}
\begin{eg}
  \[
    H(q,p)=T_2-T_0+\Pi=const
  \]

  \incfig{l2_kinetic}
  \begin{center}
  \begin{tabular}{| c | c | c |}
    \hline
     & инерциальная & неинерц \\
    \hline
    $T$ & $\frac{m}{2}[\dot{q}^{2}+\omega^{2}q^{2}]$ & $\frac{m}{2}\dot{q}^{2}$  \\
    \hline
    $\Pi$ & $0$ & $-\frac{m\omega^{2}q^{2}}{2}$ \\ 
    \hline
    $E$ & $\frac{m}{2}[\dot{q}^{2}+\omega^{2}q^{2}]$ & $\frac{m}{2}\dot{q}^{2}-\frac{m\omega^{2}q^{2}}{2}$ \\
    \hline
    $H$ & $\frac{p^{2}}{2m}-\frac{m\omega^{2}q^{2}}{2}$ & $\frac{p^{2}}{2m}-\frac{m\omega^{2}q^{2}}{2}$\\
    \hline
  \end{tabular}
  \end{center}
\end{eg}
\subsection{Интегрируемость гамильтоновых систем}
  \[
    \{\dot{x}_j=g_j(x,t)
  \]
  Пусть известно $l$ первых интегралов
  \[
    f_k(x,t)=C_k \qquad k=1,\dots ,l
  \]
  Выразим $l$ переменных $x_j$ через остальные $x_{m-l}$.
  Понизим порядок системы дифф. уравнений.
  Если система автономна
  \[
    \{\dot{x}_j=g_j(x)
    \]
    и известно $n-1$ первых инт. получим
  \[
    \frac{dx_m}{dt}=g_m(x_m,C_1,\dots ,C_{m-1}) \qquad \int_{}^{}\frac{dx_m}{g_m(x_m,C_1,\dots ,C_{m-1})}=t+C_m
  \]

\begin{enumerate}
  \item $l$ циклических инт. понижают порядок системы на $2l$
    \begin{gather*}
      \left\{\begin{aligned}
        &\dot{q}_i=\pd{H}{p_i} \\ 
        &\dot{p_i}=-\pd{H}{q_i}
      \end{aligned}\right. \\ 
      p_i = C_k \\ 
      \dot{q}_k=\pd{\tilde{H}}{C_k}=F_k(c_1,\dots ,C_l,C_{2n-2l+1},\dots ,C_{2n},t)
    \end{gather*}
  \item Если $n$ координат отделимы. То гамильт. сист инт.
    Если сист. 2-го порядка имеет 1-интеграл $\implies$ интегрируема.
\end{enumerate}

% lecture 5 

\subsection{Теорема Лиувилля (траектории в фазовом пр-ве)}

\incfig{l5_traj}
\begin{theorem}[Лиувилля о сохранении фаз. объёма] \phantom{.}

  Если
  \[
    \left\{\begin{aligned}
      \dot{x}_j=f_j(x,t)
    \end{aligned}\right.
  \]
  удовлетворяет условию
  \[
    div \vec{f}=\sum_{j=1}^{m}\pd{f_j}{x_j}=0
  \]
  в системе сохраняется фазовый объём:
  \[
    J=\iiint \limits_V \delta x_1\dots \delta x_m
  \]
\end{theorem}
\begin{remark}
  Фазовый объём в гамильтоновых системах:
  \[
    \sum_{i=1}^{n}\left(\frac{\partial^{2}H}{\partial p_i\partial q_i}-\frac{\partial^{2}H}{\partial q_i \partial p_i}\right)=0
  \]
\end{remark}

\section{Теория устойчивости}

\incfig{l5_pend}
\begin{gather*}
  \ddot{\phi}+\omega^{2}\sin \phi = 0 \qquad \omega^{2}=\frac{g}{l} \\ 
  x_1 = \phi \qquad x_2 = \dot{\phi} \\ 
  \left\{\begin{aligned}
    & \dot{x}_1=x_2 \\ 
    & \dot{x}_2 = -\omega^{2}\sin x_1
  \end{aligned}\right.
\end{gather*}
Особые точки $x_2=0$ и $\sin x_1=0$
\[
  \left\{\begin{aligned}
    & x_1=k\pi, \, k \in \mathbb{z} \\ 
    & x_2 = 0
  \end{aligned}\right.
\]

\begin{definition}[Положение равновесия]
  Такое положение механической системы при котором не изменяются
  положения мат точек при условии того,
  что в начальный момент времени она находилась в этом положении
  и скорости мат. точек были равны нулю.
\end{definition}
\begin{remark}
  В кратце:
  \[
    \vec{R}(0)=\vec{R}_0 \ \dot{\vec{R}}(0)=0 \; \implies \; \vec{R}(t)=\vec{R}(0) \ \forall t>0
  \]
\end{remark}

\subsection{Стационарные заданные системы}
\begin{definition}
  Если система стационарно задана \\ 
  ($\vec{R}=\vec{R}(q_1,\dots ,q_n)$),
  положение $q^{0}$ называется положением равновесия,
  если из $q(0)=q^{0}$ и $\dot{q}(0) = 0$ $\implies$ $q(t)=q^{0} \ \forall t>0$.
\end{definition}
\begin{theorem}[Критерий положения равновесия]
  \[
    q^{0} \text{- полож. равн} \; \Leftrightarrow \; Q_i(q^{0},\dot{q}=0)=0, \; i=1,\dots ,n
  \]
  \[
    Q_i=-\pd{\Pi}{q_i} \qquad \left\{\begin{aligned}
      \pd{\Pi}{q_i}=0
    \end{aligned}\right.
  \]
\end{theorem}
\begin{eg}

  \incfig{l5_eg1}
  
  \begin{gather*}
    \Pi_{c}=-\int_{}^{}F(x)dx=-\frac{m\omega^{2}x^{2}}{2} \\ 
    \Pi = mgy - \frac{m\omega^{2}x^{2}}{2}=-mgr\sin \phi - \frac{m\omega^{2}r^{2}\sin^{2}\phi}{2} \\ 
    \pd{\Pi}{\phi}=0 \implies mgr\sin\phi - m\omega^{2}r^{2}\sin^{2}\phi\cos\phi =0 \\
    mr\sin\phi(g-\omega^{2}r\cos \phi)=0 \\ 
    \phi_1 = 0 \qquad \phi_2=\pi \\ 
    \phi_{3,4}= \left\{\begin{aligned}
      & \pm \arccos \frac{q}{\omega^{2}r}, \; \frac{q}{\omega^{2}r} \le 1 \\ 
      & \emptyset, \; \frac{q}{\omega^{2}r}>1
    \end{aligned}\right.
  \end{gather*}
\end{eg}
\begin{definition}[Усточивость по Ляпунову]
    \begin{gather*}
        (\forall \epsilon<0)(\exists \delta>0)(\forall|\vec{x}(0)|=|\vec{x}_0|<\delta)(\forall t>0)\true |\vec{x}(t)|<\epsilon \\
        (\forall \epsilon<0)(\exists \delta>0)(\forall|q(0)|<\delta, \forall |\dot{q}(0)|<\delta)(\forall t>0) \true |q(t)|<\epsilon, |\dot{q}(t)|<\epsilon
    \end{gather*}
    Неустойчивость:
    \[
      (\exists \epsilon>0)(\forall \delta>0)(\exists|\vec{x}(0)|=\vec{x}_0<\delta)(\exists t_1>0)\true |\vec{x}(t)|>\epsilon
    \]
\end{definition}
 
% lecture 6

\subsection{Устойчивость равновесия Лагранжевых систем}
\[
  q, \dot{q}, T(\dot{q},q), Q_i(\dot{q},q) \rightarrow \frac{d}{dt}\left(\pd{T}{\dot{q}_i}\right) - \pd{T}{q_i}=Q_i, \; i =1,\dots ,n
\]
Приводим к нулю: $q=q-q^{*}$, $q^{*}$ - значение в пол. равн.

Тогда в положении равновесия: $q^{0}=0$ и $\dot{q}^{0}=0$.

\subsubsection{Устойчивость пол. равн. консервативной системы}
\[
  Q_i=-\pd{\Pi(q)}{q_i}
\]
Устойчивость определяется $\Pi(q)$ (принимаем $\Pi(q^{*})=0$)
\begin{theorem}[Лагранжа-Дирихле (дост. усл.)]
  Если в некоторой $\Delta$ - окрестности положения равновесия 
  потенциальная энергия консервативной системы имеет строгий минимум,
  это положение равновесия устойчиво.

  То есть:
  \[
    |q|<\Delta \true \left\{\begin{aligned}
        \Pi(q)=0, & \; |q|=0 \\
        \Pi(q)>0, & \; |q| \neq 0
    \end{aligned}\right.
  \]
\end{theorem}
\begin{proof}
  \incfig{l6_stab}
  $E(q,\dot{q})=T+\Pi$

  $T(\dot{q},q)>0$ при $|\dot{q}|\neq 0 \implies E(\dot{q},q)>0$ если $\{q,\dot{q}\}\neq 0, |q|<\Delta$

  Возьмём $0<\epsilon<\Delta$, $E$ достигает на границе минимум $E \ge E^{*} > 0$,
  $\exists \delta$ - окрестность: $|q|<\delta$, $|\dot{q}|<\delta$, $E(0)<E^{*}$,
  $\forall t>0 \true E(t)=E(0)<E^{*}$
\end{proof}

\begin{theorem}($\Pi(q)$ достаточное условие экстремума)
\begin{gather*}
 \Pi(q)=\underbrace{\Pi(0)}_{0} + \sum_{i=1}^{n}\underbrace{\pd{\Pi}{q_i}\Big|_{q=0}}_{0}q 
 + \frac{1}{2}\sum_{i=1}^{n}\sum_{j=1}^{n}\frac{\partial^{2}\Pi}{\partial q_i \partial q_j} \Big|_{q=0}q_i q_j + \Pi_{m}(q) \\ 
 C_{ij}=\frac{\partial^{2}\Pi}{\partial q_i \partial q_j} \qquad \Pi_2=\frac{1}{2}q^{T}Cq
\end{gather*}
Если $\Pi_2$ полож. опред., то $\Pi$ в точке $q=0$ имеет строгий минимум

$\implies$ критерий Сильвестра.
\end{theorem}

\subsubsection{Теоремы неуст. пол. равн. консервативной системы}
\begin{theorem}[Ляпунова I]
  Если в положении равновесия конс. системы у потенц. энергии $\Pi(q)$
  отсутствует минимум (в том числе нестрогий) и это можно определить по
  $\Pi_2$, то данное положение неустойчиво.
\end{theorem}
\begin{eg}
  \begin{gather*}
    \Pi(q)=\frac{1}{2}q^{T}Cq \\ 
    \Pi(q)=4q_1^{2}-2q_1q_2+2q_2^{2}=\frac{1}{2}(8q_{1}^{2}-4q_1q_2+4q_2^{2}) \\ 
    C = \begin{pmatrix}
      8 & -2 \\ 
      -2 & 4
    \end{pmatrix}
  \end{gather*}
  Критерий Сильвестра выполнен $\implies$ устойчивое положение равновесия
\end{eg}

\subsubsection{Границы теорем}
\begin{enumerate}
  \item $\Pi$ не зависит от компоненты $q$
  \item Разложение 2-ой степени не зависит от компоненты $q$
\end{enumerate}
\subsubsection{Степень неустойчивости}
$\Pi-2=\frac{1}{2}q^{T}Cq$ приводим $C$ к диаг. виду
\[
  \Pi_2=\frac{1}{2}\sum_{i=1}^{n}r_i\theta_i^{2}
\]
Число отрицательных собственный чисел $r_i$ назыв. степенью. неуст.
\subsubsection{Гироскопические и диссипативные силы}
\begin{definition}[Гироскопические силы]
  \[
    \tilde{Q}_i: \; \forall \dot{q} \true \dot{q}^{T}\tilde{Q}=\sum_{i=1}^{n}\tilde{Q}_i\dot{q}_i=0
  \]
\end{definition}
\begin{definition}[Диссипативные силы]
  \[
    Q_i^{*}: \; \sum_{i=1}^{n}Q_i^{*}\dot{q}_i \le 0
  \]
\end{definition}
\begin{definition}[Строго диссипативные силы]
  \[
    \left\{\begin{aligned}
        \sum_{i=1}^{n}Q_{i}^{*}\dot{q}_{i}=0, & \; \dot{q}_i=0 \\ 
        \sum_{i=1}^{n}Q_{i}^{*}\dot{q}_{i}<0, & \; \dot{q}_i\neq0
    \end{aligned}\right.
  \]
\end{definition}

\subsection{Асимтотическая устойчивость}
Динамическая система:
\[
  \left\{\begin{aligned}
    \dot{x}_j=f_j(x,t) \qquad j=1,\dots ,n
  \end{aligned}\right.
\]
\begin{definition}
  Нулевое решение $\vec{x}(t)=0$ называется притягивающим,
  если $\exists \Delta: \; \forall |\vec{x}(0)|<\Delta \; \exists \lim \limits_{t\to\infty}|x(0)|=0$.
\end{definition}
\begin{definition}
  Нулевое решение называется асимпт. уст., если оно уст. и притягивающее.
\end{definition}

% lecture 7

\subsection{Влияние гироскопических и диссипативных сил}
\subsubsection{Изначально устойчива}
Если в полож. равн. потенц. энергия  имеет строгий локальный минимум,
при добавл. гироскопических  и диссипативных сил оно останется устойчивым.  

При добавлении дисс. сил с полной диссипацией  полож. равн. становится асимпт. уст. 
\subsubsection{Изначально неустойчива}
Если среди коэфф. устойивости $r_i$ хотя бы один является отрицательным,
изолированное пол. равновесия на может быть стабилизировано
диссипативной силой с полной диссипацией.

Если положение равновесия стабилизировано с помощью гироскопических сил,
то добавление диссипативных сил с полной диссипацией разрушает устойчивость.

Если степень неустойчивости чётная, то возможна стабилицазия с помощью
гироскопических сил.

\section{Решения линейных систем}
\[
  \left\{\begin{aligned}
      \dot{x}_j=\sum_{k=1}^{m}a_{jk}x_k
  \end{aligned}\right. \qquad \dot{\vec{x}}=D\vec{x}
\]
$\vec{x}\equiv \vec{0}$ - тривиальное решение.

Решение $\vec{x}=\vec{u}e^{\lambda t}$, $\lambda \vec{u}e^{\lambda t}=D\vec{u}e^{\lambda t}$, $det(D-\lambda E)=0$

$\implies$ $\lambda_j$ имеет $m$ решений, если кратные корни $s$, перечисляем $s$ раз
\[
  a_0 \lambda^{m}+a_1 \lambda^{m-1} + \dots + a_{m-1}\lambda + a_m=0
\]
\begin{enumerate}
  \item Нет кратных корней
    \begin{gather*}
      (D-\lambda E)\vec{u}=0 \qquad rang(D-\lambda m) < m \\ 
      \vec{x}=\sum_{j=1}^{m}C_j \vec{u}_j e^{\lambda_j t}
    \end{gather*}
  \item Есть кратные корни
    $rang(D-\lambda E)=m-1$ присоед. вектора

    $rang(D-\lambda E)=m-s$ получим $s$ собственных векторов
\end{enumerate}
При $\lambda_j=\mu_j + i\nu_j$ общее решение:
\[
  \vec{x}=\sum_{j=1}^{m}C_j[e^{\mu_j t}(\vec{h}_j(t)\cos \nu_j t + \vec{H}_j(t)\sin \nu_j t)]
\]
\begin{itemize}
  \item $\forall j \true \mu_j=Re \lambda_j <0 \iff \vec{x}=0$ $\rightarrow$ асимпт уст
  \item $\exists j: \: \mu_j=Re \lambda_j >0 \Rightarrow \vec{x}=0$ $\rightarrow$ неуст.
  \item $\forall j \true \mu_j=Re \lambda_j <0, \, j=1,\dots ,r, \, r<m$ и $\exists j: \: \mu_j=Re \lambda_j=0, \, j=r+1,\dots ,m$
    - уст. или неуст.
\end{itemize}
\begin{theorem}[Критерий Рауса-Гурвица]
  \phantom{.}

  Матрица Гурвица $(m\times m)$:
  \[
    \begin{pmatrix}
      a_1 & a_3 & a_5 & \dots & 0 \\ 
      a_0 & a_2 & a_4 & \dots & 0 \\
      0 & a_1 & a_3 & \dots & 0 \\ 
      0 & a_0 & a_2 & \dots & 0 \\
      \dots & \dots  &\dots &\dots &\dots \\ 
      \dots & \dots &\dots &\dots & a_m
    \end{pmatrix}
  \]
  Для того, чтобы действ. части всех корней хар. ур. были отрицательными
  ($\forall j \true Re\lambda_j<0$), необходимо и достаточно чтобы все
  главные миноры матрицы Гурвица были положительны.
\end{theorem}
\begin{theorem}[Необходимое усл. устойчивости системы]
  \[
    \forall j \true a_j > 0
  \]
\end{theorem}
\begin{theorem}[Критерий Льенара-Шипара]
  При выполнении необходимого условия устойчивости для устойчивых систем
  необходимо и достаточно, чтобы все определители Гурвица с нечетными (четными)
  номерами были положительны.
\end{theorem}
\begin{theorem}[Ляпунова об устойчивости по лин. приближению]
  \[
    \left\{\begin{aligned}
      \dot{x}_j=f_j(x)
    \end{aligned}\right. \qquad \dot{\vec{x}}=D\vec{x}+g(\vec{x})
  \]
  \begin{enumerate}
    \item $\forall j \true \mu_j=Re \lambda_j<0 \iff \vec{x}=0$ - асимпт. уст. у нел. сис.
    \item $\exists j: \: Re\lambda_j>0 \Rightarrow \vec{x}=0$ - неуст. у нел. сист.
    \item $\forall j \true \mu_j=Re \lambda_j <0, \, j=1,\dots ,r, \, r<m$ и
      $\exists j: \: \mu_j=Re \lambda_j=0, \, j=r+1,\dots ,m$ 
      - критич., нельзя переходить к лин. прибл
  \end{enumerate}
\end{theorem}
\subsection{Прямой метод Ляпунова}
В $\Delta$ - окрестности $\vec{x}=0$ определяется непр. дфф. функция $V(\vec{x})$:
\begin{itemize}
  \item $V(\vec{0})=0$
  \item $V(\vec{x})>0$ при $|\vec{x}|<\Delta$ и $|\vec{x}|\neq 0$
\end{itemize}
\[
  \frac{dV(\vec{x})}{dt}=\sum_{j=1}^{m}\pd{V}{x_j}\dot{x}_j
\]
Если существует $V(x)$ и во всей окрестности $0$ выполняется:
\begin{enumerate}
  \item $\frac{dV}{dt}\le 0$ - уст.
  \item $\frac{dV}{dt}<0$ - асимпт. уст.
  \item $\frac{dV}{dt}>0$ - неуст
\end{enumerate}

% lecture 8 

\section{Малые колебания консервативных систем}
\begin{gather*}
  \dot{\vec{x}}=D\vec{x} \\ 
  \frac{d}{dt}(\pd{L}{\dot{q}_i})-\pd{L}{q_i}=0 \qquad L = T-\Pi \\ 
  T=\frac{1}{2}\dot{q}^{T}A\dot{q} \qquad \Pi=\frac{1}{2}q^{T}Cq
\end{gather*}
Кинетческая скорость может зависить не только от скоростей, но и от координат 
(полагаем $q=0$ - устойчивое положение равновесия):
\begin{gather*}
  T(q, \dot{q})=\frac{1}{2}\sum_{i,k=1}^{n}a_{ik}(q)\dot{q}_i\dot{q}_k
  = \frac{1}{2}\sum_{i,k=1}^{n}a_{ik}(0)\dot{q}_i\dot{q}_k + \dots \\ 
  \Pi(q) = \underbrace{\Pi(0)}_{0} + \underbrace{\sum_{i=1}^{n}\pd{\Pi}{q_i}\Big|_{q=0} q_i}_{0}
  + \frac{1}{2}\sum_{i,k=1}^{n}\frac{\partial^{2}\Pi}{\partial q_i \partial q_k}\Big|_{q=0} q_iq_k + \dots 
\end{gather*}
Разложив в ряды получаем $n$ уравнений:
\begin{gather*}
  \left\{\begin{aligned}
    \sum_{k=1}^{n}(a_{ik}\ddot q_k + c_{ik}q_k)=0
  \end{aligned}\right. \\ 
  A\ddot q + Cq=0
\end{gather*}
- основное уравнение малых колебаний, ищем решения в виде:
\begin{gather*}
  q=\vec{u}\sin (\omega t + \alpha) \qquad \rho=\omega^{2} \\ 
  (C-\rho A)\vec{u}=0
\end{gather*}
Тривиальное решение $\vec{u}=0$ - нахождение в положении равновесия.
\begin{gather*}
  det(C-\rho A)=0 \\ 
  a_0 \rho^{n}+a_1 \rho^{n-1} + \dots + a_n=0
\end{gather*}
- вековое уравнение, из него получаем $n$ корней для $\rho$.

Подставляя в наше уравнение получаем общее решение:
\[
  q=\sum_{i=1}^{n}C_i\vec{u}_i \sin (\sqrt{\rho_i}t+\alpha_i)
\]
Если $\rho_k$ кратности $s$, если $rang(C-\rho A)=n-s$,
можем найти $s$ амплитудных векторов $\vec{u}$.

В нашем случае это условие выполняется так как обе матрицы можно
привести к диагональному виду (одну даже к единичному):
\begin{gather*}
  q=U \theta \qquad \dot{q}=0U\dot{\theta} \\ 
  T=\frac{1}{2}\sum_{i=1}^{n}\dot{\theta}_i^{2} \qquad \Pi = \frac{1}{2}\sum_{i=1}^{n}r_i\theta_i \\ 
  \left\{\begin{aligned}
    \ddot \theta_i + \omega_i\theta_i=0
  \end{aligned}\right.
\end{gather*}

Матрица $U$ состоит из нормированных амплитудных вект.
$U=(u_1 \; \dots \; u_n)$, $u_i^{T}Au_k=\delta_{ik}$ ($\delta_{ik}$ - символ Кронекера).

\begin{eg}
  \incfig{l8_eg1}

  \begin{gather*}
    \Pi=\frac{c}{2}q_1^{2}+\frac{c}{2}(q_2-q_1)+\frac{c}{2}q_2^{2}= \frac{1}{2}(2cq_1^{2}-2cq_1q_2+2cq_2^{2})
    \qquad C = \begin{pmatrix}
      2c & -c \\ 
      -c & 2c
    \end{pmatrix} \\ 
    T=\frac{m}{2}\dot{q}_1^{2} + \frac{m}{2}\dot{q}_2^{2}+\frac{m}{4}\dot{q}_1^{2}+\frac{m}{4}\dot{q}_2^{2}=\frac{1}{2}(\frac{3m}{2}\dot{q}_1^{2}+\frac{3m}{2}\dot{q}_2^{2})
    \qquad A = \begin{pmatrix}
      \frac{3m}{2} & 0 \\ 
      0 & \frac{3m}{2}
    \end{pmatrix} \\ 
    det(C-\rho A)=0 \\ 
    det\begin{pmatrix}
      2c-\frac{3m}{2}\rho & -c \\ 
      -c & 2c-\frac{3m}{2}\rho
    \end{pmatrix}=0 \; \implies \; \rho_1=\frac{2c}{3m} \quad \rho_2=\frac{2c}{m} \\ 
    u_1=\begin{pmatrix}
      u_{11} \\ u_{21} \\ \vdots \\ u_{n1}
    \end{pmatrix} \qquad \begin{pmatrix}
      c & -c \\ 
      -c & c
    \end{pmatrix} \begin{pmatrix}
      u_{11} \\ u_{21}
    \end{pmatrix} = 0 \\ 
    cu_{11} - c u_{21} =0 \qquad u_1=\begin{pmatrix}
      1 \\ 1
    \end{pmatrix}
  \end{gather*}
  Аналогично получаем $u_2=(1 \ \ -1)^{T}$. 

  Общее решение:
  \[
    \begin{pmatrix}
      q_1 \\ q_2
    \end{pmatrix} = C_1 \begin{pmatrix}
      1 \\ 1
    \end{pmatrix} \sin(\sqrt{\frac{2c}{3m}}t+\alpha_1) + C_2 \begin{pmatrix}
      1 \\ -1
    \end{pmatrix} \sin(\sqrt{\frac{2c}{m}}t+\alpha_2)
  \]
  \begin{gather*}
    \tilde{u}_i^{T}A\tilde{u}_i=1 \qquad \tilde{u}_i=k_i u_i \\ 
    \tilde{u}_1=\frac{1}{\sqrt{3m}}\begin{pmatrix}
      1 \\ 1
    \end{pmatrix} \qquad
    \tilde{u}_2=\frac{1}{\sqrt{3m}}\begin{pmatrix}
      1 \\ -1
    \end{pmatrix} \qquad
    U=\frac{1}{\sqrt{3m}}\begin{pmatrix}
      1 & 1 \\ 
      1 & -1
    \end{pmatrix} \\ 
    \left\{\begin{aligned}
      q_1=\frac{1}{\sqrt{3m}}(\theta_1+\theta_2) \\ 
      q_2=\frac{1}{\sqrt{3m}}(\theta_1-\theta_2)
    \end{aligned}\right. \qquad \left\{\begin{aligned}
      \ddot \theta_1 + \frac{2c}{3m}\theta_1 =0 \\ 
      \ddot \theta_2 + \frac{2c}{m}\theta_2 =0
    \end{aligned}\right.
  \end{gather*}

  \incfig{l8_eg12}

  Если $\exists$ целые $k_1, k_2: \; k1_1\omega_1=k_2\omega_2$ - частотые соизмеримы,
  внутри тора есть множество схожих траекторий, иначе весь тор покрыт одной траекторией.
\end{eg}

\end{document}

