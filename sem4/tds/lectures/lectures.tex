\documentclass{article}


\usepackage{amsmath, amsthm, amsfonts, amssymb}
\usepackage[utf8]{inputenc}
\usepackage[T2A]{fontenc}
\usepackage[english, russian]{babel}

\usepackage{import}
\usepackage{pdfpages}
\usepackage{transparent}
\usepackage{xcolor}

\usepackage{parskip}
\usepackage{systeme}

\newcommand{\incfig}[2][1]{%
    \def\svgwidth{#1\columnwidth}
    \import{./figures/}{#2.pdf_tex}
}

\pdfsuppresswarningpagegroup=1

\usepackage{hyperref}
\hypersetup{
    colorlinks=true, %set true if you want colored links
    linktoc=all,     %set to all if you want both sections and subsections linked
    linkcolor=black,  %choose some color if you want links to stand out
}

\newcommand\hr{
    \noindent\rule[0.5ex]{\linewidth}{0.5pt}
}

% All the environments
\usepackage{mdframed}
\mdfsetup{skipabove=1em,skipbelow=0em}
\theoremstyle{definition}
\newmdtheoremenv[nobreak=true]{theorem}{Теорема}
\numberwithin{theorem}{section}
\newmdtheoremenv[nobreak=true]{lemma}{Лемма}
\numberwithin{lemma}{section}
\newmdtheoremenv[nobreak=true]{definition}{Определение}
\numberwithin{definition}{section}
\newmdtheoremenv[nobreak=true]{corollary}{Следствие}
\numberwithin{corollary}{section}
\newtheorem*{eg}{Пример}
\newtheorem*{remark}{Замечание}

\numberwithin{equation}{section}

% Defs
\let\phi\varphi
\let\epsilon\varepsilon
\let\kappa\varkappa
\let\implies\Rightarrow
\let\iff\Leftrightarrow
\let\true\hookrightarrow

\newcommand{\pd}[2]{\frac{\partial{#1}}{\partial{#2}}}
\newcommand{\pdd}[2]{\frac{\partial^2{#1}}{\partial{#2^2}}}
\newcommand{\pdm}[3]{\frac{\partial^2{#1}}{\partial{#2}\partial{#3}}}
\newcommand\R{\ensuremath{\mathbb{R}}}


\begin{document}

\section{Лекция 1}

\subsection{Введение}
\begin{definition}[Динамическая система]
  Система у которой
  \begin{itemize}
    \item Определено состояние как совокупность велечин или функция в данный момент времени
    \item Задан закон эволюции системы
  \end{itemize} 
  Изучаем вектор состояния $\vec{x}=\{x_1,\dots,x_n\}, \; \vec{x}\in X\in \R^n$
\end{definition}
\begin{definition}[Типы законов эволюции] /
  \begin{itemize}
    \item Системы с непрерывным временем (потоки) $\dot{\vec{x}}=\vec{f}(\vec{x})$, $\vec{x}$ - автономная
    \item Системы с дискретным временем (каскады) $\vec{x}_{k+1}=T_k \vec{x}_k$,
      здесь $T_k$ - оператор, в большинстве случаев - функция
  \end{itemize}
\end{definition}
\begin{eg}
  Для описание системы движущегося колеса используем два параметра: скорость и угол. 
  Вектор состояния $\vec{x}=\{x_0 \; x_1\}$.
  \[
    \frac{\vec{dx}}{t}=f(\vec{x},t)
  \]
  \[
    \left\{\begin{aligned}
        & x_1= f_1(\vec{x},t)  \\
        & \dots  \\
        & x_n= f_n(\vec{x},t)  \\
    \end{aligned}\right.
  \]
\end{eg}
\begin{definition}[Автономная система]
  Если правая часть системы дифф. уравнений не зависит от времени система называется автономной. 
  Когда система не автономная над ней находится сверхсистема,
  изменение которой нужно изучить и определить её воздействие на нашу систему в зависимости от времени.
\end{definition}
\begin{eg}
  Спутник на орбите - автономная система, сила воздействующая на него зависит только от координат.
  А если Земля не округлая и вращается нужно вводить начальное положение и изменение от времени, система уже не автономная.
\end{eg}

\begin{eg}
  Если мы смотрим на популяцию кроликов раз в пол года имеем следующий закон изменения. 
  \[
    x_{k+1}=f_1(x_k)
  \]
\end{eg}

\subsection{Описание системы}
\begin{definition}[Твёрдое тело]
  Система мат точек, у которых не изменяется расстояние между любыми 2мя точками.
\end{definition}
Имеем систему $N$ материальных точек, где $\nu$ - номер точки. 
Движение описывается в виде
\[
  \vec{r}_\nu(t), \; \nu=1,\dots,N
\]
\[
\vec{v}_\nu(t)=\dot{\vec{r}}_\nu(t) \qquad \vec{w}_\nu(t)=\dot{\vec{v}}_\nu(t) \
\]
Связи $f(t,\vec{r}_1,\dots,\vec{r}_\nu,\dots,\vec{r}_N, \dot{\vec{r}}_1, \dots,\dot{\vec{r}}_\nu, \dots, \dot{\vec{r}}_N) =0$, кратко $f(t,\vec{r}_\nu,\dot{\vec{r}}_N)=0$. \\ 
Конечные связи - не зависят от скорости $f(t,\vec{r}_\nu)=0$, 
стационарные - не зависят от времени $f(\vec{r}_\nu)=0$, интегрируемые $f(t,\vec{r}_\nu)=G$

\begin{eg}
\incfig{l1_wheel}
В колесе без проскальзывания
\[
  \dot{x_1}=R\dot{\phi}_1
\]
\end{eg}

\begin{eg}
  \incfig{l1_rod}
  Стационарная и нестационарная связи:
  \[
    (\vec{r}_1-\vec{r}_2)^2=l^2 \qquad\qquad (\vec{r}_1-\vec{r}_2)^2=l^2(t)
  \]
\end{eg}

\begin{definition}[Голономные системы]
  Системы мат точек, в которых нет дифференицальных неинтегрирумеых связей.
\end{definition}
\begin{remark}
  Если налолжено $d$ связей (голономных) для описание положения нужно $3N-d$ переменных.
\end{remark}
\begin{definition}[Кол-во степеней свободы]
  Минимальное количество независимых переменных которые однозначно определяют \textbf{положение} системы называется количеством степеней свободы.
\end{definition}
\begin{remark}
  Состояние отличается от положения тем, что в неё входят скорости и зная сотояние, мы может предсказать как система будет развиваться.
\end{remark}
\begin{definition}[Параметризация системы]
  Введение параметров \\ $q_1,\dots,q_n$ ($q=\{q_1,\dots,q_n\}$-обобщённые координаты) таких, что однозначнозначно описывают положение:
  \[
    \vec{r}_\nu=\vec{r}_\nu(t,q_1,\dots,q_n) \qquad \nu=1,\dots,N
  \]
\end{definition}


\end{document}

