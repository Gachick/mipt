\documentclass{article}


\usepackage{amsmath, amsthm, amsfonts}
\usepackage[utf8]{inputenc}
\usepackage[T2A]{fontenc}
\usepackage[english, russian]{babel}

\usepackage{import}
\usepackage{pdfpages}
\usepackage{transparent}
\usepackage{xcolor}

\usepackage{parskip}
\usepackage{systeme}

\newcommand{\incfig}[2][1]{%
    \def\svgwidth{#1\columnwidth}
    \import{./figures/}{#2.pdf_tex}
}

\pdfsuppresswarningpagegroup=1

\usepackage{hyperref}
\hypersetup{
    colorlinks=true, %set true if you want colored links
    linktoc=all,     %set to all if you want both sections and subsections linked
    linkcolor=black,  %choose some color if you want links to stand out
}

\newcommand\hr{
    \noindent\rule[0.5ex]{\linewidth}{0.5pt}
}

% All the environments
\usepackage{mdframed}
\mdfsetup{skipabove=1em,skipbelow=0em}
\theoremstyle{definition}
\newmdtheoremenv[nobreak=true]{theorem}{Теорема}
\newmdtheoremenv[nobreak=true]{lemma}{Лемма}
\newmdtheoremenv[nobreak=true]{definition}{Определение}
\newmdtheoremenv[nobreak=true]{corollary}{Следствие}
\newtheorem*{eg}{Пример}
\newtheorem*{remark}{Замечание}

% Defs
\let\phi\varphi
\let\epsilon\varepsilon
\let\implies\Rightarrow
\let\iff\Leftrightarrow
\let\true\hookrightarrow

\newcommand{\pd}[2]{\frac{\partial{#1}}{\partial{#2}}}
\newcommand{\pdd}[2]{\frac{\partial^2{#1}}{\partial{#2^2}}}
\newcommand{\pdm}[3]{\frac{\partial^2{#1}}{\partial{#2}\partial{#3}}}
\newcommand\R{\ensuremath{\mathbb{R}}}


\begin{document}

\section{Занятие 1}

\subsection{Основные распределения в Мат Стат}
\subsubsection{Гамма распределение}
\begin{gather*}
  \xi \sim \Gamma(\lambda,a) \; \lambda>0,a>0 \\
  P(x)=\frac{\lambda^{a}}{\Gamma(a)}x^{a-1}e^{-\lambda x} \; \{(0,+\infty)\}
\end{gather*}

\incfig{l1_g_distr}
\begin{gather*}
  \Gamma(S+1)=S \Gamma(S) \qquad \Gamma(S)= \int_{0}^{\infty}x^{s-1}e^{-x}dx \; x>0 \\
  M[ \Gamma]=\int_{-\infty}^{\infty} \rho(x)dx= \int_{0}^{\infty}\frac{\lambda^{a}}{ \Gamma(a)}(\frac{t}{\lambda})^{a}e^{-t}\frac{dt}{\lambda}=
  =\frac{1}{\lambda \Gamma(a)}\int_{0}^{\infty}t^{a}e^{-t}dt=\frac{a}{\lambda} \\
  D[\xi]=M[\xi^{2}]-M^{2}[\xi]=\frac{a^2+a}{\lambda^2}-(\frac{a}{\lambda})^2=\frac{a}{\lambda^2} \\
\end{gather*}
\begin{theorem}[Свойство суммы]
$\xi_1,\dots,\xi_n$ независимы, $\xi_i \sim \Gamma(\lambda,a_i)$, $\eta=\xi_1+\dots+\xi_n \sim \Gamma(\lambda,a_1+\dots+a_n)$
\end{theorem}
\begin{proof}
$\xi_1 \sim \Gamma(\lambda,a_1)$. $\xi_2 \sim \Gamma(\lambda,a_2)$ - независимые, $\eta=\xi_1+\xi_2$
\begin{gather*}
\Phi(y)=P(\eta < y)=P(\xi_2+\xi_2<y)=\iint_{x_1+x_2<y}p(x_1,x_2)dx_1dx_2= \\
=\int_{0}^{y}dx_2\int_{0}^{y-x_1}\frac{\lambda^{a_1}}{\Gamma(a_1)x_1^{a_1-1}e^{-\lambda x_1}}\frac{\lambda^{a_2}}{\Gamma(a_2)x_2^{a_2-2}e^{-\lambda x_2}}dx_2 \\
  \phi(y)=\Phi'(y)
\end{gather*}
\end{proof}

\subsubsection[Распределение Парсона]{Распределение Парсона $\chi^2$}
$\xi_i\sim N(0,1)$ - независимы, $\eta=\xi_1^2+\dots+\xi_n^2=\chi^2$
\begin{gather*}
  \Phi(y)=P(\xi^2<y)=\left\{\begin{aligned}
    & y\le0 \quad :0 \\
    & y > 0 \quad :P(-\sqrt{y}<\xi<\sqrt{y})
  \end{aligned}\right. \\
  \phi(x)=   \left\{\begin{aligned}
      & \frac{1}{2\sqrt{y}}F'(\sqrt{y})+\frac{1}{2\sqrt{y}}F'(-\sqrt{y}), \; y> 0 \\
      & 0, \; y<0
    \end{aligned}\right. \\
  p(x)=\frac{e^{x^2/2}}{\sqrt{2\pi}} \\
  \phi(y)=\left\{\begin{aligned}
    & \frac{1}{\sqrt{y}}\frac{e^{-y/2}}{\sqrt{2\pi}}, \; y>0  \\
    & 0, \; y\le 0
  \end{aligned}\right. \\
  p(x)=\frac{\lambda^{a}}{\Gamma(a)}x^{a-1}e^{-\lambda x}\{(0;+\infty)\} \qquad \lambda=\frac{1}{2} \qquad a=\frac{1}{2} \\
  \xi^{2}\sim \Gamma(\frac{1}{2}, \frac{1}{2}) \qquad \xi_1^{2}+\dots+\xi_n^{2} \sim \Gamma(\frac{1}{2}, \frac{n}{2})=\chi^{2}(n)
\end{gather*}
$n$ - число степеней свободы
\begin{gather*}
  M[\eta]=\frac{a}{\lambda}=\frac{n/2}{1/2}=n \\
  D[\eta]=\frac{a}{\lambda^{2}}=\frac{n/2}{1/4}=2n
\end{gather*}

\begin{theorem}[Свойство суммы]
  $\xi_1,\dots,\xi_m$ - независ, $\xi_i\sim \chi^{2}(n_i)$, $\xi_1+\dots+\xi_n\sim \chi^2(n_1+\dots+n_m)$
\end{theorem}

\subsection{Распределение Стьюдента (Госсет)}
$\xi \sim N(0,1)$, $\eta \sim \chi^2(m)$ - независимы, $\frac{\xi}{\sqrt{\eta/m}}\sim t(m)$ \\
\incfig{l1_st_distr}
\[
  p(x)=\frac{(m)^{m/2}\Gamma(\frac{m+1}{2})}{\sqrt{\pi}\Gamma(\frac{m}{2})(x^2+m)^{\frac{m+1}{2}}}
\]

\subsection{Распределение Фишера}
$\xi\sim \chi^2(n)$, $\eta\sim \chi^2(m)$ - независимые, $\frac{\xi/n}{\eta/m}\sim F(n,m)$ \\
\incfig{l1_fi_distr}
\subsection{Нормальное распределение}
\begin{gather*}
  p(\vec{x})=\frac{1}{(\sqrt{2\pi})^n}\frac{1}{\sqrt{detK}}e^{-\frac{1}{2}(\vec{x}-\vec{a})^T K^{-1}(\vec{x}-\vec{a})} \\
  \vec{\xi}\sim N(\vec{a},R)\\
\end{gather*}
Свойства:
\begin{itemize}
  \item $\xi\sim N(0,1)$, $\eta=a\xi+b \sim N(b,a^2)$
  \item $\xi\sim N(\alpha, \sigma^2)$, $\eta=a\xi+b\sim N(a\alpha+b, \sigma^2a^2)$
  \item $\xi\sim N(\vec{0},E)$,$\vec{\eta}=A\vec{\xi}+\vec{b}$, $A:n\times n,detA\neq 0$ \\ 
    \begin{gather*}
      \Phi(t_1,\dots,t_n)=P(\eta_1<t_1,\dots,\eta_n<t_n)=P(\vec{\eta}<\vec{t})=P(A\vec{\xi}+\vec{b}<\vec{t})= \\
    = \int\dots\int_{A\vec{x}+\vec{b}<\vec{t}}p(x_1,\dots,x_n)dx_1\dots dx_n= \\
      \vec{y}=A\vec{x}+\vec{b} \qquad J=\left|\frac{\partial \vec{x}}{\partial \vec{y}}\right| \qquad \frac{1}{J}=detA \\
    =\idotsint_{\vec{y}<\vec{t}}p(A^{-1}(\vec{y}-\vec{b}))\frac{1}{|detA|}dy_1\dots dy_n \\
    \phi(\vec{t})=p(A^{-1}(\vec{y}-\vec{b}))\frac{1}{|detA|} \\
    \phi(\vec{t})=\frac{1}{|detA|}\frac{1}{(\sqrt{2\pi})^n}e^{-\frac{1}{2}(A^{-1}(\vec{y}-\vec{b})^T) (A^{-1}(\vec{y}-\vec{b}))}= \\
  =\frac{1}{|detA|}\frac{1}{(\sqrt{2\pi})^n}e^{-\frac{1}{2}(\vec{t}-\vec{b})^T (A^T)^{-1} A^{-1}(\vec{t}-\vec{b})} \\
  K=AA^T \qquad \vec{\eta}=A\vec{\xi}+\vec{b}\sim N(\vec{b}, AA^T) \\
    \end{gather*}
\item $\xi\sim N(\vec{a},K)$, $\vec{\eta}=A\vec{\xi}+\vec{b}\sim N(A\vec{a}+\vec{b},AKA^{T})$, $A:n\times n$, $detA\neq 0$
\item Для $A:m\times n$ два предыдущих свойства так же верны
\item $\xi,\eta$ - независ $\implies$ $cov(\xi,\eta)=0$, в другую сторону не верно
\[
  \left\{\begin{aligned}
    \xi \sim N(a_1, \sigma_1^2) \\ 
    \eta \sim N(a_2, \sigma_2^2) \\ 
    \text{независимые}
  \end{aligned}\right.
  \Leftrightarrow
  (\xi,\eta)\sim N\left(\begin{pmatrix}
    a_1 \\ a_2
  \end{pmatrix}\begin{pmatrix}
  \sigma_1^{2} & 0 \\ 
  0 & \sigma_2^{2}
\end{pmatrix}\right)
\]
\[
  \left\{\begin{aligned}
    \xi \sim N(a_1, \sigma_1^2) \\ 
    \eta \sim N(a_2, \sigma_2^2) \\ 
    cov(\xi,\eta)=0
  \end{aligned}\right.
  \Leftarrow
  (\xi,\eta)\sim N\left(\begin{pmatrix}
    a_1 \\ a_2
  \end{pmatrix}\begin{pmatrix}
  \sigma_1^{2} & 0 \\ 
  0 & \sigma_2^{2} 
\end{pmatrix}\right)
\]
\end{itemize}
\begin{lemma}[Лемма Фишера]
Пусть $\vec{\xi}\sim N(\vec{0},E)$ и $C$ ортогональная матрица,
$\vec{\eta}=C\vec{\xi}$, тогда $\forall k=1\dots n-1$ сл. вел. 
$\kappa=\sum_{i=1}^{n}\xi_{i}^{2}-\eta_{1}^{2}-\eta_{2}^{2}-\dots -\eta_{k}^{2}\sim \chi^{2}(n-k) $
и вел $\kappa,\eta_1,\eta_2,\dots ,\eta_k$ независ.
\end{lemma}
\begin{proof}
  \begin{gather*}
    \vec{\eta}\sim N(\vec{0},\underbrace{CC^{T}}_{E}) \\ 
    \eta_1^{2}+\dots +\eta_n^{2}=\vec{\eta}^{T}\vec{\eta}=\vec{\xi}^{T}C^{T}C\vec{\xi}=\xi_1^{2}+\dots +\xi_{n}^{2} \\ 
    \kappa= \eta_{k-1}^{2}+\dots+\eta_{n}^{2} \\ 
    \kappa=\chi^{2}(n-k)
  \end{gather*}
\end{proof}

\begin{theorem}[Фишера]
  Пусть $\xi_1,\dots ,\xi_n$ независ и $\xi_i\sim N(a,\sigma^{2})$, тогда:
  \begin{enumerate}
    \item $\phi=\sqrt{n}\frac{\bar{\xi}-a}{\sigma}\sim N(0,1)$,
      $\bar{\xi}=\frac{1}{n}\sum_{1}^{n}\xi_i$
    \item $\psi=\sum_{i=1}^{n}\frac{(\xi_i-\bar{\xi})^2}{\sigma^{2}}\sim \chi^{2}(n-1)$
    \item $\phi$ и $\psi$ независ.
  \end{enumerate}
\end{theorem}
\begin{proof}
  \begin{gather*}
    \phi=\frac{1}{\sqrt{n}}\frac{\sum_{}^{n}\xi_i-na}{\sigma}=\frac{1}{\sqrt{n}}\sum_{}^{n}\left(\frac{\xi_i-a}{\sigma}\right) \\ 
    \frac{\xi_i-a}{\sigma}=\frac{1}{\sigma}\xi_i-\frac{a}{\sigma} \sim N(\frac{a}{\sigma}-\frac{a}{\sigma}, \sigma^{2}\frac{1}{\sigma^{2}})=N(0,1) \\ 
  \phi = \frac{1}{\sqrt{n}}\sum_{}^{n}\eta_i=(\frac{1}{\sqrt{n}}\dots \frac{1}{\sqrt{n}})\vec{\eta}\sim N(\vec{0}, AA^{T})=N(0,1)
  \end{gather*}
  1) Доказан 
  \begin{gather*}
    \psi = \sum_{}^{n}\left(\underbrace{\frac{\xi_i-a}{\sigma}}_{\eta_i\sim N(0,1)}-\underbrace{\frac{\bar{\xi}-a}{\sigma}}_{\bar{\eta}}\right)^{2}=
    \sum_{}^{n}(\eta_i-\bar{\eta})^{2}=\sum_{}^{n}(\eta_i^2-2\eta_i \bar{\eta}+(\bar{\eta})^2)= \\
    = \sum_{}^{n}\eta_i^{2}-2\bar{\eta}\sum_{}^{n}\eta_i+n(\bar{\eta})^{2}=\sum_{}^{n}\eta_i^{2}-n(\bar{\eta})^{2} \\ 
    \eta_i\sim N(0,1) \qquad \zeta^{2}=n\bar{\eta}^{2} \qquad \zeta=\sqrt{n}\bar{\eta}=\frac{1}{\sqrt{n}}\sum_{}^{n}\eta_i=A\vec{\eta}=\phi
  \end{gather*}
  $A=\left(\frac{1}{\sqrt{n}}\dots \frac{1}{\sqrt{n}}\right)$ $\implies$ $C$ - ортог. матрица (Грамма-Шмидта)
  
  ($A$ получается строчкой матрицы $C$ и тогда $\zeta$ - одна из координат в другом базисе и применима Лемма Фишера)

  По лемме Фишера $\psi\sim \chi^{2}(n-1)$, $\psi$ и $A\bar{\eta}$ независ
\end{proof}

\begin{theorem}[О проекции]
  Пусть $\vec{\xi}\sim N(\vec{0},\sigma^{2}E)$, $L_1:dimL_1=m_1$ и $L_2:dimL_2=m_2$ два
  ортогональных подпространства $\R^{n}$, $\vec{\eta}_1$ - проекция $\vec{\xi}$ на $L_1$,
  норм. распр., независ. и $\frac{|\eta_1|^{2}}{\sigma^{2}}\sim\chi^{2}(dimL_1)$,
  $\frac{|\eta_2|^{2}}{\sigma^{2}}\sim\chi^{2}(dimL_2)$
\end{theorem}
\begin{proof}
  $\vec{\eta}_1=A_1\vec{\xi}\sim N(\dots,\dots )$, $\vec{\zeta}=C\vec{\xi}$,
  $C$ - ортогональная. $\vec{\zeta}\sim N(\vec{0},C\sigma^{2}EC^{T})=N(\vec{0},\sigma^{2}E)$.
  Новый ортонормированный базис $e_1'\dots e_{m}'$ в $L_1$, 
  $e_{m+1}'\dots e_{n}'$ в $L_2$, $\vec{\eta_1}=\zeta_1e_1'+\dots +\zeta_{m}e_{m}'$,
  $\vec{\eta_2}=\zeta_{m+1}e_{m+1}'+\dots +\zeta_{n}e_{n}'$
\[
  \frac{\bar{\xi}}{\sigma}\sim N(\vec{0},E) \qquad \frac{|\eta_1|^{2}}{\sigma^{2}}=\sum_{}^{}\frac{\xi_i^{2}}{\sigma^{2}}\sim \chi^{2}(m_1)
\]
\end{proof}


\end{document}
