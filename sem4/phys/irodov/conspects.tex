\documentclass{article}

\usepackage{amsmath, amsthm, amsfonts}
\usepackage[utf8]{inputenc}
\usepackage[T2A]{fontenc}
\usepackage[english, russian]{babel}

\usepackage{import}
\usepackage{pdfpages}
\usepackage{transparent}
\usepackage{xcolor}

\usepackage{parskip}
\usepackage{systeme}

\newcommand{\incfig}[2][1]{%
    \def\svgwidth{#1\columnwidth}
    \import{./figures/}{#2.pdf_tex}
}

\pdfsuppresswarningpagegroup=1

\usepackage{hyperref}
\hypersetup{
    colorlinks=true, %set true if you want colored links
    linktoc=all,     %set to all if you want both sections and subsections linked
    linkcolor=black,  %choose some color if you want links to stand out
}

\newcommand\hr{
    \noindent\rule[0.5ex]{\linewidth}{0.5pt}
}

% All the environments
\usepackage{mdframed}
\mdfsetup{skipabove=1em,skipbelow=0em}
\theoremstyle{definition}
\newmdtheoremenv[nobreak=true]{theorem}{Теорема}
\newmdtheoremenv[nobreak=true]{lemma}{Лемма}
\newmdtheoremenv[nobreak=true]{definition}{Определение}
\newmdtheoremenv[nobreak=true]{corollary}{Следствие}
\newtheorem*{eg}{Пример}
\newtheorem*{remark}{Замечание}

% Defs
\let\phi\varphi
\let\epsilon\varepsilon
\let\implies\Rightarrow
\let\iff\Leftrightarrow
\let\true\hookrightarrow

\newcommand{\pd}[2]{\frac{\partial{#1}}{\partial{#2}}}
\newcommand{\pdd}[2]{\frac{\partial^2{#1}}{\partial{#2^2}}}
\newcommand{\pdm}[3]{\frac{\partial^2{#1}}{\partial{#2}\partial{#3}}}
\newcommand\R{\ensuremath{\mathbb{R}}}


\begin{document}

\section{Уравнение волны}
Упругая волна - процесс распротранения возмущения в упругой среде,
различают продольные и поперечные. \\
Для распространения волны в положительном направлении неоходимо,
чтобы аргументы $t$ и $x$ входили в функцию в виде комбинации $t-x/v$,
ведь тогда $dx/dt=v$ следовательно функция возмущения имеет следующий вид:
\[
 \xi(x,t)=f(t-x/v) 
\]
Особую роль играет гармоническая волна:
\begin{gather*}
  \xi(x,y)=a\cos \omega(t-x/v)
  T=2\pi/\omega \qquad \lambda=vT=v/\nu
\end{gather*}
Вводя волновое число $k=2\pi/\lambda$ получаем симметричный вид:
\[
  \xi=a\cos(\omega t-kx)
\]
Если волна затухающая, вводим коэфициент затухания волны $\gamma$ и получаем:
\[
  \xi=a_0 e^{-\gamma x}\cos(\omega t -kx)
\]
\subsection{Плоская волна}
Для плоской волны распростроняющейся в произвольном направлении ($\vec{n}$):
\[
  \xi=f(t-\vec{r}\vec{n}/v) \qquad \xi=a\cos(\omega t - \vec{k}\vec{r}) \qquad \vec{k}=(\omega/v)\vec{n}=(2\pi/\lambda)\vec{n}
\]
\subsection{Сферическая и цилиндрическая волны}
Продольная волна от точечного источника:
\[
  \xi=\frac{1}{r}f(t-r/v) \qquad \xi=\frac{a_0}{r}\cos(\omega t - kr)
\]
Цилиндрическая волна:
\[
  \xi=\frac{1}{\sqrt{R}}f(t-R/v) \qquad \xi=\frac{1}{\sqrt{R}}\cos(\omega t - kR)
\]

\section{Волновые уравнения}
\begin{equation} \label{eq:2.1}
  \pd{\xi}{t}=\pd{\xi}{\phi}\pd{\phi}{t}=\xi_\phi' \qquad \pd{\xi}{x}=\pd{\xi}{\phi}\pd{\phi}{x}=-\xi_\phi'/v \\ 
\end{equation}
\begin{equation} \label{eq:wave_diff}
  \pd{\xi}{x}=-\frac{1}{v}\pd{\xi}{t}
\end{equation}
Если волна распространяется в отрицательном напрвлении по оси $x$ знак справа меняется на $+$. \\ 
Физический смысл производных, $\partial\xi/\partial t=u_x$ - проекция скорости частицы среды,
$\partial\xi/\partial x=\epsilon$ - отностельная деформация среды.
\subsection{Общее волновое уравнение}
Продифференцировав (\ref{eq:2.1}) по соответствующим переменным ещё раз получим:
\begin{equation} \label {eq:2.2}
  \frac{\partial^{2}\xi}{\partial x^{2}}=\frac{1}{v^{2}}\frac{\partial^{2}\xi}{\partial t^{2}}
\end{equation}
Это уравнение удовлетворяет уравнениям распростанения в обе стороны и справедливо для однородных изотропных сред,
затухание в которых пренебрежимо мало. \\
В трёхмерном пространстве: 
\[
  \nabla^{2}\xi=\frac{1}{v^{2}}\frac{\partial^{2}\xi}{\partial t^{2}}
\]

\section{Скорость упругих волн}
\subsection{Скорость волны в тонком стержне} \label{subs:rod_speed}
По закону Гука: $\sigma=E\epsilon$, где $\epsilon=\partial\xi/\partial x$.
Примения к малому растянутому элементу второй закон Ньютона:
$\rho \Delta x S \ddot \xi =F_x(x+\Delta)+F_x(x)=S\sigma(x+\Delta x)-S\sigma(x)=S\pd{\sigma}{x}\Delta x$,
после сокращения получаем:
\[
  \rho \frac{\partial^{2}\xi}{\partial t^{2}}=E\frac{\partial^{2}\xi}{\partial x^{2}}
\]
Сопоставив с (\ref{eq:2.2}) получим $v=\sqrt{E/\rho}$ для продольной волны
\subsection{Скорость волны в гибком шнуре}
\begin{equation} \label {eq:wire_wave}
  \frac{\partial^{2}\xi}{\partial t^{2}}=\frac{F}{\rho_1}\frac{\partial^{2}\xi}{\partial x^{2}}
  \qquad v=\sqrt{F/\rho_1}
\end{equation}
где $F$ - сила натяжения, $\rho_1$ - линейная плотность шнура
\subsection{Скорость звука в жидкостях и газах}
Можно заново использовать формулу для волны с стержне, нужно только разобраться что
такое модуль Юнга. В данном случае закон Гука - связь избыточного давления с относительным
изменением длины: $\Delta p=-E \Delta\xi/\Delta x$, получаем $\Delta p = -E \Delta V/V$,
где $\Delta V/V$ - относительное приращение объёма. Так как масса - константа имеем
$dV=-V d\rho/\rho$, подставляя $v=\sqrt{dp/d\rho}$. \\ 
Учитывая что $pV^{\gamma}=const$, $E=\gamma p$, получаем $v=\sqrt{\gamma p/\rho}=\sqrt{\gamma RT/\mu}$,

\section{Энергия упругой волны}
\subsection{Плотность энергии упругой волны}
Приложим к торцу стержня растягивающую силу $F(x)$, растущую от $0$ до $F_0$.
Тогда по закону Гука $F(x)=\kappa x$, где $\kappa$ - коэф. упругости
\[
  A=\int_{0}^{x}F(x)dx=\kappa\int_{0}^{x}xd=\frac{\kappa x^{2}}{2}
\]
Работа идёт на увеличение упругой энергии $U$, тогда с учётом того что
$\kappa x=F=\sigma F$, $\sigma=E\epsilon$ и $\epsilon=x/l$, можем найти
плотность упругой энергии $w_p$
\begin{gather*}
  U=\frac{Fx}{2}=\frac{\sigma S\epsilon l}{2}=\frac{E \epsilon^{2}}{2}Sl \\ 
  w_p=\frac{U}{Sl}=\frac{E\epsilon^{2}}{2}
\end{gather*}
Но каждая единица объёма обладает и кинетической и потенциальной энергией:
\[
  w=w_k+w_p=\rho \dot{\epsilon}^{2}/2 + E \epsilon^{2}/2
\]
Согласно главе \ref{subs:rod_speed}
$E=\rho V^{2}$:
\[
  w=\frac{\rho}{2}\left[\left(\pd{\epsilon}{t}\right)^{2}+v^{2}\left(\pd{\epsilon}{x}\right)^{2}\right]
\]
Однако согласно \ref{eq:wave_diff} слогаемые равны:
\[
  w=\rho \dot{\epsilon}^{2}
\]
Тогда для гармонической волны:
\begin{gather*}
  w=\rho a^{2}\omega^{2}\sin^{2}(\omega t - kx) \\ 
  \langle w \rangle = \rho a^{2} \omega^{2} / 2
\end{gather*}

\subsection{Плотность потока энергии}
Поток энергии - кол-во энергии переносимое волной
через поверхность $S$ за единицу времени:
\[
  \Phi=dW/dt
\]
Плотность потока энергии - поток энергии через единичную площадку,
перепендикулярную направлению переноса энергии:
\[
  j=d\Phi/dS_\perp
\]
Рассматривая косой цилиндр с основанием $dS$ и длиной $vdt$:
\begin{gather*}
  dW=wdV=wvdtdS\cos\alpha=wvdtdS_\perp \\ 
  j = wv
\end{gather*}
Для определения направления вводят вектор Умова:
\[
  \vec{j}=w\vec{v}
\]
где $v$ - вектор скорости нормальный к волновой поверхности в данном месте.

В случае гармонической волны:
\begin{gather*}
  \vec{v}=(\omega/k)\vec{n} \\ 
  \langle \vec{j} \rangle=\frac{1}{2}\rho a^{2} \omega^{2} \vec{v}
\end{gather*}
Среднее по времени значение плотности потока энергии называют интенсивностью:
\[
  I=\langle j \rangle
\]

Если мы имеем дело с суперпозицией волн формула становится сложнее:
\begin{gather*}
  \vec{j}=-\sigma\vec{u} \\ 
  j_x=-E \pd{\xi}{x}\pd{\xi}{t}=-\sigma u_x
\end{gather*}
где $\sigma$ - напряжение, $\vec{u}$ - скорость частиц (вывод стр 24)

\section{Стоячие волны}
\subsection{Уравнение стоячей волны}
При наложение волн они не возмущают друг друга,
колебания частиц оказываются векторной суммой колебаний.
Рассмотрим случай когда две гармонических волны с одинаковыми
частотой и амплитудой распростроняются в противоположных направлениях
\begin{gather*}
  \xi_1=a\cos(\omega t - kx) \qquad \xi_1=a\cos(\omega t + kx) \\ 
  \xi=\xi_1+\xi_2=A\cos(kx)\cos(\omega t) \qquad A=2a
\end{gather*}
\subsection{Энергия стоячей волны}
\begin{gather*}
  \dot{\xi}=-A\omega \cos(kx)\sin(\omega t) \\ 
  \epsilon=-Ak \sin(kx)\cos(\omega t) \\ 
\end{gather*}
В стоячей волне происходят превращения энергии то польностью в потенциальную,
то полностью в кинетическую.
\subsection{Колебания струны (стержня)}
Образование стоячей волны возможно если на закреплённых концах струны
смещение $\xi=0$. То есть на длине струны $l$ должно укладываться целое число $n$
полуволн: $l=n\cdot \lambda/2$. Тогда возможные длины волн и их частоты:
\begin{gather*}
  \lambda_n=2l/n \qquad n=1,2,\dots  \\ 
  \nu_n=\frac{v}{\lambda_n}=\frac{n}{2l}n
\end{gather*}
где $v$ - фазовая скорость волны определяемая согласно \ref{eq:wire_wave}.

Частоты $\nu_n$ называеют собственными, $\nu_0$ - основной, $\nu_2,n_3,\dots $ - обертонами


\end{document}
