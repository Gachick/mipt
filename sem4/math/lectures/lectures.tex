\documentclass{article}

\usepackage{amsmath, amsthm, amsfonts}
\usepackage[utf8]{inputenc}
\usepackage[T2A]{fontenc}
\usepackage[english, russian]{babel}

\usepackage{import}
\usepackage{pdfpages}
\usepackage{transparent}
\usepackage{xcolor}

\usepackage{parskip}
\usepackage{systeme}

\newcommand{\incfig}[2][1]{%
    \def\svgwidth{#1\columnwidth}
    \import{./figures/}{#2.pdf_tex}
}

\pdfsuppresswarningpagegroup=1

\usepackage{hyperref}
\hypersetup{
    colorlinks=true, %set true if you want colored links
    linktoc=all,     %set to all if you want both sections and subsections linked
    linkcolor=black,  %choose some color if you want links to stand out
}

\newcommand\hr{
    \noindent\rule[0.5ex]{\linewidth}{0.5pt}
}

% All the environments
\usepackage{mdframed}
\mdfsetup{skipabove=1em,skipbelow=0em}
\theoremstyle{definition}
\newmdtheoremenv[nobreak=true]{theorem}{Теорема}
\newmdtheoremenv[nobreak=true]{lemma}{Лемма}
\newmdtheoremenv[nobreak=true]{definition}{Определение}
\newmdtheoremenv[nobreak=true]{corollary}{Следствие}
\newtheorem*{eg}{Пример}
\newtheorem*{remark}{Замечание}

% Defs
\let\phi\varphi
\let\epsilon\varepsilon
\let\implies\Rightarrow
\let\iff\Leftrightarrow
\let\true\hookrightarrow

\newcommand{\pd}[2]{\frac{\partial{#1}}{\partial{#2}}}
\newcommand{\pdd}[2]{\frac{\partial^2{#1}}{\partial{#2^2}}}
\newcommand{\pdm}[3]{\frac{\partial^2{#1}}{\partial{#2}\partial{#3}}}
\newcommand\R{\ensuremath{\mathbb{R}}}


\begin{document}

% lecture 1
\section{Некоторые свойства абсолютно инт функций}
\begin{definition}
  Функция $f(x)$ называется абсолютно интегрируемой на конечном или бесконечном интервале $(a,b)$,
  если выполняются два условия:
  \begin{enumerate}
    \item $\exists x_0,x_1,\dots,x_k:a=x_0<x_1<x_2<\dots<x_k=b$, что на любом отрезке
      $[\xi,\eta]$ из $(a,b)$, не содерж $x_i$ функция $f(x)$ интегрируема по Риману.
    \item $\int_{a}^{b}|f(x)|dx$ сходится
  \end{enumerate}
\end{definition}
\begin{lemma} \label{l1_abs_prod}
 Пусть $f(x)$ абсолютно интегрируема на конечном или бесконечном $(a,b)$.
 Пусть $\phi(x)$ - непрерывна и ограничена на $(a,b)$. \\
 $f(x)\phi(x)$ - абсолютно интегрируемо на $[a,b]$
\end{lemma}
\begin{proof}
  а) Рассмотрим $\forall [\xi,\eta]\subset(x_{i-1},x_i)$. На нём $f(x)$ и $\phi(x)$ - интегрируема
  по Риману, следовательно $f(x)\phi(x)$ инт. по Риману $\implies$ 1. \\
  б) Так как $\phi(x)$ - ограничена то $\exists M:|\phi(x)|\le M$ на $(a,b)$.
  Тогда $|f(x)\phi(x)|\le|f(x)|M$ на $(a,b)$. Т.к. $\int_{a}^{b}|f(x)|dx$ - сход.,
  то $\int_{a}^{b}|f(x)\phi(x)|dx$ - сход. по признаку сравнения $\implies$ 2. \\
  $\implies$ $f(x)\phi(x)$ - абс. инт. на $(a,b)$
\end{proof}
\begin{definition}
  Функция $\phi(x)$ определённая на $\R$ называется ступенчатой если
  $\exists$ числа $x_0,x_1,x_2,\dots,x_k:x_0<x_1<x_2<\dots<x_k$ и
  $c_1,c_2,\dots,c_k$:
  \[
    \phi(x) = \left\{\begin{aligned}
      & c_i, \; x \in [x_{i-1},x_i) \\ 
      & 0, \; x \in (-\infty,x_0) \cup [x_k,+\infty)
    \end{aligned}\right.
  \]
\end{definition}
\incfig{l1_step_f}
\begin{remark}
  Если положить
  \[
    \phi_i(x) = \left\{\begin{aligned}
      & 1, \; x \in [x_{i-1}, x_i) \\ 
      & 0, \; x \notin [x_{i-1}, x_i)
    \end{aligned}\right.
  \]
  то $\phi(x)=c_1\phi_1(x)+\dots+c_k\phi_k(x)$.
\end{remark}
\begin{theorem}[О приближении абс инт функций ступенчатыми]
  Пусть $f(x)$ - абс. инт. на конечном или бесконечном $(a,b)$ тогда
  $\forall \epsilon >0 \; \exists$ ступенчатая функция $\phi(x):\int_{a}^{b}|f(x)-\phi(x)|dx<\epsilon$
\end{theorem}
\begin{proof}
  Пусть для простоты записи у функции имеются только две особенности в точках $a$ и $b$,
  т.е. $f(x)$ - инт по Риману на $\forall [\xi,\eta]$ из $(a,b)$. 

  Возьмём $\forall \epsilon>0$. Из абсолютной инт. $f(x)$ следует $\exists$ таких
  $[\xi,\eta] \in (a,b)$:
  \[
    \int_{a}^{\xi}|f(x)|dx+\int_{\eta}^{b}|f(x)|dx< \frac{\epsilon}{2}
  \]
  Так как $f(x)$ инт. по Риману на $[\xi,\eta]$ то для рассмотренного $\epsilon>0$ 
  $\exists \delta:\forall$ разб. отр. $[\xi, \eta]$ $\tau=\{x_i\}_{i=0}^{n_\tau}$
  ($|\tau|<\delta$), $\forall \xi_i \in [x_{i-1}, x_i]$.

  Выполняется $\left|\int_{\xi}^{\eta}f(x)dx-\sigma_\tau\right|<\epsilon/2$, где
  $\sigma_\tau$ - сумма Дарбу. 

  $ \left|\int_{\xi}^{\eta}f(x)dx - s_\tau\right|\le \frac{\epsilon}{2}$,
  где $s_\tau=\sum_{i=1}^{n_\tau}m_i\Delta x_i$ ($m_i=\inf_{x \in [x_{i-1},x_i]} f(x), \; \Delta x_i=x_i-x_{i-1}$).

  Также $\int_{\xi}^{\eta}f(x)dx\ge s_\tau \; \implies \; 0 \le \int_{\xi}^{\eta}f(x)dx - s_\tau \le \epsilon/2$.

  Рассмотрим ступенчатую функцию:
  \[
    \phi(x) = \left\{\begin{aligned}
       & m_i, \; x \in [x_{i-1}, x_i) \\ 
       & 0, x \notin [\xi,\eta)
    \end{aligned}\right.
  \]

  \incfig{l1_step_r}
  Отметим: $s_\tau=\sum_{i=1}^{n_\tau}m_i\Delta x_i=\int_{\xi}^{\eta}f(x)dx$, $\phi(x)\le f(x)$ на $[\xi,\eta]$

  $\implies \int_{a}^{b}|f(x)-\phi(x)|dx=\int_{a}^{\xi}|f|dx+\int_{\xi}^{\eta}|f-\phi|dx+\int_{\eta}^{b}|f|dx$,
  но $\int_{\xi}^{\eta}|f-\phi|dx=\int_{\xi}^{\eta}(f-\phi)dx=\int_{\xi}^{\eta}fdx-s_\tau$

  $\implies \int_{a}^{b}|f(x)-\phi(x)|dx<\epsilon/2+\epsilon/2=\epsilon$
\end{proof}
\begin{theorem}[Римана (об осциляциях)] \label{l1_r_osc}
  Пусть $f(x)$ абс. инт. на конечном или бесконечном интервале $(a,b)$,
  тогда $\lim\limits_{\nu\to\infty}\int_{a}^{b}f(x)\cos\nu xdx=0$ и
  $\lim\limits_{\nu\to\infty}\int_{a}^{b}f(x)\sin\nu xdx=0$
\end{theorem}
\begin{proof}
  1) Если 
  \[
    \phi(x) = \left\{\begin{aligned}
      & 1, \; x \in[\xi,\eta) \\ 
      & 0, \; x \notin [\xi,\eta)
    \end{aligned}\right. ,\; [\xi,\eta] \in (a,b)
  \]
  То:
  \[
    \int_{a}^{b}\phi(x)\sin\nu xdx=\int_{\xi}^{\eta}\sin\nu xdx=
    -\frac{\cos \nu x}{\nu}\bigg|_\xi^{\eta} \underset{\nu\to\infty}{\to}0
  \]
  2) Если $\phi(x)$ - ступенчатая, то она является линейно комбинацией
  расмотренных одноступенчатых функций, поэтому для неё утверждение справедливо.

  3) Рассмотрим абс. инт. на $(a,b)$ функцию $f(x)$. Возьмём $ \underline{\forall\epsilon>0}$.

  По предыдущей теорме $\exists \phi(x)$ - ступенчатая функция: $\int_{a}^{b}|f-\phi|dx<\epsilon/2$.

  Т.к. $\lim\limits_{\nu\to \infty }\int_{a}^{b}\phi(x)\sin\nu x dx =0 $, то
  $\underline{\exists \nu_\epsilon }: \forall \nu \; (|\nu|>\nu_\epsilon) \true |\int_{a}^{b}\phi(x)\sin \nu x dx|<\epsilon/2$.
  Тогда $\underline{\forall \nu : \;(|\nu|>\nu_\epsilon}$ выполняется:
  \begin{gather*}
    \underline{|\int_{a}^{b}f(x)\sin\nu x dx|}=|\int_{a}^{b}(f(x)-\phi(x))\sin \nu x dx + \int_{a}^{b}\phi(x)\sin \nu xdx|\le  \\
    \le|\int_{a}^{b}(f(x)-\phi(x))\sin \nu xdx|+|\int_{a}^{b}\phi(x)\sin \nu xdx| < \\
    <\int_{a}^{b}|f(x)-\phi(x)|dx + \frac{\epsilon}{2} < \underline{\frac{\epsilon}{2}+\frac{\epsilon}{2}=\epsilon}
  \end{gather*}
  Подчёркнутое означает,что $\lim\limits_{\nu\to\infty}\int_{a}^{b}f(x)\sin \nu xdx=0$. Аналогично косинус.
\end{proof}
\begin{remark}
  Интервал $(a,b)$ при исследовании абсолютно интегрируемых
  на другой промежуток $[a,b],[a,b),(a,b]$
\end{remark}

% lecture 2

\section{Тригонометрические ряды Фурье}
\begin{definition}
  Ряд вида $\frac{a_0}{2}+\sum_{n=1}^{\infty}a_n\cos(nx)+b_n\sin(nx)$
  называется тригонометрическим рядом, где $a_k,b_k\in\R$
\end{definition}
\begin{definition}
  Множество функций $\{u_n(x)\}=\{1/2, \cos(x), \sin(x),$ $ \cos(2x), \sin(2x),\dots\} $
  называется тригонемтрической системой
\end{definition}
Свойства тригоном. сист.
\begin{enumerate}
  \item Триг. сист. "ортогональна" в смысле $\int_{-\pi}^{\pi}u_n(x)u_k(x)dx=0$,
    $\forall n,k:n\neq k$
  \item $\int_{-\pi}^{\pi}u_n^{2}(x)dx=\pi$, при $n \ge 2$
\end{enumerate}
\begin{proof} 1) Например
      \[
        \int_{-\pi}^{\pi}\sin nx \cos kx dx = 
        \frac{1}{2}\int_{-\pi}^{\pi}(\sin(n-k)x+\sin(n+k)x)dx=0
      \]
\end{proof}
\begin{proof} 2)
  Например \[
    \int_{-\pi}^{\pi}\cos^{2}(x)dx=\int_{-\pi}^{\pi}\frac{1+\cos(2nx)}{2}dx=\pi + 0=\pi
  \]
\end{proof}

\begin{lemma} \label{l2_f_lemma}
  Пусть
  \begin{equation} \label{l2_fourier}
    f(x)=\frac{a_0}{2}+\sum_{n=1}^{\infty}a_n\cos nx+b_n\sin nx
  \end{equation}
  и ряд сходится равномерно тогда:
  \begin{equation}
    \begin{gathered} \label{l2_f_coef}
      a_n=\frac{1}{\pi}\int_{-\pi}^{\pi}f(x)\cos nx dx, \; n=0,1,2,\dots \\ 
      b_n=\frac{1}{\pi}\int_{-\pi}^{\pi}f(x)\sin nx dx, \; n\in\mathbb{N} 
    \end{gathered}
  \end{equation}
\end{lemma}
\begin{proof}
  Домножим \ref{l2_fourier} на $\cos mx$. 
  Полученный ряд будет равномерно сходится.

  \hr
  \begin{gather*}
    \left|\sum_{k=1}^{n+p}\cos mx(a_k\cos kx + b_k \sin kx)\right|= 
    |\cos mx|\cdot\left|\sum_{k=n+1}^{n+p}a_k\cos kx + b_k\sin kx\right| \le  \\
    \left|\sum_{n+1}^{n+p}a_k\cos kx + b_k \sin kx\right|
  \end{gather*}
  Из выполнения усл. Коши равномерной сходимости для исходного ряда \ref{l2_fourier}
  следует выполнение усл. Коши равномерной сходимости полученного в результате умножения ряда.

  \hr
  Тогда имеем право интегрировать равнество (по $x$ от $-\pi$ до $\pi$)
  \begin{gather*}
    f(x)\cos mx = \frac{a_0}{2}\cos mx + \sum_{n=1}^{\infty}\cos mx (a_n \cos nx + b_n \sin nx) \\
    \implies \int_{-\pi}^{\pi}f(x)\cos mx dx = a_m \pi \\ 
    \implies a_m=\frac{1}{\pi}\int_{-\pi}^{\pi}f(x)\cos mx dx
  \end{gather*}
  Второе равество в \ref{l2_f_coef} получается аналогично.
\end{proof}
\begin{definition}
  Пусть $f(x)$ - $2\pi$ периодическая абсолютно интегрируемая на $[-\pi;\pi]$ функция.
  Тригонометрический ряд с коэффицентами \ref{l2_f_coef} называется
  тригонометрическим рядом Фурье функции $f(x)$, а коэффициенты $a_k,b_k$ - коэффициетами Фурье.
  Имеет место запись (здесь $\sim$ означает соответствие):
  \[
    f(x)\sim \frac{a_0}{2}+\sum_{n=1}^{\infty}a_n\cos nx + b_n\sin nx
  \]
\end{definition}
Перефразируем лемму:
\begin{lemma}[\ref{l2_f_lemma}']
  Рамномерно сходящийся тригонометрический ряд является рядом Фурье своей суммы. 
\end{lemma}
\begin{eg}
  \[
    \sum_{n=1}^{\infty}\frac{\cos nx}{n^{\alpha}}
  \]
  где $\alpha>1$ является рядом Фурье своей суммы, так  как он
  равномерно сходдится по признаку Веерштрасса
\end{eg}
\begin{remark}
  Если функция абсолютно интегрируема на $[-\pi,\pi]$, то интегралы
  в \ref{l2_f_coef} сходятся абсолютно по \ref{l1_abs_prod}
\end{remark}
\begin{remark}
  Если $f(x)$ - $2\pi$ периодична и абс. инт. на каком-либо $[a-\pi,a+\pi]$,
  то она будет абс. инт. на $\forall$ другом таком отрезке и
  интегралы (\ref{l2_f_coef}) не зависят от отрезка.
\end{remark}
\begin{remark}
  Любую абсолютно интегрируемую на $[a-\pi,a+\pi]$($(a-\pi,a+\pi);(a-\pi,a+\pi], [a-\pi,a+\pi)$)
  можно продолжить до $2\pi$ периодической функции,
  возможно доопределив или переопределив функцию в граничных точках.
  Интегралы при этом не меняются.
\end{remark}
\begin{corollary}
  Пусть $\{a_k\}, \{b_k\}$ - посл. коэфф. Фурье $2\pi$ периодической 
  и абсолютно инт. на $[-\pi,\pi]$ функции \[
    \implies \lim\limits_{k\to\infty} a_k=0 \qquad \lim\limits_{k\to\infty} b_k=0
  \]
  По \ref{l1_abs_prod} 
  $f(x)\cos nx$, $f(x) \sin nx $ - абс. инт.. По Т. Римана получаем нужный результат.
\end{corollary}
\begin{eg}
  \[
    \sum_{n=1}^{\infty}\sin nx
  \]
  Не может быть рядом Фурье какой-либо абсолютно инт. на $[-\pi,\pi]$ функции.
\end{eg}

\subsection{Ядро Дирихле. Принцип локализации}
Пусть $f(x)$ - $2\pi$ периодическая и абсолютно интегрируема на $[-\pi,\pi]$ и
\[
  f(x) \sim \frac{a_0}{2}+\sum_{n=1}^{\infty}a_n \cos nx + b_n \sin nx
\]
Рассмотрим частичные суммы ряда Фурье
\[
  S_n(x)=S_n(x,b)=\frac{a_0}{2}+\sum_{k=1}^{n}a_k \cos kx + b_k \sin kx
\]
Преобразуем:
\begin{gather*}
  S_n(x)= \frac{1}{2\pi}\int_{-\pi}^{\pi}f(x)dt+ \\ 
  +\sum_{k=1}^{n}\frac{1}{\pi}\int_{-\pi}^{\pi}f(t)\cos kx dt \cdot \cos kx +
  \frac{1}{\pi}\int_{-\pi}^{\pi}f(t)\sin kt dt \cdot \sin kx =\\
  \frac{1}{\pi}\int_{-\pi}^{\pi}f(t)
  \left(\frac{1}{2}+\sum_{k=1}^{n}\cos k(t-x)\right)dt \\ 
  S_n(x)=\frac{1}{\pi}\int_{-\pi}^{\pi}f(t)D_n(t-x)dt
\end{gather*}
\begin{definition}
  Функция 
  \[
    D_n(t)=\left(\frac{1}{2}+\sum_{k=1}^{n}\cos k(t)\right)
  \]
  называется ядром Дирихле
\end{definition}
\incfig{l2_dirichle}
Свойства ядра Дирихле:
\begin{enumerate}
  \item $D_n(t)$ - четная, $2\pi$ период. и непр. функция
  \item $\int_{-\pi}^{\pi}D_n(t)dt=\pi$
  \item $\max|D_n(t)|=\max D_n(t)=D_n(0)=n+\frac{1}{2}$
  \item $D_n(t)=\cfrac{\sin (n+\frac{1}{2})t}{2\sin \frac{t}{2}}$, при $t\neq 2\pi m, m\in \mathbb{Z}$
\end{enumerate}
\begin{proof} \phantom{.} 

  \begin{enumerate}
    \item Следует из аналогичных свойств слогаемых
    \item Очев
    \item $\frac{1}{2}-n\le D_n(t) \le \frac{1}{2}+n = D_n(0)$
    \item
      \begin{gather*}
        D_n(t)=\frac{1}{2}+\cos t + \cos 2t + \dots + \cos nt = \\ 
        = \cfrac{\sin \frac{t}{2}+2\sin\frac{t}{2}\cos t + 2\sin\frac{t}{2}\cos 2t + \dots + 2\sin \frac{t}{2}\cos nt}{2\sin\frac{t}{2}}= \\
        =\cfrac{\sin \frac{t}{2}-\sin \frac{t}{2}+\sin \frac{3t}{2} - \sin \frac{3t}{2} + \dots - \sin (n-\frac{1}{2})t + \sin (n+\frac{1}{2})t}{\sin\frac{t}{2}}= \\ 
        =\cfrac{\sin (n+\frac{1}{2})t}{2\sin\frac{t}{2}} 
      \end{gather*}
  \end{enumerate}
\end{proof}
\begin{theorem}[Принцип локализации]
  Пусть $f(x)$ - $2\pi$ периодическая абсолютно интегрируема на $[-\pi,\pi]$ функция.
  Пусть $x_0\in\R$, $0<\delta<\pi$. Тогда: $\lim\limits_{n\to\infty}S_n(x_0)$ и
  $\lim\limits_{n\to\infty}\frac{1}{\pi}\int_{0}^{\delta}D_n(t)(f(x_0+t)+f(x_0-t))dt$
  существуют или нет одновременно. В случае существования равны.
\end{theorem}
\begin{remark}
  Таким образом сходимость и значение суммы ряда Фурье $2\pi$ пер. и абс. инт.
  на $[-\pi,\pi]$ зависит только от свойств функции в сколь угодно малой окрестности.
\end{remark}
\begin{proof}
  Преобр. $S_n$:
  \begin{gather*}
    S_n(x_0)\underset{t=\tau+x_0}{=}\frac{1}{\pi}\int_{-\pi-x_0}^{\pi-x_0}f(x_0+\tau)D_n(\tau)d\tau 
  \end{gather*}
  \begin{equation}\label{l2_3a}
    S_n(x_0)=\frac{1}{\pi}\int_{-\pi}^{\pi}f(x_0+\tau)D_n(\tau)d\tau 
  \end{equation}
  \begin{gather*}
    S_n(x_0)=\frac{1}{\pi}(\underbrace{\int_{-\pi}^{0}}_{\tau=-t}+\underbrace{\int_{0}^{\pi}}_{\tau=t})f(x_0+\tau)D_n(\tau)d\tau = \\ 
    =\frac{1}{\pi}(\int_{0}^{\pi}f(x_0-t)D_n(-t)dt + \int_{0}^{\pi}f(x_0+t)D_n(t)dt)
  \end{gather*}
  \begin{equation}\label{l2_3b}
    S_n(x_0)=\frac{1}{\pi}\int_{0}^{\pi}D_n(t)(f(x_0-t)+f(x_0+t))dt
  \end{equation}
  \begin{gather*}
    S_n(x_0)=\frac{1}{\pi}(\int_{0}^{\delta}+\int_{\delta}^{\pi})\frac{\sin(n+\frac{1}{2}t)}{2\sin \frac{t}{2}}(f(x_0-t)+f(x_0+t))dt \\
    \frac{1}{2\sin\frac{t}{2}} \le \frac{1}{2\sin \frac{\delta}{2}} \qquad \text{на} [\delta,\pi]
  \end{gather*}
  Тогда $\frac{(f(x_0-t)+f(x_0+t))}{2\sin \frac{t}{2}}$ - абс. инт на $[\delta,\pi]$ (см \ref{l1_abs_prod}).

  Тогда 2-ой инт. $\to0$ при $n\to\infty$ (Т. Римана) и получаем нужный результат.
\end{proof}

% lecture 3 

\subsection{Признаки сходимости ряда Фурье в точке}
\begin{theorem}[Признак Дини]
  Пусть $f(x)$ - $2\pi$ периодическая и абсолютно инт. на $[-\pi,\pi]$ функция.
  Пусть  в точке $x_0$ существуют $f(x_0+0)$ и $f(x_0-0)$.
  Пусть для некоторого $\delta>0$ $\int_{0}^{\delta}\frac{|f^*_{x_0}(t)|}{t}dt$
  (где $f^*_{x_0}(t)=f(x_0+t)+f(x_0-t)-f(x_0+0)-f(x_0+0)$) сходится.
  Тогда ряд Фурье сходится в точке $x_0$ к значению $\frac{f(x_0+0)+f(x_0-0)}{2}$.
\end{theorem}
\begin{proof}
  Рассматриваем
  \begin{gather*}
    S_n(x_{0})-\frac{f(x_0+0)+f(x_0-0)}{2}= \\
     \frac{1}{\pi}\int_{0}^{\delta}D_n(t)(f(x_0+t)+f(x_0-t))dt-\frac{f(x_0+0)+f(x_0-0)}{2}\frac{2}{\pi}\int_{0}^{\pi}D_n(t)dt = \\ 
    = \frac{1}{\pi}\int_{0}^{\pi}f^*_{x_0}(t)D_n(t)dt=\frac{1}{\pi}\int_{0}^{\pi}f^*_{x_0}(t)\frac{\sin(n+.5)t}{2\sin.5t}dt = \\
    = \frac{1}{\pi}\int_{0}^{\pi}\frac{f^*_{x_0}(t)}{t}\frac{.5t}{\sin .5t}\sin(n+.5)t dt
  \end{gather*}
  Функция $f^{*}_{x_0}$ абс. инт. на $[0,\pi]$ (т.к. по сравнению с абс. инт функциями
  $f(x_0+t)$ и $f(x_0-t)$) у функции $f^{*}_{x_0}(t)$ появивлась лишь одна
  особенность при $t=0$, в окр. кот. функция по условию абс. инт.
  Функция $\frac{t/2}{\sin t/2}$ доопределима в до непрерывной и ограниченной на
  $[0,\pi]$ функции.

  По лемме \ref{l1_abs_prod} 
  $\frac{f_{x_0}^{*}}{t}\frac{t/2}{\sin t/2}$ - абс. инт. на $[0,\pi]$.
  По Т. Римана \ref{l1_r_osc} 
  интеграл $\to 0$ при $n \to \infty$
  \[
    \implies S_n(x_0) \underset{n\to\infty}{\to}\frac{f(x_0+0)+f(x_0-0)}{2}
  \]
\end{proof}
\begin{definition}
  Функция $f(x)$ удовлетворяет в точке $x_0$ правостороннему (левостороннему)
  условия Гёльдера с показателем $\alpha$ ($\alpha \in (0,1]$)
  если $\exists \delta>0, M>0 :\, \forall t\in(0,\delta) \true |f(x_0\underset{(-)}{+}t)-f(x_0\underset{(-)}{+}0)|<Mt^{\alpha}$.
\end{definition}
\begin{remark}
  При $\alpha$ это условие называется также условием Липшица.
\end{remark}
\begin{definition}
  Пусть $\exists$ $f(x_0+0)$ и $f(x_0-0)$. Введём обобщение односторонней производной
  \[
    f_{\pm}'(x_0)=\lim\limits_{t\to \pm 0} \frac{f(x_0+t)-f(x_0\pm0)}{t}
  \]
\end{definition}
\begin{lemma}
  Если $\exists$ (конечная) $f_{\underset{(-)}{+}}'(x_0)$, то $f(x)$ удовл.
  правостороннему (левостороннему) усл. Липшица.
\end{lemma}
\begin{proof}
  (для $f'(x_0)$)

  $\exists f_{+}'(x_0)=\lim\limits_{t\to+0}\frac{f(x_0+t)-f(x_0+0)}{t}$
  $\implies \frac{f(x_0+t)-f(x_0+0)}{t}=f_{+}'(x_0)+\underset{t\to +0}{o(1)}$

  $\implies$ В некоторой окр. $(0,\delta)$ выполнится
  $\frac{|f(x_0+t)-f(x_0+0)|}{t}\le |f_{+}'(x_{0})|+1$

  $\implies$ $|f(x_{0}+t)-f(x_{0}+0)|\le (|f_{+}'(x_{0})|+1)t$
\end{proof}
\begin{theorem}[Признак Липшица]
  Пусть $f(x)$ - $2\pi$ периодическая и абсолютно инт. на $[-\pi,\pi]$ функция.
  Пусть  в точке $x_0$ у функции $f(x)$ выполняются оба условия Гёльдера.
  Тогда ряд Фурье сходится в $x_{0}$ к значению $\frac{f(x_{0}+0)-f(x_{0}-0)}{2}$.
\end{theorem}
\begin{proof}
  Пусть выполняются оба условия Гёльдера.
  Тогда на $(0,\delta)$ выполняются:
  \[
    \frac{|f^{*}_{x_{0}}|}{t}\le \frac{|f(x_0+t)+f(x_0+0)|}{t}+\frac{|f(x_0-t)+f(x_0-0)|}{t}
    \le \frac{2Mt^{\alpha}}{t}=\underbrace{\frac{2M}{t^{1-\alpha}}}_{\text{абс.инт.}}
  \]
  $\implies$ по признаку сравнения $\int_{0}^{\delta}\frac{|f_{x_{0}}^{*}(t)|dt}{t}$ сход

  $\implies$ по признаку Дини  $S_n(x_{0})\underset{n\to\infty}{\to}\frac{f(x_{0}+0)-f(x_{0}-0)}{2}$
\end{proof}
\begin{corollary}
  Пусть $f(x)$ - $2\pi$ периодическая и абсолютно инт. на $[-\pi,\pi]$ функция.
  Пусть $\exists f(x_{0}\pm 0)$ и $f'_{\pm}(x_{0})$.

  $\implies$ $\lim\limits_{n\to \infty}S_n(x_{0})=\frac{f(x_{0}+0)-f(x_{0}-0)}{2}$.
\end{corollary}
\begin{proof}
  Следует из признака Липшица и леммы.
\end{proof}

\section{Суммирование рядов методом ср. арифм.}
Пусть $f(x)$ - $2\pi$ периодическая и абсолютно инт. на $[-\pi,\pi]$ функция.
\begin{definition}
  $\sigma_{n}(x)=\frac{S_0(x)+S_1(x)+\dots +S_n(x)}{n+1}$
  - сумма Фейера, где $S_k(x)$ - частичная сумма ряда Фурье.
\end{definition}
\begin{definition}
  $\Phi_n(x)=\frac{D_0(x)+D_1(x)+\dots +D_n(x)}{n+1}$ - ядро Фейера,
  где $D_k(x)$ - ядра Дирихле.
\end{definition}
Из формулы \ref{l2_3a} (принцип локализации) 
$S_n(x)=\frac{1}{\pi}\int_{-\pi}^{\pi}D_n(t)f(x+t)dt$
следует $\sigma_n(x)=\frac{1}{\pi}\int_{-\pi}^{\pi}\Phi_n(t)f(x+t)dt$.

\incfig{l3_phi_n}
Свойства ядра Фейереа:
\begin{enumerate}
  \item $\Phi_n(t)$ - четная, $2\pi$ периодичекая, непр функция
  \item $\int_{-\pi}^{\pi}\Phi_n(t)dt=\pi$
  \item $\max \Phi_n(t)=\Phi_n(0)=\frac{n+1}{2}$
  \item $\Phi_n(t)$ - неотр.
  \item $\Phi_n(t)=\frac{\sin^{2}\frac{n+1}{2}t}{2(n+1)\sin^{2}\frac{t}{2}}$
    при $t\neq 2\pi m$, $m\in \mathbb{Z}$
\end{enumerate}
\begin{proof} \phantom{.}

  \begin{enumerate}
    \item из св. ядра Дирихле
    \item из св. ядра Дирихле
    \item Т.к. $\max D_n(t)=D_n(0)=n+\frac{1}{2}$, то
      \begin{gather*}
        \max \Phi_n(t)=\Phi_n(0)=\frac{1}{n+1}(D_0(0)+D_1(0)+\dots +D_n(0))= \\
        =\frac{1}{n+1}\left(\frac{1}{2}+\left(\frac{1}{2}+1\right)+\dots +\left(\frac{1}{2}+n\right)\right)= \\
        =\frac{1}{n+1}\frac{\frac{1}{2}+\frac{1}{2}+n}{2}(n+1)=\frac{n+1}{2}
      \end{gather*}
    \item из 5)
    \item
      \begin{gather*}
        (n+1)\Phi_n=D_0(t)+D_1(t)+\dots +D_n(t)=\\
        =\frac{\sin \frac{t}{2}+\sin \frac{3}{2}t + \dots +\sin(n+\frac{1}{2})t}{2\sin \frac{t}{2}} = \\
        =\frac{2\sin \frac{t}{2} \sin \frac{t}{2}+2\sin\frac{3}{2}t\sin\frac{t}{2}+\dots +2\sin(n+\frac{1}{2})t\sin\frac{t}{2}}{4\sin^{2}\frac{t}{2}} = \\
        =\frac{\cos 0 - \cos t + \cos t - \cos 2t +\dots + \cos nt - \cos (n+1)t}{4\sin^{2}\frac{t}{2}} = \\ 
        =\frac{1-\cos (n+1)t}{4\sin^{2}\frac{t}{2}}=\frac{\sin^{2}\frac{n+1}{2}t}{2\sin^{2}\frac{t}{2}}
      \end{gather*}
  \end{enumerate}
\end{proof}

% lecture 4

\begin{theorem}[Фейера]
  Пусть $f(x)$ - $2\pi$ периодическая, непрерывная
  $\implies \sigma_n(x) \underset{\R}{\rightrightarrows} f(x)$
\end{theorem}
\begin{proof}
  \begin{gather*}
    |\sigma_n(x)-f(x)|=\frac{1}{\pi}\int_{-\pi}^{\pi}\Phi_n(t)f(x+t)dt-f(x)\int_{-\pi}^{\pi}\Phi_n(t)dt| = \\ 
  = \frac{1}{\pi}|\int_{-\pi}^{\pi}\Phi_n(t)(f(x+t)-f(x))dt|
  \le \frac{1}{\pi}\int_{-\pi}^{\pi}\Phi_n(t)|f(x+t)-f(x)|dt
  \end{gather*}
  Т.к. $f(x)$ - непр., $2\pi$ период, то она ограничена и существует $C>0:|f(x)|\le C$.
  Также $f(x)$ равномерно непрерывна на $\R$.
  
  Возьмём $\forall \epsilon>0$ (в силу равн. непр.) 
  \begin{gather*}
    \exists \delta \in (0,\pi): \forall x, x':|x-x'|<\delta \true |f(x)-f(x')|<\frac{\epsilon}{3} \\ 
    \frac{1}{\pi}\int_{-\pi}^{\pi}\Phi_n(t)|f(x+t)-f(x)|dt=\underbrace{\frac{1}{pi}\int_{-\pi}^{-\delta}\dots dt}_{I_1} + \underbrace{\frac{1}{\pi}\int_{-\delta}^{\delta}\dots dt}_{I_2} + \underbrace{\frac{1}{\pi}\int_{\delta}^{\pi}\dots dt}_{I_3} \\ 
    I_2= \frac{1}{\pi}\int_{-\delta}^{\delta}\Phi_n(t)|f(x+t)-f(x)|dt < \frac{\epsilon}{3}\frac{1}{\pi}\int_{-\delta}^{\delta}\Phi_n(t)dt 
    < \frac{\epsilon}{3}\frac{1}{\pi}\int_{-\pi}^{\pi}\Phi_n(t)dt < \frac{\epsilon}{3} \\ 
    I_3=\frac{1}{\pi}\int_{\delta}^{\pi}\Phi_n(t)|f(x+t)-f(x)|dt \le 
    \frac{1}{\pi}\int_{\delta}^{\pi}\Phi_n(t)(|f(x+t)|+|f(x)|)dt \le  \\
    \le \frac{2C}{\pi}\int_{\delta}^{\pi}\Phi_n(t)dt \le
    \frac{2C}{\pi}\pi \max \limits_{[\delta,\pi]} \Phi_n(t) \le 
    2C\frac{1}{2(n+t)\sin^{2}\frac{\delta}{2}} \underset{n\to\infty}{\to}0
  \end{gather*}
  Тогда $\exists n_3: \forall n \ge n_3 \true I_3<\frac{\epsilon}{3} \forall x$.
  Аналогично $\exists n_1: \forall n \ge n_1 \true I_1<\frac{\epsilon}{3} \forall x$.
  $\underline{\exists n_0}=\max{n_1,n_3}: \underline{\forall n \ge n_0 \true |\sigma_n(t)-f(x)|<\epsilon \, \forall x}$.

  Подчёркнутое означает $\sigma_n(t)\underset{\R}{\rightrightarrows}f(x)$
\end{proof}
\begin{corollary}
  Если ряд Фурье непр., $2\pi$ периодической функции сходится в точке $x$,
  то он сходится к $f(x)$.
\end{corollary}
\begin{proof}
  Пусть $\lim \limits_{n\to\infty} S_n(t)=A \implies \lim \limits_{n\to\infty}\sigma_n(x)=A$.

  По Т. (Фейера) $\lim\limits_{n\to\infty}\sigma_n(x)=f(x)$
  $\implies A=f(x) \implies \lim\limits_{n\to\infty}S_n(t)=f(x)$.
\end{proof}

\section{Приближение непр. функ. многочленами}
\begin{definition}
  Функция вида
  \[
    T(x)=\frac{A_0}{2}+\sum_{k=1}^{n}A_k\cos kx + B_k \sin kx
  \]
  называются тригонометрическим многочленом.
\end{definition}
\begin{theorem}[Т1 Веерштрасса]
  Пусть $f(x)$ - $2\pi$ период., непр функция.

  Тогда $\forall \epsilon>0 \exists$ триг. многочлен $T(x): \max \limits_{x\in \R} |f(x)-T(x)| < \epsilon$.
\end{theorem}
\begin{proof}
  Т.к. $\sigma_n(x)\underset{x\in \R}{\rightrightarrows} f(x)$,
  то $\underline{\forall \epsilon >0} \exists N : \forall n\ge N, \forall x\in \R \true |\sigma_n(x)-f(x)|<\epsilon$
  и $\underline{\exists T(x)}=\sigma_n(x)\underline{:\max\limits_{x\in \R} |T(x)-f(x)|<\epsilon}$.
\end{proof}
\begin{theorem}[Т1' (перефразирование)]
  Пусть $f(x)$ - непр на $[-\pi,\pi]$ и $f(-\pi)=f(\pi)$
  $\implies \forall \epsilon > \exists$ триг. многочлен $T(x): \max \limits_{x\in \R} |f(x)-T(x)| < \epsilon$.
\end{theorem}
\begin{proof}
  Такая функция продолжаема до $2\pi$ периодической, непр. функции, можем применять Т1.
\end{proof}
\begin{theorem}[Т2 Веерштрасса]
  Пусть $f(x)$- непр. на $[a,b]$.
  $\implies \forall \epsilon>0 \exists$ алгебраический многочлен
  $P(x):\max \limits_{x\in[a,b]}|f(x)-P(x)|<\epsilon$.
\end{theorem}
\begin{proof}
  Отобразим отрезок $[0,\pi]$ на отрезок $[a,b]$:
  $x=a+\frac{b-a}{\pi}t$, обозначим $f^{*}(t)=f(a+\frac{b-a}{\pi}t)$, $t\in[0,\pi]$.

  Продолжим $f^{*}(t)$ чётно на $[-\pi,\pi]$ и $2\pi$ периодически на $\R$,
  сохранив обозначение $f^{*}(t)$.

  По Т1 $\forall \epsilon>0 \exists$ тр. мн.
  $T(t): \max \limits_{t\in[0,\pi]}|f^{*}(t)-T(t)| \le \max \limits_{t\in \R}|f^{*}(t)-T(t)| < \frac{\epsilon}{2}$.

  Ряды Тейлора для $\cos kt$ и $\sin kt$ и, следовательно, для $T(t)$
  имеют радиус сходимости $+\infty$, и следовательно равномерно сходятся на любом отрезке.
  
  Таким образом $\exists$ алг. многочлен $P(t):\max\limits_{t\in[0,\pi]} |T(t)-P(t)|<\frac{\epsilon}{2}$.

  Тогда $\max \limits_{t\in[0,\pi]}|f^{*}(t)-P(t)|< \epsilon$ или 
  $\max \limits_{x\in[a,b]} |f(x)-P(\pi\frac{x-a}{b-a})|<\epsilon$.
\end{proof}
\subsection{Минимальное св-во коэфф. Фурье}
\begin{lemma}
  Если $f(x)$ инт (в несобственном смысле) на $[a,b]$ вместе с квадратом $f^{2}(x)$.
  $\implies$ $f(x)$ абс. инт на $[a,b]$.
\end{lemma}
\begin{proof}
  Следует из неравенства $|f(x)|\le \frac{1+f^{2}}{2}$.
\end{proof}
\begin{remark}
  \[
    \exists f=\left\{\begin{aligned}
      &0, & x=0 \\ 
      &\frac{1}{\sqrt{x}}, & x\in (0, 1]
    \end{aligned}\right.
  \]
  абс. инт на $[0,1]$ но не явл. инт с кв.
\end{remark}
\begin{theorem}
  Пусть $f(x))$ - $2\pi$ период. и инт. вместе с квадратом функция на $[-\pi,\pi]$.

  Пусть $S_n(x)$ - частичные суммы ряда Фурье, $a_n$ и $b_n$ - коэф. Фурье.

  Тогда:
  \begin{enumerate}
    \item $\int_{-\pi}^{\pi}(f(x)-S_n(x))^{2}dx=\min\limits_{T_n(x)}\int_{-\pi}^{\pi}(f(x)-T_n(x))^{2}dx$,
      где $T_n(x)$ - триг. многочлены степени не выше $n$.
    \item $\frac{a_0^{2}}{2}+\sum_{n=1}^{\infty}a^{2}_n+b^{2}_{n}\le \frac{1}{\pi}\int_{-\pi}^{\pi}f^{2}(x)dx$
- неравенство Бесселя
  \end{enumerate}
\end{theorem}
\begin{proof}
  Пусть $T_n(x)=\frac{A_0}{2}+\sum_{k=1}^{n}A_k\cos kx + B_k\sin kx$.

  Рассмотрим
  \begin{gather*}
    \int_{-\pi}^{\pi}(f(x)-T_n(x))^{2}dx=\int_{-\pi}^{\pi}f^{2}(x)dx - 2\int_{-\pi}^{\pi}f(x)T_n(x)dx+\int_{-\pi}^{\pi}T_n^{2}(x)dx = \\ 
    = \int_{-\pi}^{\pi} f^{2}(x)dx -2\pi (\frac{a_0A_0}{2}+\sum_{k=1}^{n}a_kA_k+b_kB_k)+\pi(\frac{A_0^{2}}{2}+\sum_{k=1}^{n}A_k^{2}+B_k^{2})= \\ 
    = \int_{-\pi}^{\pi}f^{2}(x)dx-\pi\left(\frac{a_0^{2}}{2}+\sum_{k=1}^{n}a_k^{2}+b_k^{2}\right) +\\
    +\pi\left(\frac{(A_0-a_0)^{2}}{2}+\sum_{k=1}^{n}(A_k-a_k)^{2}+(B_k-b_k)^{2}\right)
  \end{gather*}
  Видно, что последнее выражение минимально при $A_k=a_k$, $B_k=b_k$,
  что доказывает $1$.

  Для доказательства $2$ возьмём $T_n=S_n$:
  \begin{gather*}
    \int_{-\pi}^{\pi}(f(x)-S_n(x))^{2}dx=\int_{-\pi}^{\pi}f^{2}(x)dx-\pi\left(\frac{a_0}{2}+\sum_{k=1}^{n}a_k^{2}+b_k^{2}\right) \ge 0\\
    \int_{-\pi}^{\pi}f^{2}(x)dx \ge \pi\left(\frac{a_0}{2}+\sum_{k=1}^{n}a_k^{2}+b_k^{2}\right)
  \end{gather*}
  Частичн. суммы ряда $\sum_{k=1}^{\infty}a_k^{2}+b_k^{2}$ составляют неубывающую,
  ограниченную последовательность, следовательно ряд сходится.

  Переходим в последнем неравенстве к пределу при $n\to\infty$ и получаем 2.
\end{proof}

% lecture 5

\begin{theorem} [Равентсво Парсеваля]
  Пусть $f(x)$ - $2\pi$ периодична, непрерывна,
  $a_n$ и $b_n$ её коэф. Фурье, тогда
  \[
    \frac{1}{\pi}\int_{-\pi}^{\pi}f^{2}(x)dx=\frac{a_0^{2}}{2}+\sum_{n=1}^{\infty}a_n^{2}+b_n^{2}
  \]
\end{theorem}
\begin{remark}
  Равенство верно и для интегрируемой функции вместе с квадратом.
\end{remark}
\begin{remark}
  Равенство Парсеваля получается при формальной подстановке $S(x)$ вместо $f(x)$.
\end{remark}
\begin{proof}
  Возьмём $\underline{\forall \epsilon > 0}$. По Т. Вейерштрасса
  $\exists T_n(x)$ - триг. многочлен, такой что
  $\max \limits_{[-\pi,\pi]}|f(x)-T_n(x)|<\frac{\sqrt{\epsilon}}{2}$, тогда
  \[
    \frac{1}{\pi}\int_{-\pi}^{\pi}(f(x)-T_n(x))^{2}dx<\epsilon
  \]
  Из минимального свойства коэф. Фурье:
  \[
    \frac{1}{\pi}\int_{-\pi}^{\pi}(f(x)-S_n(x))^{2}dx
    \le \frac{1}{\pi}\int_{-\pi}^{\pi}(f(x)-T_n(x))^{2}dx < \epsilon \tag{$1$}
  \]
  Тогда:
  \begin{gather*}
    \underline{0\overset{\text{нер. Б.}}{\le }\frac{1}{\pi}\int_{-\pi}^{\pi}f^{2}(x)dx-\left[\frac{a_0^{2}}{2}+\sum_{n=1}^{\infty}a_n^{2}+b_n^{2}\right]} \le \\
    \le \frac{1}{\pi}\int_{-\pi}^{\pi}f^{2}(x)dx- \left[\frac{a_0^{2}}{2}+\sum_{k=1}^{n}a_k^{2}+b_k^{2}\right]
    = \frac{1}{\pi}\int_{-\pi}^{\pi}(f(x)-S_n(x))^{2}dx \underline{\overset{(1)}{<} \epsilon}
  \end{gather*}
  Из подчёркнутого получаем равенство Парсвеваля.
\end{proof}

\begin{definition}[Кусочно непр. дифф.]
  Функция $f(x)$ называется кусочно непрерывно дифф. на отрезке $[a,b]$,
  если $\exists$ такое разбиение отрезка, что на каждом отрезке разбиения,
  функция непрерывно дифф. (в концевых точках односторонне).
\end{definition}
\begin{theorem}[О почленном дифф. ряда Фурье]
  Пусть $f(x)$ - $2\pi$ период., непр на $\R$ и кусочно непр. дифф. на $[-\pi,\pi]$.

  Пусть $f(x) \sim \frac{a_0}{2}+\sum_{n=1}^{\infty}a_n\cos nx + b_n \sin nx$.

  Тогда $f'(x) \sim \sum_{n=1}^{\infty} nb_n\cos nx - na_n\sin nx$.
  (т.е. ряд можно формально дифф.)
\end{theorem}
\begin{remark}
  О сходимости ничего не говорится.
\end{remark}
\begin{proof}
  Пусть $f'(x)\sim \frac{\alpha_0}{2}+\sum_{n=1}^{\infty}\alpha_n\cos nx + \beta_n\sin nx$
  \begin{gather*}
    \alpha_0=f(\pi) - f(-\pi)= 0 \\ 
    \alpha_n=\frac{1}{\pi}\int_{-\pi}^{\pi}f'(x)\cos nx dx = \frac{1}{\pi}\int_{-\pi}^{\pi}\cos nx df(x) \\ 
    = \underbrace{\frac{1}{\pi}f(x)\cos nx \Big|_{-\pi}^{\pi}}_{0}- \frac{1}{\pi}\int_{-\pi}^{\pi}f(x)(-n)\sin nx df = nb_n \\
    \beta_n=\frac{1}{\pi}\int_{-\pi}^{\pi}f'(x)\sin nx dx = \frac{1}{\pi}\int_{-\pi}^{\pi}\sin nx df(x) \\ 
    = \underbrace{\frac{1}{\pi}f(x)\sin nx \Big|_{-\pi}^{\pi}}_{0}- \frac{1}{\pi}\int_{-\pi}^{\pi}f(x)(-n)\cos nx df = -na_n
  \end{gather*}
\end{proof}

\begin{lemma}[О порядке убывания коэф. Фурье]
  Пусть $f(x)$ - $2\pi$ периодична и непр. на $\R$.

  Пусть $f(x)$ имеет непр. производную до порядка $k-1$ включительно на $\R$
  и кусочно непр. производную порядка $k$ ($k\ge 1$) на $[-\pi,\pi]$.

  Пусть $f(x) \sim \frac{a_0}{2}+\sum_{n=1}^{\infty}a_n\cos nx + b_n \sin nx$.

  $\implies$ $|a_n| \le \frac{\epsilon_n}{n^{k}}$, $|b_n| \le \frac{\epsilon_n}{n^{k}}$,
  где $\sum_{n=1}^{\infty}\epsilon_n^{2}$ - сход.
\end{lemma}
\begin{proof}
  Пусть $f^{*}(x)\sim \sum_{n=1}^{\infty}\alpha_n\cos nx + \beta_n \sin nx$.

  Применяем предыдущую Т. $k$ раз, получим
  либо
  \[
    \alpha_n=\pm n^{k}a_n \qquad
    \beta_n=\pm n^{k} b_n \tag{$1$}
  \]
  либо
  \[
    \alpha_n=\pm n^{k}b_n \qquad
    \beta_n=\pm n^{k} a_n \tag{$2$}
  \]
  При этом $\sum_{n=1}^{\infty}\alpha_n^{2}+\beta_n^{2} \le \frac{1}{\pi}\int_{-\pi}^{\pi}(f^{(k)})^{2}dx$
  (нер. Бесселя) и ряд $\sum_{n=1}^{\infty}\epsilon_n^{2}$ (где $\epsilon_n = \sqrt{\alpha_n^{2}+\beta_n^{2}}$)
  сходится.

  Если справедл $(1)$, то
  \[
    |a_n|=\frac{\alpha_n}{n^{k}} \le \frac{\alpha_n^{2}+\beta_n^{2}}{n^{k}}=\frac{\epsilon_n}{n^{k}} \qquad |b_n| \le \frac{\epsilon_n}{n^{k}}
  \]
  Аналогично в случае $(2)$.
\end{proof}

\begin{theorem}[О скорости сходимости ряда Ф. к функ.]
  Пусть $f(x)$ - $2\pi$ периодична и непр. на $\R$.

  Пусть $f(x)$ имеет непр. производную до порядка $k-1$ включительно на $\R$
  и кусочно непр. производную порядка $k$ ($k\ge 1$) на $[-\pi,\pi]$.
  
  Тогда ряд Фурье равномерно и абсолютно сходится к $f(x)$ на $\R$
  и выполняется 
  \[
    |f(x)-S_n(x)| \le \frac{\eta_n}{n^{k-.5}}
  \]
  гле $\lim \limits_{n\to \infty} \eta_n =0$ и $\{\eta_n\}$ - числовая посл.
\end{theorem}
\begin{proof}
  Пусть $f(x) \sim \frac{a_0}{2}+\sum_{n=1}^{\infty}a_n\cos nx + b_n \sin nx$
  и $S_n(x)$ - сумма Фурье порядка $n$.

  Выполняются достаточные условия сходимости ряда Фурье к функции,
  т.е. $\lim \limits_{n\to \infty} S_n(x)=f(x)$.

  Рассмотрм остаток ряда Фурье
  \[
    r_n(x)=f(x)-S_n(x)=\sum_{m=n+1}^{\infty}a_m\cos mx + b_m \sin mx
  \]
  \hr
  При анализе остатка будем использовать неравенства Коши-Буняковского-Шварца
  для ряда
  \[
    \sum_{n=1}^{\infty}u_nv_n \le \sqrt{\sum_{n=1}^{\infty}u_n^{2}}\sqrt{\sum_{n=1}^{\infty}v_n^{2}} \tag{$1$}
  \]
  которое получается из нер. К.-Б.-Ш. для конечной суммы
  \[
    \sum_{n=1}^{N}u_nv_n \le \sqrt{\sum_{n=1}^{N}u_n^{2}}\sqrt{\sum_{n=1}^{N}v_n^{2}}
  \]
  после применения предельного перехода $N\to\infty$.

  Также будем использовать нер.
  \[
    \frac{1}{m^{p}} \le \int_{m-1}^{m}\frac{dx}{x^{p}}, \ p>0 \tag{$2$}
  \]
  которое получается инт. нерав. $\frac{1}{m^{p}}\le \frac{1}{x^{p}}$
  по отрезку $[m-1,m]$.

  \hr
  \begin{gather*}
    |r_n(x)| \le \sum_{m=n+1}^{\infty}|a_m \cos mx| +|b_m\sin mx|
    2\sum_{m=n+1}^{\infty}\frac{\epsilon_m}{m^{k}} \le \\
    \overset{(1)}{\le} \sqrt{\sum_{m=n+1}^{\infty}\epsilon_m^{2}}\sqrt{\sum_{m=n+1}^{\infty}\frac{1}{m^{2k}}}
    \overset{(1)}{\le} \sqrt{\sum_{m=n+1}^{\infty}\epsilon_m^{2}}\sqrt{\sum_{m=n+1}^{\infty}\int_{m-1}^{m}\frac{dx}{x^{2k}}} = \\
    = 2\sqrt{\sum_{m=n+1}^{\infty}\epsilon_m^{2}}\sqrt{\sum_{m=n+1}^{\infty}\int_{n}^{\infty}\frac{dx}{x^{2k}}}
    = \underbrace{\frac{2}{\sqrt{2k-1}}\sqrt{\sum_{m=n+1}^{\infty}\epsilon_m^{2}}}_{\eta_n \underset{n\to\infty}{\to}0}\sqrt{\frac{1}{n^{2k-1}}}
    = \frac{\eta_n}{n^{k-\frac{1}{2}}}
  \end{gather*}
  $\implies$ Ряд сходится равномерно, т.к. $\eta_n$ не зависит от $x$.

  Т.к. получилась оценка
  \[
    \sum_{m=n+1}^{\infty}|a_m\cos mx| + |b_m\sin mx| \le \frac{\eta_n}{n^{k-\frac{1}{2}}}
  \]
  из которой следует абсолютная сход. остатка, заключаем, что ряд сход. абс.
\end{proof}
\begin{corollary}[Достаточное усл. равн. сход. ряда]
  Пусть $f(x)$ - $2\pi$ период., непр на $\R$, имеет кус. непр.
  производную на $[-\pi,\pi]$

  $\implies$ Ряд Фурье равномерно на $\R$ и абс. сходится к $f(x)$.
\end{corollary}
\begin{proof}
  Следует из Т. при $k=1$.
\end{proof}

\begin{theorem}[О почленном инт. ряда Фурье]
  Пусть $f(x)$ - $2\pi$ период. и кусочно непр. на $[-\pi, \pi]$.

  Пусть $f(x) \sim \frac{a_0}{2}+\sum_{n=1}^{\infty}a_n\cos nx + b_n \sin nx$.

  Тогда:
  \begin{gather*}
    \int_{0}^{x}f(t)dt=\int_{0}^{x}\frac{a_0}{2}dt+\sum_{n=1}^{\infty}\int_{0}^{x}(a_n\cos nt + b_n\sin nt)dt = \\
    = \frac{a_0x}{2}+\sum_{n=1}^{\infty}\frac{a_n}{n}\sin nx + \frac{b_n}{n}(1-\cos nx)
  \end{gather*}
  и ряд в правой части сх. равн. на $\R$.

\end{theorem}

% lecture 6

\begin{proof}
  Введём $F(x)=\int_{0}^{x}f(t)dt-\frac{a_0x}{2}=\int_{0}^{x}(f(t)-\frac{a_0}{2})dt$,
  $F(x)$ - непр. на $\R$, её производная $F'(x)=f(x) - \frac{a_0}{2}$
  - кусочно непрерывная функция на $[-\pi,\pi]$.

  \[
    F(x+2\pi)=F(x)+\underbrace{\int_{x}^{x+2\pi}(f(t)-\frac{a_0}{2})dt}_{0}=F(x)
  \]
  т.е. $F(x)$ - $2\pi$ периодична

  $\implies$ ряд Фурье $F(x)$ сх. равн. к $F(x)$ на $\R$.
  \begin{gather*}
    F(x)=\frac{A_0}{2}+\sum_{1}^{\infty}A_n\cos nx + B_n\sin nx \tag{$1$} \\ 
    A_n=\frac{1}{\pi}\int_{-\pi}^{\pi}F(x)\cos nx dx=\frac{1}{\pi n}\int_{-\pi}^{\pi}F(x)d\sin nx =\\
    =\frac{1}{\pi n}F(x)\sin nx \Big|_{-\pi}^{\pi} - \frac{1}{\pi n}\int_{-\pi}^{\pi}(f(x)-\frac{a_0}{2})\sin nx dx 
    = -\frac{b_n}{n} \\ 
    B_n=\frac{a_n}{n}
  \end{gather*}
  Положим в $(1)$ $x=0$:
  \begin{gather*}
    \frac{A_0}{2}+\sum_{n=1}^{\infty}A_n=0 \implies \frac{A_0}{2}=\sum_{n=1}^{\infty}\frac{b_n}{n} \\ 
    F(x)=\sum_{n=1}^{\infty}\frac{a_n}{n}\sin nx + \frac{b_n}{n}(1-\cos nx)
  \end{gather*}
\end{proof}

\subsection{Запись р. Фурье в комплексной форме}
Рассмотрим $f(x)$ - $2\pi$ периодическую, абс. инт на $[-\pi,\pi]$ функцию
\[
  f(x) \sim \frac{a_0}{2}+\sum_{n=1}^{\infty}a_n\cos nx + b_n\sin nx
\]
Подставим $\cos nx =\frac{e^{inx}+e^{-inx}}{2}$ и $\sin nx = \frac{e^{inx}-e^{-inx}}{2}$ (ф. Эйлера)
\[
  f(x)\sim \frac{a_0}{2}+\sum_{n=1}^{\infty}\frac{a_n-ib_n}{2}e^{inx}+\frac{a_n+ib_n}{2}e^{-inx}
\]
Введём обозначения $c_0=\frac{a_{0}}{2}$; $c_n=\frac{a_n-ib_n}{2}$; $c_{-n}=\frac{a_n+ib_n}{2}$.

Тогда $f(x)\sim \sum_{n=-\infty}^{\infty}c_ne^{inx}$, где $S_n(x)=\sum_{k=-n}^{n}c_ke^{ikx}$.

Ряд сходится, если $\exists\lim \limits_{n\to\infty}S_n$.
\begin{gather*}
  c_n = \frac{1}{2\pi}\int_{-\pi}^{\pi}f(x)(\cos nx - i \sin nx)dx 
  = \frac{1}{2\pi}\int_{-\pi}^{\pi}f(x)e^{-inx}dx \\ 
  c_{-n}=\frac{1}{2\pi} \int_{-\pi}^{\pi}f(x)(\cos nx + i\sin nx)dx 
  = \frac{1}{2\pi}\int_{-\pi}^{\pi}f(x)e^{inx}dx
\end{gather*}
Общая формула:
\[
  c_n=\frac{1}{2\pi}\int_{-\pi}^{\pi}f(x)e^{-inx}dx
\]
\begin{remark}
  $c_{-n}=\bar{c}_n$, если $f(x)$ - действительное.
\end{remark}

\section{Метрические пространства}
\begin{definition}
  Множество $X$ называется метрическим пространством,
  если любой паре $x,y\in X$ поставлено в соответствие $\rho(x,y)$ (метрика или расстояние),
  такая что выполняются аксиомы:
  \begin{enumerate}
    \item $\rho(x,y)=\rho(y,x)$
    \item $\rho(x,y) \le \rho(x,z)+\rho(z,y)$ - неравенство треугольника
    \item $\rho(x,y)=0 \iff x=y$
  \end{enumerate}
\end{definition}
\begin{property}
  $\rho(x,y) \ge 0$.
\end{property} 
\begin{proof}
  $0=\rho(x,x) \le \rho(x,y)+\rho(y,x)=2\rho(x,y) \;$ $\forall x,y\in X$
\end{proof}
\begin{remark}
  Любое подмнож. метрического пространства является метрическим пространством.
\end{remark}
\begin{eg}
  Мн. $\R$ - метр. пр-во $\rho(x,y)=|x-y|$
\end{eg}
\begin{eg}
  Арифметическое евклидово пр-во $\R^{n}$, $\rho(x,y)=\sqrt{\sum_{k=1}^{n}(x_k-y_k)^{2}}$
\end{eg}
\begin{eg}
  Мн-во $B([a,b])$ ограниченных на $[a,b]$ функций,
  \begin{gather*}
    \rho(\phi(x), \psi(x))=\sup \limits_{x\in[a,b]}|\phi(x)-\psi(x)| \\ 
    |\phi(x)-\psi(x)|=|\phi(x)-\alpha(x)+\alpha(x)-\psi(x)|
    \le |\phi - \alpha| + |\alpha - \psi| \le\\
    \le \sup \limits_{[a,b]}|\phi-\alpha| + \sup \limits_{[a,b]}|\alpha-\psi| 
    =\rho(\phi,\alpha)+\rho(\alpha,\psi)
  \end{gather*}
  Перейдём в неравенстве к $\sup$:
  \begin{gather*}
  \rho(\phi,\psi)=\sup \limits_{[a,b]}|\phi -\psi| \le \rho(\phi,\alpha)+\rho(a,\psi)
  \end{gather*}
  Доказали неравенство треугольника.
\end{eg}
\begin{eg}
  Мн-во $CL([a,b])$ непр. функций на $[a,b]$ с метрикой
  $\rho(\phi, \psi)=\int_{a}^{b}|\phi(x)-\psi(x)|dx$
  является метр. пр-вом.

  2 аксиома: $|\phi-\psi|\le |\phi-\alpha|+|\alpha-\psi|\implies \rho(\phi,\psi)\le \rho(\phi,\alpha)+\rho(\alpha,\psi)$

  3 аксиома: $0=\rho(\phi,\psi)=\int_{a}^{b}|\phi(x)-\psi(x)|dx \implies \phi \equiv \psi$

  Если не требуем не непрерывности, то $\rho(\phi,\psi)=0 \not \implies \phi \equiv \psi$,
  т.к. функции могут отличатся в отдельных точках.
\end{eg}
\begin{definition}
  Посл. $\{x_{n}\}$ элементов метр. пр-ва $X$ называется сходящимся, если
  \[
    \exists x_0\in X: \forall \epsilon>0  \;\exists N: \forall n\ge N \true \rho(x_n,x_0)<\epsilon
  \]
  (или $\lim \limits_{n\to\infty} \rho(x_n,x_0)=0$).

  В этом случае считаем, что $\lim \limits_{n\to\infty}x_n=x_0$.
\end{definition}
\begin{definition}
  Посл. $\{x_n\}$ называется фундаментальной, если
  \[
    \forall \epsilon>0 \; \exists N:\forall n,m \ge N\true \rho(x_n,x_m) < \epsilon
  \]
\end{definition}
\begin{definition}
  Метр. пр-во $X$ называется полным, если в нём любая фундаментальная посл. сходится.
\end{definition}
\begin{theorem}
  Любая сходящаяся последовательность фундаментальна.
\end{theorem}
\begin{eg}
  Полные метр. пр-ва: $\R$, $\R^{n}$.
\end{eg}
\begin{eg}
  Неполные метр. пр-ва: $(0,1)$, где $\rho(x,y)=|x-y|$. 
  Здесь $\{\frac{1}{n}\}$ -фунд, но не сход.
\end{eg}
\begin{eg}
  $Q$ - мн. рац. чисел не явл. полным метр. пр.
  $\{(1+\frac{1}{n})^{n}\}$ - фунд., но не сход.
\end{eg}
\begin{remark}
  В метр. пр-ве, так же как и в $\R^{n}$, можно ввести понятия окрестности,
  внутр. точки, граничной т., т. прикосновения., предельной т.,
  замыкания, отк. множества и т.д.
\end{remark}

\subsection{Линейные нормированные пространства}
\begin{definition}
  Множество $X$ называется линейным пр-вом,
  если в нём определены сложение $(x+y)$ и умножение на число $(\alpha x)$:
  \begin{enumerate}
    \item $x+y=y+x$
    \item $(x+y)+z=x+(y+z)$
    \item $\exists 0: \: x+0=x$
    \item $\forall x \in X \; \exists (-x): \: x+(-x)=0$
    \item $(\lambda+\mu)x=\lambda x+ \mu x$
    \item $\lambda \mu)x=\lambda(\mu x)$
    \item $1 \cdot x=x$
  \end{enumerate}
  Определено вычитание $x-y=x+(-y)$.
\end{definition}
\begin{definition}
  Система векторов называется линейно независимой, если
  любая конечная подсистема линейно независима.
\end{definition}
\begin{definition}
  Линейное пространство $X$ называется нормированным, если
  на нём определенна действительная функция (функционал) $\Vert x \Vert$,
  такая что выполняются аксиомы:
  \begin{enumerate}
    \item $\Vert \lambda x \Vert=|\lambda|\Vert x \Vert   $
    \item $\Vert x+y \Vert \le \Vert x \Vert + \Vert y \Vert$
    \item $\Vert x \Vert=0 \implies x=0$
  \end{enumerate}
\end{definition}
\begin{property}
  $\Vert x \Vert \ge 0$
\end{property}
\begin{proof}
  \begin{gather*}
    0 = \Vert x-x \Vert \le \Vert x \Vert+\Vert x \Vert = 2 \Vert x \Vert
  \end{gather*}
\end{proof}

% lecture 7

\begin{definition}
  Функция, удовлетворяющая условиям 1 и 2 называется полунормой.
\end{definition}
\begin{eg}
  Мн-во действительных чисел: $\Vert x \Vert=|x|$
\end{eg}
\begin{eg}
  Геометрическое пространство: $\Vert \bar{x} \Vert=|\bar{x}|$
\end{eg}
\begin{eg}
  Мн-во $B[a,b]$ ограниченных на $[a,b]$ функций: 

  $\Vert f(x) \Vert=\sup_{[a,b]}|f(x)|$
\end{eg}
\begin{eg}
  Мн-во абс. инт. на $[a,b]$ функций $(RL_1[a,b])$: 

  $\Vert f(x) \Vert=\int_{a}^{b}|f(x)|dx$ - полунорма
\end{eg}
\begin{eg}
  Мн-во непр. на $[a,b]$ функ. $(CL_1[a,b])$: $\Vert f(x) \Vert=\int_{a}^{b}|f(x)|dx$
\end{eg}

\begin{lemma}
  Если $X$ - линейное нормированное пространство, то оно является
  метрическим пространством с метрикой $\rho(x,y)=\Vert x-y \Vert$
\end{lemma}
\begin{remark}
  В этом случае говорят, что метрика порождена нормой.
\end{remark}
\begin{remark}
  Все понятия, введённые в метрическом пространстве могут быть перенесены в нормированное пространство.
\end{remark}
\begin{definition}
  \phantom{.}

  \begin{enumerate}
    \item $\rho(x,y)=\Vert x-y \Vert=\Vert y-x \Vert=\rho(y,x)$
    \item $\rho(x,y)=\Vert x-y \Vert=\Vert (x-z)+(z-y) \Vert\le \Vert x-z \Vert + \Vert z-y \Vert=\rho(x,z)+\rho(z,y)$
    \item Если $\rho(x,y)=\Vert x-y \Vert=0$, то $x-y=0$, т.е. $x=y$

      Если $x=y$ $\implies$ $x-y=0$ $\implies$ $\Vert x-y \Vert=0$ $\implies $ $\rho(x,y)=0$
  \end{enumerate}
\end{definition}
\begin{definition}
  Линейное нормированное пространство называется полным, если оно полно в смысле
  метрики, порождённой нормой.
\end{definition}
\begin{definition}
  Полное линейное нормированное пространство называется банаховым. 
\end{definition}
\begin{definition}
  Система $\{x_{\alpha}, \alpha \in U\}$ элементов нормированного (полуномированного)
  линейного пространства $X$ называется полной в $X$, если $\forall x \in X\ \forall \epsilon>0$
  $\exists $ линейная комбинация (л.к.$(x_\alpha)$ или lc $x_\alpha$): $\Vert x- lc x_{\alpha}\Vert<\epsilon$
\end{definition}
\subsection{Теоремы Вейрештрасса о прибл. непр. функций}
\begin{theorem}[Т1 с волной]
  Система тригонометрических полна в пространстве $2\pi$-периодических,
  непрерывных функций в смысле нормы $\Vert f \Vert=\max_{[-\pi,\pi]}|f(x)|$
  (в смысле равномерного приближения)
\end{theorem}
\begin{theorem}[Т2 с волной]
  Система неотрицательных степеней $x$ $\{x,x^{2},x^{3},\dots ,x^{n},\dots \}$
  полна в пространстве $C[a,b]$ непрерывных на $[a,b]$ функций с нормой $\Vert f \Vert=\max_{[a,b]}|f(x)|$
  в смысле равномерного приближения
\end{theorem}
\subsection{Линейные пространства со скалярным произведением}
\begin{definition}
  Скалярным произведением в линейном пространстве $X$ называется функция двух переменных
  $(x,y)$, где $x,y \in X$:
  \begin{enumerate}
    \item $(x,x)>0$
    \item $(x,y)=(y,x)$
    \item $(\lambda x+ \mu y, z) = \alpha(x,z) + \mu(y,z)$
    \item $(x,x)=0 \ \implies x = 0$
  \end{enumerate}
\end{definition}
\begin{definition}
  Пространство со скалярным произведением называется предгильбертовым.
\end{definition}
\begin{definition}
  Функция $(x,y)$, удовлетворяющая условиям 1-3 называется полускалярным произведением.
\end{definition}
\begin{theorem}[Неравенство Коши-Буняковского]
  \phantom{.}

  Если $(x,y)$ -скалярное (полускалярное) произведение в линейном пространстве $X$,
  то $\forall x,y\in X$ $\true$ $(x,y)^{2}\le (x,x)(y,y)$
\end{theorem}
\begin{proof}
  \begin{gather*}
    \forall x,y \in X, \, \forall t \in R \true (tx+y,tx+y)\ge 0 \\ 
    (x,x)t^{2}+2(x,y)t+(y,y) \ge 0 \\ 
  \end{gather*}
  \begin{enumerate}
    \item $(x,x) = 0 \ \implies (x,y)=0 \implies$ нер-во верно
    \item $(x,x)\neq 0$ $\implies$ $D=(x,y)^{2}-(x,x)(y,y)\le 0$ $\implies$ нер. верно.
  \end{enumerate}
\end{proof}
\begin{corollary}[Неравенство треугольника]
  Для скалярного (полускалярного) произведения верно
  \[
    \sqrt{(x+y,x+y)} \le \sqrt{(x,x)}+\sqrt{(y,y)}
  \]
\end{corollary}
\begin{proof}
  \begin{gather*}
    (x+y,x+y)=(x,x)+2(x,y)+(y,y) \le (x,x) + 2|(x,y)|+|(y,y)| \le \\ 
    \le (x,x)+2\sqrt{(x,x)}\sqrt{(y,y)}+(y,y)=(\sqrt{(x,x)}+\sqrt{(y,y)})^{2}
  \end{gather*}
\end{proof}
\begin{lemma}
  Если $(x,y)$ - скалярное (полускалярное) произведение, то
  функция $\Vert x \Vert=\sqrt{(x,x)}$ является нормой (полунормой)
  и неравенство Коши-Буняковского имеет вид $|(x,y)|\le\Vert x \Vert\Vert y \Vert$
\end{lemma}
\begin{proof}
  \phantom{.}

  \begin{enumerate}
    \item $\sqrt{(\lambda x,\lambda x)}=|\lambda|\sqrt{(x,x)}$ 
    \item $\sqrt{(x+y,x+y)}\le \sqrt{(x,x)} + \sqrt{(y,y)}$
    \item В случае скал. произв. $\sqrt{(x,x)}=0 \ \implies \ x=0$
  \end{enumerate}
\end{proof}
\begin{eg}
  $\R \ (x,y)=xy$
\end{eg}
\begin{eg}
  Векторное пространство $(\vec{x},\vec{y})=|\vec{x}||\vec{y}|\cos \phi$
\end{eg}
\begin{eg}
  $RL_2[a,b]$ - мн-во функ., интегрируемых на $[a,b]$ в несобственном смысле вместе с квадратом

  Отметим, что если $f(x), g(x) \in RL_2[a,b]$, то $|f(x)g(x)|\le \frac{f^{2}+g^{2}}{2}$ и 
  $\int_{a}^{b}fgdx$ - сход. (абс.)

  Вводим $(f,g)=\int_{a}^{b}fgdx$ - полускалярное произв.
\end{eg}
\begin{eg}
  $CL_2[a,b]$ - мн-во непр. функций, скал. произв. $(f,g)=\int_{a}^{b}fgdx$
\end{eg}

\subsection{Сравнение норм}
Рассмотрим $C[a,b]$, $CL_1[a,b]$, $CL_2[a,b]$

В них $\Vert \phi \Vert_{\infty}=\max_{[a,b]}|\phi(t)|$, $\Vert \phi \Vert_{1}=\int_{a}^{b}|\phi(t)|dt$,
$\Vert \phi \Vert_{2}=\sqrt{\int_{a}^{b}\phi^{2}(t)dt}$

Неравенство Коши-Буняковского для $(f,g)=\int_{a}^{b}fgdx$:
\begin{gather*}
  \left( \int_{a}^{b}fgdt\right)^{2}\le \int_{a}^{b}f^{2}dt\int_{a}^{b}g^{2}dt
\end{gather*}
При $g=1$, $f(t)=|\phi(t)|$:
\begin{gather*}
  \left(\int_{a}^{b}|\phi(t)|dt\right)^{2} \le \int_{a}^{b}\phi^{2}(t)dt (b-a) \quad \Big|\sqrt{} \\
  \Vert \phi \Vert_1 \le \Vert \phi \Vert_{2}\sqrt{b-a}
\end{gather*}
Также верна: $\Vert \phi \Vert_{2}\le \Vert \phi \Vert_{\infty}\sqrt{b-a}$
 
\begin{theorem}
  Из сходимости $\{f_n(t)\}$ на норме $\Vert \cdot \Vert_{\infty}$ следует сходимость
  по норме $\Vert \cdot \Vert_{2}$, а из неё следует сходимость по норме $\Vert \cdot \Vert_{1}$
\end{theorem}
\begin{definition}
  Линейное пространство $X$ со скалярным произведением, полное в смысле метрики,
  порождённой скалярным произведением, т.е $\rho(x,y)=\Vert x-y \Vert=\sqrt{(x-y,x-y)}$ 
  называется гильбертовым пространством.
\end{definition}
\begin{theorem}
  $C[a,b]$ - пространство непрерывных функций с нормой $\Vert f(x) \Vert_{\infty}=\max_{[a,b]}$
  -полное пространство
\end{theorem}
\begin{proof}
Пусть $\{f_{n}(x)\}$ фундаментальна, т.е 

$\underline{\forall \epsilon > 0 \exists N: \forall n,m \ge N} \true \Vert f_n-f_m \Vert_{\infty}< \epsilon$

$\implies \ \underline{\forall x \in [a,b] \true |f_n(x)-f_m(x)|<\epsilon}$

Подчёркнутое означает, что выполнено условие Коши равномерной сходимости

$\implies \ \exists f$ - непр. на $[a,b]$: $f_n(x) \rightrightarrows_{[a,b]}f(x)$

$\implies \ \underline{\forall \epsilon > 0 \exists N: \forall n\ge N}, \forall x \in [a,b] \true |f_n(x)-f(x)| \epsilon$

$\implies \underline{\Vert f_n-f \Vert<\epsilon} \ \implies$ $f_n$ - сходится
по норме $\Vert \cdot \Vert_{\infty}$ $\implies$ полн. пр-ва
\end{proof}

% lecture 8

\begin{theorem}
  Пространство $CL_2[a,b]$ непрерыных функций, со скалярным произведением
  $(f,g)=\int_{a}^{b}f(x)g(x)dx$ не является полным.  
\end{theorem}
\begin{proof}
  Доказательство проведём для отрезка $[-1,1]$.

  Докажем, что последовательность
  \begin{gather*}
    f_n(x)=\left\{\begin{aligned}
      & -1, x \in [-1,\frac{-1}{n}] \\ 
      & nx, x \in (-\frac{1}{n}, \frac{1}{n}) \\ 
      & 1, x \in [\frac{1}{n},1]
    \end{aligned}\right.
  \end{gather*}
  является фундаментальной но не сходится в $CL_2[-1,1]$

  \incfig{l8_cl2}

  1) Докажем фундаментальность 

  \incfig{l8_cl2_1}

  Оценим
  \begin{gather*}
    \Vert f_{n+p}(x)-f_n(x) \Vert_2=\sqrt{\int_{-1}^{1}(f_{n+p}(x)-f_n(x))^{2}dx}= \\ 
    = \sqrt{2\int_{0}^{1/n}(f_{n+p}-f_n)^{2}dx} \le \sqrt{2\int_{0}^{1/n}(1-0)^{2}dx}= \\ 
    =\sqrt{\frac{2}{n}}\underset{n\to\infty}{\to} 0
  \end{gather*}
  Возьмём $\forall \epsilon >0$. Т.к. $\sqrt{\frac{2}{n}}\underset{n\to\infty}{\to}0$ то
  $\exists N: \forall n \ge N \true \sqrt{\frac{2}{n}}< \epsilon$.

  Тогда $\forall n \ge N, p \in N \true \Vert f_{n+p}-f_n \Vert<\epsilon$

  2) Отсутствие сходимости

  Предположим, что $\exists f(x)$ - непр. на $[-1,1]: \Vert f_n(x)-f(x) \Vert_2 \underset{n\to\infty}{\to}0$

  $\implies \sqrt{\int_{-1}^{1}(f_n-f)^{2}dx}\underset{n\to\infty}{\to}0$

  $\implies \sqrt{\int_{0}^{1}(f_n-f)^{2}dx}\underset{n\to\infty}{\to}0$

  Предположим $\exists x_0 \in (0,1): f(x_0)=1+\epsilon\neq 1$

  Тогда в некоторой $(x_0-\delta,x_0+\delta) \true |f(x)-1|> \frac{|\epsilon|}{2}$ и начиная с некоторого n выполняется
  $\sqrt{\int_{0}^{1}(f_n-f)^{2}dx}\ge \sqrt{\int_{x_0-\delta}^{x_0+\delta}\frac{\epsilon^{2}}{4}dx}=\sqrt{2\delta\frac{\epsilon^{2}}{4}} \underset{n\to\infty}{\not \to}0$

  $\implies \ f(x)=1$  на $(0,1)$, аналогично $f(x)=-1$ на $(-1,0)$

  $\implies \ f(x)$ разрывна в нуле

  $\implies \ \not \exists f(x)$ - непр. $\implies \ CL_2[-1,1]$ не явл. полным
\end{proof}

\section{Ортогональные сист. и разложение по ним}
Работаем в предгильбертовом пространстве $X_{\Pi}$ или в гильбертовом $X_{\Gamma}$
\begin{definition}
Система элементов $\{\xi_{\alpha}, \alpha \in U\}$ предгильбертого прост. $X_{\Pi}$
называется ортогональной, если $(x_{\alpha_1}, x_{\alpha_2})=0 \  \forall \alpha_,\alpha_2 \in U: \alpha_1 \neq \alpha_2$

Если при этом $\Vert x_\alpha \Vert=1 \ \forall \alpha$, то такая система называется ортонормированной.
\end{definition}
\begin{definition}
  Система элементов $X_{\Pi}$ называется незав., если $\forall $ её конечная подсистема
  - ленейно независ
\end{definition}
\begin{lemma}
  Ортог. сист. в $X_{\Pi}: x_\alpha \neq 0 \ \forall \alpha$ является лин. незав. 
\end{lemma}
\begin{proof}
  Как в ан. геометрии. (Предполагаем что система лин. завис., тогда есть нетрив. лин. комб. равная 0, 
  домнажаем скалярно на элементы и получаем что коэф. равны нулю из-за ортогональности)
\end{proof}
\begin{eg}
  Триг. система $\{\frac{1}{2}, \cos x, \sin x, \cos 2x, \sin 2x,\dots \}$ 

  - ортог. сист. в $L_2[-\pi,\pi]$, т.е. в смысле скал. произв. $\int_{-\pi}^{\pi}fgdx$

  Система $\{\frac{1}{\sqrt{2\pi}}, \frac{1}{\sqrt{\pi}}\cos x,\frac{1}{\sqrt{\pi}}\sin x,\dots  \}$ - ортонорм. сист.
\end{eg}
\begin{eg}
  Полином Лежандра
  \[
    P_0(x)=1 \qquad P_n(x)=\frac{1}{2^{n}n!}\frac{d^{n}(x^{2}-1)^{n}}{dx^{n}}
  \]
  где $ n \in N$ - ортог. сист. в $L_2[-1,1]$, где $(f,g)=\int_{-1}^{1}fgdx$. (Также $\Vert P_n \Vert=\sqrt{\frac{2}{2n+1}}$)
\end{eg}
\begin{proof}
  (Ортогональности)

  Заметим (формула Лейбница) $\frac{d^{k}(x^{2}-1)^{n}}{dx^{k}} \big|_{x=\pm 1}=0 $,
  $k=0,1,\dots ,n-1$

  Рассмотрим $Q_m(x)$ многочлен степени $m<n$.

  Тогда
  \begin{gather*}
    (Q_m,P_n)=\int_{-1}^{1}Q_m(x)\frac{d^{n}(x^{2}-1)^{n}}{dx^{n}}= \\ 
    =Q_m \frac{d^{n-1}(x^{2}-1)^{n}}{dx^{n-1}}\big|_{-1}^{1}- \int_{-1}^{1}Q_m'\frac{d^{n-1}(x^{2}-1)^{n}}{dx^{n-1}}\big|_{-1}^{1} = \\ 
    = Q_m'\frac{d^{n-2}(x^{2}-1)^{n}}{dx^{n-2}}\big|_{-1}^{1} + \int_{-1}^{1}Q_m''\frac{d^{n-2}(x^{2}-1)^{n}}{dx^{n-2}}\big|_{-1}^{1} = \\ 
    = \dots = (-1)^{m}Q_m^{(m)}\frac{d^{n-m-1}(x^{2}-1)^{n}}{dx^{n-m-1}}\big|_{-1}^{1}=0
  \end{gather*}
  В частности $\int_{-1}^{1}P_mP_ndx=0$ при $m<n$, т.е. полиномы Лежандра ортогональны.
\end{proof}

Далее рассматриваем счётные ортогональные системы $\{e_k\}: e_k\neq0$

\begin{definition}
  Пусть $\{e_1,e_2,\dots ,e_k,\dots \}$ $(e_k\neq 0)$ - ортог. сист. в $X_\Pi$,
  $x\in X_\Pi$. Тогда $\alpha_k=\frac{(x,e_k)}{\Vert e_k \Vert^{2}}$ назыв.
  коэф. Фурье $x$ по системе $\{e_k\}$.

  При этом элементу $x$ ставится в соответсвие ряд $\sum_{k=1}^{\infty}\alpha_ke_k$
  - ряд Фурье.

  Т.е. $x \sim \sum_{k=1}^{\infty}\alpha_ke_k$
\end{definition}
\begin{definition}
  Частичная сумма ряда Фурье $S_n=\sum_{k=1}^{n}\alpha_ke_k$
\end{definition}
\begin{remark}
  Триг. ряд Фурье функции, инт. с квалратов является рядом Фурье в $L_2$.
\end{remark}
\begin{remark}
  $\alpha_ke_k$ проекция $x$ на $e_k$, если система конечна $\alpha_k$ - координата
\end{remark}
\begin{definition}
  Говорят, что $x$ разложен в ряд Фурье и пишут $x=\sum_{k=1}^{\infty}\alpha_ke_k$,
  если $\Vert x-S_n \Vert \underset{n\to\infty}{\to} 0$
\end{definition}
\begin{theorem}[Т1]
  Пусть $x \in X_\Pi$, $x=\sum_{k=1}^{\infty}A_ke_k$ $\implies$ $A_k=\alpha_k$ ($\alpha_k$ - коэф. Ф.)
\end{theorem}
\begin{proof}
  $x=\sum_{k=1}^{\infty}A_ke_k$ $\implies$ $(x,e_m)\underset{(1)}{=}\sum_{k=1}^{\infty}A_k(e_k,e_m)=A_m\Vert e_m \Vert^{2}$
  $\implies$ $A_m=\frac{(x,e_m)}{\Vert e_m \Vert^{2}}=\alpha_m$

  Док-во $(1)$: $|(x,e_m) - \sum_{k=1}^{\infty}A_k(e_k,e_m)|=|(x-\sum_{k=1}^{n}A_ke_K,e_m)|$
  $\le \Vert x-\sum_{k=1}^{n}A_ke_k \Vert \Vert e_m \Vert \underset{n\to\infty}{\to}0$
  (первая норма стремится к 0)
\end{proof}

% lecture 9

\begin{theorem}[Минимальное св-во коэфф. Фурье]
  Пусть $\{e_k\}$ - ортогональная система, тогда
  \[
    \min_{A_1,\dots ,A_n}\Vert x- \sum_{k=1}^{n}A_ke_k \Vert=\Vert x- S_n(x) \Vert
  \]
  И выполняется
  \[
    \Vert x-S_n(x) \Vert^{2}=\Vert x \Vert^{2}-\sum_{k=1}^{n}\alpha_k^{2}\Vert e_k^{2} \Vert \tag{1}
  \]
\end{theorem}
\begin{proof}
  \begin{gather*}
    \underline{\Vert x-\sum_{k=1}^{n}A_ke_k \Vert^{2}}=(x-\sum_{k=1}^{n}A_ke_k,x-\sum_{k=1}^{n}A_ke_k) = \\ 
    = \Vert x \Vert^{2}-2\sum_{k=1}^{n}A_k(x,e_k) + \sum_{k=1}^{n}A_k^{2k}(e_k,e_k)= \\ 
    = \Vert x \Vert^{2}-2\sum_{k=1}^{n}A_k(x,e_k)+\sum_{k=1}^{n}A_k^{2}\Vert e_k \Vert^{2}= \\ 
    = \underline{\Vert x \Vert + \sum_{k=1}^{n}(A_k\Vert e_k \Vert-\frac{(x,e_k)}{\Vert e_k \Vert})^{2} - \sum_{k=1}^{n}\frac{(x,e_k)^{2}}{\Vert e_k \Vert^{2}}}
  \end{gather*}
  Если в качестве $A_k$ взять $\alpha_k=\frac{(x,e_k)}{\Vert e_k \Vert^{2}}$, то будет $\min$

  Подчёркнутое равенство при $A_k=\alpha_k$ даёт $(1)$
\end{proof}
\begin{corollary}
  В условиях теоремы $\Vert x-S_{n+1} \Vert \le \Vert x-S_n \Vert$
\end{corollary}
\begin{proof}
  Следует из $(1)$ (вичитаем при n+1 и n)
\end{proof}
\begin{theorem}[Т3]
  Пусть $x\in X_\Pi$; $\alpha_k$ - коэф. Фурье по ортог. системе $\{e_k\}$
  
  $\implies$ $\sum_{k=1}^{\infty}\alpha_k^{2}\Vert e_k^{2} \Vert \le \Vert x \Vert^{2}$ (Нерав. Бесселя)
\end{theorem}
\begin{proof}
  Рассмотрим $(1)$
  \begin{gather*}
    0 \le \Vert x-\sum_{k=1}^{n}\alpha_ke_k \Vert^{2}=\Vert x \Vert^{2}-\sum_{k=1}^{n}\alpha_k^{2}\Vert e_k \Vert^{2} \\ 
    \sum_{k=1}^{n}\alpha_k^{2}\Vert e_k \Vert^{2} \le \Vert x \Vert^{2}
  \end{gather*}
  Переходим к пределу при $n\to\infty$ и получаем неравенство
\end{proof}
\begin{corollary}
  В условиях теоремы $\alpha_k\Vert e_k \Vert{\to}0$ при $k\to\infty$
\end{corollary}
\begin{theorem}
  В $X_\Gamma$ ряд Фурье $\forall x \in X_\Gamma$ по любой ортогональной $\{e_k\}$ сходится.

  Если $x_0=\sum_{n=1}^{\infty}\alpha_ne_n$, то $(x-x_0,e_k)=0 \ \forall k$
\end{theorem}
\begin{proof}
  \begin{gather*}
    \Vert S_{n+p}-S_n \Vert^{2}=(\sum_{k=n+1}^{n+p}\alpha_ke_k,\sum_{k=n+1}^{n+p}\alpha_ke_k) = \sum_{k=n+1}^{n+p}\alpha_k^{2}\Vert e_k \Vert^{2} \tag{2}
  \end{gather*}
  Из неравенства Бесселя следует сходимость $\sum_{k=1}^{\infty}\alpha_k^{2}\Vert e_k \Vert^{2}$

  $\implies$ фундаментальна последовательность $\sum_{k=1}^{n}\alpha_k^{2}\Vert e_k \Vert^{2}$ 

  Из $(2) \ \implies$ фундаментальность $\{S_n\}$

  Из $X_\Gamma$ - полно $\implies$ $\{S_n\}$ сходится к $x_0 \in X_\Gamma$

  Рассмотрим $(x-x_0,e_k)=(x,e_k)-(x_0,e_k)=(x,e_k)-\sum_{n=1}^{\infty}\alpha_n(e_n,e_k)
  =(x,e_k)-\alpha_k(e_k,e_k)=(x,e_k)-\alpha_k\Vert e_k \Vert^{2}=0$
\end{proof}
\begin{theorem}[Т5]
  Ортогональная система $\{e_k\}$ является полной в $X_\Pi$ $\iff$
  $\forall x \in X_\Pi$ ряд Фурье сход. к $x$.
\end{theorem}
\begin{proof}
  1) Возьмём $x\in X_\Pi$. Пусть ряд Фурье сходится к $x$

  $\implies \lim_{n\to\infty}\Vert x-S_n(x) \Vert=0$ и $\forall \epsilon > 0 \exists n: \Vert x-S_n \Vert<\epsilon$,
  что означает полноту.

2) Пусть система $\{e_k\}$ полна.

  $\implies$ $\forall x \in X_\Pi \ \underline{ \forall \epsilon > 0 \ \exists n } \ \exists \lambda_1,\dots ,\lambda_n:
  \Vert x-(\lambda_1e_1+\dots +\lambda_ne_n) \Vert < \epsilon$

  Минимальное св-во коэф. Фурье $\implies \Vert  x - S_n \Vert < \epsilon$

  Следствие из Т2 $\implies$ $\underline{\Vert x-S_m \Vert < \epsilon \ \forall m \ge n}$

  Подчёркнутое означает что ряд Фурье сходится к $x$.
\end{proof}
\begin{theorem}[Т6]
  Ряд Фурье элемента $x \in X_\Pi$ сходится к нему $\iff$
  для него полняется равенство Парсеваля $\Vert x \Vert^{2}=\sum_{k=1}^{\infty}\alpha_k^{2}\Vert e_k \Vert^{2}$
\end{theorem}
\begin{proof}
  Формула $(1)$ из минимального св-ва коэф Фурье
  \[
    \Vert x-S_n \Vert^{2}=\Vert x \Vert^{2}-\sum_{k=1}^{n}\alpha_k^{2}\Vert e_k \Vert^{2}
  \]

  Переходя к пределу при $n\to\infty$ получаем нужный результат.
\end{proof}
\begin{corollary}
  Ортогональная система $\{e_k\}$ полна в $X_\Pi$ 

  $\iff$ $\forall $ элем. в $X_\Pi$ выполняется равенство Парсеваля 

  $\iff$ Ряд Фурье $\forall x \in X_\Pi$ сходится к $x$
\end{corollary}
\begin{theorem}[Т7]
  Если ортогональная система $\{e_k\}$ из $X_\Pi$ полная и коэф. Фурье для $x$ равны нулю,
  то $x = 0$.
\end{theorem}
\begin{proof}
  Из равентсва Парсеваля следует, что $\Vert x \Vert=0 \ \implies \ x=0$
\end{proof}
\begin{corollary}
  Если коэф. Фурье элем. $x_1$ и $x_2 \in X_\Pi$ по полной ортог. сист $\{e_k\}$
  равны между собой, то эти элементы равны.
\end{corollary}
\begin{proof}
Для элемента $x=x_2-x_1$ коэф. Фурье $\frac{(x,e_k)}{\Vert e_k \Vert^{2}}=\frac{(x_2-x_1,e_k)}{\Vert e_k \Vert^{2}}$
будут нулевыми. Тогда $x_2-x_1=0$.
\end{proof}
\begin{definition}[Замкнутость системы]
  Ортогональная система $\{e_k\}$ в $X_\Pi$ называется замкнутой, если в $X_\Pi \ \not \exists x\neq 0: (x,e_k)=0 \ \forall k$.
\end{definition}
\begin{theorem}
  В $X_\Gamma$ ортогональная система $\{e_k\}$ полна 
  $\iff$ $\{e_k\}$ - замкнута
\end{theorem}
\begin{proof}
  1) Пусть $\{e_k\}$ - полна. Пусть $\exists x \neq 0: (x,e_k)=0\ \forall k$

  $\implies$ коэф. Фурье - нулевые, по Т7 $\implies \ x= 0 \ \implies$ система замкнута
  2) Пусть система $\{e_k\}$ - замкнута. Пусть $x\in X_\Gamma$, $x \sim \sum_{k=1}^{\infty}\alpha_ke_k$ 
  и $x_0 = \sum_{ k=1}^{\infty}\alpha_ke_k$

  По Т4 $(x-x_0,e_k)=0 \ \forall k \ \implies x-x_0=0$ (т.к. замкнута)

  $\implies x = x_0 = \sum_{k=1}^{n}\alpha_ke_k$ и по Т5 система полна.
\end{proof}
\begin{definition}
  Пусть $X$ - нормированное пространство. Счётная система $\{e_k\}$ называется базисом в $X$,
  если $\forall x \in x \ \exists !$ представление $x=\sum_{k=1}^{\infty}\lambda_je_j$
\end{definition}
\begin{remark}
  Базис - полная система, но не всякая полная система - базис.
\end{remark}
\begin{eg}
  Система $\{1,x,x^{2},\dots ,x^{k},\dots \}$ не является базисом в $C[-1,1]$,
  т.к. ряд $\sum_{k=1}^{\infty}\lambda_kx^{k}$, сходящийся на этом отрезке
  - диф. функция не существует для $|x|$ (не диф. функция).
\end{eg}
\begin{theorem}[Т9]
  Пусть $\{e_k\}$ - ортог. сист. в $X_\Pi$. Если $\{e_k\}$ - полн. то она базис.
\end{theorem}
\begin{proof}
  Если $\{e_k\}$ - полная ортог. сист., то $\forall x \in X_\Pi \true x = \sum_{k=1}^{\infty}\alpha_ke_k$ по Т%.

  Единственность следует из Т1.
\end{proof}
\begin{eg}
  Полные ортогональные системы:
  \begin{enumerate}
    \item Тригонометрическая система в $L_2[-\pi,\pi]$
    \item Система полиномов Лежандра в $L_2[-1,1]$
  \end{enumerate}
\end{eg}

\section{Собственные интегралы, зависящие от параметра}
\begin{theorem}[Т1]
  Пусть $f(x,y)$ - непр. в $[a,b]\times [c,d]$

  $\implies \ I(y)=\int_{a}^{b}f(x,y)dx$ - непр. на $[c,d]$ (здесь y - параметр)
\end{theorem}
\begin{proof}
  $f$ - непр. на компакте $[a,b]\times [c,d]$, значит $f$ - равномерно непр. на нём 

  $\implies \ \omega(\delta) \underset{\delta\to\infty}{\to}0$ (модуль. непр.)

  $\implies \ |I(y+\Delta y)-I(y)| \le \int_{a}^{b}|f(x,y+\Delta y)-f(x,y)|dx \le 
  (b-a)\omega(\Delta y) \underset{\Delta y \to 0}{\to} 0$
  $\implies \ I(y)$ - непр. на $[c,d]$.
\end{proof}
\incfig{l10_t2}
\begin{theorem}[Т2]
  Пусть $\phi(y)$ и $\psi(y)$ - непр. на $[c,d]$. $\phi(y) \le \psi(y)$ на $[c,d]$

  $E=\{(x,y)\}: \phi(y)\le x \le \psi(y), \, y\in [c,d]$, f(x,y) - непр. на $E$

  $\implies \ I(y)=\int_{\phi(y)}^{\psi(y)}f(x,y)dx$ - непр. на $[c,d]$
\end{theorem}
\begin{proof}
  Замена пременной $x=\phi(y)+t(\psi(y)-\phi(y))$

  $I(y)=\int_{0}^{1}f(\phi(y)+t(\psi(y)-\phi(y)), y)(\psi(y)-\phi(y))dy$ - непр. на $[0,1]\times [c,d]$ как суперпоз. непр.

  По Т1 $I(y)$ - непр. функ. на $[c,d]$.
\end{proof}
\begin{theorem}[Т3 Об интегрировании под знаком интеграла]
  \phantom{.}

  \begin{enumerate}
  \item $f(x,y)$ - инт. по Риману на $[a,b]\times [c,d]$
  \item $\int_{a}^{b}f(x,y)dx$ существувет $\forall y \in [c,d]$
  \item $\int_{c}^{d}f(x,y)dy$ существувет $\forall x \in [a,b]$
  \end{enumerate}
  $\implies$ $\int_{c}^{d}(\int_{a}^{b}f(x,y)dx)dy=\int_{a}^{b}(\int_{c}^{d}f(x,y)dy)dx$
\end{theorem}
\begin{proof}
  Утверждение следует из равенства кратного интегрела $\iint_{[a,b]\times [c,d]}f(x,y)dxdy$ двум повторным
\end{proof}
\begin{remark}
  Если $f(x,y)$ непр. на $[a,b]\times [c,d]$, то верно заключение теоремы.
\end{remark}
\begin{theorem}[Т4 О дифференцировании интеграла]
  \phantom{.}

  \begin{enumerate}
    \item $f(x,y), f_y'(x,y)$ - непр. в $[a,b]\times [c,d]$
    \item $E:\{(x,y): \phi(y) \le x \le \psi(y), y\in [c,d]\}\subset [a,b]\times [c,d]$
    \item $\exists \phi'(y),\psi'(y)$ на $[c,d]$
  \end{enumerate}
  Тогда существует

  $\left(\int_{\phi(y)}^{\psi(y)}f(x,y)dx\right)'=\psi'(y)f(\psi(y),y)-\phi'(y)f(\phi(y),y)+\int_{\phi(y)}^{\psi(y)}f_y'(x,y)dx$
\end{theorem}
\begin{proof}
  Пусть $y,y_0\in[c,d]$. Обозначим $I(y)=\int_{\phi(y)}^{\psi(y)}d(x,y)dx$
  \begin{gather*}
    I(y)-I(y_0)=(\int_{\phi(y)}^{\phi(y_0)}+\int_{\phi(y_0)}^{\psi(y_0)}+\int_{\psi(y_0)}^{\psi(y)})f(x,y)dx - \int_{\phi(y_0)}^{\psi(y_0)}f(x,y_0)dx = \\ 
    = \int_{\phi(y)}^{\phi(y_0)}f(x,y)dx+\int_{\psi(y_0)}^{\psi(y)}f(x,y)dx+\int_{\phi(y_0)}^{\psi(y_0)}(f(x,y)-f(x,y_0))dx \\ 
    \frac{1}{y-y_0}\int_{\phi(y)}^{\phi(y_0)}f(x,y)dx= \\ 
    = \frac{1}{y-y_0}(\phi(y_0)-\phi(y))f(\tilde{x},y) \underset{y\to y_0}{\to} -\phi'(y_0)f(\phi(y_0), y_0)
  \end{gather*}
  (Использовали теорему о среднем, $\tilde{x} \in (\phi(y), \phi(y_0)$)

  Аналогично для второго интеграла.
  \begin{gather*}
    \frac{1}{y-y_0}\int_{\phi(y_0)}^{\psi(y_0)}(f(x,y)-f(x,y_0))dx= \\ 
     = \frac{1}{y-y_0}\int_{\phi(y_0)}^{\psi(y_0)}f_y'(x,y_0+\theta(y-y_0))(y-y_0)dx = \\ 
     = \int_{\phi(y_0)}^{\psi(y_0)}f_y'(x,y_0+\theta(y-y_0))dx \underset{y\to y_0}{\to}\int_{\phi(y_0)}^{\psi(y_0)}f_y'(x,y_0)dx
  \end{gather*}
  Воспользовались Т1.
\end{proof}

\section{Несобственные интегралы, зависящие от параметра}
Будем рассматривать несобственные интегралы
\[
  I(y)=\int_{a}^{b}f(x,y)dx \quad-\infty < a < b \le +\infty, \ y\in Y \tag{1}
\]
с единственной особенностью на верхнем пределе.

Это предполагает, что для любых $\eta \in (a,b)$ и $\forall y \in Y$ 
существует Риманов интеграл $\int_{a}^{\eta}f(x,y)dx$.

По умолчанию считаем, что условие выполняется.

В случае сходимости интеграла:
\[
  I(y)=\lim_{\eta \to b-0}\int_{a}^{\eta}f(x,y)dx \qquad
  \lim_{\eta\to b-0}\int_{\eta}^{b}f(x,y)dx=0
\]
\begin{definition}
  Сходящийся на $Y$ интеграл называется равномерно сходящимся на $Y$, если
  \[
    \forall \epsilon > 0 \ \exists \eta_{\epsilon}\in(a,b): \ \forall \eta \in (\eta_\epsilon,b); \ \forall y \in Y 
    \true |\int_{\eta}^{b}f(x,y)dx|<\epsilon
  \]
\end{definition}
\begin{definition}[Перефразирование]
  Сходящийся на $Y$ называется равномерно сходящимся, если 
  \[
    \lim_{\eta\to b-0}\sup_{y\in Y}|\int_{\eta}^{b}f(x,y)dx|=0
  \]
\end{definition}
\begin{theorem}[Критерий Коши равномерной сходимости несобст. инт.]
  Интеграл $(1)$ сходится равномерно на $Y$ 
  $\iff$ выполняется условие Коши равномерной сходимости интеграла
  \[
    \forall \epsilon>0 \ \exists \eta_\epsilon \in (a,b): \ \forall \eta',\eta'' \in (\eta,b), \ \forall y \in Y 
    \true |\int_{\eta'}^{\eta''}f(x,y)dx| < \epsilon
  \]
\end{theorem}
\begin{proof}
  \phantom{.}

  1) Пусть интеграл сходится равномерно, т.е.
  \begin{gather*}
    \underline{\forall \epsilon > 0 \ \exists \eta_{\epsilon}\in(a,b)}: \ \forall \eta \in (\eta_\epsilon,b); \ \forall y \in Y 
    \true |\int_{\eta}^{b}f(x,y)dx|<\frac{\epsilon}{2}
  \end{gather*}
  Возьмём  $\underline{\forall \eta',\eta'' \in (\eta_\epsilon,b), \ \forall y \in Y}$

  Тогда
  \begin{gather*}
    \underline{|\int_{\eta'}^{\eta''}f(x,y)dx|}=|(\int_{\eta'}^{b}-\int_{\eta''}^{b})f(x,y)dx|  \le \\
    \le |\int_{\eta'}^{b}f(x,y)dx| + |\int_{\eta''}^{b}f(x,y)dx|
    \underline{< \frac{\epsilon}{2}+\frac{\epsilon}{2}=\epsilon}
  \end{gather*}
  2) Пусть выполняется условие Коши равномерной сходимости, т.е.
  \begin{gather*}
    \underline{\forall \epsilon>0 \ \exists \eta_\epsilon \in (a,b): \ \forall \eta'},\eta'' \underline{\in (\eta,b), \ \forall y \in Y}
    \true |\int_{\eta'}^{\eta''}f(x,y)dx| < \frac{\epsilon}{2}
  \end{gather*}
  Перейдём к пределу при $\eta''\to b-0$ (предел существует при $\forall y \in Y$, т.к 
  из выполнения условия Коши равномерной сходимости следует условия Коши сходимости 
  и, следовательно, сходимость)
  \begin{gather*}
    \underline{|\int_{\eta'}^{b}f(x,y)dx|\le \frac{\epsilon}{2} < \epsilon}
  \end{gather*}
  Подчёркнутое означает сходимость интеграла.
\end{proof}
\begin{theorem}[Т2 Признак Вейерштрасса равн. сход. несобст. инт.]
  Пусть $\exists \phi(x): \ \phi(x)$ - инт. по Риману на $[a,\eta] \ \forall \eta \in (a,b)$
  $|f(x,y)| \le \phi(x) \ \forall x \in [a,b), \ \forall y \in Y$,
  $\int_{a}^{b}\phi(x)dx$ - сходится

  $\implies$ $(1)$ сходится равномерно на $Y$.
\end{theorem}
\begin{proof}
  Возьмём $\underline{\forall \epsilon >0}$. Из сходимости $\int_{a}^{b}\phi(x)dx$ следует (кр. Коши),
  что $\underline{\exists \eta_\epsilon \in (a,b): \ \forall \eta',\eta'' \in (\eta,b)} 
    \true |\int_{\eta'}^{\eta''}f(x,y)dx| < \epsilon$
  
    Но $\underline{\forall y \in Y \true |\int_{\eta'}^{\eta''}f(x,y)dx|} \le |\int_{\eta'}^{\eta''}|f(x,y)|dx|
    \le |\int_{\eta'}^{\eta''}\phi(x)dx|\underline{<\epsilon}$

    Подчёркнутое означает равномерную сходимость интеграла $(1)$.
\end{proof}

% lecture 11

\begin{definition}[Равномерная сходимость функции]
  $f(x,y)\rightrightarrows \phi(y)$ при $x\to x_0$,
  если $\lim_{x\to x_0}\sup_{y\in E}|f(x,y)-\phi(y)|=0$
\end{definition}
\begin{theorem}[Признак Дирихле равномерной сход. инт.]
  Пусть
  \begin{enumerate}
    \item $f(x,y)$ непр. на $[a,+\infty)\times Y$ по $x$
      и $\exists M>0: \ |\int_{a}^{\eta}f(x,y)dx|<M \ \forall \eta>a, \ \forall y \in Y$
      (интеграл $\int_{a}^{\eta}$ - равномерно ограничен)
    \item $\pd{g}{x}$ - непр. по $x$ на $[a,+\infty)\times Y$

      $g(x,y)$ - монот. по $x$ на $[a,+\infty)\times Y$
    \item $g(x,y) \underset{y\in Y}{\rightrightarrows} 0$ при $x \to +\infty$ 
  \end{enumerate}
  $\implies \ \int_{a}^{+\infty}f(x,y)g(x,y)dx$ сх. равн. на $Y$
\end{theorem}
\begin{proof}
  Возьмём $\underline{\forall \epsilon > 0}$. Т.к. $g(x,y) \underset{y\in Y}{\rightrightarrows}0$ при $x\to \infty$,
  то $\underline{\exists \eta_\epsilon >a} : \forall \eta>\eta_\epsilon \ \forall y \in Y \true |g(\eta,y)| < \frac{\epsilon}{6M}$

  Возьмём $\underline {\forall \eta', \eta'' > \eta_\epsilon, \ \forall y \in Y}$.
  Тогда
  \begin{gather*}
    \underline{|\int_{\eta'}^{\eta''}f(x,y)g(x,y)dx|}=\int_{\eta'}^{\eta''}g(x,y)d\int_{\eta'}^{x}f(\xi,y)d\xi|= \\ 
    = |g(x,y)\int_{\eta'}^{x}f(\xi,y)d\xi\big|_{\eta'}^{\eta''}-\int_{\eta'}^{\eta''}(\int_{\eta'}^{x}f(\xi,y)d\xi)g_x'(x,y)dx| \le \\ 
    (\text{т.к.} \int_{\eta'}^{x}=\int_{a}^{x}-\int_{a}^{\eta'}) \quad \le |g(\eta'',y)|\cdot 2M + 2M|\int_{\eta'}^{\eta''}|g_x'(x,y)|dx| = \\ 
    =|g(\eta'',y)2M+2M|\int_{\eta'}^{\eta''}g_x'(x,y)dx \le 2M (2|g(\eta'',y)|+|g(\eta')|) < \\ 
    \underline{< 2M(2\frac{\epsilon}{6M}+\frac{\epsilon}{6M}) = \epsilon}
  \end{gather*}
  Подчёркнутое означает равномерную сходимость интеграла.
\end{proof}
\begin{theorem}[Признак Абеля (вне программы)]
  Пусть
  \begin{enumerate}
    \item $f(x,y)$ - непр. по $x$ на $[a,+\infty)\times Y$, $\int_{a}^{+\infty}f(x,y)dx$ сх-ся равн. на $Y$
    \item $\pd{g}{x}$ - непр. по $x$ на $[a,+\infty)\times Y$, $g(x,y)$ монот. по $x$ на $[a,+\infty)\times Y$

      $g(x,y)$ - равн. огр. (т.е. огр. на совокупн. пер. $x$ и $y$ на $[a,+\infty)\times Y$)
  \end{enumerate}
  $\implies$ $\ \int_{a}^{\infty}fgdx$ - равн. сход. на $Y$
\end{theorem}

\subsection{Свойства равномерно сходящихся интегралов}
\begin{theorem}[О непрерывности]
  Пусть $f(x,y)$ непр. на $[a,b)\times [c,d]$

  $I(y)=\int_{a}^{b}f(x,y)dx$ - равн. сход. на $[c,d]$

  $\implies \ I(y)$ - непр. на $[c,d]$.
\end{theorem}
\begin{proof}
  Возьмём $\underline{\forall y_0 \in [c,d]}$. Возьмём $\underline{\forall \epsilon>0}$.
  Т.к. $\int_{a}^{b}f(x,y)dx$ - сх. равн. на $[c,d]$, то
  $\exists \eta \in (a,b): \ \forall y \in [c,d] \true |\int_{\eta}^{b}f(x,y)dx|<\frac{\epsilon}{3}$

  Теперь $\int_{a}^{\eta}f(x,y)dx$ - собст. интеграл и непр. на $[c,d]$.

  Тогда $\underline{\exists \delta >0: \ \forall y \in U_\delta(y_0)} \true |\int_{a}^{\eta}f(x,y)dx - \int_{a}^{\eta}f(x,y_0)dx|<\frac{\epsilon}{3}$

  Оценим
  \begin{gather*}
    \underline{|I(y) - I(y_0)|}\le |\int_{a}^{\eta}f(x,y)dx-\int_{a}^{\eta}f(x,y_0)dx| + \\ 
    + |\int_{\eta}^{b}f(x,y)dx|+|\int_{\eta}^{b}f(x,y_0)dx|\underline{< \frac{\epsilon}{3}\cdot 3 = \epsilon}
  \end{gather*}
\end{proof}
\begin{theorem}[Об интегрировании несобственного интеграла]
  В условиях предыдущей теоремы:
  \begin{gather*}
    \int_{c}^{d}I(y)dy=\int_{c}^{d}(\int_{a}^{b}f(x,y)dx)dy=\int_{a}^{b}(\int_{c}^{d}f(x,y)dy)dx
  \end{gather*}
\end{theorem}
\begin{proof}
  Т.к. $f(x,y)$ - непр., то
  \[
    \int_{c}^{d}(\int_{a}^{\eta}f(x,y)dx)dy=\int_{a}^{\eta}(\int_{c}^{d}f(x,y)dy)dx \tag{1}
  \]
  Но
  \begin{gather*}
    |\int_{c}^{d}(\int_{a}^{\eta}f(x,y)dx)dy-\int_{c}^{d}(\int_{a}^{b}f(x,y)dx)dy| \le \\ 
    \le \int_{c}^{d}|\int_{\eta}^{b}f(x,y)dx|dy \le (d-c)\sup_{y\in [c,d]}|\int_{\eta}^{b}f(x,y)dx|\underset{\eta\to b-0}{\to}0
  \end{gather*}
  Т.к. инт. $I(y)$ сход. равномерно.

  Доказали, что $\lim_{\eta\to b-0}\int_{c}^{d}(\int_{a}^{\eta}f(x,y)dx)dy=\int_{c}^{d}(\int_{a}^{b}fdx)dy$

  Переходя в (1) к пределу при $\eta\to b-0$ получаем нужный результат.
\end{proof}
\begin{theorem}[О дифф. собственного интеграла]
  \phantom{.}

  Пусть $f(x,y), \pd{f}{y}$ - непр. на $[a,b)\times [c,d]$.

  Пусть $\exists y_0 \in [c,d]: \int_{a}^{b}f(x,y_0)dx$ - сход.

  Пусть $\int_{a}^{b}\pd{f}{y}dx$ - сход. равномерно на $[c,d]$

  $\implies \ \frac{d}{dy}\int_{a}^{b}f(x,y)dx=\int_{a}^{b}\pd{f(x,y)}{y}dx$
\end{theorem}
\begin{proof}
  По предыдущей теореме
  \begin{gather*}
    \int_{y_0}^{y}(\int_{a}^{b}f_y'(x,t)dx)dt=\overbrace{\int_{a}^{b}(f(x,y)-f(x,y_0))dx}^{(1)} = \\ 
    =\underbrace{\int_{a}^{b}f(x,y)dx}_{(3)} - \underbrace{\int_{a}^{b}f(x,y_0)dx}_{(2)}
  \end{gather*}
  (3) сходится т.к. сходятся (1) и (2).
  
  Дифф. полученное равенство
  \[
    \int_{a}^{b}f_y'(x,y)dx=\frac{d}{dy}\int_{a}^{b}f(x,y)dx
  \]
\end{proof}
\section{Эйлеровы интегралы}
\begin{definition}
  Гамма-функция Эйлера
  \[
    \Gamma(s)=\int_{0}^{+\infty}x^{s-1}e^{-x}dx, \ s>0
  \]
\end{definition}
Св-ва:
\begin{enumerate}
  \item Инт. сходится при $s>0$ и расх. при $s\le0$

    \begin{proof}
      \phantom{.}

      $x^{s-1}e^{-x} \underset{x\to 0}{\sim}x^{s-1} \ \implies \ \int_{0}^{1}x^{s-1}e^{-x}dx$ - сх. $s>0$, расх. $s\le0$

    $x^{s-1}e^{-x}=o(e^{-\frac{x}{2}}) \ \implies \ \int_{1}^{+\infty}x^{s-1}e^{-x}dx$ - сходится $\forall s$
    \end{proof}
  \item $\Gamma(s)$ - непр. при $s>0$

    \begin{proof}
      \phantom{.}

      \begin{enumerate}
        \item $x^{s-1}e^{-x}\le x ^{s_0-1}$, где $x\in (0,1]$, $s_1\ge s \ge s_0> 0$

          $\implies \ I_1(s)=\int_{0}^{1}x^{s-1}e^{-x}dx$ - сх. равн. на $[s_0,s_1]$

          $\implies \ I_1(s)$ - непр. на $[s_0,s_1]$ $\implies$ $I_1(s)$ - непр. на $(0,+\infty)$
        \item $x^{s-1}e^{-x}\le x^{s_1-1}e^{-x}$, где $x\in[1,+\infty)$, $0<s_0\le s \le s_1$

          $\implies \ I_2(s)=\int_{0}^{1}x^{s-1}e^{-x}dx$ - сх. равн. на $[s_0,s_1]$

          $\implies \ I_2(s)$ - непр. на $[s_0,s_1]$ $\implies$ $I_2(s)$ - непр. на $(0,+\infty)$
      \end{enumerate}
      $\implies \ \Gamma(s)$ - непр. на $(0,+\infty)$
    \end{proof}
% lecture 12 
  \item $\Gamma(s+1)=s\Gamma(s)$
    \begin{proof}
      \begin{gather*}
        \Gamma(s+1)=\int_{0}^{+\infty}x^{s}e^{-x}dx= -\int_{0}^{+\infty}x^{s}de^{-x} = \\
        =\underbrace{-x^{s}e^{-x}\big|_{0}^{\infty}}_{0}+\int_{0}^{+\infty}sx^{s-1}e^{-x}dx=s\Gamma(s)
      \end{gather*}
    \end{proof}
    ($\Gamma(1)=\Gamma(2)=1, \Gamma(3)=2\dots \Gamma(n+1)=n!$)
\end{enumerate}
\begin{definition}
  Бета-функция Эйлера:
  \[
    B(p,q)=\int_{0}^{1}x^{p-1}(1-x)^{q-1}dx
  \]
\end{definition}
Св-ва:
\begin{enumerate}
  \item Интеграл сходится при $p>0,q>0$
    \begin{proof}
      \phantom{.}

      $x^{p-1}(1-x)^{p-1}\underset{x\to +0}{\sim} x^{p-1} \ \implies \ \int_{0}^{1/2}x^{p-1}(1-x)^{q-1}dx$ - сх. при $p>0$

      $x^{p-1}(1-x)^{p-1}\underset{x\to 1-0}{\sim} (1-x)^{q-1} \ \implies \ \int_{1/2}^{1}x^{p-1}(1-x)^{q-1}dx$ - сх. при $q>0$
    \end{proof}
  \item $B(p,q)=B(q,p)$
    \begin{proof}
      \phantom{.}

      Замена переменных $1-x=t$, $dx=-dt$

      $B(p,q)=\int_{0}^{1}x^{p-1}(1-x)^{q-1}dx=-\int_{1}^{0}(1-t)^{p-1}t^{q-1}dt=\int_{0}^{1}t^{q-1}(1-t)^{p-1}dt=B(q,p)$
    \end{proof}
  \item $B(p,q)$ - непр. на $(0,+\infty)\times(0,+\infty)$
    \begin{proof}
      \phantom{.}

      Рассмотрим $[p_0,+\infty)\times[q_0,+\infty)$, где $p_0>0,q_0>0$

      Верно $x^{p-1}(1-x)^{q-1} \le x^{p_0-1}(1-x)^{q_0-1}$ при $x \in (0,1)$, $p\ge p_0, q \ge q_0$,
      при этом $\int_{0}^{1}x^{p_0-1}(1-x)^{q_0-1}dx$ - сход.

      $\implies B(p,q)$ - сх. равн. на $[p_0,+\infty)\times [q_0,+\infty)$ (пр. Вейер.)

      $B(p,q)$ - непр. на $[p_0,+\infty)\times [q_0,+\infty)$
      
      $\implies \ B(p,q)$ - непр. на $(0,+\infty)\times(0,+\infty)$
    \end{proof}
    \begin{remark}
      Здесь применили теоремы для инт., завис. от двух параметров
    \end{remark}
  \item $B(p,q)=\frac{\Gamma(p)\Gamma(q)}{\Gamma(p+q)}$ (без док.)
  \item Формула дополнения $B(a,1-a)=\frac{\pi}{\sin a\pi}$ $(a\in(0,1)$ (без док.)
  
\end{enumerate}
\section{Интеграл Дирихле}
\begin{lemma}
  \[
    \int_{0}^{+\infty}\frac{\sin \alpha x}{x}dx=\frac{\pi}{2} sign(\alpha)
  \]
\end{lemma}
\begin{proof}
  (Через ядро Дирихле)
  \begin{gather*}
    \frac{1}{2}+\sum_{k=1}^{n}\cos kx = \frac{\sin(n+\frac{1}{2}x)}{2\sin \frac{x}{2}} \qquad \Big|\int_{0}^{\pi}\dots dx \\ 
    \frac{\pi}{2}=\int_{0}^{\pi}\frac{\sin(n+\frac{1}{2}x)}{2\sin \frac{x}{2}} = \\
    = \underbrace{\int_{0}^{\pi}\overbrace{\left(\frac{1}{2\sin \frac{x}{2}}-\frac{1}{x}\right)}^{(1)}\sin(n+\frac{1}{2})xdx}_{(2)} + \int_{0}^{\pi}\frac{\sin (n+\frac{1}{2})x}{x}dx
  \end{gather*}
  (1) имеет предел при $x\to 0$ (расскладываем по Тейлору) $\implies$ можно доопределить до непр. на $[0,\pi]$ 

  Тогда интеграл (2) при $n\to +\infty$ стремится к 0 (теорема Римана)

  Переходим в последнем равенстве к пределеу при $n\to+\infty$ (замена $(n+\frac{1}{2})x=t$)
  \begin{gather*}
    \frac{\pi}{2}=\lim_{n\to \infty}\int_{0}^{\pi}\frac{\sin(n+\frac{1}{2})x}{x}dx=\lim_{n\to\infty}\int_{0}^{(n+\frac{1}{2})\pi}\frac{\sin t}{t}dt
  \end{gather*}
  Но $\int_{0}^{\infty}\frac{\sin t}{t}dt$ - сход. $\implies \ \int_{0}^{+\infty}\frac{\sin t}{t}dt=\frac{\pi}{2}$

  Из нечётности и постоянства незименности знака $\alpha$ $\implies \ \int_{0}^{+\infty}\frac{\sin \alpha x}{x}dx = \frac{\pi}{2}sign(\alpha)$
\end{proof}

\section{Интеграл Фурье}
\begin{lemma}
  Пусть $f(x)$ - абс. инт. на $(a,b)$ (конечном или беск.), $\phi(x,y)$ - непр. и огран.
  на $(a,b)\times [c,d]$. Тогда
  \begin{enumerate}
    \item $I(y)=\int_{a}^{b}f(x)\phi(x,y)dx$ - непр. на $[c,d]$
    \item $\int_{c}^{d}(\int_{a}^{b}f(x)\phi(x,y)dx)dy=\int_{a}^{b}(\int_{c}^{d}f(x)\phi(x,y)dy)dx$
  \end{enumerate}
\end{lemma}
\incfig{l12_l1}
\begin{proof}(Для 2-х особ. на концах)

  1) Пусть $\phi(x,y)\le M$ на $(a,b)\times [c,d]$

  Возьмём $\underline{\forall \epsilon > 0}$. Т.к. $f(x)$ - абс. инт., то
  $\exists \xi, \eta \in (a,b): \ \int_{a}^{\xi}|f(x)|dx < k \epsilon$,
  $\int_{\eta}^{b}|f(x)|dx < k \epsilon$

  Рассмотрим $\Delta I =I(y+\Delta y) - I(y) = (\int_{a}^{\xi}+\int_{\xi}^{\eta}+\int_{\eta}^{b})f(x)(\phi(x,y+\Delta y) - \phi(x,y))dx$

  Оценка $|\Delta I| < 2Mk\epsilon + \omega(\Delta y, \phi, \Pi) \int_{a}^{b}|f(x)|dx + 2Mk\epsilon$ ($\Pi = [\xi,\eta]\times [c,d]$)

  Т.к. $\phi(x,y)$ - непр. на $\Pi$, то $\phi(x,y)$ - равн. непр. на $\Pi$

  $\underline{\exists \delta >0}: \ \forall \Delta y < \delta \true |\omega(\Delta y, \phi, \Pi)| < k\epsilon$
  
  Тогда $|\Delta I|< 4Mk\epsilon + \int_{a}^{b}|f(x)|dx k\epsilon = k\epsilon(4M+\int_{a}^{b}|f(x)|dx)$
  
  Если $k=\frac{1}{4M + \int_{a}^{b}|f(x)|dx}$, то получим $\underline{|\Delta I| < \epsilon}$

  Подчёркнутое означает непрерывность $I(y)$.

  2) Доказана теорема о том, что любая абс. инт. функция $f(x)$ может быть приближена на $[a,b]$
  ступенчатой функцией $\psi(x)$ по норме $\Vert \cdot \Vert_{1}$ (интеграл от модуля) сколь 
  угодно точно. Но ступенчатая функция может быть приближена непрерывной $f_{\epsilon}(x)$ по норме
  $\Vert \cdot \Vert_{1}$, сколь угодно точно, если "исправить" ступеньку на трапецию.
  ($\_|\bar{}|\_ \rightarrow \_/\bar{}\backslash \_$)

  Таким образом $\forall \epsilon > 0 \ \exists f_\epsilon(x): \int_{a}^{b}|f_\epsilon(x)-f(x)|dx<\epsilon$

  При этом $f_{\epsilon}(x)$ - непр. и равна нулю вне некоторого $[\alpha,\beta]\subset(a,b)$

  Тогда $f_\epsilon(x)\phi(x,y)$ - непр. на $[\alpha,\beta]\times [c,d] \subset (a,b)\times [c,d]$ и
  \[
    \int_{c}^{d}(\int_{a}^{b}f_\epsilon(x)\phi(x,y)dx)dy=\int_{a}^{b}f_\epsilon(x)(\int_{c}^{d}\phi(x,y)dy)dx
  \]
  Покажем, что перейдя к пределу при $\epsilon \to 0$ в последнем равенстве получим
  утверждение леммы.
  Действительно
  \[
    \int_{c}^{d}(\int_{a}^{b}f_{\epsilon}(x)\phi(x,y)dx)dy \underset{\epsilon\to 0}{\to}\int_{c}^{d}(\int_{a}^{b}f(x)\phi(x,y)dx)dy
  \]
  т.к. $|\int_{c}^{d}(\int_{a}^{b}(f(x)-f_\epsilon(x))\phi(x,y)dx)dy| \le M(d-c)\int_{a}^{b}|f(x)-f_\epsilon(x)|dx \le M(d-c)\epsilon \underset{\epsilon\to 0}{\to}0$
  
  Аналогично $\int_{a}^{b}(f_\epsilon(x)\int_{c}^{d}\phi(x,y)dy)dx \underset{\epsilon\to 0}{\to}\int_{a}^{b}(\int_{c}^{d}f(x)\phi(x,y)dy)dx$
\end{proof}

% lecture 13

\begin{definition}
  Пусть $f(x)$ - абс. инт. на $(-\infty,+\infty)$. Тогда:
  \begin{gather*}
    S(x)=S(x,f)=\int_{0}^{+\infty}(a(y)\cos xy+ b(y) \sin xy)dy \\ 
    a(y)=\frac{1}{\pi}\int_{-\infty}^{+\infty}f(t)\cos ty dt \qquad
    b(y)=\frac{1}{\pi}\int_{-\infty}^{+\infty}f(t)\sin ty dt
  \end{gather*}
  называется интегралом Фурье функции $f(x)$
\end{definition}
\begin{remark}
  Интеграл Фурье - аналог ряда Фурье для функции определённой на $(-\infty,+\infty)$,
  $a(y),b(y)$ - аналоги коэф. Фурье
\end{remark}
\begin{lemma}
  Пусть $f(x)$ - абс. инт. $(-\infty,+\infty)$ - непр. на $(-\infty,+\infty)$. Тогда
  \begin{enumerate}
    \item $a(y),b(y)$ - непр. на $(-\infty,+\infty$
    \item $\lim_{y\to \pm\infty} a(y)=\lim_{y\to \pm\infty} b(y) = 0$
  \end{enumerate}
\end{lemma}
\begin{proof}
  \phantom{.}

  \begin{enumerate}
    \item Следует из леммы 1 
    \item Следует из Теоремы Римана
  \end{enumerate}
\end{proof}
\begin{theorem}[О сходимости интеграла Фурье к функции]
  Пусть $f(x)$ - абс. инт. на $(-\infty,+\infty)$.

  Пусть в точке $x_0 \ \exists f(x_0\pm 0), f_{\pm}'(x_0)$

  $\implies \ S(x_0)=\frac{f(x_0+0)+f(x_0-0)}{2}$
\end{theorem}
\begin{proof}
  Преобразуем инт. Фурье (подставляем $a$ и $b$):
  \[
    S(x)=\frac{1}{\pi}\int_{0}^{+\infty}\left(\int_{-\infty}^{+\infty}f(y)\cos y (x-t)dt\right)dy
  \]
  Рассмотрим $S(x)=\frac{1}{\pi}\int_{0}^{\eta}\left(\int_{-\infty}^{+\infty}f(y)\cos y (x-t)dt\right)dy$ - аналог. част. суммы Фурье

  Применяя лемму 1, получим
  \begin{gather*}
    S_\eta=\int_{-\infty}^{+\infty}f(t)\left(\int_{0}^{\eta}\cos y(x-t)dt\right)dt  = \\
    = \frac{1}{\pi}\int_{-\infty}^{+\infty}f(t)\frac{\sin \eta(x-t)}{x-t}dt = \\ 
    \langle t-x=u \rangle \ =\frac{1}{\pi}\int_{-\infty}^{+\infty}f(x+u)\frac{\sin \eta u}{u}du= \\ 
    = \frac{1}{\pi}\left(\underset{t=-u}{\int_{-\infty}^{0}}+\underset{t=u}{\int_{0}^{+\infty}}\right)f(x+u)\frac{\sin \eta u}{u}du =  \\ 
    = \frac{1}{\pi}\int_{0}^{+\infty}(f(x+t)+f(x-t))\frac{\sin \eta t}{t}dy
  \end{gather*}
  В точке $x_0 \ \exists f(x_0\pm 0), f_{\pm}'(x_0)$. Тогда (используя инт. Дирихле)
  \begin{gather*}
    S_\eta(x_0)-\frac{f(x_0+0)+f(x_0-0)}{2}= \\ 
    = \frac{1}{\pi}\int_{0}^{+\infty}(f(x_0+t)+f(x_0-t))\frac{\sin \eta t}{t}dt - \\ -  \frac{1}{\pi}\int_{0}^{+\infty}(f(x_0+0)+f(x_0-0))\frac{\sin \eta t}{t}dt = \\ 
    = \underbrace{\frac{1}{\pi}\int_{0}^{+\infty}(f(x_0+t)-f(x_0+0))\frac{\sin \eta t}{t}dt}_{I^{+}} + \\ + \underbrace{\frac{1}{\pi}\int_{0}^{+\infty}(f(x_0-t)-f(x_0-0))\frac{\sin \eta t}{t}dt}_{I^{-}}
  \end{gather*}
  \begin{gather*}
    I^{+}=\left(\int_{0}^{1}+\int_{1}^{+\infty}\right)(f(x_0+t)-f(x_0+0))\frac{\sin\eta t}{t}dt = \\ 
    = \underbrace{\int_{0}^{1}\frac{f(x_0+t)-f(x_0+0)}{t}\sin \eta t dt}_{I^{+}_{1}} + \\ + \underbrace{\int_{1}^{+\infty}\frac{f(x_0+t)}{t}\sin \eta t dt}_{I_{2}^{+}} - \underbrace{\int_{1}^{+\infty}f(x_0+0)\frac{\sin \eta t}{t}dt}_{I_{3}^{+}}
  \end{gather*}
  По теореме Римана $I_{1}^{+} \underset{\eta \to \infty}{\to } 0$ (т.к. $\exists f_{+}'(x_0$), $I_{2}^{+} \underset{\eta \to \infty}{\to } 0$

  Преобразуем $I_{3}^{+}\underset{\eta t = u}{=}f(x_0+0)\int_{\eta}^{+\infty}\frac{\sin u}{u}du \underset{\eta\to \infty}{0}$,
  т.к. это остаток сходящегося интеграла Дирихле.

  $\implies \ I^{+}=I_{1}^{+}+ I_{2}^{+} + I_{3}^{+} \underset{\eta \to \infty}{\to} 0$. Аналогичо $I^{-}$.

  $\implies S_\eta(x_0) \underset{\eta \to \infty}{\to} \frac{f(x_0+0)+f(x_0-)}{2}$
\end{proof}
\begin{definition}
  Пусть $f(x)$ - инт. в собственном или несобственном смысле на произвольном $[\eta,\eta]$.

  Тогда интеграл $v.p.\int_{-\infty}^{+\infty}f(x)dx=\lim_{\eta \to +\infty}\int_{-\eta}^{\eta}f(x)dx$
  называется интегралом в смысле главного значения.
\end{definition}
\begin{remark}
  Если $\exists \ \int_{-\infty}^{+\infty}f(x)dx$, то $\exists \ v.p. \int_{-\infty}^{+\infty}f(x)dx$, но не наоборот
\end{remark}
\begin{eg}
  Функции $\sin x$ и $x$
\end{eg}
Пусть $f(x)$ - абс. инт. на $(-\infty,+\infty)$, непр., в любой точке существует одностор. производная.

Тогда по теореме:
\begin{gather*}
  f(x)=\frac{1}{\pi}\int_{0}^{+\infty}\left(\int_{-\infty}^{+\infty}f(t)\cos(x-t)dt\right)dy = \\ 
  = \frac{1}{2\pi}\int_{-\infty}^{+\infty}\left(\int_{-\infty}^{+\infty}f(t)\cos y (x-t)dt\right)dy
\end{gather*}
Также:
\[
  0=v.p.\frac{1}{2\pi}\int_{-\infty}^{+\infty}\left(\int_{-\infty}^{+\infty}f(t)\sin y(x-t)dt\right)dy
\]
Умножаем на $i$ и складываем с предыдущим:
\begin{definition}
\[
  f(x)=\frac{1}{2\pi}v.p. \int_{-\infty}^{+\infty}\left(\int_{-\infty}^{+\infty}f(t)e^{iy(x-t)}dt\right)dy \tag{0}
\]
- комплексная запись инт. Фурье
\end{definition}

\section{Преобразование Фурье}
Пусть $f(x)$ - абс. инт и непр. на $(-\infty,+\infty)$, в каждой точке существуют $f_{\pm}'$

Тогда перепишем $(0)$ в виде
\[
  f(x)=v.p. \frac{1}{\sqrt{2\pi}}\int_{-\infty}^{+\infty}\left(\frac{1}{\sqrt{2\pi}}\int_{-\infty}^{+\infty}f(t)e^{-iyt}dt\right)e^{iyx}dy \tag{1}
\]
\begin{definition}
  \[
    F[f](y)=v.p. \frac{1}{\sqrt{2\pi}}\int_{-\infty}^{+\infty}f(x)e^{-ixy}dx
  \]
  называется преобразованием Фурье функции $f(x)$
\end{definition}
\begin{definition}
  \[
    F^{-1}[f](y)=v.p. \frac{1}{\sqrt{2\pi}}\int_{-\infty}^{+\infty}f(x)e^{ixy}dx
  \]
  называется обратным преобразованием Фурье функции $f(x)$
\end{definition}
\begin{theorem}[Т1]
  Пусть $f(x)$ абс. инт. на $(-\infty,+\infty)$, непр. и меет обе одностор. произв. в любой точке

  $\implies \ F^{-1}[F[f]]=f; F[F^{-1}[f]]=f$
\end{theorem}
\begin{proof}
  Первая формула совпадает с $(1)$.

  Формулу $(0)$ можно записать поменяв $(x-t)$ на $(t-x)$ т.е.
  \begin{gather*}
    f(x) = v.p. \frac{1}{2\pi}\int_{-\infty}^{+\infty}\int_{-\infty}^{+\infty}f(t)e^{iy(t-x)}dtdy = \\ 
    = v.p. \frac{1}{\sqrt{2\pi}}\int_{-\infty}^{+\infty}e^{-iyx}\left(\frac{1}{\sqrt{2\pi}}\int_{-\infty}^{+\infty}f(t)e^{iyt}dt\right)dt = F[F^{-1}[f]]
  \end{gather*}
\end{proof}
\subsection{Св-ва преобр. Фурье абс. инт. функций}
\begin{enumerate}
  \item $F[\lambda_1 f_1+\lambda_2f_2]=\lambda_1F[f_1]+\lambda_2F[f_2]$
  \item $F[f](y)$ - непр. на $(-\infty,+\infty)$
  \item $F[f](y) \underset{y\to \infty}{\to}0$
  \item $F[f](y)$ - огран. на $(-\infty,+\infty)$
\end{enumerate}
\begin{proof}
  \phantom{.}

  \begin{enumerate}
    \item Из свойства интеграла
    \item Из леммы 2 раздела 'Интеграл Фурье'
      , т.к. $F[f](y)=(a(t)-ib(t))\sqrt{\frac{\pi}{2}}$
    \item Аналогично второму
    \item $|F[f](y)|\le \frac{1}{\sqrt{2\pi}}\int_{-\infty}^{+\infty}|f(x)|\underbrace{|e^{-ixy}|}_{1}dx=\frac{1}{\sqrt{2\pi}}\int_{-\infty}^{+\infty}|f(x)|dx$
  \end{enumerate}
\end{proof}
\subsection{Преобразование производной}
\begin{theorem}[Т2 Преобразование Фурье производной]
  Пусть $f(x)$ - абс. инт. на $(-\infty,+\infty)$, $f'(x)$ - непр. и абс. инт. на $(-\infty,+\infty)$

  $\implies \ F[f'](y)=iyF[f](y), \ y \in (-\infty,+\infty)$
\end{theorem}
\begin{proof}
  \begin{gather*}
    f(x)=f(0)+\int_{0}^{x}f'(t)dt
  \end{gather*}
  Т.к. $\int_{-\infty}^{+\infty}f'(t)dt$ - сход., то $\exists \lim_{x\to +\infty}f(x),\ \lim_{x\to -\infty}f(x)$

  Эти пределы равны $0$, т.к. $\int_{-\infty}^{+\infty}f(x)dx$ - сход.

  Тогда
  \begin{gather*}
    F[f'](y)=\frac{1}{\sqrt{2\pi}\int_{-\infty}^{+\infty}f'(x)e^{-ixy}dx}= \\ 
    = \underbrace{\frac{1}{\sqrt{2\pi}}f(x)e^{-ixy}\big|_{-\infty}^{\infty}}_{=0}+\frac{iy}{\sqrt{2\pi}}\int_{-\infty}^{+\infty}f(x)e^{-ixy}dx = iyF[f](y)
  \end{gather*}
\end{proof}
\begin{corollary}
  Пусть $f(x)$ абс. инт. на $(-\infty,+\infty)$ вместе со своими производными до порядка n влюкчительно и
  $f^{(n)}(x)$ - непр. как $(-\infty,+\infty)$. Тогда

  \begin{gather*}
    F[f^{(n)}](y)=(iy)^{n}F[f](y) \tag {2} \\ 
    |F[f](y)| \le \frac{M}{|y|^{n}} \qquad M=\sup_{(-\infty,+\infty)}|F[f^{(n)}]| \tag{3}
  \end{gather*}
\end{corollary}
\begin{proof}
  Применяя $n$ раз теорему получаем $(2)$. Оценка следует из $(2)$
\end{proof}
\begin{theorem}[Т3 Производная от преобразования]
  \phantom{.}

  Пусть $f(x)$ - непр. на $(-\infty,+\infty)$, а $xf(x)$ - абс. инт на $(-\infty,+\infty)$

  $\implies \ \exists \frac{d}{dy}F[f](y)=F[-ixfx()](y)$
\end{theorem}
\begin{proof}
  Из абс. сход $xf(x)$ и непр. $f(x)$ следует абс. инт. $f(x)$.

  Тогда $F[f](y)=\frac{1}{\sqrt{2\pi}}\int_{-\infty}^{+\infty}f(x)e^{-ixy}dx$.

Продифф. по $y$:
\[
  \frac{d}{dy}F[f](y)=\frac{1}{\sqrt{2\pi}}\int_{-\infty}^{+\infty}f(x)(-ix)e^{-iyx}dx
\]
Дифф. законно, т.к. инт. справа сх. равн. по признаку Вейерштрасса 

($|f(x)(-ix)e^{-ixy}|\le|f(x)x|$)
\end{proof}
\begin{corollary}
  Пусть $f(x)$ - непр. на $(-\infty,+\infty)$. $x^{n}f(x)$ - абс. инт. на $(-\infty,+\infty)$

  $\implies \ \exists \ \frac{d^{n}}{dy^{n}}F[f](y)=F[(-ix)^{n}f(x)](y)$
\end{corollary}
\begin{proof}
  Применяем теорему $n$ раз.
\end{proof}

% lecture 14

\section{Обобщённые функции}
\begin{definition}
  Носителем функции $\phi(x)$ $(x\in \R)$ называется замыкание множества,
  на котором $\phi(x) \neq 0$ и обозначается $supp \phi$
\end{definition}
\begin{definition}
  Функция $\phi(x)$ называется финитной, если её носитель ограничен.
\end{definition}
\begin{definition}
  Пространство $D$ основных (пробных) функций - множество беск. дифф. финитных функций
  со сходимость, определённой следующим образом.
\end{definition}
\begin{definition}
  Последовательность $\{\phi_n(x)\}$ называется сход. в $D$ к функции $\phi(x)$, если 
  \begin{enumerate}
    \item $\exists [a,b]: \ supp \phi_n(x) \in [a,b] \ \forall n$
    \item $\sup_{[a,b]}|\phi_n^{(s)}(x)-\phi^{(s)}(x)| \underset{n\to \infty}{\to} 0 \ \forall s =0,1,2,\dots $
  \end{enumerate}
\end{definition}
\begin{eg}
  \[
    \phi_a(x)=\left\{\begin{aligned}
      &e^{-\frac{a^{2}}{a^{2}-x^{2}}}, |x|<a, a>0 \\ 
      &0, |x|\ge a
    \end{aligned}\right.
  \]
  Докозательство беск. дифф. функции $\phi_a(x)$ проводится аналогично док-ву беск. дифф. функции
  \[
    f(x) = \left\{\begin{aligned}
      & e^{-\frac{1}{x^{2}}}, \neq 0 \\ 
      & 0, x=0
    \end{aligned}\right.
  \]
  (пределы всех производных в точках $a$ равны $0$)
\end{eg}
\begin{definition}
  Отображение $f: D \to \R$, будем называть функционалом.

  Значение $f$ на $\phi$ будем обозначать $(f,\phi)$
\end{definition}
\begin{definition}
  Функционал $f$ на $D$ будем называть линейным, если $(f,\alpha_1\phi_1+\alpha_2\phi_2)=\alpha_1(f,\phi_1)+\alpha_2(f,\phi_2)$,
  для $\forall \alpha_1,\alpha_2 \in \R; \forall \phi_1,\phi_2 \in D$
\end{definition}
\begin{definition}
  Функционал $f$ на $D$ называется непрерывным, если $\forall \{\phi_n\}$ из $D: \ \phi_n \underset{n\to \infty}{\to}\phi$ в $D$
  $\true (f,\phi_n) \underset{ n \to \infty}{\to }(f,\phi)$
\end{definition}
\begin{definition}
  Всякий линейный непрерывный функционал на $D$ называется обобщённой функцией.
\end{definition}
\begin{definition}
  Пространством обобщённых функций $D'$ называется множество всех обобщённых функций
  с введённымы в нём операциями сложение и умножение на число и сходимостью по следующим правилам
  \begin{enumerate}
    \item $(\alpha_1f_1+\alpha_2f_2,\phi)=\alpha_1(f_1,\phi)+\alpha_2(f_2,\phi)\ \forall \alpha_1,\alpha_2 \in \R, \forall \phi \in D$
    \item Последовательность $\{f_n\}$ называется сходящейся в $D'$ к $f\in D'$, если
      $(f_n,\phi)\underset{n\to \infty}{\to}(f,\phi) \ \forall \phi\in D$

      Сходимость записываться $f_n \underset{n\to \infty}{\to }f $ в $D'$
  \end{enumerate}
\end{definition}
\begin{definition}
  Функция $f(x)$ называется локально абс. инт., если $f(x)$ абс. инт на любом $[a,b]$
\end{definition}
\begin{definition}
  Функционал порождённый локально абс. инт. функцией $f(x)$ по правилу 
  $(f,\phi)=\int_{-\infty}^{+\infty}f(x)\phi(x)dx$ называется регулярной обобщённой функцией
  и обозначается также $f$
\end{definition}
\begin{proof}(Коректности определения)
  \begin{enumerate}
    \item $(f,\phi)\exists$ (см. лемму перед теоремой Римана)
    \item Функционал линеен (следует из линейности интеграла)
    \item Функционал непрерывен, т.к. из того, что $\phi_n(x)\underset{n\to\infty}{\to}\phi(x)$ в $D$
      следует $|(f,\phi_n)-(f,\phi)|\le \int_{a}^{b}|f||\phi-\phi_n|dx \le \sup_{[a,b]}|\phi-\phi_n|\int_{a}^{b}|f|dx \underset{n\to\infty}{\to}0$

      $\implies \ (f,\phi_n) \underset{ n\to\infty}{\to}(f,\phi)$
  \end{enumerate}
\end{proof}
\begin{definition}
  Обобщённая функция, которая не является регульрной называется сингулярной.
\end{definition}
\begin{definition}
  Функционал вида $(\delta(x),\phi(x))=\phi(0)$ называется $\delta$-функцией (Дирака)
\end{definition}
\begin{theorem}
  $\delta$-функция является сингулярной обобщённой функцией
\end{theorem}
\begin{proof}
  \phantom{.}

  \begin{enumerate}
    \item Линейность $(\delta,\alpha_1\phi_1+\alpha_2\phi_2)=\alpha_1\phi(0)+\alpha_2\phi_2(0)=\alpha_1(\delta,\phi_1)+\alpha_2(\delta,\phi_2)$
    \item Непрерывность $\phi_n(x) \underset{n\to \infty}{\to}\phi(x)$ в $D$

      $\implies \ \phi_n(0)\underset{n\to\infty}{\to}\phi(0) \iff (\delta,\phi_n)\underset{n\to\infty}{\to}(\delta,\phi)$
    \item Сингулярность

      Предположим, что $\delta$ - регулярная и порождена локально абс. инт. функ. $f(x)$
      и $\phi(0)=\int_{-\infty}^{+\infty}f(x)\phi(x)dx$

      Пусть $\phi(x)=\phi_\epsilon(x)=\left\{\begin{aligned}
        & e^{-\frac{\epsilon^{2}}{\epsilon^{2}-x^{2}}}, |x|<\epsilon, \epsilon>0 \\ 
        & 0, |x|> \epsilon
      \end{aligned}\right.$

      Тогда $\frac{1}{e}=\int_{-\infty}^{+\infty}f(x)\phi_\epsilon(x)dx$.

      Но $|\int_{-\infty}^{+\infty}f(x)\phi_\epsilon(x)dx| \le \int_{-\epsilon}^{+\epsilon}|f(x)|\frac{1}{f}dx \underset{\epsilon\to 0}{\to}0$ (?!)
      
      $\implies \ \delta$ - синг. обобщ. функ.
  \end{enumerate}
\end{proof}
\begin{remark}
  Допускается вместо $(\delta,\phi)$ писать $\int_{-\infty}^{+\infty}\delta(x)\phi(x)dx$
\end{remark}
\begin{eg}
  \begin{gather*}
    \delta_n(x)=\left\{\begin{aligned}
      & \frac{n}{2}, x \in [-\frac{1}{n},\frac{1}{n}] \\ 
      & 0, x \not \in [-\frac{1}{n},\frac{1}{n}]
    \end{aligned}\right.
  \end{gather*}
  Доказать, что $\delta_n(x) \underset{n\to\infty}{\to}\delta(x)$ в $D'$
  \begin{proof}
    \begin{gather*}
      (\delta_n,\phi)=\int_{-\infty}^{+\infty}\delta_n(x)\phi(x)dx=\int_{-\frac{1}{n}}^{\frac{1}{n}}\frac{n}{2}\phi(x)dx = \\ 
      = \frac{n}{2}\frac{2}{n}\phi(\xi_n)\underset{n\to\infty}{\to}\phi(0)=(\delta,\phi)
    \end{gather*}
    Здесь $\xi_n\in(-\frac{1}{n},\frac{1}{n})$
  \end{proof}
\end{eg}
\begin{definition}
  $(\delta(x-x_0),\phi(x))=\phi(x_0)$
\end{definition}
\begin{definition}
  Пусть $f\in D'$. Тогда $(f',\phi)=-(f,\phi')$
\end{definition}
\begin{proof}[Обоснование]
  Если $f$ - регулярная обобщённая функция, порожд. непр. дифф. функцией $f(x)$, то
  $f'(x)$ - локально. абс. инт. и $(f',\phi)=\int_{-\infty}^{+\infty}f'\phi dx=f(x)\phi(x)\big|_{-\infty}^{+\infty}-\int_{-\infty}^{+\infty}f\phi' dx = -(f,\phi')$

  Докажем, что $f'\in D'$

  \begin{enumerate}
    \item $f' \in D \ \implies \ \exists (f,\phi')$
    \item $f'$ - линейный, т.к. $f$ - линейный
    \item Если $\phi_n \underset{n\to\infty}{\phi}$ в $D$, то

      $\phi_n' \to \phi'$ в $D$ $\ \implies \ $ (т.к. $f$ - непр) $(f,\phi_n')\underset{n\to\infty}{\to}(f,\phi')$ 

      $\implies (f',\phi_n)\underset{n\to\infty}{\to}(f',\phi)$- непр.
  \end{enumerate}
\end{proof}
\begin{remark}
  Любая обобщённая функция имеет производную.
\end{remark}
\begin{eg}
  Функция Хевисайда
  \begin{gather*}
    \theta(x)= \left\{\begin{aligned}
      & 1, x \ge 0 \\ 
      & 0, x < 0
    \end{aligned}\right. \\ 
    (\theta',\phi)=-(\theta,\phi')=-\int_{-\infty}^{+\infty}\theta(x)\phi'(x)dx = \\ 
    = -\int_{0}^{+\infty}\phi'(x)dx=-\phi(x)\big|_{0}^{+\infty}=\phi(0)=(\delta,\phi)
  \end{gather*}
  $\implies \ \theta'(x)=\delta(x)$ в $D'$
\end{eg}
\begin{definition}
  Пусть $f \in D'$, $g(x)$ - беск. дифф. функция. Тогда $(fg, \phi)=(f,\phi g)$
\end{definition}
\begin{eg}
  Упростить в $D'$: $\delta'(x)x$
  \begin{gather*}
    (\delta'(x),\phi(x))=(\delta'(x),\phi(x)x)=-(\delta(x),\phi'(x)+\phi(x))= \\ 
    =-\phi'(0)0-\phi(0)=-(\delta,\phi)
  \end{gather*}
  $\implies \delta'(x)x=-\delta(x)$ в $D'$
\end{eg}


\end{document}
