\documentclass{article}

\usepackage{amsmath, amsthm, amsfonts, amssymb}
\usepackage[utf8]{inputenc}
\usepackage[T2A]{fontenc}
\usepackage[english, russian]{babel}

\usepackage{import}
\usepackage{pdfpages}
\usepackage{transparent}
\usepackage{xcolor}

\usepackage{parskip}
\usepackage{systeme}

\newcommand{\incfig}[2][1]{%
    \def\svgwidth{#1\columnwidth}
    \import{./figures/}{#2.pdf_tex}
}

\pdfsuppresswarningpagegroup=1

\usepackage{hyperref}
\hypersetup{
    colorlinks=true, %set true if you want colored links
    linktoc=all,     %set to all if you want both sections and subsections linked
    linkcolor=black,  %choose some color if you want links to stand out
}

\newcommand\hr{
    \noindent\rule[0.5ex]{\linewidth}{0.5pt}
}

% All the environments
\usepackage{mdframed}
\mdfsetup{skipabove=1em,skipbelow=0em}
\theoremstyle{definition}
\newmdtheoremenv[nobreak=true]{theorem}{Теорема}
\numberwithin{theorem}{section}
\newmdtheoremenv[nobreak=true]{lemma}{Лемма}
\numberwithin{lemma}{section}
\newmdtheoremenv[nobreak=true]{definition}{Определение}
\numberwithin{definition}{section}
\newmdtheoremenv[nobreak=true]{corollary}{Следствие}
\numberwithin{corollary}{section}
\newtheorem*{eg}{Пример}
\newtheorem*{remark}{Замечание}

\numberwithin{equation}{section}

% Defs
\let\phi\varphi
\let\epsilon\varepsilon
\let\kappa\varkappa
\let\implies\Rightarrow
\let\iff\Leftrightarrow
\let\true\hookrightarrow

\newcommand{\pd}[2]{\frac{\partial{#1}}{\partial{#2}}}
\newcommand{\pdd}[2]{\frac{\partial^2{#1}}{\partial{#2^2}}}
\newcommand{\pdm}[3]{\frac{\partial^2{#1}}{\partial{#2}\partial{#3}}}
\newcommand\R{\ensuremath{\mathbb{R}}}


\begin{document}

% lecture 1
\section{Некоторые свойства абсолютно инт функций}
\begin{definition}
  Функция $f(x)$ называется абсолютно интегрируемой на конечном или бесконечном интервале $(a,b)$,
  если выполняются два условия:
  \begin{enumerate}
    \item $\exists x_0,x_1,\dots,x_k:a=x_0<x_1<x_2<\dots<x_k=b$, что на любом отрезке
      $[\xi,\eta]$ из $(a,b)$, не содерж $x_i$ функция $f(x)$ интегрируема по Риману.
    \item $\int_{a}^{b}|f(x)|dx$ сходится
  \end{enumerate}
\end{definition}
\begin{lemma}
 Пусть $f(x)$ абсолютно интегрируема на конечном или бесконечном $(a,b)$.
 Пусть $\phi(x)$ - непрерывна и ограничена на $(a,b)$. \\
 $f(x)\phi(x)$ - абсолютно интегрируемо на $[a,b]$
\end{lemma}
\begin{proof}
  а) Рассмотрим $\forall [\xi,\eta]\subset(x_{i-1},x_i)$. На нём $f(x)$ и $\phi(x)$ - интегрируема
  по Риману, следовательно $f(x)\phi(x)$ инт. по Риману $\implies$ 1. \\
  б) Так как $\phi(x)$ - ограничена то $\exists M:|\phi(x)|\le M$ на $(a,b)$.
  Тогда $|f(x)\phi(x)|\le|f(x)|M$ на $(a,b)$. Т.к. $\int_{a}^{b}|f(x)|dx$ - сход.,
  то $\int_{a}^{b}|f(x)\phi(x)|dx$ - сход. по признаку сравнения $\implies$ 2. \\
  $\implies$ $f(x)\phi(x)$ - абс. инт. на $(a,b)$
\end{proof}
\begin{definition}
  Функция $\phi(x)$ определённая на $\R$ называется ступенчатой если
  $\exists$ числа $x_0,x_1,x_2,\dots,x_k:x_0<x_1<x_2<\dots<x_k$ и
  $c_1,c_2,\dots,c_k$:
  \[
    \phi(x) = \left\{\begin{aligned}
      & c_i, \; x \in [x_{i-1},x_i) \\ 
      & 0, \; x \in (-\infty,x_0) \cup [x_k,+\infty)
    \end{aligned}\right.
  \]
\end{definition}
\incfig{l1_step_f}
\begin{remark}
  Если положить
  \[
    \phi_i(x) = \left\{\begin{aligned}
      & 1, \; x \in [x_{i-1}, x_i) \\ 
      & 0, \; x \notin [x_{i-1}, x_i)
    \end{aligned}\right.
  \]
  то $\phi(x)=c_1\phi_1(x)+\dots+c_k\phi_k(x)$.
\end{remark}
\begin{theorem}[О приближении абс инт функций ступенчатыми]
  Пусть $f(x)$ - абс. инт. на конечном или бесконечном $(a,b)$ тогда
  $\forall \epsilon >0 \; \exists$ ступенчатая функция $\phi(x):\int_{a}^{b}|f(x)-\phi(x)|dx<\epsilon$
\end{theorem}
\begin{proof}
  Пусть для простоты записи у функции имеются только две особенности в точках $a$ и $b$,
  т.е. $f(x)$ - инт по Риману на $\forall [\xi,\eta]$ из $(a,b)$. 

  Возьмём $\forall \epsilon>0$. Из абсолютной инт. $f(x)$ следует $\exists$ таких
  $[\xi,\eta] \in (a,b)$:
  \[
    \int_{a}^{\xi}|f(x)|dx+\int_{\eta}^{b}|f(x)|dx< \frac{\epsilon}{2}
  \]
  Так как $f(x)$ инт. по Риману на $[\xi,\eta]$ то для рассмотренного $\epsilon>0$ 
  $\exists \delta:\forall$ разб. отр. $[\xi, \eta]$ $\tau=\{x_i\}_{i=0}^{n_\tau}$
  ($|\tau|<\delta$), $\forall \xi_i \in [x_{i-1}, x_i]$.

  Выполняется $\left|\int_{\xi}^{\eta}f(x)dx-\sigma_\tau\right|<\epsilon/2$, где
  $\sigma_\tau$ - сумма Дарбу. 

  $ \left|\int_{\xi}^{\eta}f(x)dx - s_\tau\right|\le \frac{\epsilon}{2}$,
  где $s_\tau=\sum_{i=1}^{n_\tau}m_i\Delta x_i$ ($m_i=\inf_{x \in [x_{i-1},x_i]} f(x), \; \Delta x_i=x_i-x_{i-1}$).

  Также $\int_{\xi}^{\eta}f(x)dx\ge s_\tau \; \implies \; 0 \le \int_{\xi}^{\eta}f(x)dx - s_\tau \le \epsilon/2$.

  Рассмотрим ступенчатую функцию:
  \[
    \phi(x) = \left\{\begin{aligned}
       & m_i, \; x \in [x_{i-1}, x_i) \\ 
       & 0, x \notin [\xi,\eta)
    \end{aligned}\right.
  \]

  \incfig{l1_step_r}
  Отметим: $s_\tau=\sum_{i=1}^{n_\tau}m_i\Delta x_i=\int_{\xi}^{\eta}f(x)dx$, $\phi(x)\le f(x)$ на $[\xi,\eta]$

  $\implies \int_{a}^{b}|f(x)-\phi(x)|dx=\int_{a}^{\xi}|f|dx+\int_{\xi}^{\eta}|f-\phi|dx+\int_{\eta}^{b}|f|dx$,
  но $\int_{\xi}^{\eta}|f-\phi|dx=\int_{\xi}^{\eta}(f-\phi)dx=\int_{\xi}^{\eta}fdx-s_\tau$

  $\implies \int_{a}^{b}|f(x)-\phi(x)|dx<\epsilon/2+\epsilon/2=\epsilon$
\end{proof}
\begin{theorem}[Римана (об осциляциях)]
  Пусть $f(x)$ абс. инт. на конечном или бесконечном интервале $(a,b)$,
  тогда $\lim\limits_{\nu\to\infty}\int_{a}^{b}f(x)\cos\nu xdx=0$ и
  $\lim\limits_{\nu\to\infty}\int_{a}^{b}f(x)\sin\nu xdx=0$
\end{theorem}
\begin{proof}
  1) Если 
  \[
    \phi(x) = \left\{\begin{aligned}
      & 1, \; x \in[\xi,\eta) \\ 
      & 0, \; x \notin [\xi,\eta)
    \end{aligned}\right. ,\; [\xi,\eta] \in (a,b)
  \]
  То:
  \[
    \int_{a}^{b}\phi(x)\sin\nu xdx=\int_{\xi}^{\eta}\sin\nu xdx=
    -\frac{\cos \nu x}{\nu}\bigg|_\xi^{\eta} \underset{\nu\to\infty}{\to}0
  \]
  2) Если $\phi(x)$ - ступенчатая, то она является линейно комбинацией
  расмотренных одноступенчатых функций, поэтому для неё утверждение справедливо.

  3) Рассмотрим абс. инт. на $(a,b)$ функцию $f(x)$. Возьмём $ \underline{\forall\epsilon>0}$.

  По предыдущей теорме $\exists \phi(x)$ - ступенчатая функция: $\int_{a}^{b}|f-\phi|dx<\epsilon/2$.

  Т.к. $\lim\limits_{\nu\to \infty }\int_{a}^{b}\phi(x)\sin\nu x dx =0 $, то
  $\underline{\exists \nu_\epsilon }: \forall \nu \; (|\nu|>\nu_\epsilon) \true |\int_{a}^{b}\phi(x)\sin \nu x dx|<\epsilon/2$.
  Тогда $\underline{\forall \nu : \;(|\nu|>\nu_\epsilon}$ выполняется:
  \begin{gather*}
    \underline{|\int_{a}^{b}f(x)\sin\nu x dx|}=|\int_{a}^{b}(f(x)-\phi(x))\sin \nu x dx + \int_{a}^{b}\phi(x)\sin \nu xdx|\le  \\
    \le|\int_{a}^{b}(f(x)-\phi(x))\sin \nu xdx|+|\int_{a}^{b}\phi(x)\sin \nu xdx| < \\
    <\int_{a}^{b}|f(x)-\phi(x)|dx + \frac{\epsilon}{2} < \underline{\frac{\epsilon}{2}+\frac{\epsilon}{2}=\epsilon}
  \end{gather*}
  Подчёркнутое означает,что $\lim\limits_{\nu\to\infty}\int_{a}^{b}f(x)\sin \nu xdx=0$. Аналогично косинус.
\end{proof}
\begin{remark}
  Интервал $(a,b)$ при исследовании абсолютно интегрируемых
  на другой промежуток $[a,b],[a,b),(a,b]$
\end{remark}

\end{document}
