\documentclass{article}

\usepackage{amsmath, amsthm, amsfonts}
\usepackage[utf8]{inputenc}
\usepackage[T2A]{fontenc}
\usepackage[english, russian]{babel}

\usepackage{import}
\usepackage{pdfpages}
\usepackage{transparent}
\usepackage{xcolor}

\usepackage{parskip}
\usepackage{systeme}

\newcommand{\incfig}[2][1]{%
    \def\svgwidth{#1\columnwidth}
    \import{./figures/}{#2.pdf_tex}
}

\pdfsuppresswarningpagegroup=1

\usepackage{hyperref}
\hypersetup{
    colorlinks=true, %set true if you want colored links
    linktoc=all,     %set to all if you want both sections and subsections linked
    linkcolor=black,  %choose some color if you want links to stand out
}

\newcommand\hr{
    \noindent\rule[0.5ex]{\linewidth}{0.5pt}
}

% All the environments
\usepackage{mdframed}
\mdfsetup{skipabove=1em,skipbelow=0em}
\theoremstyle{definition}
\newmdtheoremenv[nobreak=true]{theorem}{Теорема}
\newmdtheoremenv[nobreak=true]{lemma}{Лемма}
\newmdtheoremenv[nobreak=true]{definition}{Определение}
\newmdtheoremenv[nobreak=true]{corollary}{Следствие}
\newtheorem*{eg}{Пример}
\newtheorem*{remark}{Замечание}

% Defs
\let\phi\varphi
\let\epsilon\varepsilon
\let\implies\Rightarrow
\let\iff\Leftrightarrow
\let\true\hookrightarrow

\newcommand{\pd}[2]{\frac{\partial{#1}}{\partial{#2}}}
\newcommand{\pdd}[2]{\frac{\partial^2{#1}}{\partial{#2^2}}}
\newcommand{\pdm}[3]{\frac{\partial^2{#1}}{\partial{#2}\partial{#3}}}
\newcommand\R{\ensuremath{\mathbb{R}}}


\begin{document}

% lecture 1
\section{Некоторые свойства абсолютно инт функций}
\begin{definition}
  Функция $f(x)$ называется абсолютно интегрируемой на конечном или бесконечном интервале $(a,b)$,
  если выполняются два условия:
  \begin{enumerate}
    \item $\exists x_0,x_1,\dots,x_k:a=x_0<x_1<x_2<\dots<x_k=b$, что на любом отрезке
      $[\xi,\eta]$ из $(a,b)$, не содерж $x_i$ функция $f(x)$ интегрируема по Риману.
    \item $\int_{a}^{b}|f(x)|dx$ сходится
  \end{enumerate}
\end{definition}
\begin{lemma} \label{l1_abs_prod}
 Пусть $f(x)$ абсолютно интегрируема на конечном или бесконечном $(a,b)$.
 Пусть $\phi(x)$ - непрерывна и ограничена на $(a,b)$. \\
 $f(x)\phi(x)$ - абсолютно интегрируемо на $[a,b]$
\end{lemma}
\begin{proof}
  а) Рассмотрим $\forall [\xi,\eta]\subset(x_{i-1},x_i)$. На нём $f(x)$ и $\phi(x)$ - интегрируема
  по Риману, следовательно $f(x)\phi(x)$ инт. по Риману $\implies$ 1. \\
  б) Так как $\phi(x)$ - ограничена то $\exists M:|\phi(x)|\le M$ на $(a,b)$.
  Тогда $|f(x)\phi(x)|\le|f(x)|M$ на $(a,b)$. Т.к. $\int_{a}^{b}|f(x)|dx$ - сход.,
  то $\int_{a}^{b}|f(x)\phi(x)|dx$ - сход. по признаку сравнения $\implies$ 2. \\
  $\implies$ $f(x)\phi(x)$ - абс. инт. на $(a,b)$
\end{proof}
\begin{definition}
  Функция $\phi(x)$ определённая на $\R$ называется ступенчатой если
  $\exists$ числа $x_0,x_1,x_2,\dots,x_k:x_0<x_1<x_2<\dots<x_k$ и
  $c_1,c_2,\dots,c_k$:
  \[
    \phi(x) = \left\{\begin{aligned}
      & c_i, \; x \in [x_{i-1},x_i) \\ 
      & 0, \; x \in (-\infty,x_0) \cup [x_k,+\infty)
    \end{aligned}\right.
  \]
\end{definition}
\incfig{l1_step_f}
\begin{remark}
  Если положить
  \[
    \phi_i(x) = \left\{\begin{aligned}
      & 1, \; x \in [x_{i-1}, x_i) \\ 
      & 0, \; x \notin [x_{i-1}, x_i)
    \end{aligned}\right.
  \]
  то $\phi(x)=c_1\phi_1(x)+\dots+c_k\phi_k(x)$.
\end{remark}
\begin{theorem}[О приближении абс инт функций ступенчатыми]
  Пусть $f(x)$ - абс. инт. на конечном или бесконечном $(a,b)$ тогда
  $\forall \epsilon >0 \; \exists$ ступенчатая функция $\phi(x):\int_{a}^{b}|f(x)-\phi(x)|dx<\epsilon$
\end{theorem}
\begin{proof}
  Пусть для простоты записи у функции имеются только две особенности в точках $a$ и $b$,
  т.е. $f(x)$ - инт по Риману на $\forall [\xi,\eta]$ из $(a,b)$. 

  Возьмём $\forall \epsilon>0$. Из абсолютной инт. $f(x)$ следует $\exists$ таких
  $[\xi,\eta] \in (a,b)$:
  \[
    \int_{a}^{\xi}|f(x)|dx+\int_{\eta}^{b}|f(x)|dx< \frac{\epsilon}{2}
  \]
  Так как $f(x)$ инт. по Риману на $[\xi,\eta]$ то для рассмотренного $\epsilon>0$ 
  $\exists \delta:\forall$ разб. отр. $[\xi, \eta]$ $\tau=\{x_i\}_{i=0}^{n_\tau}$
  ($|\tau|<\delta$), $\forall \xi_i \in [x_{i-1}, x_i]$.

  Выполняется $\left|\int_{\xi}^{\eta}f(x)dx-\sigma_\tau\right|<\epsilon/2$, где
  $\sigma_\tau$ - сумма Дарбу. 

  $ \left|\int_{\xi}^{\eta}f(x)dx - s_\tau\right|\le \frac{\epsilon}{2}$,
  где $s_\tau=\sum_{i=1}^{n_\tau}m_i\Delta x_i$ ($m_i=\inf_{x \in [x_{i-1},x_i]} f(x), \; \Delta x_i=x_i-x_{i-1}$).

  Также $\int_{\xi}^{\eta}f(x)dx\ge s_\tau \; \implies \; 0 \le \int_{\xi}^{\eta}f(x)dx - s_\tau \le \epsilon/2$.

  Рассмотрим ступенчатую функцию:
  \[
    \phi(x) = \left\{\begin{aligned}
       & m_i, \; x \in [x_{i-1}, x_i) \\ 
       & 0, x \notin [\xi,\eta)
    \end{aligned}\right.
  \]

  \incfig{l1_step_r}
  Отметим: $s_\tau=\sum_{i=1}^{n_\tau}m_i\Delta x_i=\int_{\xi}^{\eta}f(x)dx$, $\phi(x)\le f(x)$ на $[\xi,\eta]$

  $\implies \int_{a}^{b}|f(x)-\phi(x)|dx=\int_{a}^{\xi}|f|dx+\int_{\xi}^{\eta}|f-\phi|dx+\int_{\eta}^{b}|f|dx$,
  но $\int_{\xi}^{\eta}|f-\phi|dx=\int_{\xi}^{\eta}(f-\phi)dx=\int_{\xi}^{\eta}fdx-s_\tau$

  $\implies \int_{a}^{b}|f(x)-\phi(x)|dx<\epsilon/2+\epsilon/2=\epsilon$
\end{proof}
\begin{theorem}[Римана (об осциляциях)]
  Пусть $f(x)$ абс. инт. на конечном или бесконечном интервале $(a,b)$,
  тогда $\lim\limits_{\nu\to\infty}\int_{a}^{b}f(x)\cos\nu xdx=0$ и
  $\lim\limits_{\nu\to\infty}\int_{a}^{b}f(x)\sin\nu xdx=0$
\end{theorem}
\begin{proof}
  1) Если 
  \[
    \phi(x) = \left\{\begin{aligned}
      & 1, \; x \in[\xi,\eta) \\ 
      & 0, \; x \notin [\xi,\eta)
    \end{aligned}\right. ,\; [\xi,\eta] \in (a,b)
  \]
  То:
  \[
    \int_{a}^{b}\phi(x)\sin\nu xdx=\int_{\xi}^{\eta}\sin\nu xdx=
    -\frac{\cos \nu x}{\nu}\bigg|_\xi^{\eta} \underset{\nu\to\infty}{\to}0
  \]
  2) Если $\phi(x)$ - ступенчатая, то она является линейно комбинацией
  расмотренных одноступенчатых функций, поэтому для неё утверждение справедливо.

  3) Рассмотрим абс. инт. на $(a,b)$ функцию $f(x)$. Возьмём $ \underline{\forall\epsilon>0}$.

  По предыдущей теорме $\exists \phi(x)$ - ступенчатая функция: $\int_{a}^{b}|f-\phi|dx<\epsilon/2$.

  Т.к. $\lim\limits_{\nu\to \infty }\int_{a}^{b}\phi(x)\sin\nu x dx =0 $, то
  $\underline{\exists \nu_\epsilon }: \forall \nu \; (|\nu|>\nu_\epsilon) \true |\int_{a}^{b}\phi(x)\sin \nu x dx|<\epsilon/2$.
  Тогда $\underline{\forall \nu : \;(|\nu|>\nu_\epsilon}$ выполняется:
  \begin{gather*}
    \underline{|\int_{a}^{b}f(x)\sin\nu x dx|}=|\int_{a}^{b}(f(x)-\phi(x))\sin \nu x dx + \int_{a}^{b}\phi(x)\sin \nu xdx|\le  \\
    \le|\int_{a}^{b}(f(x)-\phi(x))\sin \nu xdx|+|\int_{a}^{b}\phi(x)\sin \nu xdx| < \\
    <\int_{a}^{b}|f(x)-\phi(x)|dx + \frac{\epsilon}{2} < \underline{\frac{\epsilon}{2}+\frac{\epsilon}{2}=\epsilon}
  \end{gather*}
  Подчёркнутое означает,что $\lim\limits_{\nu\to\infty}\int_{a}^{b}f(x)\sin \nu xdx=0$. Аналогично косинус.
\end{proof}
\begin{remark}
  Интервал $(a,b)$ при исследовании абсолютно интегрируемых
  на другой промежуток $[a,b],[a,b),(a,b]$
\end{remark}

% lecture 2

\section{Тригонометрические ряды Фурье}
\begin{definition}
  Ряд вида $\frac{a_0}{2}+\sum_{n=1}^{\infty}a_n\cos(nx)+b_n\sin(nx)$
  называется тригонометрическим рядом, где $a_k,b_k\in\R$
\end{definition}
\begin{definition}
  Множество функций $\{u_n(x)\}=\{1/2, \cos(x), \sin(x),$ $ \cos(2x), \sin(2x),\dots\} $
  называется тригонемтрической системой
\end{definition}
Свойства тригоном. сист.
\begin{enumerate}
  \item Триг. сист. "ортогональна" в смысле $\int_{-\pi}^{\pi}u_n(x)u_k(x)dx=0$,
    $\forall n,k:n\neq k$
  \item $\int_{-\pi}^{\pi}u_n^{2}(x)dx=\pi$, при $n \ge 2$
\end{enumerate}
\begin{proof} 1) Например
      \[
        \int_{-\pi}^{\pi}\sin nx \cos kx dx = 
        \frac{1}{2}\int_{-\pi}^{\pi}(\sin(n-k)x+\sin(n+k)x)dx=0
      \]
\end{proof}
\begin{proof} 2)
  Например \[
    \int_{-\pi}^{\pi}\cos^{2}(x)dx=\int_{-\pi}^{\pi}\frac{1+\cos(2nx)}{2}dx=\pi + 0=\pi
  \]
\end{proof}

\begin{lemma} \label{l2_f_lemma}
  Пусть
  \begin{equation} \label{l2_fourier}
    f(x)=\frac{a_0}{2}+\sum_{n=1}^{\infty}a_n\cos nx+b_n\sin nx
  \end{equation}
  и ряд сходится равномерно тогда:
  \begin{equation}
    \begin{gathered} \label{l2_f_coef}
      a_n=\frac{1}{\pi}f(x)\int_{-\pi}^{\pi}\cos nx dx, \; n=0,1,2,\dots \\ 
      b_n=\frac{1}{\pi}\int_{-\pi}^{\pi}f(x)\sin nx dx, \; n\in\mathbb{N} 
    \end{gathered}
  \end{equation}
\end{lemma}
\begin{proof}
  Домножим \ref{l2_fourier} на $\cos mx$. 
  Полученный ряд будет равномерно сходится.

  \hr
  \begin{gather*}
    \left|\sum_{k=1}^{n+p}\cos mx(a_k\cos kx + b_k \sin kx)\right|= 
    |\cos mx|\cdot\left|\sum_{k=n+1}^{n+p}a_k\cos kx + b_k\sin kx\right| \le  \\
    \left|\sum_{n+1}^{n+p}a_k\cos kx + b_k \sin kx\right|
  \end{gather*}
  Из выполнения усл. Коши равномерной сходимости для исходного ряда \ref{l2_fourier}
  следует выполнение усл. Коши равномерной сходимости полученного в результате умножения ряда.

  \hr
  Тогда имеем право интегрировать равнество (по $x$ от $-\pi$ до $\pi$)
  \begin{gather*}
    f(x)\cos mx = \frac{a_0}{2}\cos mx + \sum_{n=1}^{\infty}\cos mx (a_n \cos nx + b_n \sin nx) \\
    \implies \int_{-\pi}^{\pi}f(x)\cos mx dx = a_m \pi \\ 
    \implies a_m=\frac{1}{\pi}\int_{-\pi}^{\pi}f(x)\cos mx dx
  \end{gather*}
  Второе равество в \ref{l2_f_coef} получается аналогично.
\end{proof}
\begin{definition}
  Пусть $f(x)$ - $2\pi$ периодическая абсолютно интегрируемая на $[-\pi;\pi]$ функция.
  Тригонометрический ряд с коэффицентами \ref{l2_f_coef} называется
  тригонометрическим рядом Фурье функции $f(x)$, а коэффициенты $a_k,b_k$ - коэффициетами Фурье.
  Имеет место запись (здесь $\sim$ означает соответствие):
  \[
    f(x)\sim \frac{a_0}{2}+\sum_{n=1}^{\infty}a_n\cos nx + b_n\sin nx
  \]
\end{definition}
Перефразируем лемму:
\begin{lemma}[\ref{l2_f_lemma}']
  Рамномерно сходящийся тригонометрический ряд является рядом Фурье своей суммы. 
\end{lemma}
\begin{eg}
  \[
    \sum_{n=1}^{\infty}\frac{\cos nx}{n^{\alpha}}
  \]
  где $\alpha>1$ является рядом Фурье своей суммы, так  как он
  равномерно сходдится по признаку Веерштрасса
\end{eg}
\begin{remark}
  Если функция абсолютно интегрируема на $[-\pi,\pi]$, то интегралы
  в \ref{l2_f_coef} сходятся абсолютно по \ref{l1_abs_prod}
\end{remark}
\begin{remark}
  Если $f(x)$ - $2\pi$ периодична и абс. инт. на каком-либо $[a-\pi,a+\pi]$,
  то она будет абс. инт. на $\forall$ другом таком отрезке и
  интегралы (\ref{l2_f_coef}) не зависят от отрезка.
\end{remark}
\begin{remark}
  Любую абсолютно интегрируемую на $[a-\pi,a+\pi]$($(a-\pi,a+\pi);(a-\pi,a+\pi], [a-\pi,a+\pi)$)
  можно продолжить до $2\pi$ периодической функции,
  возможно доопределив или переопределив функцию в граничных точках.
  Интегралы при этом не меняются.
\end{remark}
\begin{corollary}
  Пусть $\{a_k\}, \{b_k\}$ - посл. коэфф. Фурье $2\pi$ периодической 
  и абсолютно инт. на $[-\pi,\pi]$ функции \[
    \implies \lim\limits_{k\to\infty} a_k=0 \qquad \lim\limits_{k\to\infty} b_k=0
  \]
  По \ref{l1_abs_prod} 
  $f(x)\cos nx$, $f(x) \sin nx $ - абс. инт.. По Т. Римана получаем нужный результат.
\end{corollary}
\begin{eg}
  \[
    \sum_{n=1}^{\infty}\sin nx
  \]
  Не может быть рядом Фурье какой-либо абсолютно инт. на $[-\pi,\pi]$ функции.
\end{eg}

\subsection{Ядро Дирихле. Принцип локализации}
Пусть $f(x)$ - $2\pi$ периодическая и абсолютно интегрируема на $[-\pi,\pi]$ и
\[
  f(x) \sim \frac{a_0}{2}+\sum_{n=1}^{\infty}a_n \cos nx + b_n \sin nx
\]
Рассмотрим частичные суммы ряда Фурье
\[
  S_n(x)=S_n(x,b)=\frac{a_0}{2}+\sum_{k=1}^{n}a_k \cos kx + b_k \sin kx
\]
Преобразуем:
\begin{gather*}
  S_n(x)= \frac{1}{2\pi}\int_{-\pi}^{\pi}f(x)dt+ \\ 
  +\sum_{k=1}^{n}\frac{1}{\pi}\int_{-\pi}^{\pi}f(t)\cos kx dt \cdot \cos kx +
  \frac{1}{\pi}\int_{-\pi}^{\pi}f(t)\sin kt dt \cdot \sin kx =\\
  \frac{1}{\pi}\int_{-\pi}^{\pi}f(t)
  \left(\frac{1}{2}+\sum_{k=1}^{n}\cos k(t-x)\right)dt \\ 
  S_n(x)=\frac{1}{\pi}\int_{-\pi}^{\pi}f(t)D_n(t-x)dt
\end{gather*}
\begin{definition}
  Функция 
  \[
    D_n(t)=\left(\frac{1}{2}+\sum_{k=1}^{n}\cos k(t)\right)
  \]
  называется ядром Дирихле
\end{definition}
\incfig{l2_dirichle}
Свойства ядра Дирихле:
\begin{enumerate}
  \item $D_n(t)$ - четная, $2\pi$ период. и непр. функция
  \item $\int_{-\pi}^{\pi}D_n(t)dt=\pi$
  \item $\max|D_n(t)|=\max D_n(t)=D_n(0)=n+\frac{1}{2}$
  \item $D_n(t)=\cfrac{\sin (n+\frac{1}{2})t}{2\sin \frac{t}{2}}$, при $t\neq 2\pi m, m\in \mathbb{Z}$
\end{enumerate}
\begin{proof} \phantom{.} 

  \begin{enumerate}
    \item Следует из аналогичных свойств слогаемых
    \item Очев
    \item $\frac{1}{2}-n\le D_n(t) \le \frac{1}{2}+n = D_n(0)$
    \item
      \begin{gather*}
        D_n(t)=\frac{1}{2}+\cos t + \cos 2t + \dots + \cos nt = \\ 
        = \cfrac{\sin \frac{t}{2}+2\sin\frac{t}{2}\cos t + 2\sin\frac{t}{2}\cos 2t + \dots + 2\sin \frac{t}{2}\cos nt}{2\sin\frac{t}{2}}= \\
        =\cfrac{\sin \frac{t}{2}-\sin \frac{t}{2}+\sin \frac{3t}{2} - \sin \frac{3t}{2} + \dots - \sin (n-\frac{1}{2})t + \sin (n+\frac{1}{2})t}{\sin\frac{t}{2}}= \\ 
        =\cfrac{\sin (n+\frac{1}{2})t}{2\sin\frac{t}{2}} 
      \end{gather*}
  \end{enumerate}
\end{proof}
\begin{theorem}[Принцип локализации]
  Пусть $f(x)$ - $2\pi$ периодическая абсолютно интегрируема на $[-\pi,\pi]$ функция.
  Пусть $x_0\in\R$, $0<\delta<\pi$. Тогда: $\lim\limits_{n\to\infty}S_n(x_0)$ и
  $\lim\limits_{n\to\infty}\frac{1}{\pi}\int_{0}^{\delta}D_n(t)(f(x_0+t)+f(x_0-t))dt$
  существуют или нет одновременно. В случае существования равны.
\end{theorem}
\begin{remark}
  Таким образом сходимость и значение суммы ряда Фурье $2\pi$ пер. и абс. инт.
  на $[-\pi,\pi]$ зависит только от свойств функции в сколь угодно малой окрестности.
\end{remark}
\begin{proof}
  Преобр. $S_n$:
  \begin{gather*}
    S_n(x_0)\underset{t=\tau+x_0}{=}\frac{1}{\pi}\int_{-\pi-x_0}^{\pi-x_0}f(x_0+\tau)D_n(\tau)d\tau \\ 
    S_n(x_0)=\frac{1}{\pi}\int_{-\pi}^{\pi}f(x_0+\tau)D_n(\tau)d\tau \\ 
    S_n(x_0)=\frac{1}{\pi}(\underbrace{\int_{-\pi}^{0}}_{\tau=-t}+\underbrace{\int_{0}^{\pi}}_{\tau=t})f(x_0+\tau)D_n(\tau)d\tau = \\ 
    =\frac{1}{\pi}(\int_{0}^{\pi}f(x_0-t)D_n(-t)dt + \int_{0}^{\pi}f(x_0+t)D_n(t)dt) \\ 
    S_n(x_0)=\frac{1}{pi}\int_{0}^{\pi}D_n(t)(f(x_0-t)+f(x_0+t))dt \\ 
    S_n(x_0)=\frac{1}{\pi}(\int_{0}^{\delta}+\int_{\delta}^{\pi})\frac{\sin(n+\frac{1}{2}t)}{2\sin \frac{t}{2}}(f(x_0-t)+f(x_0+t))dt \\
    \frac{1}{2\sin\frac{t}{2}} \le \frac{1}{2\sin \frac{\delta}{2}} \qquad \text{на} [\delta,\pi]
  \end{gather*}
  Тогда $\frac{(f(x_0-t)+f(x_0+t))}{2\sin \frac{t}{2}}$ - абс. инт на $[\delta,\pi]$ (см \ref{l1_abs_prod}).

  Тогда 2-ой инт. $\to0$ при $n\to\infty$ (Т. Римана) и получаем нужный результат.
\end{proof}

\end{document}
