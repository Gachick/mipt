\documentclass{article}

\usepackage{amsmath, amsthm, amsfonts, amssymb}
\usepackage[utf8]{inputenc}
\usepackage[T2A]{fontenc}
\usepackage[english, russian]{babel}

\usepackage{import}
\usepackage{pdfpages}
\usepackage{transparent}
\usepackage{xcolor}

\usepackage{parskip}
\usepackage{systeme}

\newcommand{\incfig}[2][1]{%
    \def\svgwidth{#1\columnwidth}
    \import{./figures/}{#2.pdf_tex}
}

\pdfsuppresswarningpagegroup=1

\usepackage{hyperref}
\hypersetup{
    colorlinks=true, %set true if you want colored links
    linktoc=all,     %set to all if you want both sections and subsections linked
    linkcolor=black,  %choose some color if you want links to stand out
}

\newcommand\hr{
    \noindent\rule[0.5ex]{\linewidth}{0.5pt}
}

% All the environments
\usepackage{mdframed}
\mdfsetup{skipabove=1em,skipbelow=0em}
\theoremstyle{definition}
\newmdtheoremenv[nobreak=true]{theorem}{Теорема}
\numberwithin{theorem}{section}
\newmdtheoremenv[nobreak=true]{lemma}{Лемма}
\numberwithin{lemma}{section}
\newmdtheoremenv[nobreak=true]{definition}{Определение}
\numberwithin{definition}{section}
\newmdtheoremenv[nobreak=true]{corollary}{Следствие}
\numberwithin{corollary}{section}
\newtheorem*{eg}{Пример}
\newtheorem*{remark}{Замечание}

\numberwithin{equation}{section}

% Defs
\let\phi\varphi
\let\epsilon\varepsilon
\let\kappa\varkappa
\let\implies\Rightarrow
\let\iff\Leftrightarrow
\let\true\hookrightarrow

\newcommand{\pd}[2]{\frac{\partial{#1}}{\partial{#2}}}
\newcommand{\pdd}[2]{\frac{\partial^2{#1}}{\partial{#2^2}}}
\newcommand{\pdm}[3]{\frac{\partial^2{#1}}{\partial{#2}\partial{#3}}}
\newcommand\R{\ensuremath{\mathbb{R}}}


\begin{document}

% lecture 1
\section{Некоторые свойства абсолютно инт функций}
\begin{definition}
  Функция $f(x)$ называется абсолютно интегрируемой на конечном или бесконечном интервале $(a,b)$,
  если выполняются два условия:
  \begin{enumerate}
    \item $\exists x_0,x_1,\dots,x_k:a=x_0<x_1<x_2<\dots<x_k=b$, что на любом отрезке
      $[\xi,\eta]$ из $(a,b)$, не содерж $x_i$ функция $f(x)$ интегрируема по Риману.
    \item $\int_{a}^{b}|f(x)|dx$ сходится
  \end{enumerate}
\end{definition}
\begin{lemma} \label{l1_abs_prod}
 Пусть $f(x)$ абсолютно интегрируема на конечном или бесконечном $(a,b)$.
 Пусть $\phi(x)$ - непрерывна и ограничена на $(a,b)$. \\
 $f(x)\phi(x)$ - абсолютно интегрируемо на $[a,b]$
\end{lemma}
\begin{proof}
  а) Рассмотрим $\forall [\xi,\eta]\subset(x_{i-1},x_i)$. На нём $f(x)$ и $\phi(x)$ - интегрируема
  по Риману, следовательно $f(x)\phi(x)$ инт. по Риману $\implies$ 1. \\
  б) Так как $\phi(x)$ - ограничена то $\exists M:|\phi(x)|\le M$ на $(a,b)$.
  Тогда $|f(x)\phi(x)|\le|f(x)|M$ на $(a,b)$. Т.к. $\int_{a}^{b}|f(x)|dx$ - сход.,
  то $\int_{a}^{b}|f(x)\phi(x)|dx$ - сход. по признаку сравнения $\implies$ 2. \\
  $\implies$ $f(x)\phi(x)$ - абс. инт. на $(a,b)$
\end{proof}
\begin{definition}
  Функция $\phi(x)$ определённая на $\R$ называется ступенчатой если
  $\exists$ числа $x_0,x_1,x_2,\dots,x_k:x_0<x_1<x_2<\dots<x_k$ и
  $c_1,c_2,\dots,c_k$:
  \[
    \phi(x) = \left\{\begin{aligned}
      & c_i, \; x \in [x_{i-1},x_i) \\ 
      & 0, \; x \in (-\infty,x_0) \cup [x_k,+\infty)
    \end{aligned}\right.
  \]
\end{definition}
\incfig{l1_step_f}
\begin{remark}
  Если положить
  \[
    \phi_i(x) = \left\{\begin{aligned}
      & 1, \; x \in [x_{i-1}, x_i) \\ 
      & 0, \; x \notin [x_{i-1}, x_i)
    \end{aligned}\right.
  \]
  то $\phi(x)=c_1\phi_1(x)+\dots+c_k\phi_k(x)$.
\end{remark}
\begin{theorem}[О приближении абс инт функций ступенчатыми]
  Пусть $f(x)$ - абс. инт. на конечном или бесконечном $(a,b)$ тогда
  $\forall \epsilon >0 \; \exists$ ступенчатая функция $\phi(x):\int_{a}^{b}|f(x)-\phi(x)|dx<\epsilon$
\end{theorem}
\begin{proof}
  Пусть для простоты записи у функции имеются только две особенности в точках $a$ и $b$,
  т.е. $f(x)$ - инт по Риману на $\forall [\xi,\eta]$ из $(a,b)$. 

  Возьмём $\forall \epsilon>0$. Из абсолютной инт. $f(x)$ следует $\exists$ таких
  $[\xi,\eta] \in (a,b)$:
  \[
    \int_{a}^{\xi}|f(x)|dx+\int_{\eta}^{b}|f(x)|dx< \frac{\epsilon}{2}
  \]
  Так как $f(x)$ инт. по Риману на $[\xi,\eta]$ то для рассмотренного $\epsilon>0$ 
  $\exists \delta:\forall$ разб. отр. $[\xi, \eta]$ $\tau=\{x_i\}_{i=0}^{n_\tau}$
  ($|\tau|<\delta$), $\forall \xi_i \in [x_{i-1}, x_i]$.

  Выполняется $\left|\int_{\xi}^{\eta}f(x)dx-\sigma_\tau\right|<\epsilon/2$, где
  $\sigma_\tau$ - сумма Дарбу. 

  $ \left|\int_{\xi}^{\eta}f(x)dx - s_\tau\right|\le \frac{\epsilon}{2}$,
  где $s_\tau=\sum_{i=1}^{n_\tau}m_i\Delta x_i$ ($m_i=\inf_{x \in [x_{i-1},x_i]} f(x), \; \Delta x_i=x_i-x_{i-1}$).

  Также $\int_{\xi}^{\eta}f(x)dx\ge s_\tau \; \implies \; 0 \le \int_{\xi}^{\eta}f(x)dx - s_\tau \le \epsilon/2$.

  Рассмотрим ступенчатую функцию:
  \[
    \phi(x) = \left\{\begin{aligned}
       & m_i, \; x \in [x_{i-1}, x_i) \\ 
       & 0, x \notin [\xi,\eta)
    \end{aligned}\right.
  \]

  \incfig{l1_step_r}
  Отметим: $s_\tau=\sum_{i=1}^{n_\tau}m_i\Delta x_i=\int_{\xi}^{\eta}f(x)dx$, $\phi(x)\le f(x)$ на $[\xi,\eta]$

  $\implies \int_{a}^{b}|f(x)-\phi(x)|dx=\int_{a}^{\xi}|f|dx+\int_{\xi}^{\eta}|f-\phi|dx+\int_{\eta}^{b}|f|dx$,
  но $\int_{\xi}^{\eta}|f-\phi|dx=\int_{\xi}^{\eta}(f-\phi)dx=\int_{\xi}^{\eta}fdx-s_\tau$

  $\implies \int_{a}^{b}|f(x)-\phi(x)|dx<\epsilon/2+\epsilon/2=\epsilon$
\end{proof}
\begin{theorem}[Римана (об осциляциях)] \label{l1_r_osc}
  Пусть $f(x)$ абс. инт. на конечном или бесконечном интервале $(a,b)$,
  тогда $\lim\limits_{\nu\to\infty}\int_{a}^{b}f(x)\cos\nu xdx=0$ и
  $\lim\limits_{\nu\to\infty}\int_{a}^{b}f(x)\sin\nu xdx=0$
\end{theorem}
\begin{proof}
  1) Если 
  \[
    \phi(x) = \left\{\begin{aligned}
      & 1, \; x \in[\xi,\eta) \\ 
      & 0, \; x \notin [\xi,\eta)
    \end{aligned}\right. ,\; [\xi,\eta] \in (a,b)
  \]
  То:
  \[
    \int_{a}^{b}\phi(x)\sin\nu xdx=\int_{\xi}^{\eta}\sin\nu xdx=
    -\frac{\cos \nu x}{\nu}\bigg|_\xi^{\eta} \underset{\nu\to\infty}{\to}0
  \]
  2) Если $\phi(x)$ - ступенчатая, то она является линейно комбинацией
  расмотренных одноступенчатых функций, поэтому для неё утверждение справедливо.

  3) Рассмотрим абс. инт. на $(a,b)$ функцию $f(x)$. Возьмём $ \underline{\forall\epsilon>0}$.

  По предыдущей теорме $\exists \phi(x)$ - ступенчатая функция: $\int_{a}^{b}|f-\phi|dx<\epsilon/2$.

  Т.к. $\lim\limits_{\nu\to \infty }\int_{a}^{b}\phi(x)\sin\nu x dx =0 $, то
  $\underline{\exists \nu_\epsilon }: \forall \nu \; (|\nu|>\nu_\epsilon) \true |\int_{a}^{b}\phi(x)\sin \nu x dx|<\epsilon/2$.
  Тогда $\underline{\forall \nu : \;(|\nu|>\nu_\epsilon}$ выполняется:
  \begin{gather*}
    \underline{|\int_{a}^{b}f(x)\sin\nu x dx|}=|\int_{a}^{b}(f(x)-\phi(x))\sin \nu x dx + \int_{a}^{b}\phi(x)\sin \nu xdx|\le  \\
    \le|\int_{a}^{b}(f(x)-\phi(x))\sin \nu xdx|+|\int_{a}^{b}\phi(x)\sin \nu xdx| < \\
    <\int_{a}^{b}|f(x)-\phi(x)|dx + \frac{\epsilon}{2} < \underline{\frac{\epsilon}{2}+\frac{\epsilon}{2}=\epsilon}
  \end{gather*}
  Подчёркнутое означает,что $\lim\limits_{\nu\to\infty}\int_{a}^{b}f(x)\sin \nu xdx=0$. Аналогично косинус.
\end{proof}
\begin{remark}
  Интервал $(a,b)$ при исследовании абсолютно интегрируемых
  на другой промежуток $[a,b],[a,b),(a,b]$
\end{remark}

% lecture 2

\section{Тригонометрические ряды Фурье}
\begin{definition}
  Ряд вида $\frac{a_0}{2}+\sum_{n=1}^{\infty}a_n\cos(nx)+b_n\sin(nx)$
  называется тригонометрическим рядом, где $a_k,b_k\in\R$
\end{definition}
\begin{definition}
  Множество функций $\{u_n(x)\}=\{1/2, \cos(x), \sin(x),$ $ \cos(2x), \sin(2x),\dots\} $
  называется тригонемтрической системой
\end{definition}
Свойства тригоном. сист.
\begin{enumerate}
  \item Триг. сист. "ортогональна" в смысле $\int_{-\pi}^{\pi}u_n(x)u_k(x)dx=0$,
    $\forall n,k:n\neq k$
  \item $\int_{-\pi}^{\pi}u_n^{2}(x)dx=\pi$, при $n \ge 2$
\end{enumerate}
\begin{proof} 1) Например
      \[
        \int_{-\pi}^{\pi}\sin nx \cos kx dx = 
        \frac{1}{2}\int_{-\pi}^{\pi}(\sin(n-k)x+\sin(n+k)x)dx=0
      \]
\end{proof}
\begin{proof} 2)
  Например \[
    \int_{-\pi}^{\pi}\cos^{2}(x)dx=\int_{-\pi}^{\pi}\frac{1+\cos(2nx)}{2}dx=\pi + 0=\pi
  \]
\end{proof}

\begin{lemma} \label{l2_f_lemma}
  Пусть
  \begin{equation} \label{l2_fourier}
    f(x)=\frac{a_0}{2}+\sum_{n=1}^{\infty}a_n\cos nx+b_n\sin nx
  \end{equation}
  и ряд сходится равномерно тогда:
  \begin{equation}
    \begin{gathered} \label{l2_f_coef}
      a_n=\frac{1}{\pi}f(x)\int_{-\pi}^{\pi}\cos nx dx, \; n=0,1,2,\dots \\ 
      b_n=\frac{1}{\pi}\int_{-\pi}^{\pi}f(x)\sin nx dx, \; n\in\mathbb{N} 
    \end{gathered}
  \end{equation}
\end{lemma}
\begin{proof}
  Домножим \ref{l2_fourier} на $\cos mx$. 
  Полученный ряд будет равномерно сходится.

  \hr
  \begin{gather*}
    \left|\sum_{k=1}^{n+p}\cos mx(a_k\cos kx + b_k \sin kx)\right|= 
    |\cos mx|\cdot\left|\sum_{k=n+1}^{n+p}a_k\cos kx + b_k\sin kx\right| \le  \\
    \left|\sum_{n+1}^{n+p}a_k\cos kx + b_k \sin kx\right|
  \end{gather*}
  Из выполнения усл. Коши равномерной сходимости для исходного ряда \ref{l2_fourier}
  следует выполнение усл. Коши равномерной сходимости полученного в результате умножения ряда.

  \hr
  Тогда имеем право интегрировать равнество (по $x$ от $-\pi$ до $\pi$)
  \begin{gather*}
    f(x)\cos mx = \frac{a_0}{2}\cos mx + \sum_{n=1}^{\infty}\cos mx (a_n \cos nx + b_n \sin nx) \\
    \implies \int_{-\pi}^{\pi}f(x)\cos mx dx = a_m \pi \\ 
    \implies a_m=\frac{1}{\pi}\int_{-\pi}^{\pi}f(x)\cos mx dx
  \end{gather*}
  Второе равество в \ref{l2_f_coef} получается аналогично.
\end{proof}
\begin{definition}
  Пусть $f(x)$ - $2\pi$ периодическая абсолютно интегрируемая на $[-\pi;\pi]$ функция.
  Тригонометрический ряд с коэффицентами \ref{l2_f_coef} называется
  тригонометрическим рядом Фурье функции $f(x)$, а коэффициенты $a_k,b_k$ - коэффициетами Фурье.
  Имеет место запись (здесь $\sim$ означает соответствие):
  \[
    f(x)\sim \frac{a_0}{2}+\sum_{n=1}^{\infty}a_n\cos nx + b_n\sin nx
  \]
\end{definition}
Перефразируем лемму:
\begin{lemma}[\ref{l2_f_lemma}']
  Рамномерно сходящийся тригонометрический ряд является рядом Фурье своей суммы. 
\end{lemma}
\begin{eg}
  \[
    \sum_{n=1}^{\infty}\frac{\cos nx}{n^{\alpha}}
  \]
  где $\alpha>1$ является рядом Фурье своей суммы, так  как он
  равномерно сходдится по признаку Веерштрасса
\end{eg}
\begin{remark}
  Если функция абсолютно интегрируема на $[-\pi,\pi]$, то интегралы
  в \ref{l2_f_coef} сходятся абсолютно по \ref{l1_abs_prod}
\end{remark}
\begin{remark}
  Если $f(x)$ - $2\pi$ периодична и абс. инт. на каком-либо $[a-\pi,a+\pi]$,
  то она будет абс. инт. на $\forall$ другом таком отрезке и
  интегралы (\ref{l2_f_coef}) не зависят от отрезка.
\end{remark}
\begin{remark}
  Любую абсолютно интегрируемую на $[a-\pi,a+\pi]$($(a-\pi,a+\pi);(a-\pi,a+\pi], [a-\pi,a+\pi)$)
  можно продолжить до $2\pi$ периодической функции,
  возможно доопределив или переопределив функцию в граничных точках.
  Интегралы при этом не меняются.
\end{remark}
\begin{corollary}
  Пусть $\{a_k\}, \{b_k\}$ - посл. коэфф. Фурье $2\pi$ периодической 
  и абсолютно инт. на $[-\pi,\pi]$ функции \[
    \implies \lim\limits_{k\to\infty} a_k=0 \qquad \lim\limits_{k\to\infty} b_k=0
  \]
  По \ref{l1_abs_prod} 
  $f(x)\cos nx$, $f(x) \sin nx $ - абс. инт.. По Т. Римана получаем нужный результат.
\end{corollary}
\begin{eg}
  \[
    \sum_{n=1}^{\infty}\sin nx
  \]
  Не может быть рядом Фурье какой-либо абсолютно инт. на $[-\pi,\pi]$ функции.
\end{eg}

\subsection{Ядро Дирихле. Принцип локализации}
Пусть $f(x)$ - $2\pi$ периодическая и абсолютно интегрируема на $[-\pi,\pi]$ и
\[
  f(x) \sim \frac{a_0}{2}+\sum_{n=1}^{\infty}a_n \cos nx + b_n \sin nx
\]
Рассмотрим частичные суммы ряда Фурье
\[
  S_n(x)=S_n(x,b)=\frac{a_0}{2}+\sum_{k=1}^{n}a_k \cos kx + b_k \sin kx
\]
Преобразуем:
\begin{gather*}
  S_n(x)= \frac{1}{2\pi}\int_{-\pi}^{\pi}f(x)dt+ \\ 
  +\sum_{k=1}^{n}\frac{1}{\pi}\int_{-\pi}^{\pi}f(t)\cos kx dt \cdot \cos kx +
  \frac{1}{\pi}\int_{-\pi}^{\pi}f(t)\sin kt dt \cdot \sin kx =\\
  \frac{1}{\pi}\int_{-\pi}^{\pi}f(t)
  \left(\frac{1}{2}+\sum_{k=1}^{n}\cos k(t-x)\right)dt \\ 
  S_n(x)=\frac{1}{\pi}\int_{-\pi}^{\pi}f(t)D_n(t-x)dt
\end{gather*}
\begin{definition}
  Функция 
  \[
    D_n(t)=\left(\frac{1}{2}+\sum_{k=1}^{n}\cos k(t)\right)
  \]
  называется ядром Дирихле
\end{definition}
\incfig{l2_dirichle}
Свойства ядра Дирихле:
\begin{enumerate}
  \item $D_n(t)$ - четная, $2\pi$ период. и непр. функция
  \item $\int_{-\pi}^{\pi}D_n(t)dt=\pi$
  \item $\max|D_n(t)|=\max D_n(t)=D_n(0)=n+\frac{1}{2}$
  \item $D_n(t)=\cfrac{\sin (n+\frac{1}{2})t}{2\sin \frac{t}{2}}$, при $t\neq 2\pi m, m\in \mathbb{Z}$
\end{enumerate}
\begin{proof} \phantom{.} 

  \begin{enumerate}
    \item Следует из аналогичных свойств слогаемых
    \item Очев
    \item $\frac{1}{2}-n\le D_n(t) \le \frac{1}{2}+n = D_n(0)$
    \item
      \begin{gather*}
        D_n(t)=\frac{1}{2}+\cos t + \cos 2t + \dots + \cos nt = \\ 
        = \cfrac{\sin \frac{t}{2}+2\sin\frac{t}{2}\cos t + 2\sin\frac{t}{2}\cos 2t + \dots + 2\sin \frac{t}{2}\cos nt}{2\sin\frac{t}{2}}= \\
        =\cfrac{\sin \frac{t}{2}-\sin \frac{t}{2}+\sin \frac{3t}{2} - \sin \frac{3t}{2} + \dots - \sin (n-\frac{1}{2})t + \sin (n+\frac{1}{2})t}{\sin\frac{t}{2}}= \\ 
        =\cfrac{\sin (n+\frac{1}{2})t}{2\sin\frac{t}{2}} 
      \end{gather*}
  \end{enumerate}
\end{proof}
\begin{theorem}[Принцип локализации]
  Пусть $f(x)$ - $2\pi$ периодическая абсолютно интегрируема на $[-\pi,\pi]$ функция.
  Пусть $x_0\in\R$, $0<\delta<\pi$. Тогда: $\lim\limits_{n\to\infty}S_n(x_0)$ и
  $\lim\limits_{n\to\infty}\frac{1}{\pi}\int_{0}^{\delta}D_n(t)(f(x_0+t)+f(x_0-t))dt$
  существуют или нет одновременно. В случае существования равны.
\end{theorem}
\begin{remark}
  Таким образом сходимость и значение суммы ряда Фурье $2\pi$ пер. и абс. инт.
  на $[-\pi,\pi]$ зависит только от свойств функции в сколь угодно малой окрестности.
\end{remark}
\begin{proof}
  Преобр. $S_n$:
  \begin{gather*}
    S_n(x_0)\underset{t=\tau+x_0}{=}\frac{1}{\pi}\int_{-\pi-x_0}^{\pi-x_0}f(x_0+\tau)D_n(\tau)d\tau 
  \end{gather*}
  \begin{equation}\label{l2_3a}
    S_n(x_0)=\frac{1}{\pi}\int_{-\pi}^{\pi}f(x_0+\tau)D_n(\tau)d\tau 
  \end{equation}
  \begin{gather*}
    S_n(x_0)=\frac{1}{\pi}(\underbrace{\int_{-\pi}^{0}}_{\tau=-t}+\underbrace{\int_{0}^{\pi}}_{\tau=t})f(x_0+\tau)D_n(\tau)d\tau = \\ 
    =\frac{1}{\pi}(\int_{0}^{\pi}f(x_0-t)D_n(-t)dt + \int_{0}^{\pi}f(x_0+t)D_n(t)dt)
  \end{gather*}
  \begin{equation}\label{l2_3b}
    S_n(x_0)=\frac{1}{\pi}\int_{0}^{\pi}D_n(t)(f(x_0-t)+f(x_0+t))dt
  \end{equation}
  \begin{gather*}
    S_n(x_0)=\frac{1}{\pi}(\int_{0}^{\delta}+\int_{\delta}^{\pi})\frac{\sin(n+\frac{1}{2}t)}{2\sin \frac{t}{2}}(f(x_0-t)+f(x_0+t))dt \\
    \frac{1}{2\sin\frac{t}{2}} \le \frac{1}{2\sin \frac{\delta}{2}} \qquad \text{на} [\delta,\pi]
  \end{gather*}
  Тогда $\frac{(f(x_0-t)+f(x_0+t))}{2\sin \frac{t}{2}}$ - абс. инт на $[\delta,\pi]$ (см \ref{l1_abs_prod}).

  Тогда 2-ой инт. $\to0$ при $n\to\infty$ (Т. Римана) и получаем нужный результат.
\end{proof}

% lecture 3 

\subsection{Признаки сходимости ряда Фурье в точке}
\begin{theorem}[Признак Дини]
  Пусть $f(x)$ - $2\pi$ периодическая и абсолютно инт. на $[-\pi,\pi]$ функция.
  Пусть  в точке $x_0$ существуют $f(x_0+0)$ и $f(x_0-0)$.
  Пусть для некоторого $\delta>0$ $\int_{0}^{\delta}\frac{|f^*_{x_0}(t)|}{t}dt$
  (где $f^*_{x_0}(t)=f(x_0+t)+f(x_0-t)-f(x_0+0)-f(x_0+0)$) сходится.
  Тогда ряд Фурье сходится в точке $x_0$ к значению $\frac{f(x_0+0)+f(x_0-0)}{2}$.
\end{theorem}
\begin{proof}
  Рассматриваем
  \begin{gather*}
    S_n(x_{0})-\frac{f(x_0+0)+f(x_0-0)}{2}= \\
     \frac{1}{\pi}\int_{0}^{\delta}D_n(t)(f(x_0+t)+f(x_0-t))dt-\frac{f(x_0+0)+f(x_0-0)}{2}\frac{2}{\pi}\int_{0}^{\pi}D_n(t)dt = \\ 
    = \frac{1}{\pi}\int_{0}^{\pi}f^*_{x_0}(t)D_n(t)dt=\frac{1}{\pi}\int_{0}^{\pi}f^*_{x_0}(t)\frac{\sin(n+.5)t}{2\sin.5t}dt = \\
    = \frac{1}{\pi}\int_{0}^{\pi}\frac{f^*_{x_0}(t)}{t}\frac{.5t}{\sin .5t}\sin(n+.5)t dt
  \end{gather*}
  Функция $f^{*}_{x_0}$ абс. инт. на $[0,\pi]$ (т.к. по сравнению с абс. инт функциями
  $f(x_0+t)$ и $f(x_0-t)$) у функции $f^{*}_{x_0}(t)$ появивлась лишь одна
  особенность при $t=0$, в окр. кот. функция по условию абс. инт.
  Функция $\frac{t/2}{\sin t/2}$ доопределима в до непрерывной и ограниченной на
  $[0,\pi]$ функции.

  По лемме \ref{l1_abs_prod} 
  $\frac{f_{x_0}^{*}}{t}\frac{t/2}{\sin t/2}$ - абс. инт. на $[0,\pi]$.
  По Т. Римана \ref{l1_r_osc} 
  интеграл $\to 0$ при $n \to \infty$
  \[
    \implies S_n(x_0) \underset{n\to\infty}{\to}\frac{f(x_0+0)+f(x_0-0)}{2}
  \]
\end{proof}
\begin{definition}
  Функция $f(x)$ удовлетворяет в точке $x_0$ правостороннему (левостороннему)
  условия Гёльдера с показателем $\alpha$ ($\alpha \in (0,1]$)
  если $\exists \delta>0, M>0 :\, \forall t\in(0,\delta) \true |f(x_0\underset{(-)}{+}t)-f(x_0\underset{(-)}{+}0)|<Mt^{\alpha}$.
\end{definition}
\begin{remark}
  При $\alpha$ это условие называется также условием Липшица.
\end{remark}
\begin{definition}
  Пусть $\exists$ $f(x_0+0)$ и $f(x_0-0)$. Введём обобщение односторонней производной
  \[
    f_{\pm}'(x_0)=\lim\limits_{t\to \pm 0} \frac{f(x_0+t)-f(x_0\pm0)}{t}
  \]
\end{definition}
\begin{lemma}
  Если $\exists$ (конечная) $f_{\underset{(-)}{+}}'(x_0)$, то $f(x)$ удовл.
  правостороннему (левостороннему) усл. Липшица.
\end{lemma}
\begin{proof}
  (для $f'(x_0)$)

  $\exists f_{+}'(x_0)=\lim\limits_{t\to+0}\frac{f(x_0+t)-f(x_0+0)}{t}$
  $\implies \frac{f(x_0+t)-f(x_0+0)}{t}=f_{+}'(x_0)+\underset{t\to +0}{o(1)}$

  $\implies$ В некоторой окр. $(0,\delta)$ выполнится
  $\frac{|f(x_0+t)-f(x_0+0)|}{t}\le |f_{+}'(x_{0})|+1$

  $\implies$ $|f(x_{0}+t)-f(x_{0}+0)|\le (|f_{+}'(x_{0})|+1)t$
\end{proof}
\begin{theorem}[Признак Липшица]
  Пусть $f(x)$ - $2\pi$ периодическая и абсолютно инт. на $[-\pi,\pi]$ функция.
  Пусть  в точке $x_0$ у функции $f(x)$ выполняются оба условия Гёльдера.
  Тогда ряд Фурье сходится в $x_{0}$ к значению $\frac{f(x_{0}+0)-f(x_{0}-0)}{2}$.
\end{theorem}
\begin{proof}
  Пусть выполняются оба условия Гёльдера.
  Тогда на $(0,\delta)$ выполняются:
  \[
    \frac{|f^{*}_{x_{0}}|}{t}\le \frac{|f(x_0+t)+f(x_0+0)|}{t}+\frac{|f(x_0-t)+f(x_0-0)|}{t}
    \le \frac{2Mt^{\alpha}}{t}=\underbrace{\frac{2M}{t^{1-\alpha}}}_{\text{абс.инт.}}
  \]
  $\implies$ по признаку сравнения $\int_{0}^{\delta}\frac{|f_{x_{0}}^{*}(t)|dt}{t}$ сход

  $\implies$ по признаку Дини  $S_n(x_{0})\underset{n\to\infty}{\to}\frac{f(x_{0}+0)-f(x_{0}-0)}{2}$
\end{proof}
\begin{corollary}
  Пусть $f(x)$ - $2\pi$ периодическая и абсолютно инт. на $[-\pi,\pi]$ функция.
  Пусть $\exists f(x_{0}\pm 0)$ и $f'_{\pm}(x_{0})$.

  $\implies$ $\lim\limits_{n\to \infty}S_n(x_{0})=\frac{f(x_{0}+0)-f(x_{0}-0)}{2}$.
\end{corollary}
\begin{proof}
  Следует из признака Липшица и леммы.
\end{proof}

\section{Суммирование рядов методом ср. арифм.}
Пусть $f(x)$ - $2\pi$ периодическая и абсолютно инт. на $[-\pi,\pi]$ функция.
\begin{definition}
  $\sigma_{n}(x)=\frac{S_0(x)+S_1(x)+\dots +S_n(x)}{n+1}$
  - сумма Фейера, где $S_k(x)$ - частичная сумма ряда Фурье.
\end{definition}
\begin{definition}
  $\Phi_n(x)=\frac{D_0(x)+D_1(x)+\dots +D_n(x)}{n+1}$ - ядро Фейера,
  где $D_k(x)$ - ядра Дирихле.
\end{definition}
Из формулы \ref{l2_3a} (принцип локализации) 
$S_n(x)=\frac{1}{\pi}\int_{-\pi}^{\pi}D_n(t)f(x+t)dt$
следует $\sigma_n(x)=\frac{1}{\pi}\int_{-\pi}^{\pi}\Phi_n(t)f(x+t)dt$.

\incfig{l3_phi_n}
Свойства ядра Фейереа:
\begin{enumerate}
  \item $\Phi_n(t)$ - четная, $2\pi$ периодичекая, непр функция
  \item $\int_{-\pi}^{\pi}\Phi_n(t)dt=\pi$
  \item $\max \Phi_n(t)=\Phi_n(0)=\frac{n+1}{2}$
  \item $\Phi_n(t)$ - неотр.
  \item $\Phi_n(t)=\frac{\sin^{2}\frac{n+1}{2}t}{2(n+1)\sin^{2}\frac{t}{2}}$
    при $t\neq 2\pi m$, $m\in \mathbb{Z}$
\end{enumerate}
\begin{proof} \phantom{.}

  \begin{enumerate}
    \item из св. ядра Дирихле
    \item из св. ядра Дирихле
    \item Т.к. $\max D_n(t)=D_n(0)=n+\frac{1}{2}$, то
      \begin{gather*}
        \max \Phi_n(t)=\Phi_n(0)=\frac{1}{n+1}(D_0(0)+D_1(0)+\dots +D_n(0))= \\
        =\frac{1}{n+1}\left(\frac{1}{2}+\left(\frac{1}{2}+1\right)+\dots +\left(\frac{1}{2}+n\right)\right)= \\
        =\frac{1}{n+1}\frac{\frac{1}{2}+\frac{1}{2}+n}{2}(n+1)=\frac{n+1}{2}
      \end{gather*}
    \item из 5)
    \item
      \begin{gather*}
        (n+1)\Phi_n=D_0(t)+D_1(t)+\dots +D_n(t)=\\
        =\frac{\sin \frac{t}{2}+\sin \frac{3}{2}t + \dots +\sin(n+\frac{1}{2})t}{2\sin \frac{t}{2}} = \\
        =\frac{2\sin \frac{t}{2} \sin \frac{t}{2}+2\sin\frac{3}{2}t\sin\frac{t}{2}+\dots +2\sin(n+\frac{1}{2})t\sin\frac{t}{2}}{4\sin^{2}\frac{t}{2}} = \\
        =\frac{\cos 0 - \cos t + \cos t - \cos 2t +\dots + \cos nt - \cos (n+1)t}{4\sin^{2}\frac{t}{2}} = \\ 
        =\frac{1-\cos (n+1)t}{4\sin^{2}\frac{t}{2}}=\frac{\sin^{2}\frac{n+1}{2}t}{2\sin^{2}\frac{t}{2}}
      \end{gather*}
  \end{enumerate}
\end{proof}

% lecture 4

\begin{theorem}[Фейера]
  Пусть $f(x)$ - $2\pi$ периодическая, непрерывная
  $\implies \sigma_n(x) \underset{\R}{\rightrightarrows} f(x)$
\end{theorem}
\begin{proof}
  \begin{gather*}
    |\sigma_n(x)-f(x)|=\frac{1}{\pi}\int_{-\pi}^{\pi}\Phi_n(t)f(x+t)dt-f(x)\int_{-\pi}^{\pi}\Phi_n(t)dt| = \\ 
  = \frac{1}{\pi}|\int_{-\pi}^{\pi}\Phi_n(t)(f(x+t)-f(x))dt|
  \le \frac{1}{\pi}\int_{-\pi}^{\pi}\Phi_n(t)|f(x+t)-f(x)|dt
  \end{gather*}
  Т.к. $f(x)$ - непр., $2\pi$ период, то она ограничена и существует $C>0:|f(x)|\le C$.
  Также $f(x)$ равномерно непрерывна на $\R$.
  
  Возьмём $\forall \epsilon>0$ (в силу равн. непр.) 
  \begin{gather*}
    \exists \delta \in (0,\pi): \forall x, x':|x-x'|<\delta \true |f(x)-f(x')|<\frac{\epsilon}{3} \\ 
    \frac{1}{\pi}\int_{-\pi}^{\pi}\Phi_n(t)|f(x+t)-f(x)|dt=\underbrace{\frac{1}{pi}\int_{-\pi}^{-\delta}\dots dt}_{I_1} + \underbrace{\frac{1}{\pi}\int_{-\delta}^{\delta}\dots dt}_{I_2} + \underbrace{\frac{1}{\pi}\int_{\delta}^{\pi}\dots dt}_{I_3} \\ 
    I_2= \frac{1}{\pi}\int_{-\delta}^{\delta}\Phi_n(t)|f(x+t)-f(x)|dt < \frac{\epsilon}{3}\frac{1}{\pi}\int_{-\delta}^{\delta}\Phi_n(t)dt 
    < \frac{\epsilon}{3}\frac{1}{\pi}\int_{-\pi}^{\pi}\Phi_n(t)dt < \frac{\epsilon}{3} \\ 
    I_3=\frac{1}{\pi}\int_{\delta}^{\pi}\Phi_n(t)|f(x+t)-f(x)|dt \le 
    \frac{1}{\pi}\int_{\delta}^{\pi}\Phi_n(t)(|f(x+t)|+|f(x)|)dt \le  \\
    \le \frac{2C}{\pi}\int_{\delta}^{\pi}\Phi_n(t)dt \le
    \frac{2C}{\pi}\pi \max \limits_{[\delta,\pi]} \Phi_n(t) \le 
    2C\frac{1}{2(n+t)\sin^{2}\frac{\delta}{2}} \underset{n\to\infty}{\to}0
  \end{gather*}
  Тогда $\exists n_3: \forall n \ge n_3 \true I_3<\frac{\epsilon}{3} \forall x$.
  Аналогично $\exists n_1: \forall n \ge n_1 \true I_1<\frac{\epsilon}{3} \forall x$.
  $\underline{\exists n_0}=\max{n_1,n_3}: \underline{\forall n \ge n_0 \true |\sigma_n(t)-f(x)|<\epsilon \, \forall x}$.

  Подчёркнутое означает $\sigma_n(t)\underset{\R}{\rightrightarrows}f(x)$
\end{proof}
\begin{corollary}
  Если ряд Фурье непр., $2\pi$ периодической функции сходится в точке $x$,
  то он сходится к $f(x)$.
\end{corollary}
\begin{proof}
  Пусть $\lim \limits_{n\to\infty} S_n(t)=A \implies \lim \limits_{n\to\infty}\sigma_n(x)=A$.

  По Т. (Фейера) $\lim\limits_{n\to\infty}\sigma_n(x)=f(x)$
  $\implies A=f(x) \implies \lim\limits_{n\to\infty}S_n(t)=f(x)$.
\end{proof}

\section{Приближение непр. функ. многочленами}
\begin{definition}
  Функция вида
  \[
    T(x)=\frac{A_0}{2}+\sum_{k=1}^{n}A_k\cos kx + B_k \sin kx
  \]
  называются тригонометрическим многочленом.
\end{definition}
\begin{theorem}[Т1 Веерштрасса]
  Пусть $f(x)$ - $2\pi$ период., непр функция.

  Тогда $\forall \epsilon>0 \exists$ триг. многочлен $T(x): \max \limits_{x\in \R} |f(x)-T(x)| < \epsilon$.
\end{theorem}
\begin{proof}
  Т.к. $\sigma_n(x)\underset{x\in \R}{\rightrightarrows} f(x)$,
  то $\underline{\forall \epsilon >0} \exists N : \forall n\ge N, \forall x\in \R \true |\sigma_n(x)-f(x)|<\epsilon$
  и $\underline{\exists T(x)}=\sigma_n(x)\underline{:\max\limits_{x\in \R} |T(x)-f(x)|<\epsilon}$.
\end{proof}
\begin{theorem}[Т1' (перефразирование)]
  Пусть $f(x)$ - непр на $[-\pi,\pi]$ и $f(-\pi)=f(\pi)$
  $\implies \forall \epsilon > \exists$ триг. многочлен $T(x): \max \limits_{x\in \R} |f(x)-T(x)| < \epsilon$.
\end{theorem}
\begin{proof}
  Такая функция продолжаема до $2\pi$ периодической, непр. функции, можем применять Т1.
\end{proof}
\begin{theorem}[Т2 Веерштрасса]
  Пусть $f(x)$- непр. на $[a,b]$.
  $\implies \forall \epsilon>0 \exists$ алгебраический многочлен
  $P(x):\max \limits_{x\in[a,b]}|f(x)-P(x)|<\epsilon$.
\end{theorem}
\begin{proof}
  Отобразим отрезок $[0,\pi]$ на отрезок $[a,b]$:
  $x=a+\frac{b-a}{\pi}t$, обозначим $f^{*}(t)=f(a+\frac{b-a}{\pi}t)$, $t\in[0,\pi]$.

  Продолжим $f^{*}(t)$ чётно на $[-\pi,\pi]$ и $2\pi$ периодически на $\R$,
  сохранив обозначение $f^{*}(t)$.

  По Т1 $\forall \epsilon>0 \exists$ тр. мн.
  $T(t): \max \limits_{t\in[0,\pi]}|f^{*}(t)-T(t)| \le \max \limits_{t\in \R}|f^{*}(t)-T(t)| < \frac{\epsilon}{2}$.

  Ряды Тейлора для $\cos kt$ и $\sin kt$ и, следовательно, для $T(t)$
  имеют радиус сходимости $+\infty$, и следовательно равномерно сходятся на любом отрезке.
  
  Таким образом $\exists$ алг. многочлен $P(t):\max\limits_{t\in[0,\pi]} |T(t)-P(t)|<\frac{\epsilon}{2}$.

  Тогда $\max \limits_{t\in[0,\pi]}|f^{*}(t)-P(t)|< \epsilon$ или 
  $\max \limits_{x\in[a,b]} |f(x)-P(\pi\frac{x-a}{b-a})|<\epsilon$.
\end{proof}
\subsection{Минимальное св-во коэфф. Фурье}
\begin{lemma}
  Если $f(x)$ инт (в несобственном смысле) на $[a,b]$ вместе с квадратом $f^{2}(x)$.
  $\implies$ $f(x)$ абс. инт на $[a,b]$.
\end{lemma}
\begin{proof}
  Следует из неравенства $|f(x)|\le \frac{1+f^{2}}{2}$.
\end{proof}
\begin{remark}
  \[
    \exists f=\left\{\begin{aligned}
      &0, & x=0 \\ 
      &\frac{1}{\sqrt{x}}, & x\in (0, 1]
    \end{aligned}\right.
  \]
  абс. инт на $[0,1]$ но не явл. инт с кв.
\end{remark}
\begin{theorem}
  Пусть $f(x))$ - $2\pi$ период. и инт. вместе с квадратом функция на $[-\pi,\pi]$.

  Пусть $S_n(x)$ - частичные суммы ряда Фурье, $a_n$ и $b_n$ - коэф. Фурье.

  Тогда:
  \begin{enumerate}
    \item $\int_{-\pi}^{\pi}(f(x)-S_n(x))^{2}dx=\min\limits_{T_n(x)}\int_{-\pi}^{\pi}(f(x)-T_n(x))^{2}dx$,
      где $T_n(x)$ - триг. многочлены степени не выше $n$.
    \item $\frac{a_0^{2}}{2}+\sum_{n=1}^{\infty}a^{2}_n+b^{2}_{n}\le \frac{1}{\pi}\int_{-\pi}^{\pi}f^{2}(x)dx$
- неравенство Бесселя
  \end{enumerate}
\end{theorem}
\begin{proof}
  Пусть $T_n(x)=\frac{A_0}{2}+\sum_{k=1}^{n}A_k\cos kx + B_k\sin kx$.

  Рассмотрим
  \begin{gather*}
    \int_{-\pi}^{\pi}(f(x)-T_n(x))^{2}dx=\int_{-\pi}^{\pi}f^{2}(x)dx - 2\int_{-\pi}^{\pi}f(x)T_n(x)dx+\int_{-\pi}^{\pi}T_n^{2}(x)dx = \\ 
    = \int_{-\pi}^{\pi} f^{2}(x)dx -2\pi (\frac{a_0A_0}{2}+\sum_{k=1}^{n}a_kA_k+b_kB_k)+\pi(\frac{A_0^{2}}{2}+\sum_{k=1}^{n}A_k^{2}+B_k^{2})= \\ 
    = \int_{-\pi}^{\pi}f^{2}(x)dx-\pi\left(\frac{a_0^{2}}{2}+\sum_{k=1}^{n}a_k^{2}+b_k^{2}\right) +\\
    +\pi\left(\frac{(A_0-a_0)^{2}}{2}+\sum_{k=1}^{n}(A_k-a_k)^{2}+(B_k-b_k)^{2}\right)
  \end{gather*}
  Видно, что последнее выражение минимально при $A_k=a_k$, $B_k=b_k$,
  что доказывает $1$.

  Для доказательства $2$ возьмём $T_n=S_n$:
  \begin{gather*}
    \int_{-\pi}^{\pi}(f(x)-S_n(x))^{2}dx=\int_{-\pi}^{\pi}f^{2}(x)dx-\pi\left(\frac{a_0}{2}+\sum_{k=1}^{n}a_k^{2}+b_k^{2}\right) \ge 0\\
    \int_{-\pi}^{\pi}f^{2}(x)dx \ge \pi\left(\frac{a_0}{2}+\sum_{k=1}^{n}a_k^{2}+b_k^{2}\right)
  \end{gather*}
  Частичн. суммы ряда $\sum_{k=1}^{\infty}a_k^{2}+b_k^{2}$ составляют неубывающую,
  ограниченную последовательность, следовательно ряд сходится.

  Переходим в последнем неравенстве к пределу при $n\to\infty$ и получаем 2.
\end{proof}

% lecture 5

\begin{theorem} [Равентсво Парсеваля]
  Пусть $f(x)$ - $2\pi$ периодична, непрерывна,
  $a_n$ и $b_n$ её коэф. Фурье, тогда
  \[
    \frac{1}{\pi}\int_{-\pi}^{\pi}f^{2}(x)dx=\frac{a_0^{2}}{2}+\sum_{n=1}^{\infty}a_n^{2}+b_n^{2}
  \]
\end{theorem}
\begin{remark}
  Равенство верно и для интегрируемой функции вместе с квадратом.
\end{remark}
\begin{remark}
  Равенство Парсеваля получается при формальной подстановке $S(x)$ вместо $f(x)$.
\end{remark}
\begin{proof}
  Возьмём $\underline{\forall \epsilon > 0}$. По Т. Вейерштрасса
  $\exists T_n(x)$ - триг. многочлен, такой что
  $\max \limits_{[-\pi,\pi]}|f(x)-T_n(x)|<\frac{\sqrt{\epsilon}}{2}$, тогда
  \[
    \frac{1}{\pi}\int_{-\pi}^{\pi}(f(x)-T_n(x))^{2}dx<\epsilon
  \]
  Из минимального свойства коэф. Фурье:
  \[
    \frac{1}{\pi}\int_{-\pi}^{\pi}(f(x)-S_n(x))^{2}dx
    \le \frac{1}{\pi}\int_{-\pi}^{\pi}(f(x)-T_n(x))^{2}dx < \epsilon \tag{$1$}
  \]
  Тогда:
  \begin{gather*}
    \underline{0\overset{\text{нер. Б.}}{\le }\frac{1}{\pi}\int_{-\pi}^{\pi}f^{2}(x)dx-\left[\frac{a_0^{2}}{2}+\sum_{n=1}^{\infty}a_n^{2}+b_n^{2}\right]} \le \\
    \le \frac{1}{\pi}\int_{-\pi}^{\pi}f^{2}(x)dx- \left[\frac{a_0^{2}}{2}+\sum_{k=1}^{n}a_k^{2}+b_k^{2}\right]
    = \frac{1}{\pi}\int_{-\pi}^{\pi}(f(x)-S_n(x))^{2}dx \underline{\overset{(1)}{<} \epsilon}
  \end{gather*}
  Из подчёркнутого получаем равенство Парсвеваля.
\end{proof}

\begin{definition}[Кусочно непр. дифф.]
  Функция $f(x)$ называется кусочно непрерывно дифф. на отрезке $[a,b]$,
  если $\exists$ такое разбиение отрезка, что на каждом отрезке разбиения,
  функция непрерывно дифф. (в концевых точках односторонне).
\end{definition}
\begin{theorem}[О почленном дифф. ряда Фурье]
  Пусть $f(x)$ - $2\pi$ период., непр на $\R$ и кусочно непр. дифф. на $[-\pi,\pi]$.

  Пусть $f(x) \sim \frac{a_0}{2}+\sum_{n=1}^{\infty}a_n\cos nx + b_n \sin nx$.

  Тогда $f'(x) \sim \sum_{n=1}^{\infty} nb_n\cos nx - na_n\sin nx$.
  (т.е. ряд можно формально дифф.)
\end{theorem}
\begin{remark}
  О сходимости ничего не говорится.
\end{remark}
\begin{proof}
  Пусть $f'(x)\sim \frac{\alpha_0}{2}+\sum_{n=1}^{\infty}\alpha_n\cos nx + \beta_n\sin nx$
  \begin{gather*}
    \alpha_0=f(\pi) - f(-\pi)= 0 \\ 
    \alpha_n=\frac{1}{\pi}\int_{-\pi}^{\pi}f'(x)\cos nx dx = \frac{1}{\pi}\int_{-\pi}^{\pi}\cos nx df(x) \\ 
    = \underbrace{\frac{1}{\pi}f(x)\cos nx \Big|_{-\pi}^{\pi}}_{0}- \frac{1}{\pi}\int_{-\pi}^{\pi}f(x)(-n)\sin nx df = nb_n \\
    \beta_n=\frac{1}{\pi}\int_{-\pi}^{\pi}f'(x)\sin nx dx = \frac{1}{\pi}\int_{-\pi}^{\pi}\sin nx df(x) \\ 
    = \underbrace{\frac{1}{\pi}f(x)\sin nx \Big|_{-\pi}^{\pi}}_{0}- \frac{1}{\pi}\int_{-\pi}^{\pi}f(x)(-n)\cos nx df = -na_n
  \end{gather*}
\end{proof}

\begin{lemma}[О порядке убывания коэф. Фурье]
  Пусть $f(x)$ - $2\pi$ периодична и непр. на $\R$.

  Пусть $f(x)$ имеет непр. производную до порядка $k-1$ включительно на $\R$
  и кусочно непр. производную порядка $k$ ($k\ge 1$) на $[-\pi,\pi]$.

  Пусть $f(x) \sim \frac{a_0}{2}+\sum_{n=1}^{\infty}a_n\cos nx + b_n \sin nx$.

  $\implies$ $|a_n| \le \frac{\epsilon_n}{n^{k}}$, $|b_n| \le \frac{\epsilon_n}{n^{k}}$,
  где $\sum_{n=1}^{\infty}\epsilon_n^{2}$ - сход.
\end{lemma}
\begin{proof}
  Пусть $f^{*}(x)\sim \sum_{n=1}^{\infty}\alpha_n\cos nx + \beta_n \sin nx$.

  Применяем предыдущую Т. $k$ раз, получим
  либо
  \[
    \alpha_n=\pm n^{k}a_n \qquad
    \beta_n=\pm n^{k} b_n \tag{$1$}
  \]
  либо
  \[
    \alpha_n=\pm n^{k}b_n \qquad
    \beta_n=\pm n^{k} a_n \tag{$2$}
  \]
  При этом $\sum_{n=1}^{\infty}\alpha_n^{2}+\beta_n^{2} \le \frac{1}{\pi}\int_{-\pi}^{\pi}(f^{(k)})^{2}dx$
  (нер. Бесселя) и ряд $\sum_{n=1}^{\infty}\epsilon_n^{2}$ (где $\epsilon_n = \sqrt{\alpha_n^{2}+\beta_n^{2}}$)
  сходится.

  Если справедл $(1)$, то
  \[
    |a_n|=\frac{\alpha_n}{n^{k}} \le \frac{\alpha_n^{2}+\beta_n^{2}}{n^{k}}=\frac{\epsilon_n}{n^{k}} \qquad |b_n| \le \frac{\epsilon_n}{n^{k}}
  \]
  Аналогично в случае $(2)$.
\end{proof}

\begin{theorem}[О скорости сходимости ряда Ф. к функ.]
  Пусть $f(x)$ - $2\pi$ периодична и непр. на $\R$.

  Пусть $f(x)$ имеет непр. производную до порядка $k-1$ включительно на $\R$
  и кусочно непр. производную порядка $k$ ($k\ge 1$) на $[-\pi,\pi]$.
  
  Тогда ряд Фурье равномерно и абсолютно сходится к $f(x)$ на $\R$
  и выполняется 
  \[
    |f(x)-S_n(x)| \le \frac{\eta_n}{n^{k-.5}}
  \]
  гле $\lim \limits_{n\to \infty} \eta_n =0$ и $\{\eta_n\}$ - числовая посл.
\end{theorem}
\begin{proof}
  Пусть $f(x) \sim \frac{a_0}{2}+\sum_{n=1}^{\infty}a_n\cos nx + b_n \sin nx$
  и $S_n(x)$ - сумма Фурье порядка $n$.

  Выполняются достаточные условия сходимости ряда Фурье к функции,
  т.е. $\lim \limits_{n\to \infty} S_n(x)=f(x)$.

  Рассмотрм остаток ряда Фурье
  \[
    r_n(x)=f(x)-S_n(x)=\sum_{m=n+1}^{\infty}a_m\cos mx + b_m \sin mx
  \]
  \hr
  При анализе остатка будем использовать неравенства Коши-Буняковского-Шварца
  для ряда
  \[
    \sum_{n=1}^{\infty}u_nv_n \le \sqrt{\sum_{n=1}^{\infty}u_n^{2}}\sqrt{\sum_{n=1}^{\infty}v_n^{2}} \tag{$1$}
  \]
  которое получается из нер. К.-Б.-Ш. для конечной суммы
  \[
    \sum_{n=1}^{N}u_nv_n \le \sqrt{\sum_{n=1}^{N}u_n^{2}}\sqrt{\sum_{n=1}^{N}v_n^{2}}
  \]
  после применения предельного перехода $N\to\infty$.

  Также будем использовать нер.
  \[
    \frac{1}{m^{p}} \le \int_{m-1}^{m}\frac{dx}{x^{p}}, \ p>0 \tag{$2$}
  \]
  которое получается инт. нерав. $\frac{1}{m^{p}}\le \frac{1}{x^{p}}$
  по отрезку $[m-1,m]$.

  \hr
  \begin{gather*}
    |r_n(x)| \le \sum_{m=n+1}^{\infty}|a_m \cos mx| +|b_m\sin mx|
    2\sum_{m=n+1}^{\infty}\frac{\epsilon_m}{m^{k}} \le \\
    \overset{(1)}{\le} \sqrt{\sum_{m=n+1}^{\infty}\epsilon_m^{2}}\sqrt{\sum_{m=n+1}^{\infty}\frac{1}{m^{2k}}}
    \overset{(1)}{\le} \sqrt{\sum_{m=n+1}^{\infty}\epsilon_m^{2}}\sqrt{\sum_{m=n+1}^{\infty}\int_{m-1}^{m}\frac{dx}{x^{2k}}} = \\
    = 2\sqrt{\sum_{m=n+1}^{\infty}\epsilon_m^{2}}\sqrt{\sum_{m=n+1}^{\infty}\int_{n}^{\infty}\frac{dx}{x^{2k}}}
    = \underbrace{\frac{2}{\sqrt{2k-1}}\sqrt{\sum_{m=n+1}^{\infty}\epsilon_m^{2}}}_{\eta_n \underset{n\to\infty}{\to}0}\sqrt{\frac{1}{n^{2k-1}}}
    = \frac{\eta_n}{n^{k-\frac{1}{2}}}
  \end{gather*}
  $\implies$ Ряд сходится равномерно, т.к. $\eta_n$ не зависит от $x$.

  Т.к. получилась оценка
  \[
    \sum_{m=n+1}^{\infty}|a_m\cos mx| + |b_m\sin mx| \le \frac{\eta_n}{n^{k-\frac{1}{2}}}
  \]
  из которой следует абсолютная сход. остатка, заключаем, что ряд сход. абс.
\end{proof}
\begin{corollary}[Достаточное усл. равн. сход. ряда]
  Пусть $f(x)$ - $2\pi$ период., непр на $\R$, имеет кус. непр.
  производную на $[-\pi,\pi]$

  $\implies$ Ряд Фурье равномерно на $\R$ и абс. сходится к $f(x)$.
\end{corollary}
\begin{proof}
  Следует из Т. при $k=1$.
\end{proof}

\begin{theorem}[О почленном инт. ряда Фурье]
  Пусть $f(x)$ - $2\pi$ период. и кусочно непр. на $[-\pi, \pi]$.

  Пусть $f(x) \sim \frac{a_0}{2}+\sum_{n=1}^{\infty}a_n\cos nx + b_n \sin nx$.

  Тогда:
  \begin{gather*}
    \int_{0}^{x}f(t)dt=\int_{0}^{x}\frac{a_0}{2}dt+\sum_{n=1}^{\infty}\int_{0}^{x}(a_n\cos nt + b_n\sin nt)dt = \\
    = \frac{a_0x}{2}+\sum_{n=1}^{\infty}\frac{a_n}{n}\sin nx + \frac{b_n}{n}(1-\cos nx)
  \end{gather*}
  и ряд в правой части сх. равн. на $\R$.

\end{theorem}

% lecture 6

\begin{proof}
  Введём $F(x)=\int_{0}^{x}f(t)dt-\frac{a_0x}{2}=\int_{0}^{x}(f(t)-\frac{a_0}{2})dt$,
  $F(x)$ - непр. на $\R$, её производная $F'(x)=f(x) - \frac{a_0}{2}$
  - кусочно непрерывная функция на $[-\pi,\pi]$.

  \[
    F(x+2\pi)=F(x)+\underbrace{\int_{x}^{x+2\pi}(f(t)-\frac{a_0}{2})dt}_{0}=F(x)
  \]
  т.е. $F(x)$ - $2\pi$ периодична

  $\implies$ ряд Фурье $F(x)$ сх. равн. к $F(x)$ на $\R$.
  \begin{gather*}
    F(x)=\frac{A_0}{2}+\sum_{1}^{\infty}A_n\cos nx + B_n\sin nx \tag{$1$} \\ 
    A_n=\frac{1}{\pi}\int_{-\pi}^{\pi}F(x)\cos nx dx=\frac{1}{\pi n}\int_{-\pi}^{\pi}F(x)d\sin nx =\\
    =\frac{1}{\pi n}F(x)\sin nx \Big|_{-\pi}^{\pi} - \frac{1}{\pi n}\int_{-\pi}^{\pi}(f(x)-\frac{a_0}{2})\sin nx dx 
    = -\frac{b_n}{n} \\ 
    B_n=\frac{a_n}{n}
  \end{gather*}
  Положим в $(1)$ $x=0$:
  \begin{gather*}
    \frac{A_0}{2}+\sum_{n=1}^{\infty}A_n=0 \implies \frac{A_0}{2}=\sum_{n=1}^{\infty}\frac{b_n}{n} \\ 
    F(x)=\sum_{n=1}^{\infty}\frac{a_n}{n}\sin nx + \frac{b_n}{n}(1-\cos nx)
  \end{gather*}
\end{proof}

\subsection{Запись р. Фурье в комплексной форме}
Рассмотрим $f(x)$ - $2\pi$ периодическую, абс. инт на $[-\pi,\pi]$ функцию
\[
  f(x) \sim \frac{a_0}{2}+\sum_{n=1}^{\infty}a_n\cos nx + b_n\sin nx
\]
Подставим $\cos nx =\frac{e^{inx}+e^{-inx}}{2}$ и $\sin nx = \frac{e^{inx}-e^{-inx}}{2}$ (ф. Эйлера)
\[
  f(x)\sim \frac{a_0}{2}+\sum_{n=1}^{\infty}\frac{a_n-ib_n}{2}e^{inx}+\frac{a_n+ib_n}{2}e^{-inx}
\]
Введём обозначения $c_0=\frac{a_{0}}{2}$; $c_n=\frac{a_n-ib_n}{2}$; $c_{-n}=\frac{a_n+ib_n}{2}$.

Тогда $f(x)\sim \sum_{n=-\infty}^{\infty}c_ne^{inx}$, где $S_n(x)=\sum_{k=-n}^{n}c_ke^{ikx}$.

Ряд сходится, если $\exists\lim \limits_{n\to\infty}S_n$.
\begin{gather*}
  c_n = \frac{1}{2\pi}\int_{-\pi}^{\pi}f(x)(\cos nx - i \sin nx)dx 
  = \frac{1}{2\pi}\int_{-\pi}^{\pi}f(x)e^{-inx}dx \\ 
  c_{-n}=\frac{1}{2\pi} \int_{-\pi}^{\pi}f(x)(\cos nx + i\sin nx)dx 
  = \frac{1}{2\pi}\int_{-\pi}^{\pi}f(x)e^{inx}dx
\end{gather*}
Общая формула:
\[
  c_n=\frac{1}{2\pi}\int_{-\pi}^{\pi}f(x)e^{-inx}dx
\]
\begin{remark}
  $c_{-n}=\bar{c}_n$, если $f(x)$ - действительное.
\end{remark}

\section{Метрические пространства}
\begin{definition}
  Множество $X$ называется метрическим пространством,
  если любой паре $x,y\in X$ поставлено в соответствие $\rho(x,y)$ (метрика или расстояние),
  такая что выполняются аксиомы:
  \begin{enumerate}
    \item $\rho(x,y)=\rho(y,x)$
    \item $\rho(x,y) \le \rho(x,z)+\rho(z,y)$ - неравенство треугольника
    \item $\rho(x,y)=0 \iff x=y$
  \end{enumerate}
\end{definition}
\begin{property}
  $\rho(x,y) \ge 0$.
\end{property} 
\begin{proof}
  $0=\rho(x,x) \le \rho(x,y)+\rho(y,x)=2\rho(x,y) \;$ $\forall x,y\in X$
\end{proof}
\begin{remark}
  Любое подмнож. метрического пространства является метрическим пространством.
\end{remark}
\begin{eg}
  Мн. $\R$ - метр. пр-во $\rho(x,y)=|x-y|$
\end{eg}
\begin{eg}
  Арифметическое евклидово пр-во $\R^{n}$, $\rho(x,y)=\sqrt{\sum_{k=1}^{n}(x_k-y_k)^{2}}$
\end{eg}
\begin{eg}
  Мн-во $B([a,b])$ ограниченных на $[a,b]$ функций,
  \begin{gather*}
    \rho(\phi(x), \psi(x))=\sup \limits_{x\in[a,b]}|\phi(x)-\psi(x)| \\ 
    |\phi(x)-\psi(x)|=|\phi(x)-\alpha(x)+\alpha(x)-\psi(x)|
    \le |\phi - \alpha| + |\alpha - \psi| \le\\
    \le \sup \limits_{[a,b]}|\phi-\alpha| + \sup \limits_{[a,b]}|\alpha-\psi| 
    =\rho(\phi,\alpha)+\rho(\alpha,\psi)
  \end{gather*}
  Перейдём в неравенстве к $\sup$:
  \begin{gather*}
  \rho(\phi,\psi)=\sup \limits_{[a,b]}|\phi -\psi| \le \rho(\phi,\alpha)+\rho(a,\psi)
  \end{gather*}
  Доказали неравенство треугольника.
\end{eg}
\begin{eg}
  Мн-во $CL([a,b])$ непр. функций на $[a,b]$ с метрикой
  $\rho(\phi, \psi)=\int_{a}^{b}|\phi(x)-\psi(x)|dx$
  является метр. пр-вом.

  2 аксиома: $|\phi-\psi|\le |\phi-\alpha|+|\alpha-\psi|\implies \rho(\phi,\psi)\le \rho(\phi,\alpha)+\rho(\alpha,\psi)$

  3 аксиома: $0=\rho(\phi,\psi)=\int_{a}^{b}|\phi(x)-\psi(x)|dx \implies \phi \equiv \psi$

  Если не требуем не непрерывности, то $\rho(\phi,\psi)=0 \not \implies \phi \equiv \psi$,
  т.к. функции могут отличатся в отдельных точках.
\end{eg}
\begin{definition}
  Посл. $\{x_{n}\}$ элементов метр. пр-ва $X$ называется сходящимся, если
  \[
    \exists x_0\in X: \forall \epsilon>0  \;\exists N: \forall n\ge N \true \rho(x_n,x_0)<\epsilon
  \]
  (или $\lim \limits_{n\to\infty} \rho(x_n,x_0)=0$).

  В этом случае считаем, что $\lim \limits_{n\to\infty}x_n=x_0$.
\end{definition}
\begin{definition}
  Посл. $\{x_n\}$ называется фундаментальной, если
  \[
    \forall \epsilon>0 \; \exists N:\forall n,m \ge N\true \rho(x_n,x_m) < \epsilon
  \]
\end{definition}
\begin{definition}
  Метр. пр-во $X$ называется полным, если в нём любая фундаментальная посл. сходится.
\end{definition}
\begin{theorem}
  Любая сходящаяся последовательность фундаментальна.
\end{theorem}
\begin{eg}
  Полные метр. пр-ва: $\R$, $\R^{n}$.
\end{eg}
\begin{eg}
  Неполные метр. пр-ва: $(0,1)$, где $\rho(x,y)=|x-y|$. 
  Здесь $\{\frac{1}{n}\}$ -фунд, но не сход.
\end{eg}
\begin{eg}
  $Q$ - мн. рац. чисел не явл. полным метр. пр.
  $\{(1+\frac{1}{n})^{n}\}$ - фунд., но не сход.
\end{eg}
\begin{remark}
  В метр. пр-ве, так же как и в $\R^{n}$, можно ввести понятия окрестности,
  внутр. точки, граничной т., т. прикосновения., предельной т.,
  замыкания, отк. множества и т.д.
\end{remark}

\subsection{Линейные нормированные пространства}
\begin{definition}
  Множество $X$ называется линейным пр-вом,
  если в нём определены сложение $(x+y)$ и умножение на число $(\alpha x)$:
  \begin{enumerate}
    \item $x+y=y+x$
    \item $(x+y)+z=x+(y+z)$
    \item $\exists 0: \: x+0=x$
    \item $\forall x \in X \; \exists (-x): \: x+(-x)=0$
    \item $(\lambda+\mu)x=\lambda x+ \mu x$
    \item $\lambda \mu)x=\lambda(\mu x)$
    \item $1 \cdot x=x$
  \end{enumerate}
  Определено вычитание $x-y=x+(-y)$.
\end{definition}
\begin{definition}
  Система векторов называется линейно независимой, если
  любая конечная подсистема линейно независима.
\end{definition}
\begin{definition}
  Линейное пространство $X$ называется нормированным, если
  на нём определенна действительная функция (функционал) $\Vert x \Vert$,
  такая что выполняются аксиомы:
  \begin{enumerate}
    \item $\Vert \lambda x \Vert=|\lambda|\Vert x \Vert   $
    \item $\Vert x+y \Vert \le \Vert x \Vert + \Vert y \Vert$
    \item $\Vert x \Vert=0 \implies x=0$
  \end{enumerate}
\end{definition}
\begin{property}
  $\Vert x \Vert \ge 0$
\end{property}
\begin{proof}
  \begin{gather*}
    0 = \Vert x-x \Vert \le \Vert x \Vert+\Vert x \Vert = 2 \Vert x \Vert
  \end{gather*}
\end{proof}

\end{document}
